\documentclass[autodetect-engine,dvipdfmx-if-dvi,a5paper,ja=standard,twoside,titlepage,final,twocolumn]{ltjtbook}
\usepackage{luatexja}
\usepackage{luatexja-fontspec}
\usepackage{luatexja-ruby}
\usepackage{lscape}
\usepackage{pifont}
\usepackage{subfiles}
\usepackage{color}
\usepackage{luatexja-otf}
\usepackage{comment}
\usepackage{multicol}

%\setmainfont{ipaexm}

\setmainjfont[
  TateFeatures={JFM=propv}, %JFMを指定する
]{ipaexm.ttf}


%\setlength{\columnsep}{4\zw}
%\columnseprule=0.4pt

\begin{document}

\title{\tt \Huge Isolation Heart}
\author{}
\date{}
\maketitle


\begin{landscape}
    {
    \centering
    Earth Coincidence Control Office\\
    ECCO is always near to you.\\
    We are given myself by our sense,\\
    we are been to be tied to it.\\

    \if0
      A long long time ago, I lost my body.
      But I'll have felt a pain.
      Ever ever ever ever ever...
      I...I can't bear this ache any longer.
    \fi

    \ding{"76}
    \begin{center}
    \parbox[t]{0.5\linewidth}{\tt
    QSBsb25nIGxvbmcgdGltZSBhZ28
    sIEkgbG9zdCBteSBib2R5LiBCdX
    QgSSdsbCBoYXZlIGZlbHQgYSBwY
    WluLiBFdmVyIGV2ZXIgZXZlciBl
    dmVyIGV2ZXIuLi4gSS4uLkkgY2F
    uJ3QgYmVhciB0aGlzIGFjaGUgYW
    55IGxvbmdlci4=}
    \end{center}}
\end{landscape}

\newpage

\subfile{sections/IHSection1}
寂しさに潰れそうな私を、\ruby{瀬玲奈}{セレナ}がそっと手を握ってくれる。
友達同士で手をつなぐことなんて、今までになかったから、その暖かさに驚く。

そして、寒さに身を寄せ合う私たち二人を前に、彼は語り始めた。\\
「君たちには、悪夢を狩ってもらいたいんだ」\\
子ども番組に出てくるキャラクターのような愛くるしい声と、その異質な姿。
「悪夢?それって、あの時のヤツみたいな?」\\
漂う光る球体に、瀬玲奈がそう聞いた。

テスト\colorbox{black}{adsf}

\parbox<y>{\linewidth}{
\begin{landscape}
\jfontspec{g_pencilkaisho_free.ttf}
「私たちは感覚によって自らをあたえられ、そしてしばり付けられている。」
どこかで聞いた覚えのある言葉だけど、よく覚えていない。\\

痛みが消えて、自分を失う。\\
そんな人間を、私は何人も見てきた。
\end{landscape}
}
%{\jfontspec{g_pencilkaisho_free.ttf}
%あああああ}
%あああああ

\chapter{\rm 夜の始まりへ}
\section{\rm 狩猟の街}
人気のない路地裏。
街灯も消え寝静まった夜には、昼の街並みとは違う何かがそこにあった。\\
「今日から君たちは『狩人』になる。そしてそれには必ず危険が伴う。まずはそれを理解してほしい」\\
目の間に浮かぶ球体が幼い子どもの様な声色でそう喋った。
可愛らしいマスコットのようなそれは、しかし私達の常識の外側にいる者。
『星の使者』彼は自らをそう名乗っている。
些か信じがたいが、彼らはこの星の外からの来訪者だという。

宇宙人の言葉を信じるのなら、私たちは今から未来を守る為に戦うらしい。
身の毛もよだつ恐ろしい何かと、
私たちは彼らのもたらす『武器』あるいは『力』を手に取り、夜の静かな狩りを行うのだ。\\
「\ruby{華南}{かなん}、これが君の武器、戦うため守るための力だよ」\\
「これが私の武器?」\\
渡されたモノは、夜目にも一際目立つ真っ黒な武器だった。
一つは小さい、というかよくテレビとかで見る拳銃そのもので、ずっしりと重たいが不思議と持ちにくさはない。
まるで何年も使い古され、完全に自分の手に馴染んでしまった様な感触。
引き金を引くまでの所作を違和感なく行える。
もう一つは私の身長の半分はあるかもしれない長さと、威圧感の装飾を纏った銃だった。
片方よりも更に重たく、素人目に見ても持ち上げて撃つものではないことは分かった。
長方体の集合、直線によってのみ構成された\ruby{番}{つがい}の武器は、
その銃把を手に握ると瞬く間に思考がクリアになり、
それぞれの持つ特性、適切な運用方法が頭の中に入ってくる。
それをマニュアルによる情報と言うより、
\ruby{先天的}{アプリオリ}な事実として理解させられたことに、彼らの持つ技術力の高さを思い知らされる。
無機質なそれは外見を裏切らず、二つとも戦う以外の役割を一切捨てていた。

そう、私たちは戦うのだ。
その為に私たちは彼との契約を結び、『使命』を背負う。
そしてその完遂の暁には各々の願いを叶えるという『対価』を支払うという。

だけど、私はまだその対価を決め兼ねている。
誰しもが初めから定めているわけではない、そう彼らは言っている。
けれど今の私にはこれと言って不満があるわけでもない。
叶えたい夢もない。
だから、立ち向かうことに戸惑いを残してしまう、決意みたいな物が全くない。
だったら、私がどうしてこんなことをするのか。……はっきり言うと自分でもまだよくわからない。
成り行きでなってしまった以上、そんな言い訳を言える立場ではないけれど、
それが今の嘘偽りのない気持ちだった。
ただ少しだけ思うのは、今、私の胸に巣食うこの迷いが晴れていくこと、それを願っているのかもしれない。
生きてきた中で望むことはしなかったけど、誰かのために何かをしたこともなかった。
頼るわけでも頼られるわけでもない、宙ぶらりんなこの心を何処かに落ち着かせたい。
カッコつけるつもりはないけれど、守りたいものが欲しい、大切にするべきものに気づきたい。
しいて言えばそれが理由だろう。
それに、私の傍らに立つ\ruby{瀬玲奈}{せれな}が一緒だというのも後押しになった。

……何がどうであれ最早戻ることは出来ない。
星の使者が語る言葉は、まるで戦地に赴く兵士に向けられた煽り文句に
聞こえてならない。
一抹の不安が、私の体を固くさせる。\\
「緊張するでしょ。でも大丈夫。初めから出来る人なんて誰もいないわ。
私だって成り立ての時は失敗ばかりだったから」\\
そう言って私を勇気づけてくれたのは、奇しくも同じ高校の先輩である\ruby{彩芽}{あやめ}だった。\\
「でも、彩芽さんは私なんかよりもずっと勇気があるじゃないですか。
あの時、私達を助けてくれた時みたいに」\\
「そうね、でも勇気なんて慣れみたいなもの。何事も経験あるのみ。
しっかり私について来れば大丈夫。後輩にはかっこいいとこ見せないとね」\\
「じゃあ期待してますよ!アヤ先輩」\\
まるで子供のように目を輝かせてはしゃぐ瀬玲奈。
その姿を見れば、彼女は本当に自分で望んだことなんだと分かる。
彼女はいつも優柔不断な私と比べて、はっきりとモノを言う性格で、どこまでも前向きだ。
だから彼女は後悔をしないし、いつも最後までやり通す。
時々それが作り物に思えてしまうほど、真っ直ぐで力強い彼女の生き方。
きっと私なんかとは比べようもなく強くなっていくだろう。
だから私もそれを見習って、たくましく生きていきたい。
その為に今私は闘いに身を興じるのだろう。

「さあ、そろそろ行きましょうか」\\
彩芽は踵を返し、歩き始める。
それについていく私と瀬玲奈。
夜の暗闇が無性に怖かった。
ふと後ろを振り返ると、そこにさっきまでいたはずの星の使者の姿はなく、ただ声だけが残っていた。\\
「二人共目覚めることを忘れないように。明日の光は常に訪れるのだから」\\
希望に満ちた激励かあるいは警句か、どちらにせよ少しは気が晴れる、そんな気がした。

\section{初戦}
真夜中の大通り。
建物の隙間に隠れ、獲物を偵察している彩芽。\\
「見て、あそこにいる」\\
彼女が目を配る先には、『悪夢』がいた。\\
点々と光る電灯だけが頼りのこの狩場で、私は初めて獲物を眼の前にする興奮を覚える。
「あれが、私達の獲物」\\
自然と口から溢れる言葉。\\
「そう、あれが『悪夢』。ヒトの心に巣食って、最後には食い潰す。
あれに取り憑かれればひとたまりもないわ」\\
「あんなに大きいなんて……。あの時のはもっと小さかったのに」\\
瀬玲奈が驚くのも無理はなかった。
私達の身長の3倍ほどはある巨体。
恐怖を感じないほうがおかしいだろう。
幽かに揺らめくその体は、電灯に照らされ異質な姿を私達に示す。
まるで抽象画の世界からひょっこりと出てきたような化物。
緩やかな楕円と鋭利な三角形が組み合わさった胴体に、波のように幾何学的な模様が絶えず動き回って、眼が痛い。
そして、現実離れした異型からところどころ生えたヒトの手足。
ただそれだけが纏う現実感が、私の頭を混乱させる。\\
「そうね。あれはかなり育っている奴よ。たぶんここ最近悪夢を見なかったのも、あれが共食いしていたからだと思う。
……初めての相手にしては少し強すぎるかも」\\
けれど、彩芽はいたって冷静だった。
影に身を潜め、獲物の動きを見極めている。\\
「アイツ、かなり太ってるから、動きは鈍いようね。
一発一発の攻撃は重たいかもしれないけれど、避けることはそんなに難しくない」\\
「でも、私たちは\scalebox{3}[1]{―}」\\
「分かってる。だから華南は私を援護してくれればいい。
丁度それが出来る武器だし、相手の注意を私が引いていれば攻撃される事はないでしょうし」\\
「あの、私はどうすれば?」\\
「瀬玲奈は私とついて来て。相手の後ろに回り込んで、とにかく斬りつけるの。大丈夫、戦い方は武器が教えてくれる」\\
「分かりました。頑張ります」\\
さすがの瀬玲奈も緊張しているのだろう、額に汗がにじみ出ている。
私も心臓がバクバクして、息をするのが辛い。\\
「3つ数えたらいくわよ。準備はいい?」\\
二人共頷いて返事をする。
3,2,彩芽が指で数える。
\scalebox{3}[1]{―}1。\\
「出るわよ!瀬玲奈走って!」\\
「はい!」\\
物陰から飛び出し、彩芽と瀬玲奈は一目散に悪夢へと飛びかかる。
私は銃を、銃に教えられるがまま、見よう見まねで構える。
彩芽はまるで動物のように速く、すでにその間合いを数歩のところまでに詰めている。
洗練されたその動きにはある種の美しさを感じる。
背後に迫る人影に気付いたのか、悪夢はその図体のっそりと動かして、私達を見る。


一瞬、何故か目があった。
そんな気がする。
前線の二人ではなく私に狙いをつけたのだろうか。
だとしたら、やられる。
背中から汗が吹き出る。
あるはずのない眼に追われている。
焦燥感はじわじわと私を締め付けていく。

怖い。

……いや、そう感じただけだ。
恐怖を押し込んで私も前に出る。
銃を構えて照星に目標を合わせる。
すると、震える手は自然と静まり、
\section{\rm 邂逅}

「危なかったわね」\\
大人びた少女の声に、私は我に返る。
地味な衣装の上に暗い緑色のロングコートを着込んだ女性の姿は、おおよそ現代、特にこの国では目にすることの
ない格好、しいて言えばおとぎ話の中にいる人物のようだった。\\
「もう少し遅かったら駄目だったかもしれない。二人共怪我はない?」\\
なんというかとても親近感のわく人だ、そう感じた。\\
「あ、はい。大丈夫です」\\
まだ状況をうまく理解出来ていない、そんな感じで返事をする瀬玲奈。
私もまだ何も把握できていない。特にどうして何も出来なかったのか、あんなにも怖かったのか。
安堵の言葉や感謝の礼よりも先に、その疑問の解決を私は望んだ。\\
「……あの、さっきのアレって何だったんですか。私、なにも\scalebox{3}[1]{―}」\\
「逃げることさえできなかった、なんて当然ね。あれはあなた達の常識が通用しない相手。
私たちはあれを『悪夢』と呼んでいるけど、正直に言えばほとんど何も分からない。
形状も千差万別、知能がありそうで無さそうな不可思議な行動。
それでいて人の精神に干渉、理解していると言ってもいい、あなた達がここに誘い込まれたのもそのせいよ」\\
「誘い込まれた?」\\
「ええ、そうよ。あなた達最近悪い夢なんか見たりしてないかしら?暗い、陰鬱な、逃げ出したくなる様な夢」\\
「見たことあります。ていうか、今日見ました。
なんかこう、暗闇に引きずり込まれて、ずっと溺れてるみたいな夢でした」\\
瀬玲奈の言う夢と私の見た夢はほとんど変わらない。
同じ夢を見た、と言うと彼女はやっぱり、と言った。\\

\chapter{\rm 狩人、その使命}
\section{\tt \large after\_awakening}
夜の出来事がまるで嘘だったような、苦痛のない目覚め。むしろ、普段よりも快い。
ふと脚を触る。擦り傷などどこにもなかった。

学校からの帰り道、血染めだった道に触れる。
あの夜、初めての獲物を狩ったあの時と、見たもの触れたものは全て同じだった。

まるで変わらない、アスファルトのザラザラとした痛み。\\

〈それが、過去と現在とを共通する\ruby{感覚}{ワタシ}だと誰が証明できるのだろうか〉

\chapter{\rm 溺れる魂、付随する肉体}
\section{\tt \large isolation, break}
突き刺さった剣、血まみれの体。
コンクリートの壁にもたれ掛かるエナの先には、突っ伏せた瀬玲奈の体が夥しい血とともに転がっていた。\\
「痛い、痛い痛い痛い痛いイタイイタイイダイ、イダイイダイ……。なんで、こんなに、痛いのよ。
どうして、はぁ、ぁ、目が醒めないのよ!」\\
腹の底から湧き上がった彼女の憎しみの声。
聴くものを道連れにしようとする怨嗟。\\
「嫌だ。死にたくない。死にたく、ない」\\
生きることを望み死を嘆く声は、やがて生きとし生けるものへの呪詛となり、その狂気にも似た生への執着を
露わにする。\\
「\scalebox{3}[1]{―}殺してやる。殺してやる。殺してやる。殺してやる。殺してやる。
殺す、殺す、殺す、ころす、コロス、コロス、コロス……!」\\
苦しみの声は全てを呪い、理想に侵された体は死にゆくばかり。

そう、瀬玲奈は死んだ。
肉体は紛れもなく死に絶え、すでに冷たく、生きることを諦めている。
それなのにまだ動く。彼女の執念。短くしかし強大な妄執が、現実すら冒し、歪め始めている。

\scalebox{3}[1]{―}心は、魂は、何かは、
『紅上瀬玲奈の生存する』世界への収縮を渇望する\scalebox{3}[1]{―}。\\
「うぅ、ヴァァァァァァ、アアアアア。アアアアア、ッ……」\\
理性を失い、宿痾に敗れ、心を引き裂かれた悲鳴。

華南はもう耐えられなかった。ただそれだけだ。けれど彼女が瀬玲奈を殺めるには十分過ぎる理由だった。

瀬玲奈の首にそっと手をかける。

力を込める。

吐息を感じる。

生ぬるく。

……冷たい。

\scalebox{3}[1]{―}。

紅上瀬玲奈はここで死んだ。少なくともそれ以外の\ruby{可能性}{セカイ}は、ない。\\

息の荒い、魂の底から生きることを望んでいる、彼女には似合わないその呼吸は、けれど死の間際を克明に記している。
何人であろうと、耳と目を閉じ口を噤みその事実から意識を逸らすことなど、許されるはずはない。
それは死にゆく二人を看取る華南にとっては尚更だろう。
だが、それはあまりにも酷い。

流れ出す血を飲み込みながら、エナは華南に向けて話し始める。\\
「これが、現実から逃げ続けて来た愚か者の末路\scalebox{3}[1]{―}置いてきたはずの体もいつの間にかここにいる」\\
まるで自嘲の様な文言は、彼女の諦めを鮮明にしていく。\\
「ああ、こんなに血がいっぱい。\scalebox{3}[1]{―}体が冷たい。全部、夢だったのに。
流れる血も、痛みも、体も、全部、全部、目覚めれば何もかも消え去っていく。
それで気づけばよかったんだ。
本当の自分、その在処を。だったら、もう少しマシな生き方が出来たかもしれないのに」\\
息を大きく吸うエナ。血反吐を吐きながら咳き込む姿は、枯れていく老人の様だった。\\
「ああ、でも、そんなこと考えないのが普通、よね」\\


\chapter{\rm 幼年期の終わり、ヒトの終わり}
\section{\tt \large endless\_guilt}
終端の広間に舞い落ちる小さな球体。その表面は\ruby{水面}{みなも}のように揺らいでいる。まるで原初の海、すべてが
混沌とした暗い色に溶け込んでいたように。
それこそが最後の悪夢、幼年期の楔。
これを解き放てばすべてが終わる。文字通り、すべてが。
だから私はそれを殺さなければならない。
だがそれはあまりにも弱弱しく、かつての獣のような悪夢たちに比べれば、愛くるしさすら感じる。
まるで自らの赤子のように、子供を産んだことすらない私にさえそう感じるのだ。
その感情に刹那、心は揺らぎ、刃を握る手から力が零れ落ちそうになる。
だが、やらなければならない。
私の後ろに立つ彼女……彼女たちとの契約を果たさなければならないのだから。\\
「万城目華南、君の使命を全うするべき時だ。その身に誓った約束、忘れたわけではないだろう」\\
「ええ、わかってるわ。私が、すべてを終わらせる。その罪を背負って……」\\
「君の罪ではないよ。それは僕たちが贖うべき罪だ。僕たちの終わりなき殉教。
狂信にも似た、けれど取り返しのつかないと理解しているこの螺旋運動を、僕たち以外の知性体が続ける必要はない」\\
彼らの独白はいつも哀しみに溢れている。
厭世観と、しかし使命感に満ちたその言葉の重さを、私は今になってようやく理解できる。
彼らとの対話にあった気だるさはその相互理解の欠落だったのだ。

\chapter{\rm 彼方の断章}
\section{}
かすかに聞こえる銃声、爆音。無論、無機質なスピーカーからの音でしかない。
生き残るべき人間を選定するための戦争。
理論上の最大値、百万人へと近似させる為の計画的殺戮。
自由を謳う連合も、秩序を敷く共同体も、互いに争う中、その使命を共にしているにすぎない。
欺瞞に満ちた、無意味なこの行為。
多重の防壁に囲まれたこの聖域に座する私達十六人は、許されるべきではない。
私はふと思った。
今すぐこの扉を開き、武器を手に取り、淘汰の世界に身を窶すべきなのだろうか。
いや、それはダメだ。
計画の末、新たな領域に引きずりあげられた人々を導く存在が必要なのだ。
その為に私たちはわざわざこの冷たい棺に引きこもり、淘汰を免れているのだ。
残された時間も、あと僅かになる。
ターミナルにはシステムの各通知が夥しく表示され、刻一刻と刻まれるカウントダウンに思える。
そして、コードの入力を求められる。
事前に決めた任意の文字列を入力し、承認をする。
それが十六人分完了すれば、全てが終わり、そして始まる。

\chapter{\rm 構想}
\section{}
不意に近づくエナ。重ねられた唇はしっとりと濡れている。
状況を理解出来ないまま続けられる行為は、拒否する暇を与えない。
押し入ってくる舌が私の舌と絡まり、そして抜けていく。
瞬間、漂う風味に私は咽る。
血だ。紛れもなくそれは血の味。
それが彼女の血であることは疑いようもなく、だからこそ、私は得も言えぬ嫌悪感を抱く。\\
「これが血。死ぬこと傷つくことの味」\\
噛みちぎられた下唇から流れる舐め、神妙な顔つきを見せるエナ。
何かを達観したかように、彼女は語り始める。\\
「小さい時、ふざけて錆びた鉄棒を舐めてみた事があった。
錆びた鉄の味は、血の味とよく似てる。でも、何か違う。
同じ鉄の味なのに、生き物の味は、生きている味はこうも私達の感情を刺激する。
今アンタが気持ち悪いって思ったように」

\section{}
天と地を結ぶ扉。
かつて私達が辿り着いたと思い込んでいたそれは、その実妄想に過ぎず、結局私たちは物質に囚われたままだった。
凍えた光の格子の中に、その意識を移したとしても、その本質は変わらなかったのだ。
だが今は違う。天上に開く禍々しい、まるで獣の口の様な孔は、
けれどこの世界と『何か』をつなぐ唯一の形であり、我々には辿り着けなかった尊きものなのだ。

シナプス\scalebox{3}[1]{―}タンパク質の壁に囲まれ、
束縛されていたヒトの魂は、その姿を容易には晒さなかった。
かと言って頭蓋を切り開き、脳を弄っても誰もそのカタチを見ることは出来ない。
それは名状し難い、言語的説明の付かない方法によって原始の生命と癒着し、
過去、現在そして不確かな未来をも結ぶ、意志の不可視な苗床となったのだ。
だが今やそれは我々にも見える形となって天上へと昇っていく。
深青の尾を引き、純白の衣を棚引かせ、新たなる世界を祝福する彼らの姿を見て、私たちは得も言えぬ感慨に浸った。

かつての私ならば、この結末に憤慨し、落胆するのだろうか。
少なくとも己の無力を恥じるだろう。
まさにそれは今私が感じているのだから。
だが嘆きはしない。怒りもしない。
むしろ歓びに近い高揚が、私の心を支配していた。

本体との通信が途絶した。極点に近いほど崩壊の度合いは緩やかになるのだろう。
爛々とし、世界を遍く天使の輪。
そのあまりの美しさに見惚れている中で、ほんの僅かに、聴覚のノイズを感じた。

明らかに人工的な音色。

周期的に揺らめく流れ。

不思議と心地よかった。

瞬間、それを理解した。

新たなるを讃える頌歌。

それに違いない。

ああ、見たまえ。

\scalebox{3}[1]{―}今宵はこんなにも星空のきれいな夜だ。
\section{}
それは現状の有機物を構成する水素や酸素といった非金属元素や、
生命活動に必須なナトリウムなどの金属元素の性質と極めて類似している。
しかしそれの内部構造はまったくもって確認できない。
現行の観測手段を尽く拒絶し、その神秘を隠し通している。
ただ、水素原子との相対比較により分かる質量はおおよそ電子の整数倍と等しく、しかも不確定性原理から逸脱し、
位置情報と質量を同時に確定することが出来ている。

\section{}
「本日未明、河川敷の高架橋下にて是枝彩芽の死体が発見された。
我々の検分では遺体に目立った外傷はなく、また薬物の服用も認められなかった。
死因はおそらく餓死、死後一ヶ月ほどは経過している。
推測するに何らかの理由により意識不明となり、その後死亡したと考えられる。
すでに警察に通報し、遺体は回収。
事件性は認められず、問題なく事故という形で処理されるだろう。

だが、我々の方は大問題だ。このような事例は今まで確認されていない。
理論上考慮不可能な状態の発生に、委員会は現在でも集中協議を重ね、対応を模索している。
尤も依然として重要視されるのは計画の遂行性であり、それが保証されれば別段この事件を気にかける必要はない。
しかし現在我々が注視している集団、橘春香を中心とするグループについては早急な対処が求められる。
現状狩人にとって最も脅威となる存在である彼女たちは、我々の計画において、明確な不穏分子である。

特例によって開示された情報を持つ君にとっても、最早他人事では済まされない。
彼女たちは君をあからさまな殺意を持って狙うだろう。
何のための君に我々の記録を明かしたのか、理解出来ていない訳ではないだろう」

\section{}
ざーざーと耳に刺さるシャワーの音、規則的な秒針の音、胸の内から響く鼓動でさえも、
この静かな夜の世界では私の意識を乱すに事足りる。
ふと気がつけば、ベッドに伏せながら、じっと一点を見つめている自分がいた。

風呂場から漏れ出る光。
中には瀬玲奈が入っている。
私は彼女のあとにシャワーを浴びるつもりだったが、彼女はやけに長く入ったままだった。
日頃、彼女はもっと早く、長くても十分ほどで出て来るはずなのに、今日はもう二十分以上も経っている。
普段ならそんなことを気にする必要はないのだろうけれど、
私の頭の中には、あの時のうなだれていた瀬玲奈の姿が離れなかった。
彼女は今、苦しんでいるのだろうか。
だとしたら自分は何をすれば良いのだろうか。

ためらいはしたけれど、やはり、居ても立ってもいられなかった。
ベッドから起き上がり、風呂場で服を脱ぐ。\\
「……はいるよ」\\
声をかけても、中から返事はなかった。
それでも扉を開ける。
垂れ流されているシャワー。
熱気が充満した室内の、浴槽の縁に、瀬玲奈は足を抱えて座り込み、
その綺麗な白金色の長髪を垂らして俯いていた。
入り込んだ冷気に気付いたのか、顔を上げ、驚いたように私の方を見る瀬玲奈。
その顔は心底疲れ果てていた。\\
「あっ、先輩。……すいません、お湯出しっぱなしにしてて」\\
あくまでも彼女は普段通りのままでいたいのだろう。
でも。\\
「そんなことじゃないでしょ」\\
「えっ\scalebox{3}[1]{―}」\\
「嫌なこと辛いこと、なんでもかんでも一人で抱え込まないで。

\scalebox{3}[1]{―}何のために、瀬玲奈は今、ここにいるの?」\\

Sometimes, we had been thinking a one thing; 
About that this world which we live in 
was made by either some great one like a god or god himself.

「私達は時折、神秘主義的な実在論者になるんだ。
この世界は確かに、数学や論理的記述によって完全に普遍的に表されて、私達自身もまたその一部であると。
そしてそれらは、不可知で一意な創造者によって設計され、彼もまたその中に内包されていると。
私達はただそれの自覚をし続けているだけで、
言い換えればプラトンのイデア界、その地図を作っているだけだと。
だったら、こんなことに意味はあるのだろうか。
私達のこの繰り返しさえも、それらに予め記述された順路だとしたら。
いや、これ以上はやめよう。
ただ一つ言いたいことは、
私達は決して君たちよりも遥かに優れているということではないということ。
この世界の真理を知らない、ただの愚者であるということ。
そして、未来も過去もない、\ruby{孤児}{みなしご}であるということ。」

「宇宙的慈善活動家。誰が言ったのかは覚えていないが、私はこの言葉を大いに気に入っている。
まさしく我々そのものだ。一歩道を踏み外せば、それは偽善にもなるし、独善にもなるし、
だが正しくあれば本当の善にもなりる。」

この世界にはなぜか、この世には自分と同等か或いはそれ以上に賢い人間しか存在しないという
不確かな事実を盲信していて、それに当てはまらないと『勝手に見なした』人間を、
馬鹿だの愚か者だの、アイツは古いなどと不必要に攻撃する人間がいるのだ。
そしてそれは、本人の賢さには依拠しない。

Similar to that vegetables is got to rot, we will falling down to endless distance.
But fragment of our universe will continue to remain, as the shell of the egg does not rot.

孵卵主義者によると、現在の宇宙のエネルギー準位は理論上よりも早く下がっているという。
彼らに言わせてみれば、それは卵の中の栄養が枯渇しつつあることの証拠だという。
まったく馬鹿らしい。

\section*{\tt Wound's reality}
最初は単なる事故だった。
本当に、本当に、本当に、本当に。
手を滑らせてカッターを落としてしまった。
驚くほどきれいに、刃は私の手首から腕にかけてを切り裂いた。
5センチ程だろうか。
幸い傷は浅く、病院には行かなくても良かった。
傷もいずれ消える。
けれど、表皮の下を切り裂かれた痛みは、『痛くない』とやせ我慢できるほど、
生易しいものではなかった。

左腕の痛みに悶えて、涙目になる。

痛い。

だけどその時に気づいてしまったのだ。

痛みは、私の心を晴らしてくれるということを。

それはもう嫌というほどに、生きている心地がするのだ。
それと共に、またあの『あやふや』の中に引き戻されることを恐れるようになった。
生きているのかも、死んでいるのかも分からない、永遠の中へ。

恐れはいつか、もっと現実的な行動へと変化していった。\\

\scalebox{3}[1]{―}生傷は疼き、
かさぶたを剥がせば、また血は滲み出す。
痛みが戻る。

結局、その傷が癒えることは、なかった。

心の鎮痛剤。
痛みは万能の処方。
けれど、耐性はあらゆる薬に対して生まれる。
痛みにしてもだ。

ダメだとは分かっていた。

カッターを取り出して、刃を出す。
……血に錆びていた。
歯切れの悪い刃は、強く押し当てるだけでも痛かった。
皮膚が引っかかるような感覚。
ピリピリとする。
いつの間にか切れていた。

恍惚な私。
血の赤は鮮烈で、私を酔わせる。
それは確かに私の心を握りしめ、放さない。
同時に私は噛み締めているのだ。
ああ、生きている、と。\\

血は固まって、赤黒い。
またやっちゃった。
後ろめたさを無視することが出来ない。
血を拭って、まだ暑いのにパーカーを羽織る。

白いパーカーだ。

血が滲み出して、赤いシミにならないだろうか。
誰かにこれが、バレてしまうのだろうか。

そう思うと私は\scalebox{3}[1]{―}。\\

これ以上はやめようと思った。\\

じゃないと私は、誰からも愛されないから。

\subsection*{}
自殺は「心理的視野狭窄
(逃れられない心理的苦痛から解放されるには自殺しか無いと考えたりすること)」
の末に行われるものだという。
しかし自傷は、「自殺が、脱出困難な苦痛を解決するために、
『意識を永遠に終焉させる』方法であるのに対し、
自傷は、自分の意識状態を変容させることで何とか
苦痛を『一時的にしのぎ』、その瞬間を『生き延びるため』に行われる。」という。
或いは「自殺とは『苦痛しか存在しない世界からの脱出』」であり、
「自傷とは『苦痛に満ちた世界を耐えしのぶこと』」である。\\

「自傷とは、自殺以外の意図から、非致死性の予測をもって、
故意に、そして直接的に自らの身体に対して非致死的な損害を加えること。」\\

自傷者の自殺リスクを高める原因は、辛いときに周囲に援助を求めることが出来ないからである。
援助者は自傷行為を頭ごなしに否定してはいけない。それは自傷者の否定であり、援助希求能力を潰すことになってしまう。
援助者はまず、自傷行為を告白したこと、治療の場に赴いたことを肯定するべきである。
自傷は本人にとって望ましいことではないが、そうしなければ他者に暴力を振るったり、自殺してしまったりしていただろう。
他人を傷つけるよりも自分を傷つける事が悪いと思う人間はいない。
たとえ自傷者が何ら深刻さのないあっけらかんな態度をとっていても、それは自傷によって一時的に辛さを抑えているだけである。
\begin{quote}
たとえば「切っちゃった、テヘッ」といった態度を示す彼らの真意は、「たしかに自分を傷つけてしまったけれど、それでも自分を大切にしたい」という気持ちがあるのだ、と理解すべきなのです。
ですから、傷の手当てに訪れた彼らに対する第一声は、こんな言葉にするべきです。「よく来たね」。
\end{quote}
「自傷・自殺する子どもたち」松本俊彦

\end{document}