\documentclass[../IHMain]{subfiles}

\begin{document}

\chapter{人の記憶}
\section*{\tt if\_you\_love\_me}
呼び出しは成功した。

昼休み、I科棟の裏で。

メッセージには確かに、既読を表すマークが付いている。
正確な時刻は指定していないけれど、きっと来てくれると信じていた。

「これ、呼んでください」\\
明らかに震えている声だったが、口から出たことを評価してほしい。
渡した紙に書いたのは、詩だ。
下手くそだとは自分でも分かっているが、それでも、なんというか、
直接的な表現は気恥ずかしく、悶てしまうのだ。
\begin{verse}
    怖いんだ\\
    さめてしまうのが\\
    だからこの思いを\\
    小箱にしまって\\
    鍵をかけたい\\ \\

    けれど、恐怖はいずれ鍵穴から\\
    ひっそりと抜け出して\\
    戻ってくるんだ\\ \\

    だから君と一緒に閉じたい\\
    鍵穴を二つにわけて\\
    出口を細めて\\ \\

    そうすれば、\\
    わずかに溢れるだけだから\\ \\

    繰り返しくりかえし\\
    鍵を回して\\
    穴を狭めて\\ \\
    
    僕は安心して\\
    君と眠りたい\\
\end{verse}
やっぱりこういう表現も恥ずかしい。
だが、渡してしまったものを奪い取ることは出来ない。
私は俯くことしかできなかった。
彼女の顔を見ることが出来ない。
凛々しい、沈黙の似合う彼女の顔を。

人生で一番長い時間だった。
バクバクと鼓動する心臓の流れは早く、しかし彼女の返事は永遠を待っても帰ってこない。\\
「これって\scalebox{3}[1]{―}」\\
「はい」\\
「告白なの」\\
「はい。そうです……」\\
頷く彼女。
私は見下されているのだろうか。気取った人間だと。失敗した、そう後悔した。
抑えきれない震え。もはや細かすぎて速すぎて、ただ静止しているようにしか見えないだろう。

ああ、だめだ。

私には、早すぎたのだ。

諦めよう。

「いいよ」\\
「えっ」\\
私は顔を上げた。
自分でも驚くぐらい、真っ赤に染まっているだろう。\\
「あの、なんて」\\
「いいよ、って」\\
「ほんと、ですか?」\\
「それ以外ある?」\\
返しに困った。
いきなり過ぎて、頭が真っ白になった。
この答えを望んでいたはずなのに、いざそうなると何もできなくなるのはどうしてなんだろう。

そんな私の目の前に、彼女は迫ってきた。

う……。

息が苦しくなる。

ほんの一瞬。

ほのかな甘い匂いと一緒に。

私は今度こそ、もう何も考えられなくなった。

「キスって、したことあった?」\\
「いえ、全然……」\\
「じゃあ私が初めてね。うれしい」\\
彼女はそのまま帰っていこうとする。
私を通り過ぎて。\\
「それじゃあ、次ちょっとやることがあるから、もう帰るね」\\
「あ、あの! 先輩は、初めてですか」\\
彼女は笑った。
ささやかな笑顔だが、とてもエロチックだ。\\
「ひみつ」\\
少しがっかりした。
はぐらかされたこと、少しでも可能性のあることに、私は嫉妬しているのか。\\
「また放課後! 一緒に\scalebox{3}[1]{―}」\\
「学生玄関って分かる? 自販機が置いてあるところ」\\
「たぶん」\\
「そこで待ってて。一緒に帰ろう」\\
「はい! 待ってます」\\
「ええ、またね。セレナ」\\

まだ一人きり。だけど私はひとりじゃない。
心が躍るとよく言うが、確かにこれは踊っている。

世界は少し、明るく映っていた。

\section{}
\subsection*{(1)}
遠くに見えるのはアヤメだった。
朝の時間帯、購買に失くしてしまった消しゴムを買いにいった時に、彼女を見つけた。
私たちが二年生までいる棟と、彼女たち上級生がいる専門科棟は、階段で繋がっている。
その一つ、教職員用の駐車場側にある階段、その下にいる。

移動の途中だろうかと思ったが、違うようだった。
何か言い争っているような感じだ。
丁度木に隠れていて、相手を見ることが出来ない。

ただかなり大きな声で言い合っていて、その内容は断片的ながらも聞き取ることが出来た。\\

「……アンタには関係ない」\\
「まだ続けるつもりなの? もうやめて。いい加減にしないと」\\
「私は私の意思で続けてる。私にはこれしかない。エナには関係ない」\\
エナというのが、どうやら相手の名前らしい。\\
「……も巻き込んで、一体、どうするつもりなの。自分のやってることを、理解しているの」\\
「だから、関係ない。もう関わらないで。昔とは違う」\\
「待って、アヤメ……」\\
そこで会話は途切れたようだった。

一体誰だろう。少し興味を持ってしまって、私は覗き込んでみた。
私服\scalebox{3}[1]{―}ブラウスだと思うが\scalebox{3}[1]{―}の女性。
ということは四年生以上だ。
髪型は長いポニーテールで、両側には房が垂らされている。
前髪は遠目に見る限り、切りそろえられている。

アヤメの友達だろうか。
ただ、アヤメは誰かと言い争うタイプには見えなかった。
適当にあしらうのが、いつもの彼女だろう。
でもそれをしなかったというのは、相手がそれほどの人間だということだろうか。

やめておこう。これ以上は個人の問題だろうし、下手に首を突っ込まない方がいい。
好奇心はこういった場合、抑えておくのが好ましい。

私はそのまま、消しゴムを購買で買って、教室へ戻った。

\subsection{(2)}
今日は一人でごはんを食べなくてはいけないようだった。\\
「ゴメン、私今日ちょっと予定あるから……。今日は一人でもいい?」\\
「ああ、別にいいよ。行ってきて」\\
「うん、ごめんね、ホント。終わったらすぐ帰ってくるから」\\
「そんな忙しくしないでもいいよ。一人でもごはんぐらい食べられるし」\\
「じゃあ」\\
「うん、いってらっしゃい」\\
彼女の背中は、どこか憂鬱なものに見えた。
彼女のそんな姿を見たのは、いつぶりだろうか。

思い出す。彼女と初めて出会った時のことを。
確かそのときにも、あんな雰囲気を出していた気がする。

どうだっただろうか。

一人で暇だし、せっかくだから振り返ってみよう。
思い出に浸って悦に入る歳でもないと思うのだが、今はそんな気分だった。\\

セレナと出会ったのは、塾だった。
夏休みから通い始めたので、受験勉強を春先、いやそれ以上前から始めていた人たちからすれば、
何を今さらと思われていたかもしれない。

夜まで勉強詰めというのは、想像以上に大変で、私はすぐに行く気を失せていた。
友達も居ないし、モチベーションもない。
志望校は適当に、県内の偏差値順の上から三つ目か四つ目のどれか。
志は低く、一向に勉強に身が入らなかった。

そんな状況を変えようと、私は、同志を探し出そうとしていた。
かと言っても、大勢でつるむのは嫌だった。
最高でも自分を合わせて三人までのコミュニティーが、私には丁度よかった。

でも大抵、みんなは寄ってたかって集まりたがる。
ここもそんな人間が大半を締めていて、私はこの課題に難儀していた。

だがそこに、一人だけの少女がいた。

観察するに、いつもひとりきりだった。
弁当を食べるときも、自習のときも。
彼女は周囲を拒絶していたし、周囲もまた彼女を拒絶していた。
それは彼女の纏う雰囲気のせいだろう。
大凡私たちの周りには存在しない、その美しい白金の長髪、
それに強弱のはっきりした顔は、
彼女の異質さを際立て、孤立を助長させていた。

だから私は彼女に話しかけた。
勝手に、彼女は私と同じ人間だと思っていたのだ。
周囲に馴染めず、またその努力もしない。
どうすればいいかわからないからだ。

正直に言おう、私は彼女を見下していたのかもしれない。
孤独な彼女に手を伸ばす、唯一の善人を演じていただけ。

その時の私だけは、浅ましさの塊だっただろう。\\
「あの、一緒に食べてもいい?」\\
休憩時間、みんなが夜ご飯を食べに出かけたり、弁当を食べたりする中で、
やはり彼女は一人だった。\\
「なに?」\\
「ごめんなさい、急に。でもなんか、あなたと一緒に食べたくて」\\
「誰か知らないし、知りたくない。ごめん。一人にさせて。
あなたにかまってる暇はないの」\\
想像以上に彼女は冷たかった。\\
「あ、あの!」\\
そのまま距離を取られてしまった。
彼女は席を離れ、どこかへ消えた。

結局、私は今日も、一人きりだった。\\

塾が終わった。
夜の十時過ぎ頃だった。
窓の外を見れば、雨が降っている。
幸いにも傘を用意していて、私には何ら問題ではなかったが、
ちらほらと傘の不用意を嘆いている人間がいた。

黙って教室を出ていく。
階段を降りて、玄関を出ようとする。

人がいる。
雨宿りでもしているのだろうか、と思ったがすぐにその考えは捨てた。
彼女だ、さっきの。
彼女も傘を持ってこなかったのだろう。
なんだか気まずかった。

でも入り口はそこしかないのだし、私は前へと進んだ。\\
「あのー」\\
意外にも、声をかけてきたのは彼女だった。
下駄箱から靴を出して、履いている私を彼女はその一言で捕まえた。\\
「さっきはごめん」\\
謝っている。そんな必要はないのに。\\
「別に、私のほうこそ、急にあんなこと言って……」\\
「けど、せっかく話しかけてくれたのに、あんなふうに言っちゃったから。
本当は、その\scalebox{3}[1]{―}すこしうれしかった。」\\
「え、あ、ありがとう」\\
「でもどうして私なんかに声かけたの?」\\
「えーなんかなんとなく」\\
「嘘でしょ」\\
「ええ? そんなことないよ」\\
「嘘。私がボッチだったからでしょ。分かるんだよお、だって私もそうするし」\\
「まあ、その通りなんだけど」\\
「正直でよろしい」\\
「許してくれる?」\\
「いいとも。そのかわり、傘貸してくれない?」\\
「傘? 一つしかないけど」\\
「そうじゃなくて、一緒に帰ろうって意味」\\
なんだか恥ずかしそうだった。\\
「ああ、はいはい、いいよ。狭いけど」\\
「やった。じゃあお邪魔しまーす」\\
彼女は何の躊躇いもなく私の傘の中に入ってきた。
おそらく彼女は、一歩をなかなか踏み出せないタイプの人間なんだろう。
でもその先は、流れるように進んでいくんだ。
私とは、若干似ていると思った。私は踏み出した後もなかなか進めないのだが。

「中学ってどこ?」\\
「えっと\scalebox{3}[1]{―}」\\
帰り道の会話は、単純なものだった。
通っている中学校のことだったり、好きなものとか、
どこの高校を目指しているのかというものだった。
「ほんと雨って最悪だよね」\\
「そう? 私はそうでもないけど。だって、雨がないと水は流れないよ」\\
「カナンは難しいこと考えるね」\\
「それほどじゃないと思うけど」\\
「私なんて、自分以外のものに興味なんて湧かない」\\
「それって自己中ってこと」\\
「そうじゃない。嫌なの。自分と誰かが関わってるって、思いたくない。
一人で生きて、一人で死にたい」\\
「それこそ、今考えるべきことじゃないと思うけどなあ」\\
「そうかもね。でも、まあ、いいや。なんか、あなたとならいいかもね」\\
「いいってなにが?」\\
「友達、になること」\\
「それ、本当?」\\
「うん。嫌?」\\
「嫌じゃない。こちらこそだよ」\\
「あ、ここ曲がるんだけど、どう?」\\
「私は、まっすぐ行かないと」\\
「そう。じゃあまたね、あの\scalebox{3}[1]{―}」\\
「カナン、詠華南」\\
「じゃあねカナン」\\
「ちょっと待って!」\\
「ああ、ごめん忘れてた。私はセレナね。上は時国」\\
「なんか珍しい」\\
「そっちこそ珍しいよ」\\
「そうかもしれない」\\
「珍しいもの同士、仲良くしようね」\\
「珍しい者同士って、ふふ、うん、そうだね。じゃあ、また今度」\\
「またねー」\\
セレナは信号を渡っていった。
夜だから、すぐにその姿は見えなくなった。

ああ、なんだか足取りが軽い。
胸に突っかかっていたなにかが、すっぽりと抜け落ちていった気分だ。

とにかく、私は安心した。\\

今思えば、あれから先になって、やっと自分の進路が順調に決まっていった気がする。
セレナがここを志望していたから、私も一緒にした。
目的が生まれたから、勉強する意味ができたし、あの日々は充実していた。
もちろん今もそうだが、あのときはいろいろと無茶も出来た。

なんだかんだで、私はセレナにぞっこんなんだ。
本当に、出会えてよかった。

「なにニヤニヤしてるの?」\\
振り向くと、そこにはセレナが居た。
なんだか彼女は嬉しそうだった。
幸福の洪水を抑えきれなくて、たまらずに溢れ出ている、そんな笑みを浮かべている。
それを無理矢理にも真顔に矯正しようとしているから、尚更に面白い顔になっている。\\
「そっちこそ、なんか気持ち悪い顔になってる」\\
「あ、えっ! ウソ!」\\
顔を覆ってグチャクチャにする。
ほっぺを手のひらで回してほぐしているつもりらしい。\\
「どう、これで大丈夫?」\\
「全然。口の端が上がってる。なにかあったの? いいことでも」\\
「まあね。ちょっと」\\
「ふーん」\\
「カナンは?」\\
「え、なにが」\\
「カナンだってにやけてたじゃん。なんかあったの」\\
「別に。昔のこと思い出してただけ」\\
「昔って、いつ」\\
「塾に行ってた頃。ほら、セレナと初めて会った時の」\\
「ああ、あの塾最悪だったね。大してわかりやすくもなっかったし」\\
「自習用でしょ」\\
「そうだけどね」\\
「って、そういうことじゃなくて」\\
「私と出会ったってこと?」\\
「そう。セレナと友達になれてよかったなあ、って」\\
「ミートゥ―」\\
「なにそれ、変な英語やめてよ」\\
「面白いじゃん」\\
「……ちょっとは笑ったけど」\\
「じゃあ私の勝ち」\\
「じゃあ笑ってない」\\
「なにそれー」\\

狩人になっても、私たちは所詮、等身大の女子高生だ。

ただ私は今も探している。

自分の生きる意味。
それを狩人になれば見つけられると思って、今までやってきた。
ただそれは難しい。
なにが正しいのかも判断できずに、けれど何度も立ち向かって、戦ってきた。\\

だけど気づくだろう。
悪夢とは、恐怖と不安を運ぶ不吉な使者だと。
自分がただちっぽけな存在、他者に関わることも感知することもできないしされない、
ちっぽけで孤独な存在であるという自覚。
それを植え付けるのが、悪夢という存在であるということを。

それほどまでに、今回の悪夢は恐ろしかったのだ。どこまでも。

\section{}
\subsection*{(?)}
「狩人というものは、相変わらず物騒なものだね」どこからの声。\\
「それ、私へのあてつけのつもり?」\\
「君も狩人だろう? そういうことだよ」\\
「確かにそうね」\\
夜景に紛れる人影。
ビルの屋上に座っている。
ありきたりな格好付けだろう。
街を見上げる場所は、心の優越感をもたらす。
「今日はどこに出る?」\\
「さあ? 片割れが一生懸命探してくれてるし、僕はそれを盗み見するだけだよ」\\
「最低ね。仲間じゃないの?」\\
「違うね。君たちがそうじゃないように」\\
「あっそう」\\
「どうするんだい。今夜こそ、やりあってみるのかい?」\\
「そろそろ止めないと。マズいことになる。アンタもそれはわかってるでしょ」\\
「そうだね。準備をしておこう」\\

少女は立ち上がった。
特異な衣装は狩人のそれで、彼女の出で立ちは長いマントにくるまれている。
もちろん顔も、深いフードで隠されている。\\
「今夜は長くなりそうね」\\
彼女は夜の闇に消え去った。

\subsection*{(1)}
雨の夜、水たまりを踏んで、跳ねる飛沫を踝に浴びながら、私は長い路地を走っていた。
微かに混じるネオンの光。その色彩を頼りに、私は悪夢を追いかける。

今宵の悪夢の素早さは段違いで、何度も私の銃撃を躱して逃げていく。
ここ、と狙いをつけて撃つが、聞こえるのは不発の鈍い音。\\
「速すぎる!」\\
角を曲がる。
雨で滑りそうになった。
壁にもたれかかって、その勢いで肩をぶつける。
それでも気にしている暇はなく、走り続けるしかない。
道端に落ちるゴミも構いなしにぶちまけて、ただ足を動かして。
狭い道は所々の出っ張りが、意地汚い罠に思えるほど生えている。
意識を割く先が多くあるというのは、それだけで労力を倍増させ、疲労を誘うのだ。

はあ、はあ、はあ。

息が上がってくる。
無意識に口を上に向けて、より多くの空気を吸い込もうとする。
あまり意味のないことだろうが。

庇の先から滴る大きな雨粒が頬に目掛けて落ちてきた。
目に水が入る。
痛い。
たまらずに目をつむった。
異物感が眼球を圧迫して、瞼を縫い付ける。
しかしすぐに、無理矢理にでも閉じた眼をこじ開ける。
ぼやけた視界で、かろうじて曲がり角を認識できた。

私は立ち止まった。

慣性が私の背中を押したが、なんとか抑え、転ぶこと防げた。

だが目の前にあるものを見え、私は結局後ろへ尻もちをついてしまった。

悪夢は息絶えていた。
それは私の弾丸が命中したわけでも、ましてや狩人によるものでもなかった。

残骸に佇む、ひょろ長い真っ黒な人型。悪夢にそっくりな雰囲気。
まるであのときの、モエの姿を模った外形を携えていた悪夢の、
その部分だけをそのまま切り離したようなもの。

だがすぐにその既視感は捨てた。
あれは完全に人だ。立ち振舞がそうだった。

剣のようなものが突き刺され、その先端は悪夢を貫いて、
悪夢の死骸は泥水のように側溝へと流れていく。

雨の音がうるさい。

抜き上がる剣。私に気づいたのだ。ゆっくりとその面を私に向けて、
その虚空の表情をまじまじと見せつける。
のっぺりとした眼孔は、夜目を働かせてやっと色味を見せた。
初めは全くの闇。黒い、どこまでも深い延長のみの空間があるだけだが、
しばらくすると寂しく輝く星は、死にかけの発光、燻った白色を放つ。
それは目元だ。覆われた布は私達のそれと何ら変わりなく、
顔を隠す目的だとわかったからだ。
布と同じく、深々と被された帽子、重厚な衣装。
コートに似た服装は、ボロボロだが何か執念を感じる。
そしてそのどれもが暗く淀んでいて、夜に沈む配色をしていた。

眼とも判断できない、その朧な顔はなおも私に合わせられ、人型はおもむろに背を伸ばした。
獲物を見定める猟師のように。
私は獣だ。弱々しい、怯えて動けない哀れな犠牲者だ。
少なくとも今は。
硬い筋が口に引っかかるような気がして、うまく息ができない。

一瞬だった。
瞬きを終えて、まだゴワゴワした瞼を開けば、
もはや私の寿命は物理的な距離に置き換えられていた。
剣先は確実に胴体を貫こうと迫ってくる。

だがここで諦める私ではなかった。
正確には、体が勝手に動いたのだが。

動やったのかは自分でもよくわからなかったが、私は一撃を受け流した。
ただ手首には相当な負荷が掛かって、おかしな方向へ向いてしまった。
銃を盾にしたのだろう。
摩擦に熱せられて、当たる雨水が湯気になっている。

関節をひねるモーメントをそのままに、私はすぐに方向を転換して逃げ出した。
振り向いたあとも運動し続ける手を自制して、体勢を立て直し、
しっかりと地面を掴んで、開く限界まで股下を広げて無心で走る。

足音がする。とても早いリズムを刻みながら、昔よく歌った歌のように、音を鳴らしている。
楽しくともなんともない雨の日だ。

汗ばんできた。最近は夜も暖かく、またじめじめしている。
何もかもが最悪だ。体力も尽きかけている。
きた道を逆順して、所々に撹乱を見込んで角を曲がったりするが、
どうしてか人型のアレは私の後を追いかけてくる。
臭いでも嗅いでいるのか。
まさか。

人混みに出ることは避けなければならなかった。十二時を超えたとしても、繁華街から人波は早々に消えない。
たとえそこが廃れかけであってもだ。
狩人の姿は狩人にしか見えない。悪夢もまた然りだ。
これは経験則だが、おそらく正しい。
だけど危害が及ぶことはある。
それは今までもそうだったし、私達が悪夢を狩る重要な意味の一つだ。
だから避けたかった。
大勢の人を相手にして、私はリスクを負担できない。
できるだけ路地を縫うように走るが、終点はどこにでもある。

どうしよう。

そろそろ足も限界だ。
このまま逃げ続けるのも、面白くない。
不利な状況なのだろうか。
私は慎重に考える。
今の距離であれば、もしかすれば、
倒すことができるかもしれない。
先手必勝で弾丸を撃ち込む。
そうするしかない。

いちにのさん、で振り返るんだ。
ホルダーに入った銃を握って、心の中で数え始める。

いち。

目線を後ろに移動させる。

にの。

腕に振りをつけて、体を捻らせる。

さん。

両手で握り、脇を締めて、照星をあわせる。
まだ間に合う。案の定、アレは剣を持ち上げ、私に突撃してこようとする。
得物の有効距離に私が入るには、まだ一メートルほどもある。

私は引き金を引いた。
一発、二発、三発、いやそれ以上に。
自動装填される弾丸すべてを放ちきった。
空になった弾倉を捨てて、新しいものを装填する。
常々、この作業を略せないのかと思う。無限の弾数を持つ銃があればどれほど心強いだろう。
だがそんなものは実現しない。
その限り、リロードという作業は最も油断できない、ある意味命取りな行動になるし、
一過性の安心を与えてくれる。
立ち上がる恐怖を抑えるためにも、備えることは重要だ。

足元を見る。倒れる黒い塊に気づいた。
ぴくりとも動かないそれは、おそらく死んだのだろう。
私はそれでも監視を続ける。

いずれ\scalebox{3}[1]{―}悪夢だったのだろうか\scalebox{3}[1]{―}人型をした物体は、
波にさらわれる砂の城のように、徐々に雨に溶けて、
小さな悪夢と同じように、水流の中を不純物として流れていった。

完全に消滅したのを確認して、私は始めて自らを許した。

………………。

「カナン後ろ!」

頭上から響く声に従って、私は後ろを振り向いた。

不意打ちは防げず、振りかざされた刃は私の肩を掠め、傷を刻んで、
そのままに過ぎ去っていく。

痛みは鈍く、私は、何が起こったのかを理解出来なかった。
その直後に、恐怖とともに痛みは傷口から湧き上がる血に乗って体から流れ出す。
血の抜けていく感覚が気持ち悪い。
痺れる肩口。

腰も砕けて、私は立ち上がることもままならない。
だが、逃げる必要はすぐに無くなった。

セレナが上から落ちてくる。
体重をかけて相手を押しつぶして、それはそのまま致命傷となった。
いともたやすく地に伏して、今度こそ、霧散して消えていくのを見届ける。

「ありがとう。助かったよ、セレナ」\\
「大丈夫、それ」\\
「ああ、大丈夫。そんなに深くないし。なんともないよ」\\
若干の強がりはあったが、そこまで焦っていないのも事実だった。
なぜなら目覚めると、この傷はすぐに癒える\scalebox{3}[1]{―}というより、ほぼ消滅している
としたほうがいいかもしれない\scalebox{3}[1]{―}のだから。\\
「でもごめん、もうちょっと早く仕留められたら、怪我なんてしなくてもよかったのに」\\
「あ、え、ちょっとまって、セレナもコイツを追いかけてたの?」\\
「え、違うの?」\\
セレナは私の左肩を持ってくれた。
その場に留まるのもよくないと思って、私達は移動することにした。\\
「私も一匹、ていうか一回、倒したんだけど、蘇ったのかな。それで油断してて」\\
私の報告を聞いて、黙り込むセレナ。
考える間はかなり長く、私は嘔吐きそうになる。神妙な顔で、彼女は口を開いた。\\
「カナン、たぶんそれ違うよ。最初から二体いたんだ」\\
「だとしたら、それって」\\
「もしかしたら、もっといるかも」\\
これは全くの直感、なんの理論的な考察もないものだ。
だが、二度あることは三度あるとよく言うし、身構えておくことは必要だと思う。\\
「アヤメさんも追ってた気がする」\\
「最初から?」\\
「私はそうだったけど」\\
「こっちは、小さいやつを追ってたら、急に出てきたの。
それで、悪夢を殺して、私を襲ってきた」\\
「殺したって、共食い?」\\
「わからない。そもそもこれが悪夢なのかもはっきりしないし」\\
「妙に人の形してるしね」\\
「動きもそうだったし、とにかく普通じゃない」\\
「後でアオタにでも聞いてみよっか」\\
「そうするしかないしね」\\
「まあ、答えてくれるかどうかの確証はないけど……
たぶん無理だろうなあー」\\
繁華街を離れて、住宅街に出る。
流石にここまで来ると光は少なく、電灯の点滅する明かりだけが頼りになった。



\end{document}
