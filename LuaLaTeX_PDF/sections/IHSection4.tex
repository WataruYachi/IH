\documentclass[../IHMain]{subfiles}

\begin{document}

\chapter{人の記憶}
\section*{\tt if\_you\_love\_me}
呼び出しは成功した。

昼休み、I科棟の裏で。

メッセージには確かに、既読を表すマークが付いている。
正確な時刻は指定していないけれど、きっと来てくれると信じていた。

「これ、呼んでください」\\
明らかに震えている声だったが、口から出たことを評価してほしい。
渡した紙に書いたのは、詩だ。
下手くそだとは自分でも分かっているが、それでも、なんというか、
直接的な表現は気恥ずかしく、悶てしまうのだ。
\begin{verse}
    怖いんだ\\
    さめてしまうのが\\
    だからこの思いを\\
    小箱にしまって\\
    鍵をかけたい\\ \\

    けれど、恐怖はいずれ鍵穴から\\
    ひっそりと抜け出して\\
    戻ってくるんだ\\ \\

    だから君と一緒に閉じたい\\
    鍵穴を二つにわけて\\
    出口を細めて\\ \\

    そうすれば、\\
    わずかに溢れるだけだから\\ \\

    繰り返しくりかえし\\
    鍵を回して\\
    穴を狭めて\\ \\
    
    僕は安心して\\
    君と眠りたい\\
\end{verse}
やっぱりこういう表現も恥ずかしい。
だが、渡してしまったものを奪い取ることは出来ない。
私は俯くことしかできなかった。
彼女の顔を見ることが出来ない。
凛々しい、沈黙の似合う彼女の顔を。

人生で一番長い時間だった。
バクバクと鼓動する心臓の流れは早く、しかし彼女の返事は永遠を待っても帰ってこない。\\
「これって\scalebox{3}[1]{―}」\\
「はい」\\
「告白なの」\\
「はい。そうです……」\\
頷く彼女。
私は見下されているのだろうか。気取った人間だと。失敗した、そう後悔した。
抑えきれない震え。もはや細かすぎて速すぎて、ただ静止しているようにしか見えないだろう。

ああ、だめだ。

私には、早すぎたのだ。

諦めよう。

「いいよ」\\
「えっ」\\
私は顔を上げた。
自分でも驚くぐらい、真っ赤に染まっているだろう。\\
「あの、なんて」\\
「いいよ、って」\\
「ほんと、ですか?」\\
「それ以外ある?」\\
返しに困った。
いきなり過ぎて、頭が真っ白になった。
この答えを望んでいたはずなのに、いざそうなると何もできなくなるのはどうしてなんだろう。

そんな私の目の前に、彼女は迫ってきた。

う……。

息が苦しくなる。

ほんの一瞬。

ほのかな甘い匂いと一緒に。

私は今度こそ、もう何も考えられなくなった。

「キスって、したことあった?」\\
「いえ、全然……」\\
「じゃあ私が初めてね。うれしい」\\
彼女はそのまま帰っていこうとする。
私を通り過ぎて。\\
「それじゃあ、次ちょっとやることがあるから、もう帰るね」\\
「あ、あの! 先輩は、初めてですか」\\
彼女は笑った。
ささやかな笑顔だが、とてもエロチックだ。\\
「ひみつ」\\
少しがっかりした。
はぐらかされたこと、少しでも可能性のあることに、私は嫉妬しているのか。\\
「また放課後! 一緒に\scalebox{3}[1]{―}」\\
「学生玄関って分かる? 自販機が置いてあるところ」\\
「たぶん」\\
「そこで待ってて。一緒に帰ろう」\\
「はい! 待ってます」\\
「ええ、またね。セレナ」\\

まだ一人きり。だけど私はひとりじゃない。
心が躍るとよく言うが、確かにこれは踊っている。

世界は少し、明るく映っていた。

\section{}
\subsection*{(1)}
遠くに見えるのはアヤメだった。
朝の時間帯、購買に失くしてしまった消しゴムを買いにいった時に、彼女を見つけた。
私たちが二年生までいる棟と、彼女たち上級生がいる専門科棟は、階段で繋がっている。
その一つ、教職員用の駐車場側にある階段、その下にいる。

移動の途中だろうかと思ったが、違うようだった。
何か言い争っているような感じだ。
丁度木に隠れていて、相手を見ることが出来ない。

ただかなり大きな声で言い合っていて、その内容は断片的ながらも聞き取ることが出来た。\\

「……アンタには関係ない」\\
「まだ続けるつもりなの? もうやめて。いい加減にしないと」\\
「私は私の意思で続けてる。私にはこれしかない。エナには関係ない」\\
エナというのが、どうやら相手の名前らしい。\\
「……も巻き込んで、一体、どうするつもりなの。自分のやってることを、理解しているの」\\
「だから、関係ない。もう関わらないで。昔とは違う」\\
「待って、アヤメ……」\\
そこで会話は途切れたようだった。

一体誰だろう。少し興味を持ってしまって、私は覗き込んでみた。
私服\scalebox{3}[1]{―}ブラウスだと思うが\scalebox{3}[1]{―}の女性。
ということは四年生以上だ。
髪型は長いポニーテールで、両側には房が垂らされている。
前髪は遠目に見る限り、切りそろえられている。

アヤメの友達だろうか。
ただ、アヤメは誰かと言い争うタイプには見えなかった。
適当にあしらうのが、いつもの彼女だろう。
でもそれをしなかったというのは、相手がそれほどの人間だということだろうか。

やめておこう。これ以上は個人の問題だろうし、下手に首を突っ込まない方がいい。
好奇心はこういった場合、抑えておくのが好ましい。

私はそのまま、消しゴムを購買で買って、教室へ戻った。

\subsection{(2)}
今日は一人でごはんを食べなくてはいけないようだった。\\
「ゴメン、私今日ちょっと予定あるから……。今日は一人でもいい?」\\
「ああ、別にいいよ。行ってきて」\\
「うん、ごめんね、ホント。終わったらすぐ帰ってくるから」\\
「そんな忙しくしないでもいいよ。一人でもごはんぐらい食べられるし」\\
「じゃあ」\\
「うん、いってらっしゃい」\\
彼女の背中は、どこか憂鬱なものに見えた。
彼女のそんな姿を見たのは、いつぶりだろうか。

思い出す。彼女と初めて出会った時のことを。
確かそのときにも、あんな雰囲気を出していた気がする。

どうだっただろうか。

一人で暇だし、せっかくだから振り返ってみよう。
思い出に浸って悦に入る歳でもないと思うのだが、今はそんな気分だった。\\

セレナと出会ったのは、塾だった。
夏休みから通い始めたので、受験勉強を春先、いやそれ以上前から始めていた人たちからすれば、
何を今さらと思われていたかもしれない。

夜まで勉強詰めというのは、想像以上に大変で、私はすぐに行く気を失せていた。
友達も居ないし、モチベーションもない。
志望校は適当に、県内の偏差値順の上から三つ目か四つ目のどれか。
志は低く、一向に勉強に身が入らなかった。

そんな状況を変えようと、私は、同志を探し出そうとしていた。
かと言っても、大勢でつるむのは嫌だった。
最高でも自分を合わせて三人までのコミュニティーが、私には丁度よかった。

でも大抵、みんなは寄ってたかって集まりたがる。
ここもそんな人間が大半を締めていて、私はこの課題に難儀していた。

だがそこに、一人だけの少女がいた。

観察するに、いつもひとりきりだった。
弁当を食べるときも、自習のときも。
彼女は周囲を拒絶していたし、周囲もまた彼女を拒絶していた。
それは彼女の纏う雰囲気のせいだろう。
大凡私たちの周りには存在しない、その美しい白金の長髪、
それに強弱のはっきりした顔は、
彼女の異質さを際立て、孤立を助長させていた。

だから私は彼女に話しかけた。
勝手に、彼女は私と同じ人間だと思っていたのだ。
周囲に馴染めず、またその努力もしない。
どうすればいいかわからないからだ。

正直に言おう、私は彼女を見下していたのかもしれない。
孤独な彼女に手を伸ばす、唯一の善人を演じていただけ。

その時の私だけは、浅ましさの塊だっただろう。\\
「あの、一緒に食べてもいい?」\\
休憩時間、みんなが夜ご飯を食べに出かけたり、弁当を食べたりする中で、
やはり彼女は一人だった。\\
「なに?」\\
「ごめんなさい、急に。でもなんか、あなたと一緒に食べたくて」\\
「誰か知らないし、知りたくない。ごめん。一人にさせて。
あなたにかまってる暇はないの」\\
想像以上に彼女は冷たかった。\\
「あ、あの!」\\
そのまま距離を取られてしまった。
彼女は席を離れ、どこかへ消えた。

結局、私は今日も、一人きりだった。\\

塾が終わった。
夜の十時過ぎ頃だった。
窓の外を見れば、雨が降っている。
幸いにも傘を用意していて、私には何ら問題ではなかったが、
ちらほらと傘の不用意を嘆いている人間がいた。

黙って教室を出ていく。
階段を降りて、玄関を出ようとする。

人がいる。
雨宿りでもしているのだろうか、と思ったがすぐにその考えは捨てた。
彼女だ、さっきの。
彼女も傘を持ってこなかったのだろう。
なんだか気まずかった。

でも入り口はそこしかないのだし、私は前へと進んだ。\\
「あのー」\\
意外にも、声をかけてきたのは彼女だった。
下駄箱から靴を出して、履いている私を彼女はその一言で捕まえた。\\
「さっきはごめん」\\
謝っている。そんな必要はないのに。\\
「別に、私のほうこそ、急にあんなこと言って……」\\
「けど、せっかく話しかけてくれたのに、あんなふうに言っちゃったから。
本当は、その\scalebox{3}[1]{―}すこしうれしかった。」\\
「え、あ、ありがとう」\\
「でもどうして私なんかに声かけたの?」\\
「えーなんかなんとなく」\\
「嘘でしょ」\\
「ええ? そんなことないよ」\\
「嘘。私がボッチだったからでしょ。分かるんだよお、だって私もそうするし」\\
「まあ、その通りなんだけど」\\
「正直でよろしい」\\
「許してくれる?」\\
「いいとも。そのかわり、傘貸してくれない?」\\
「傘? 一つしかないけど」\\
「そうじゃなくて、一緒に帰ろうって意味」\\
なんだか恥ずかしそうだった。\\
「ああ、はいはい、いいよ。狭いけど」\\
「やった。じゃあお邪魔しまーす」\\
彼女は何の躊躇いもなく私の傘の中に入ってきた。
おそらく彼女は、一歩をなかなか踏み出せないタイプの人間なんだろう。
でもその先は、流れるように進んでいくんだ。
私とは、若干似ていると思った。私は踏み出した後もなかなか進めないのだが。

「中学ってどこ?」\\
「えっと\scalebox{3}[1]{―}」\\
帰り道の会話は、単純なものだった。
通っている中学校のことだったり、好きなものとか、
どこの高校を目指しているのかというものだった。
「ほんと雨って最悪だよね」\\
「そう? 私はそうでもないけど。だって、雨がないと水は流れないよ」\\
「カナンは難しいこと考えるね」\\
「それほどじゃないと思うけど」\\
「私なんて、自分以外のものに興味なんて湧かない」\\
「それって自己中ってこと」\\
「そうじゃない。嫌なの。自分と誰かが関わってるって、思いたくない。
一人で生きて、一人で死にたい」\\
「それこそ、今考えるべきことじゃないと思うけどなあ」\\
「そうかもね。でも、まあ、いいや。なんか、あなたとならいいかもね」\\
「いいってなにが?」\\
「友達、になること」\\
「それ、本当?」\\
「うん。嫌?」\\
「嫌じゃない。こちらこそだよ」\\
「あ、ここ曲がるんだけど、どう?」\\
「私は、まっすぐ行かないと」\\
「そう。じゃあまたね、あの\scalebox{3}[1]{―}」\\
「カナン、詠華南」\\
「じゃあねカナン」\\
「ちょっと待って!」\\
「ああ、ごめん忘れてた。私はセレナね。上は時国」\\
「なんか珍しい」\\
「そっちこそ珍しいよ」\\
「そうかもしれない」\\
「珍しいもの同士、仲良くしようね」\\
「珍しい者同士って、ふふ、うん、そうだね。じゃあ、また今度」\\
「またねー」\\
セレナは信号を渡っていった。
夜だから、すぐにその姿は見えなくなった。

ああ、なんだか足取りが軽い。
胸に突っかかっていたなにかが、すっぽりと抜け落ちていった気分だ。

とにかく、私は安心した。\\

今思えば、あれから先になって、やっと自分の進路が順調に決まっていった気がする。
セレナがここを志望していたから、私も一緒にした。
目的が生まれたから、勉強する意味ができたし、あの日々は充実していた。
もちろん今もそうだが、あのときはいろいろと無茶も出来た。

なんだかんだで、私はセレナにぞっこんなんだ。
本当に、出会えてよかった。

「なにニヤニヤしてるの?」\\
振り向くと、そこにはセレナが居た。
なんだか彼女は嬉しそうだった。
幸福の洪水を抑えきれなくて、たまらずに溢れ出ている、そんな笑みを浮かべている。
それを無理矢理にも真顔に矯正しようとしているから、尚更に面白い顔になっている。\\
「そっちこそ、なんか気持ち悪い顔になってる」\\
「あ、えっ! ウソ!」\\
顔を覆ってグチャクチャにする。
ほっぺを手のひらで回してほぐしているつもりらしい。\\
「どう、これで大丈夫?」\\
「全然。口の端が上がってる。なにかあったの? いいことでも」\\
「まあね。ちょっと」\\
「ふーん」\\
「カナンは?」\\
「え、なにが」\\
「カナンだってにやけてたじゃん。なんかあったの」\\
「別に。昔のこと思い出してただけ」\\
「昔って、いつ」\\
「塾に行ってた頃。ほら、セレナと初めて会った時の」\\
「ああ、あの塾最悪だったね。大してわかりやすくもなっかったし」\\
「自習用でしょ」\\
「そうだけどね」\\
「って、そういうことじゃなくて」\\
「私と出会ったってこと?」\\
「そう。セレナと友達になれてよかったなあ、って」\\
「ミートゥ―」\\
「なにそれ、変な英語やめてよ」\\
「面白いじゃん」\\
「……ちょっとは笑ったけど」\\
「じゃあ私の勝ち」\\
「じゃあ笑ってない」\\
「なにそれー」\\

狩人になっても、私たちは所詮、等身大の女子高生だ。

ただ私は今も探している。

自分の生きる意味。
それを狩人になれば見つけられると思って、今までやってきた。
ただそれは難しい。
なにが正しいのかも判断できずに、けれど何度も立ち向かって、戦ってきた。\\

だけど気づくだろう。
悪夢とは、恐怖と不安を運ぶ不吉な使者だと。
自分がただちっぽけな存在、他者に関わることも感知することもできないしされない、
ちっぽけで孤独な存在であるという自覚。
それを植え付けるのが、悪夢という存在であるということを。

それほどまでに、今回の悪夢は恐ろしかったのだ。どこまでも。

\section{}
\subsection*{(?)}
「狩人というものは、相変わらず物騒なものだね」どこからの声。\\
「それ、私へのあてつけのつもり?」\\
「君も狩人だろう? そういうことだよ」\\
「確かにそうね」\\
夜景に紛れる人影。
ビルの屋上に座っている。
ありきたりな格好付けだろう。
街を見上げる場所は、心の優越感をもたらす。
「今日はどこに出る?」\\
「さあ? 片割れが一生懸命探してくれてるし、僕はそれを盗み見するだけだよ」\\
「最低ね。仲間じゃないの?」\\
「違うね。君たちがそうじゃないように」\\
「あっそう」\\
「どうするんだい。今夜こそ、やりあってみるのかい?」\\
「そろそろ止めないと。マズいことになる。アンタもそれはわかってるでしょ」\\
「そうだね。準備をしておこう」\\

少女は立ち上がった。
特異な衣装は狩人のそれで、彼女の出で立ちは長いマントにくるまれている。
もちろん顔も、深いフードで隠されている。\\
「今夜は長くなりそうね」\\
彼女は夜の闇に消え去った。

\subsection*{(1)}
雨の夜、水たまりを踏んで、跳ねる飛沫を踝に浴びながら、私は長く細い路地を走っていた。
微かに混じるネオンの光。その色彩を頼りに、私は悪夢を追いかける。

今宵の悪夢の素早さは、今までの個体とは段違いで、何度も私の銃撃を躱して逃げていく。
ここ、と狙いをつけて撃つが、聞こえるのは不発の鈍い音。\\
「速すぎる!」\\
角を曲がる。
雨で滑りそうになった。
壁にもたれかかって、その勢いで肩をぶつける。
それでも気にしている暇はなく、走り続けるしかない。
道端に落ちるゴミも構いなしにぶちまけて、ただ足を動かして。
狭い道は所々の出っ張りが、意地汚い罠に思えるほど生えている。
意識を割く先が多数あるというのは、それだけで労力を倍増させ、疲労を誘うのだ。

はあ、はあ、はあ。

息が上がってくる。
無意識に口を上に向けて、より多くの空気を吸い込もうとする。
あまり意味のないことだろうが。

庇の先から滴る大きな雨粒が頬を目掛けて落下してくる。
目に水が入る。
痛い。
たまらずに目をつむった。
異物感が眼球を圧迫して、瞼を縫い付ける。
しかしすぐに、無理矢理にでも閉じた眼をこじ開ける。
ぼやけた視界で、かろうじて曲がり角を認識できた。

今度はうまく曲がれた。
だけど、私は立ち止まった。先が行き止まりに見えたからだ。実際には違うのだろうけれど、
何故か靄がかった世界で立ち止まろうと思ったのだ。事実、その先は開けていた。

慣性が私の背中を押したが、背中を反って打ち消し、前方へ転ぶことは防げた。

だが目の前にあるものに驚嘆し、私は後ろへ尻もちをついてしまった。

結果、悪夢は息絶えていた。
それは私の弾丸が命中したわけでも、ましてや狩人によるものでもなかった。

残骸に佇む、ひょろ長い真っ黒な人型。悪夢にそっくりな雰囲気。
まるであのときの、モエの姿を模った外形を携えていた悪夢の、
その部分だけをそのまま切り離したようなもの。

けれども、すぐにその既視感は捨てた。
あれは完全に人だ。立ち振舞がそうだった。

剣のようなものが突き刺され、その先端は悪夢を貫いて、
悪夢の死骸は泥水のように側溝へと流れていく。

雨の音がうるさい。

抜き上がる剣。私に気づいたのだ。ゆっくりとその面を私に向けて、
その虚空の表情をまじまじと見せつける。
のっぺりとした眼孔は、夜目を働かせてやっと色味を見せた。
初めは全くの闇。黒い、どこまでも深い延長のみの空間があるだけだが、
しばらくすると寂しく輝く星は、死にかけの発光、燻った白色を放つ。
それは目元だった。覆われた布は私達のそれと何ら変わりなく、
顔を隠す目的だとわかったからだ。
布と同じく、深々と被された帽子、重厚な衣装。
コートに似た服装は、ボロボロだが何か執念を感じる。
そしてそのどれもが暗く淀んでいて、夜に沈む配色をしていた。

眼とも判断できない、その朧な顔はなおも私に合わせられ、人型はおもむろに背を伸ばした。
獲物を見定める猟師のように。
私は獣だ。弱々しい、怯えて動けない哀れな犠牲者だ。
少なくとも今は。
硬い筋が口に引っかかるような気がして、うまく息ができない。

一瞬だった。
瞬きを終えて、まだゴワゴワした瞼を開けば、
もはや私の寿命は物理的な距離に置き換えられていた。
剣先は確実に胴体を貫こうと迫ってくる。

だがここで諦める私ではなかった。
正確には、体が勝手に動いたのだが。

動やったのかは自分でもよくわからなかったが、私は一撃を受け流した。
とっさに立ち上がってだ。
ただ手首には相当な負荷が掛かって、おかしな方向へ向いてしまった。
銃を盾にしたのだろう。
摩擦に熱せられて、当たる雨水が湯気になっている。

関節をひねるモーメントをそのままに、私はすぐに方向を転換して逃げ出した。
振り向いたあとも運動し続ける手を自制して、体勢を立て直し、
しっかりと地面を掴んで、開く限界まで股下を広げて無心で走る。

足音がする。とても早いリズムを刻みながら、昔よく歌った歌のように、音を鳴らしている。
楽しくともなんともない雨の日だ。

汗ばんできた。最近は夜も暖かく、またじめじめしている。
何もかもが最悪だ。体力も尽きかけている。
きた道を逆順して、所々に撹乱を見込んで角を曲がったりするが、
どうしてか人型のアレは私の後を執拗に追いかけてくる。
臭いでも嗅いでいるのか。
まさか。

人混みに出ることは避けなければならなかった。十二時を超えたとしても、繁華街から人波は早々に消えない。
たとえそこが廃れかけであってもだ。
狩人の姿は狩人にしか見えない。悪夢もまた然りだ。
これは経験則だが、おそらく正しい。
だけど危害が及ぶことはある。
それは今までもそうだったし、私達が悪夢を狩る重要な意味の一つだ。
だから避けたかった。
大勢の人を相手にして、私はリスクを負担できない。
できるだけ路地を縫うように走るが、終点はどこにでもある。

どうしよう。

そろそろ足も限界だ。
このまま逃げ続けるのも、面白くない。
不利な状況なのだろうか。
私は慎重に考える。
今の距離であれば、もしかすれば、
倒すことができるかもしれない。
先手必勝で弾丸を撃ち込む。
そうするしかない。

いちにのさん、で振り返るんだ。
ホルダーに入った銃を握って、心の中で数え始める。

いち。

目線を後ろに移動させる。

にの。

腕に振りをつけて、体を捻らせる。

さん。

両手で握り、脇を締めて、照星をあわせる。
まだ間に合う。案の定、アレは剣を持ち上げ、私に突撃してこようとする。
得物の有効距離に私が入るには、まだ一メートルほどもある。

私は引き金を引いた。
一発、二発、三発、いやそれ以上に。
自動装填される弾丸すべてを放ちきった。
空になった弾倉を捨てて、新しいものを装填する。
常々、この作業を略せないのかと思う。無限の弾数を持つ銃があればどれほど心強いだろう。
だがそんなものは実現しない。
その限り、リロードという作業は最も油断できない、ある意味命取りな行動になるし、
また一方で、一過性の安心を与えてくれる。
立ち上がる恐怖を抑えるためにも、備えることは重要だ。

足元を見る。倒れる黒い塊に気づいた。
ぴくりとも動かないそれは、おそらく死んだのだろう。
私はそれでも監視を続ける。

いずれ\scalebox{3}[1]{―}悪夢だったのだろうか\scalebox{3}[1]{―}人型をした物体は、
波にさらわれる砂の城のように、徐々に雨に溶けて、
小さな悪夢と同じように、水流の中を不純物として流れていった。

完全に消滅したのを確認して、私は始めて自らを許した。

………………。

「カナン後ろ!」

頭上から響く声に従って、私は後ろを振り向いた。

不意打ちは防げず、振りかざされた刃は私の肩を掠め、傷を刻んで、
そのままに地面へ叩きつけられる。

痛みは鈍く、私は、何が起こったのかを瞬時には理解出来なかった。
混乱した脳の機能回復。
その直後に、恐怖とともに痛みは傷口から湧き上がる血に乗って体から流れ出す。
体液が皮膚の上を染めていく温さと、
血の抜けていく感覚が気持ち悪い。
赤が占める面積は、疼痛の支配率と比例している。
痺れる肩口。

腰も砕けて、私は立ち上がることもままならない。
だが、逃げる必要はすぐに無くなった。

セレナが上から落ちてくる。
体重をかけて相手を押しつぶして、それはそのまま致命傷となった。
いともたやすく地に伏して、今度こそ、霧散して消えていくのを見届ける。
どうして死んでいなかったのかと疑問に思うが、
すぐに痛みでかき消されてしまう。
こんなに大きな怪我は始めてだ。
夥しい流血は、いくら血を見慣れていると言っても、やはり辛い。

だが忘れるな。これは夢に等しい。
夢の中の死は、目覚めと等価だ。

「ありがとう。助かったよ、セレナ」\\
「大丈夫、それ」\\
「ああ、大丈夫。そんなに深くないと思うし。なんともないよ」\\
若干の強がりはあったが、そこまで焦っていないのも事実だった。
なぜなら目覚めると、この傷はすぐに癒える\scalebox{3}[1]{―}というより、ほぼ消滅している
としたほうがいいかもしれない\scalebox{3}[1]{―}のだから。
その自覚は一応の冷静さをもたらしてくれる。
「でもごめん、もうちょっと早く仕留められたら、怪我なんてしなくてもよかったのに」\\
「あ、え、ちょっとまって、セレナもコイツを追いかけてたの?」\\
「え、違うの?」\\
セレナは私の左肩を持ってくれた。
引っ張られる手に力を任せて立ち上がる。
その場に留まるのもよくないと思って、私達は移動することにした。
傷口を手で押さえる。
出血が怖かった。
幸い、歩くことはできた。
目覚めると治る、と言っても現状の痛みを打ち消す効果はない。
なるべく慎重になろうと、セレナは私の歩幅に合わせてくれた。\\
「私も一匹、というか一回、倒したんだけど、蘇ったのかな。それで油断してて」\\
私の報告を聞いて、黙り込むセレナ。
考える間はかなり長く、私は嘔吐きそうになる。神妙な顔で、彼女は口を開いた。\\
「カナン、たぶんそれ違うよ。最初から二体いたんだ」\\
「だとしたら、それって」\\
「もしかしたら、もっといるかも」\\
これは全くの直感、なんの理論的な考察もないものだ。
だが、二度あることは三度あるとよく言うし、身構えておくことは必要だと思う。\\
「アヤメさんも追ってた気がする」\\
「最初から?」\\
「私はそうだったけど」\\
「こっちは、小さいやつを追ってたら、急に出てきたの。
それで、悪夢を殺して、私を襲ってきた」\\
「殺したって、共食い?」\\
「わからない。そもそもこれが悪夢なのかもはっきりしないし」\\
「妙に人の形してるしね」\\
「動きもそうだったし、とにかく普通じゃない」\\
「後でアオタにでも聞いてみよっか」\\
「そうするしかないしね」\\
「まあ、答えてくれるかどうかの確証はないけど……
たぶん無理だろうなあー」
正直、それは私もそう思った。
アオタの知ることは、案外少ないのだ。
それこそ、私達の想像を超えたほど、狭い範囲の知識しか持たないようだった。\\

繁華街を離れて、住宅街に出る。
やっとそこそこ広い道に出れた。
用水路の上に架かった橋の並び。
近くには武家屋敷の跡があるため、この周辺はかなり綺麗に整備されていた。
古きを称える町だ。
流石にここまで来ると光も少なく、電灯の明かりだけが頼りになった。
静かな街を、どこに行くあてもなく歩く二人。
願わくばアヤメと合流できればいいのだが。

「まって」\\
セレナの手が私の道を遮った。\\
「あれ、見える?」\\
セレナは静かに問う。\\
「なにが?」\\
「あそこに、何かいる」\\
彼女が何を見ているのか全く見当がつかなかった。
私も、夜に目は十分なれているはずなのに、おそらく彼女は私と違って、
更に奥深くを見据えている。

何かがちらついた。
逆を言うと、それしかわからなかった。\\
「ごめん、カナン!」\\
私を突き放すセレナ。
同時に\scalebox{3}[1]{―}本当に完全に同時だ\scalebox{3}[1]{―}また人型が剣を構え、
私達に向かって突撃をしてきたのだ。

私達をまたも襲う人型は、やはり悪夢と言うよりは狩人に近い存在に見える。
服装も同系で、日常からはかけ離れている。それでもやはり、あれは悪夢だと断定する。

長いマントを翻して、悪夢は、また突進の準備を始める。

懐に携帯した短剣を取り出して、セレナは防御した。
擦れる金属同士が火花を散らす。
両者とも相当な力をかけていることは明らかだった。

倒れた私には目もくれず、人型、いやもはや悪夢であろうそれは、
セレナと戦いを続けようとした。
接近しては離れを繰り返す悪夢の動きに、セレナは翻弄され、
しかし決してその剣筋を見失うことはなかった。
長剣を鞘から抜いて応戦する。
リーチを得たセレナが勢いを増していくが、
剣先をすれすれで避ける悪夢の柔軟さと、セレナの愚直さとは、単純に相性が悪い。

常軌を逸した運動を絶えず取り続け、悪夢はもはや重力など無視しているようにも思える。
それでもセレナは拮抗を保つ。
私はそれにこそ驚きを隠せなかった。セレナは確実に成長していた。
私よりも遥かに、一挙手一投足に迷いはなく、悪夢を仕留めにかかっている。
その攻撃は必ず、人体の急所とされる場所に向かっていた。
それに今思えば、あのときセレナは空からの攻撃を仕掛けていた。
かなりの高さがあったはずだ。
私は彼女を見くびっていたようだった。
一体どうして、彼女はこんなにも強いのか。

回転しながら中空を飛び、セレナとの間合いを引き離す悪夢。
人の肢体は人の動きをして、全体的な調和を持った運動は、熟練した体操選手のようだ。

私も何かをしないといけない。

着地の不意打ちを狙おうと目論む。

幸い、私は何ら目をつけられていないようだし、まさか飛び道具をこちらが持っていると、
あちら側が予想できる可能性は低いと打算した。

いける。そう思った。

私はホルダーから銃を取り出して、銃を構えた。
痛む肩が邪魔をする。片手で撃つしかない。

けれど\scalebox{3}[1]{―}私は観念した。この場所に私の居場所はないのだと。
私は、自分でも気づかないほど、未熟だったのだ。

追えない移動。瞬きなど一瞬たりともしなかったはずなのに、
悪夢がどうやって自分の眼前に現れたのか、何も捉えることができなかった。
体当たりだ。首の骨を折ろうとしているのだろう。到達地点が私の首筋であることは、
すぐに検討がついた。

何もなく過ぎていく時間。

静止した精神の中で。

私は、横からの力を受けた。

アスファルトに頭を打ち付ける。

その横を飛んでいくセレナ。

腹を蹴り上げられたセレナの体は、そのまま鉄柵に衝突した。
人体から鳴ってはいけないような音が、彼女の脊髄から漏れ出る。\\
「……あ、あ、ぁぁ」\\
うまく呼吸が出来ていなさそうだった。
体がすっかり息の吸い方を忘れてしまっていて、彼女の神経は衝撃に支配されている。
苦痛に歪んだセレナの顔を見るのは耐え難く、私はどうにかしてこの状況を早く
改善しなければならないと躍起になった。
思考を巡らして、最善手を模索する。
だめだ。彼女のことで頭がいっぱいだ。
自分の痛みにはある程度無頓着になれるのに、他人の痛みにはやはり気を使わずにはいられない。

そうやっている内にも、悪夢は勝ち誇ったかのような足取りで、セレナに向かっている。
あれの驚異判定はセレナを優先している。今の彼女位抵抗できるはずもなく、
確実に殺されるだろう。もし、そうでなくとも、良くないことに変わりはない。
死ぬという結末がこの夢に用意されているのかは、信じていないが、最悪の場合を考慮する必要は、
どんなことでも当てはまるだろう。

どうすればいいのか。

私は、叫ぶことにした。\\
「こっちにこい!」\\
声に気づいて、悪夢は私の方を見る。
目の役割をしている、底知れぬ\ruby{儚}{くら}い穴。
痩けた眼窩。遠くに見える星の煌めき。

悪夢は、私を目標にした。\\
「来なさいよ! クソ野郎!」\\
思いつく限りの罵倒を浴びせる。
人語を理解するかは期待していないが、異言語間の会話であっても、その語気は伝わるだろう。
怒っていたり、恨んでいたり、悲しんでいたり、喜んでいたり。
それを見込んだのだ。

もはや逃げる気もない。

重いのか軽いのかもどっちつかずな歩みは、中心線の定まらない、覇気のない姿だ。
なにかの依存症患者のような、自己を持たない、あるいは何かに寄りかからないと立てない
人間にそっくりだ。
歩くのは転ぶことの繰り返し。

それでもいつかはたどり着くのが世界の常だ。
無限は有限に、有限は無限に帰着して、前者が悪夢の運動で、後者が私の緊張だ。

私の前に立ち止まる悪夢は、足元の私を見下す。

振り上げられる手には、剣が握られている。きっとその先は私の心臓だろう。

この夢に死の可能性はないという、かつてのアオタの言葉は承知している。

それでも死の覚悟はある。持たなくてはいけないと、心の何処かが警告を発している。
歩く威圧感は、私の精神を握り潰そうとしてくる。

きっと倒すことは叶わないだろう。

ただ、ゆらゆらしたマントが気に食わない。

私は、一発食らわせてやろうと思った。

もう一度、後ろに忍ばせた拳銃を前に出す。
傲慢な悪夢は、もはやそんな私の行動など、羽虫が集る程度だと思っているのだろう。
確かに、その通りかもしれない。先の戦いを見れば、それは私でもわかる。
だがうざったいのも事実だ。眼の中にでも入れば、無論痛む。
まあ、今回は足首だが。

とっさに足を避けようとする悪夢。
だが遅い。その前に私は引き金を引いた。
なにせ、ほぼゼロ距離だ。初速の減衰はどうなるのか、いまいち理解していないが、
弾丸は肉を引き裂いてくれるだろう。
片手で反動を受ける。腕がどこかに飛んでいきそうだった。
それはあちらも同じだ。
放った弾丸は、ブーツに似せた衣装も破って、悪夢の肉をそのエネルギーでズタズタにして、
最後には地面に転がっていく。

酷く人に似た悪夢だ。
血を吹き出している、足首に穿たれた穴は、虚空ではなく赤黒い。
焼けた肉の臭いがする。火薬の痕が黒く焦げ付いている。
妙に生々しい。だがそれは悪いことではない。
だとすれば、悪夢は間違いなく痛みを感じているだろう。
その希望は、十分に抱ける。

そしてそれは叶った。
願いは届き、悪夢の足は崩れていく。
おそらく立ってはいられないはずだ。
蹌踉めく体を、重力が地面に引きずり下ろす。
たまらずに剣を杖代わりに応急する。
スキは出来た。

\scalebox{3}[1]{―}それだけだった。

悪あがきは成功した。しかし、それ以降の建設的な行動は、まったくもって予定されていなかった。
セレナに託す、なんてことは、やはり頼ることの出来ない手だ。
彼女は未だ、治癒していない。

もはやここまでか。

あと何発か、叩き込んでやろうと思ったが、腕が言うことを聞かない。
やはり杜撰な目論見は、災いしか呼ばないのか。
激昂した悪夢は何をするのかわからない。
ただその行く末は、私の破滅だ。

今度こそ、終わりだ。

\scalebox{6}[1]{―}。

ドン、と響く音。
目を瞑っていて、何が起こったのか把握できなかった。
リセットされた視界は、まっさらだ。
殺意を丸出しにした悪夢など、どこにもいない。

あたりを見渡す。
少し離れたところに、何かがあった。

人だ。色がある。
くすんだ白色の外套に身を包んで、悪夢の首に得物を突き刺している。
吹き出す血は動脈からの飛沫。
誰だ? 顔がよく見えない。狩人であることは想定できるが、知り合いではないだろう。
それよりも、いつの間にか決着はついていて、悪夢はとっくに消滅していた。
遠くからでも、その風景は確かに捉えられた。
ナイフが、アスファルトに食い込んでいた。
墓標のように残されたそれを抜いて、その人影は私達の方に向かって歩きだした。

「カナン……あれ」\\
振り絞る声で、私に訴えるセレナ。\\
「誰!」\\
大声を出すのも辛かった。
肉体的にも、精神的にも消耗しきっていた。
そんなときに、見知らぬ乱入者の出現。\\
「答えて、じゃないと」\\
脅しは得意でもないし、そもそもやったことがない。
いざ銃口を人に向けようとしても、良心が自制する。
それとも臆病なだけか。どちらかだとしても、利き腕の痛みは構えを邪魔するし、
反対は言うまでもなく、使い物にならない。
人であろうと悪夢であろうと、得体の知れない存在というのは、
恐怖以外の何物でもなかった。\\
「答ろ!」\\
無視される。

やがて距離は、その顔を覗けるほどに近づいて、やっと人影は言葉を発した。
それが人であるという簡易な保証。少なくとも、話の通じないことよりは、
前向きに捉えることができる。\\
「災難だね」\\
若い女性の声。年齢的には、そう変わらなさそうだ。
いや、違う。アヤメと同じだ。疲れ切ったようで、またどこか達観した、
常に斜に構える姿勢を滲ませている声。
放置されたセレナの剣を取り上げて、彼女の目の前に落とす。
その姿は、私達に類似している。
服装はどこか古臭くて、いかにも戦闘用のものだ。
顔もフードで隠れ、よく見えない。
ただ、白が目立つ。
夜には似合わない色だ。\\
「アンタ、何者?」\\
「無理しなくてもいいよ。アンタたちにはまだアレは早すぎた。
頑張ったんじゃないの。
まあ、あそこで相方を見捨てれば、もっと早く終わったかもしれないけど」\\
「どう、いうこと」\\
「あそこで倒れてる女のことだって。どう見ても足手まといだったでしょ」\\
「うるさい!」\\
「そうやって怒鳴るの、良くないと思うけど」\\
「\scalebox{3}[1]{―}見捨てるなんて、そんなのできるわけない」\\
「でも、教えられてるんでしょ。ここで死ぬことはないって」\\
「……」\\
「へえ、怖いんだ」\\
「そうよ、怖いわよ」\\
私は我慢ならなかった。
どうして、私達が責められなければならないのだろうか。
それも、いきなり現れた人間に説教されるほど、不条理に思うことはない。\\
「狩人が怖がってるとか、それでも戦えるの?」\\
嫌な笑顔。引きつったにやけ顔が、その口元からわかる。\\
「戦える。私達は獣じゃない。なりふり構わず突進するほうが、馬鹿じゃないの?」\\
「あっそう。それは考え方の違いでしょ。
アンタたちと哲学論争するために、私は狩人に成ったわけじゃない」\\
飽きたのだ、と言いたげに、立ち去ろうとする。\\
「待って!」\\
「なに?」\\
「あなた、何者?」\\
「ふ、ふふふ」\\
「何がおかしいの」\\
「いや、呑気だなって。結局、肩の傷より私の名前のほうが気になってるんじゃん」\\
「それは……」\\
「教えない」\\
弁明の機会を与えずに、彼女は断言した。\\
「どうして」\\
「教えたくないから。さようなら。もうアンタとは会いたくないね」\\
吐き捨てた台詞そのままに、私は相手に言い返したかった。

が、その前に眠気が襲う。
目覚めの合図だ。
小さくぼやけていく姿は、先の悪夢と違いがない。
あれと人とを分かつものは、一体どこにあるのだろうか。

そして待っているのは、いつもめざましの音だけだ。

\section{}
アヤメもまた、同じく怪異な悪夢と対峙していた。
最初は、いつもと変わらない小さな悪夢を狩るだけの、簡単な作業だったはずなのに、
カナン、セレナと別れた途端に、わけのわからない人型が襲ってきた。

長い髪がたなびいている悪夢。黒い髪だ。
夜空を湛えたその幻想的な疑似餌に釣られて、アヤメは混乱した。
一体アレはなんだというのか。首を跳ねるに十分な機会を逃してしまった。
人か悪夢か、それとも狩人か。
アヤメはカナンとは違って、相手は狩人に近い存在だと判断した。
あれを通常の手順で狩ることは出来ない。
一対一で、やり合うしかない。
カナンたちを巻き込めば、逆に足手まといになるだけだろうと。

淡い光を背に、アヤメと悪夢は町の上を駆け回っていた。
悪夢の攻撃手段は剣のみだ。それに対して、アヤメには弓がある。
変形は一度も見せていないから、これはここぞという時の切り札にしたかった。
逃げるだけに見せかけて、どこかでスキを作らせるか。
行動に指針を定めて、アヤメは下に降りた。

込み入った道だ。
ここはそういったところが多い。
戦時中に空爆をあまり受けなかったので、古い町並みがそのまま残っているのだ。
現代人にしてみれば、車を通らせるのに、狭いは角が多いはで苦労することしかないが、
今のアヤメにとっては好都合だった。

待ち伏せは功を奏した。
角に身を潜め、過ぎ去っていく悪夢の背後をとる。
異常な探査能力は、しかし視界に頼ったものではないらしい。

背中を突く。
完全な不意打ちだ。しかし相手も只者ではなかった。
切り込まれる刃を、横に反れるように体を動かして、最小限の裂傷に抑える。
受け身を取り、立ち上がり、反撃に転じる。
金属のしなるほど、高速な斬りつけ。
なんとか弾くが、振動は柄を伝ってくる。
脳が揺れる。
それでも歴戦の狩人は違った。
好機を悟り、力任せに大振りな攻撃を連発する悪夢。
左、右、右、左……。
だが一向に刃は致命に届かない。

予測のしやすい攻撃だった。
腹ががら空きだ。
内臓\scalebox{3}[1]{―}その有無はこの際考慮しない\scalebox{3}[1]{―}に一発。
柄を肝臓に、破壊するつもりで。
ただ剣を斬り込む時間的余裕はないと判断した。
だから徹底的に殴る。

怯む悪夢を見て、まだまだだと思いながらまた一発。
今度は心臓に目掛けて。
悪夢は胸部を殴打されて、肋骨が軋む。

このままでは死ぬ。
打撃が致命傷にはならないだろうが、いずれ、この傷は命取りになる。
本能がそう囁く。警告は興奮剤でもある。
それは、人も悪夢も変わらないだろう。生命としての、獣の本質だ。

握る凶器も忘れて、悪夢はアヤメの肩に噛み付いた。
右肩に食い込む犬歯。肉を噛みちぎろうとする。
たまらずに振り払おうと、アヤメは悪夢をがむしゃらに叩くが、打ち込まれた楔は、
一向に抜ける気がしない。
死に物狂いの行動かと思いきや、しっかりと右の手首を掴んで、剣を使わせまいとしている。
腐っても知性はあるのだというのか。闘争本能に過ぎない、単なる悪戯の類なのか。

もういい。肩の肉なんざくれてやる。
彼女の沸点はとっくに到達していた。
わがままに地団駄する子供に対して、しびれを切らして引きずっていこうとする親のように、
噛みつく顎を強引に引き剥がそうとする。
自分から前に出て、下顎を揺らす。
同時に悪夢を足で突き飛ばして、トラバサミのような歯牙から解放される。
抉られる肉を対価に、アヤメは再び攻勢に出ようとしたのだ。

そのまま膝で腹を打ち、悪夢の体勢を崩す。
強力な一撃だ。なまじ人間の構造で繰り出せる技ではない。
ひしひしと凹んでいく柔らかい肉の感覚。
それすらも行き詰まって、たどり着く硬い骨の硬さ。
衝動に身を任せ、ただひたすらに暴力を繰り返すアヤメの顔は、
禍々しい形相を晒している。目元だけでも、憎しみが伝わってくる。
なぜ、彼女はこれほどまでに固執するのか。もはや悪夢に戦意はなく、
死に体だというのに。

動かなくなった悪夢。ぐったりとしたそれの頭を鷲掴みにして、
片方で剣を展開して、弓を番う。
その後に、悪夢を地面に叩きつけて胸を踏む。
もはや逃げる気力もない悪夢は、ただ呆然と矢の先を見つめている。

耳障りな音がする。

赤子が喚き散らすような声。
しかし所々が枯れていて、老人のようにも聞こえる。
響く雑音では、アヤメの心を微動だにしない。
慈悲などない。弓を引く。
わざわざ剣でとどめを刺さずに、弓で決着をつけようとするほど、彼女はこの戦いに
執着していたのだろう。

至近距離からの一矢は、完全に頭部を貫通して、悪夢の生命を絶った。\\

欺瞞に満ちた流血を、雨水が洗い落としていく。
遺体は珍しく、長いこと残留していた。
漂白された骨格を、暗闇から抉り出して、アヤメは宙に掲げる。
大腿骨あたりのものだろうか。
興奮はとっくに冷めて、彼女は冷徹な観測者に立ち戻った。\\
「小さい」\\
子供のものだった。女児か男児かを判定できるほど、彼女に解剖学的な知識はなかったが、
それでも、違和感を感じ取れるだけのものはあった。

胸部の変形した肋骨の配置。

陥没した頭蓋。

そのどれもが、人の骨格図にそのまま透かして見れるだろう。

不意に、彼女は悪夢の顔を抑えた。
人の顔に見えたのだ。
しっかりとした人格を持った表情に。
眼球を放り出そうとばかりに、見開く瞳孔の開ききった眼差し。

それを最後に、悪夢は蕩けていった。

\section{}
色あせていく写真を透かして、天井を仰ぐ。
腕の重なる重みが眼球を圧迫する。
寝台に身を預けて、私は古い記憶に思い巡らそうと思った。
いや、絶対的な時間など現実にはないのだから、この表現は間違っているのかもしれない。
もしかすれば、これはつい昨日の出来事なのかもしれないという疑問は、誰にも否定できない。
ただ感じるのだ。どこか遠くへ過ぎ去っていく流れを。

けれど、色調を失って、茶色く変色していく風景の中で、
それでも彼女はその色彩をはっきりと保っていた。

自らの意思で遠のいていく記憶の中の少女。
手を伸ばさないといけない。
明日、明後日と日をまたぐほど、彼女との距離は開いていく。
単純な線形だが、だからこそ残酷。

私は焦っていた。喧嘩をしたわけでもない。
なのに、いつの間にか疎遠になっていた。
私も彼女も、接点を失ってしまったのだ。
自分ではそれを望んでいなかったはずなのに。
話しかける勇気がなかった。私は小心者だった。
だから黙って通り過ぎる。
今日も明日も明後日も。
憂鬱だ。学校に行きたくない。あの空気を吸うのはもうごめんだ。
肺が重くなる空気。同じ空気を吸っているはずなのに。

煙たい空は嫌味に写る。
何もかもが私の敵に見える。
記憶の一部、灰色に侵された世界。
立ち入ることも憚れる領域。
人の顔はじゃみじゃみに塗りつぶされて、口を引きつって笑っている。
人のまばらな廊下を歩いていく。

トイレに行きたかったのだ。

狭い空間。

彼女と鉢合わせた。

その中で唯一覚えている色がある。

赤だ。

線。

やめて。

私は、そのまま寝た。




\end{document}
