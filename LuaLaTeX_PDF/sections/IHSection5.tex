\documentclass[../IHMain]{subfiles}

\begin{document}
\chapter{無限\\という病}
\section{}
夕暮れに照らされて、呆然と立ち尽くす私の前を、多くの瓦礫が飛んでいく。
何が起こったのか。耳が聞こえない。耳鳴りに支配されている。

早すぎる狩りの夜、それは、不吉の前兆だったのだ。

虹彩は取捨選択を忘れて、過剰な光を眼球に送り込んでいた。
赤色の光。散乱した太陽光は、単純に眩しくて、世界の認識を阻む。

体がうまく動かない。手足が拘束されている。重たい住居の残骸を、いや待って、
どうして瓦礫が生じているのか。今までそんなことは、一度たりともなかったのに。
小石すら蹴飛ばせない悪夢が、どうしてこんな破壊をもたらしたのか。
よく考えると、例外が一つだけあった。いつかの人型、その剣先が刺さり、窪んだアスファルト。
どういうことだ。私は、混乱した。

微細な埃が、口の中に入っているようだった。
舌がジャリジャリして、噛みあわせる度に硬質な音が響く。
吐き出してまだましになったが、願わくば、水で濯ぎたかった。

現実への介入は、セレナの言葉から始まった。\\
「カナン! アヤメさんを!」\\
彼女は戦っていた。大きな口の悪夢。もはや口だけなのではないだろうか。
怪魚のように空を跳ね回って、セレナを翻弄している。
私はセレナを助けようと思った。
しかし、彼女の言葉を今更に理解した。

私は辺りを見渡す。
瓦礫から身を掘り出して、立ち上がる。
擦りむいた肌は埃の充満した空間で、赤を晒して痛みを誘う。
アヤメを探す。
私の更に奥、もっと混雑した崩落現場。
その中心に、彼女は閉ざされていた。

目覚めない彼女。
私は急いで作業を開始した。
掘削は困難を極めた。ギザギザと凶器に変貌した、コンクリートの重たい破片を持つことは、
今の不注意な脳では、どこをどう持てば安全だとか、
どうすれば楽に撤去できるなどといったことを判断できないから、私の手を容易く傷つける。
人差し指と親指の間に、刃のような破片が高速で滑る。
血が出た。落としてしまったのだ。
だが、気にしている場合ではない。

やっと彼女の体が見えてきた。

私は、手を止めた。

彼女の左の瓦礫が、尽く朱に染まっていた。夕日の朱よりも、更に際どい朱。
鮮血の色。私は直感した。信じたくはなかったが、認めざるを得ない。
恐る恐る、血に染まったタイルの名残を避ける。
血で滑りかける。

「アヤメさん……」\\
「うぅ……ぅ……ぁ」\\
微かなうめき声。彼女自身は、まだ気付いていないようだった。
それがずっと続けばよかったのに。

アヤメは目を開いた。
ぼやけて虚ろな視界にも、その朱は鮮烈に光るだろう。\\
「痛い……」\\
「アヤメさん、大丈夫ですか」\\
そんなわけないことぐらい、誰でも分かるはずだった。\\
「痛い、痛い、ない、ない、ないよ、カナン。どこにあるの? どこに? ねえ、ねえ、ねえってば!」\\
「\scalebox{3}[1]{―}わかりません」\\
私の胸ぐらをつかんで、立ち上がろうとする。
左腕を失った相補だろうか、彼女の腕力は驚くほど大きかった。体幹がいとも容易く崩される。
けれど彼女が立ち上がることは出来なかった。痛みに耐えかねて、破断した肩口を手で押さえる。

解れた腕の接合部。筋肉繊維がそのままに人形の縫糸に見える。
断面を注視する度胸はなかったが、見た限り骨も砕けている。
素人目に見ても、それは永久的な欠損以外の何物でもないことが分かった。

だが、ここは狩りの夜、夢に極めて近い現実。
傷は継承されす、ただその記憶のみが保存される。
それが決まり。この世界で最も信頼度の高い規則。
疑うことは腐るほどあっても、その格率は常に検証を耐えてきたし、私達はそれによく従っていた。
私は急いで、彼女を目覚めさせようとした。
嫌がる彼女の顔を掌で覆って、視界を奪おうとする。
暗闇がいつも目覚めを誘ってくれる。そのはずだ。

眼球が景色=世界との接点つまり、今と私を繋ぐものであるのだから、それを断ち切ればいい。
安直な考えだが、妙に哲学的で権威的な説得力を持っていた。
格式的な表現を以て、自分を納得させるしかない。

アヤメの眼球の形を直に感じる。
ゴムに包まれたビー玉を転がすよう。
なるべく圧をかけないようにと心がけるが、手加減が出来ない。
瞼を閉じさせながらも、まるでそれが、
嫌なものからは目を逸らさせるような、愚かな行為なのではないかと勘ぐった。
ただそれ以外に、私に持てる術はない。

治療はできるはずない。抱えて逃げることも、できそうにない。
右足を骨折していた。
程度はわからないが、腫れ上がった脹脛を鑑みて、いや、
それを知ってしまったから歩けなくなってしまったのか。
踏ん張ることが、人生で初めて嫌になった。

だからこうするしかないのだ。私の手を払おうと、アヤメの腕が傷口から離れる。\\
「ああ、だめ! 押さえててください!」\\
止血の方法は原始的な方法しか知らない。
暴れるアヤメの腕は、もはや手がつけられない。
無理矢理にも肘を肩に当てる。
根本を圧迫すれば\scalebox{3}[1]{―}鼻血の対処法のようにだ\scalebox{3}[1]{―}
とりあえず止まってくれるはず。
死というものがこの夜に保証されているのかは、まだ断定できない要素の一つだが、
緊迫した彼女の顔を見れば、そんな悠長に目覚めを待つだけにはいかない。\\
「あぁアアアァ\scalebox{3}[1]{―}! 腕! 私のうで」\\
動かない足を動かす。結局それは咆哮に変化する。\\
「痛い\scalebox{3}[1]{―}」\\
「頑張ってください、アヤメさん。
早く、早く目覚めて!」\\
「ああ! 何も見えない。腕、見えない。どこにあるの」\\
「アヤメさん! 動かないで! 血が、血が出ちゃう!」\\
体勢は限界に近づいていた。
角度はもはや重力に屈服する間近だ。

アヤメの顔が、文字通り白くなっていく。
血の気が引くとはこういうことか。
本当に肌が白くなっている。血の色味が消えた肌は、これほどに冷たいなんて。
それに伴って、彼女の呼吸は不規則なものになっていった。
ついには音を出すことも辛くなったのだろう、口からは空吹きしか聞こえない。

嘘だ。この先は死じゃないか。

私は激昂した。約束と違うじゃないかと。誰に怒るわけでもなく、
ただ心の中で毒ついた。ここで取り乱している場合ではない。
焦りは疲労に変換されて、私はついにアヤメの顔から手を離した。

目を開くアヤメ。

左腕を嘆く。

涙もなく、声もない啼泣。

見てて辛くなる。心の奥底が突き上げられて、吐瀉物として吐き出しそうになる。
無力さに手が震える。

「あーあ、だから言ったじゃん。\scalebox{3}[1]{―}気をつけなよって」\\
頭上からの声。聞き慣れない子供の声。
中性的で、二次性徴の迎えていない純粋無垢な、『人』の声だ。
人が生まれるままに持つ音、識別子。
それはすぐに、私の頭の中の手帳に記憶された。
警戒するべき人間として、あるいは、敵として。\\
「可哀想に、腕ちぎれちゃって」\\
「可哀想? 何言ってるのよ! アンタにそんなこと言う権利なんてない」\\
「人間としての思いやる気持ちを否定するんだ……」\\
「だとしても、そこで黙ってみてる人は信用できない」\\
「そりゃあそうだけどさ。何も出来ないじゃん。意味のないことをするなんて、馬鹿だなあ」\\
無意味だと? 確かに、目覚めればすべて丸く収まると高をくくっていた私もいる。だが、
眼の前に苦しんでいる人間を、嘲ることなどできるだろうか。
彼にとってはどれほど愉快であっても、これは私の最も重要な問題なのだ。
彼女を救えなければ、自らを信じることが出来ない。
狩人を続けることも、これ以上は無理かもしれない。
私は弱いからだ。死を背負うことは出来ない。

「身勝手なんだだなあ、結局。自分が可愛いだけじゃないか。
人が死ねば、自分も死ぬ。帰納的だね\scalebox{3}[1]{―}あれ、違うかな? まあいいか」\\
「そんなこと……ない」\\
「嘘だよ。だってそのほうが自然でしょ?」\\
「私は、死んでほしくないの。生きてほしいの。それが当然。助けようとする心がある」\\
「寂しいから?」\\
「はあ?」\\
「怖いんでしょ。認めなよ。アヤメが死ぬと、自分で何をやれば良いのかわからなくなるから。
自分が孤独に苛まれるから、逃げることが出来ないから。頼ることが出来ないから」\\
「気取ったこと言ってるつもりかもしれないけど\scalebox{3}[1]{―}」\\
何一つ理解できない、と反論を試みようとしたが、そのまえに、
二人の意識は大きなものに斥力を受けて、この会話は中断された。

それこそ、必死になって看病していたはずのアヤメのことすら忘れて、
私は眼前の事象に注目していた。

あの時の女性だ。
白い、てるてる坊主のような装いが、空中を闊歩している。
体の芯に響く轟音を唸らせながら、悪夢を攻め立てていく。
まっすぐ歩みを進めると、自然と空へ昇っていくように、
彼女の運動\scalebox{3}[1]{―}全くの道具を用いない\scalebox{3}[1]{―}は三次元に拡張されいる。
そして大きな仕事を、悪夢に対して果たす。
上に登る悪夢の図体。
重力に逆らう力の向き。
口を突かれ、嗚咽を漏らす。
セレナが手詰まっていた相手を、あっさりと蹂躙していく。
まるで、かつてアヤメが私達を襲う悪夢にしたように。

巨大な暗黒は、斜陽に向かって飛んでいった。
赤に塗りつぶされる黒。
色環の序列からは考えられない反逆だ。
駆逐される虚空。世界の穴。
修正は逐次行われて、ついには埋め尽くされてしまった。

違う。こんなことに意識を割いている暇はないのだ。
今は、一刻を争う状況だ。
アヤメの顔を見る。意識は薄れかけているようだが、
そのおかげでか呼吸は落ち着いていた。
本能的な領域へと、やっと制御権が委譲されたようだ。
止血、そうだ血は止まったのか。
凝固した血液の成す結晶、赤黒い鉱石が彼女の末端に形成されていた。
早すぎると一瞬思ったが、知識のない以上どうこう言うことは出来ないし、
今は純粋に落ち着くべきだと思った。だが\scalebox{3}[1]{―}。

アヤメは目を瞑っている。

私も、目を瞑りそうになる。

待て、待て、待て、待て、待て、待て。ここで目覚めるべきではないだろう。
必死に言い聞かせるが、すでに自重によって瞼は下がっていく。
低きに落ちる、重力に逆らえない物体の性。私はそれが恨めしい。
悪夢の消滅は、確かに目覚めの告知でもある。だがここで去ることができる人間が、一体、
何をできるのか。この場所に居たい。狂い出す私の呼吸。荒い、悪夢にうなされる呼吸。
息の熱い流れ、対流する空気のもどかしさ。汗のじれったさ。聞こえる生活音。
水の音。蛇口から流れる音。

……嫌だ。

目覚めるな。

薄目を堅く閉ざす。

……既に目覚めていた。

暗闇はただの瞼の裏側。私は、目を見開いた。\\

「あ、おはよう」\\
お姉ちゃんの声。私はリビングのソファーの上で、眠っていたようだ。
時計を見れば、まだ六時を過ぎたばかり。
冬ならまだしも、夏の今頃はまだまだ小学生でも活動可能な時間帯だろう。\\
「アンタが昼寝なんて珍しいね」\\
何も知らないお姉ちゃんの、呑気な声。
コーヒーを淹れながらの、他愛ない雑談。

私は構わずに、二階へ駆け上がった。

学校から帰った後、すぐに眠ったのか。
曖昧な記憶を頼りに、バッグを放り投げた場所を探る。
私の部屋の、ちょうど真ん中に、無造作に置かれていた。
ファスナーを開いて、中の携帯を取り出す。

なにげに、自分から電話の機能を使うのは、初めてな気がする。
だから、使い方がわからなかった。
適当に、想定されたデザインを頼りに、私は通話を試みる。
電話帳から、たしか受け取ったはずの電話番号を探す。

あった。

私はそれを叩いて、電話をかけた。

自動的なアナウンス。

電話がつながる、その一途の期待はあっさりと潰えた。

私は、いてもたってもいられずに駆け出した。
転がりそうになりながら、階段を降りていく。
一段一段が遠く感じる。

「どうしたのカナン?」\\
お姉ちゃんは聞いてきた。
そんなことにいちいち反応している余裕もない。
玄関に飛び出て、靴を履こうとする。
かかと部分が潰れて、うまく履けない。
ああ、こんなことでもたついている暇はないのに。\\
「なんか急いでるんなら、送ってあげようか?」\\
「いらない!」\\
やっと足を靴の中に収めて、私は玄関を体重で開けた。

掠れた記憶を頼りに、私は地面を蹴って走り出す。
狩りの夢、その地理的な情報は、十分に覚えている。
おそらくだが、それほど遠くない場所だと思う。
たとえそれが無駄足だとしても、止まっているよりはマシだ。

帰宅する児童や就労者の列を、逆走していく。

\scalebox{3}[1]{―}たどり着けない。

間違いなく景色は覚えている。ただやはり、その光景が今私の居る場所から、
どうやって接続しているのかが、わからない。

どうしよう。こんなところで、ただ町並みを眺めることしか出来ない。

夕日の赤色がうざったい。

だから、とにかく走った。

やけくそになって、私は足を動かすことしか脳になかった。
この加速が、どこか目的地に導いてくれるはずだと、有りもしない希望を抱いて、
私の加速度は指数関数的に増加していく。
体力の尽きるまで。
筋肉の断裂するまで。

しかし限界は驚くほど早く、私の体をアスファルトに縛り付けた。
反作用に耐えきれず、足裏は過度に変形したような痛みを発している。

これが夢の中なら、もっと走れるはず。
息も切れずに、疲れも、痛みも知らずに、ただ心折れるまで遠くに行けるはずなのに。

「どこに行くつもりなの」\\
眼の前からの声。苦しくて、下を向いていた私には、それが自分に投げかけられた問である
ということに、若干の遅れをもってから理解した。\\
「あなた、さっきの……」\\
服装は違うが雰囲気は全く同じだ。
すべてを見下して、自分は一人別の世界に生きていると鼻にかけている態度。\\
「覚えてくれてたんだ。意外」\\
覚えるも何も、私は、今彼女を知った。
その顔を目に焼き付ける。
たとえ名前を知っていなくても、私はその顔を頼りに、いつか雲隠れする彼女でも、
暴き立てるつもりだ。

黒髪のロングヘアー。年齢は私達と同じぐらいだろう。
耳にはピアスを付けている。十字架を象ったように見える。
黒いTシャツに、デニムのホットパンツ。地味だがそこそこ目に留まる服装をしている。
元は端正な顔だと思う。ただおそらくは、年齢の割に大人びている顔だと言われるだろう。
目と口を結ぶ綺麗な三角形。
薄い唇。
荒んだ目元。
少々痩けた頬。
私は、ヘロインシックという言葉を思い出した。
ただそれに実年齢がついていけていない。
どこかぎこちない、わざとそうしているように見せている、無表情でぶっきらぼうな顔の雰囲気。
徹しきれていないのは口元だ。
歪んだ口角を隠しきれていない。恥ずかしいのだ。
自分の態度が、自分に相応しくないと悶ている。\\
「どうしたの。なんでそんなに急いでいるの?」\\
「アヤメさんはどうなったの」\\
かなり低い声が出た。\\
「さあ? あんな腕になっちゃったら、もう終わりだよ」\\
「死んでないよね」\\
「それは多分」\\
彼女ははぐらかそうとした。明言を避けている理由が知りたい。\\
「ねえ、おかしくない? どうして夢の中の怪我なのに、アヤメさんの腕は治らないの?
あなた、嘘でもついているんじゃないの。私達をどうしたいのか分からないけど、
いい加減なことは\scalebox{3}[1]{―}」\\
「彼女はもう、戻らない。これは確定事項。
いつまでもうじうじ言ってる必要もないこと」\\
「なんて、言ったの」\\
「聞いてたくせに」\\
「そんなわけない!」\\
私は激情に身を任せようとした。手は勝手に前進して、彼女の胸ぐらを掴もうとした。
だが抑える。食い込む爪が食い止める。\\
「治らないの? 本当に?」\\
「さあね? でもそうなんじゃないの。
だって、彼女は目覚めなかったから」\\
「\scalebox{3}[1]{―}だったら、私の願いで治す」\\
「それは無理だね」\\
「どうして」\\
「きっとアイツは言うでしょうね。それは出来ないって。
本当に心苦しくて、申し訳も出来ないが、僕たちにも限界があると。
安っぽい御託を並べて、適当にあしらうだけ」\\
「そんな、そんなはずない。願いは叶えられる。
人の腕ぐらい、治せるはず」\\
「傲慢だね。そもそも、アイツと何回喋ったの? そんなに信頼できるほど、
考えてみれば、会話もしていないはず。狩人に煽てられて、
君ならできるよって無責任に言って、その気になったら後は放置じゃない」\\
私は、確かにそうだと頷いてしまった。
心の中でだが。
アオタ\scalebox{3}[1]{―}その名前すらも、今では訝しいが\scalebox{3}[1]{―}との会話というか
会うこと自体が、片手で数える程度、
それも狩人になりたての頃にしかないことは、紛れもない事実だった。\\
「願いを叶えるのは、契約の対価の……はず」\\
「契約を果たせなければ、報酬を払う必要もない」\\
「何が言いたいの」\\
「使い潰せばいいってこと。\scalebox{3}[1]{―}騙されたんだよ。
あのホラ吹きが、あなたたちだけと契約しているという確証はない。
きっとどこかで、第二第三の狩人を勧誘している。
その先は決まっているのに、夢をちらつかせたり、啓発セミナーじみたことを言っている」\\
「それは、私達が弱いから。アオタは励ましてくれてる。
そもそも、それと約束を守ること云々は関係ない。
私は認めない。きっと叶えてみせる」\\
「自分が必死になって、それこそ、死にそうになっても、まだそんな事言うんだ。
全部ウソだって、どうして疑わないのか。わからないね」\\
「私は戦う。あなたのおかげね。何を知ったかぶりして、偉そうにしているのか知らないけど、
アンタを信じないし、いつか思い知らせてあげる。それでいい。それが戦う意味。
これはまやかしじゃない。誰かの為になるって。証明してみせる」\\
「へえ、じゃあ、頑張ってね」\\
「そう言うアンタは、何がしたいの?」\\
「何も変わらない。私も、誰かの為になりたい。でもずっと現実的よ。
この世界を救う、正しいやり方。それをするだけ」\\
「正しいやり方? アヤメさん見捨てるような奴に何が」\\
「見捨ててない。彼女は死んでないって、言ったよね。
それは私が助けたから。きっと、うまくいってる」\\
「言葉だけなら、どうにだって言えるでしょう」\\
「もういい。アンタ、意外と自己中っぽいんだね。まあ、狩人にはお似合いだね」\\
「ああ、そうですか。それは、ありがとうございます」\\
完全に投げやりだった。\\
「じゃあ、さようなら」\\
彼女は後ろを向いて、歩いていった。

二度と会いませんように、と陰口を叩いて、私も家に帰ることにした。

会話を反芻する。

とりあえず、反省する。

アヤメのことは、もう、諦めるしかないだろう。
彼女の腕は、既に失われてしまったという現実を、私は理解しなければならない。
だが\scalebox{3}[1]{―}都合よいことは安々と信じるのかと自分でも思うが\scalebox{3}[1]{―}
あの女が言う限りは、死んではいないはず。
助ける方法はある。
私は、狩人になるための願いを留保していた。
自分のために、何かを望むことに躊躇してしまったからだ。
だが今は違う。目的が出来た。私は、叶えるべき願いを得た。

迷うことはもう、やめよう。
かつての志は捨てて、今はただ、俗物へと成り下がる。
私には、その覚悟があった。人の為というのは、決して高潔なことではないのだ。
私はやっと気付いた。

だから、強くなれるのだ。

その自覚が愚かなものだとしても、恐怖を塗り替えるには十分なものだった。

\section{}
遠い記憶に思いを馳せる。
普段私は、これを良しとしない。
思い出の固執は、世界に盲た人間か、あるいは本当に古くなってしまった者のよすがだと
思っているからだ。それは、今の私ではない。少なくとも今は、前へ歩むことを諦めてはいない。

けれど、ときたまに、どうしようもなく、
部屋の四隅に縮こまって、ただ俯いて時間が過ぎるのを待ちたい、
そんな誘惑に打ち勝てないとき、私はその記憶を頼る。部屋と言っても、心の中の部屋だ。
現実的な物差しで測ることの出来ない距離、その壁が直行する角。
そこだけが、夕日に照らされている。

それにしても、遠いというのは、どういうことなのだろうか。
現在地からの純粋な距離なのか、起立した足の長さなのか、床の低さなのか、
それとも私の視座の高さなのだろうか。

私は、心の距離だと考えた。沈殿して堆積している日々の記憶、日常の残滓、
その基底からの距離。そこが私の基盤で、だからこそ侵し難いほどに遠い。

だからこそ、侵犯の許されるときは、
私にとって心身の危機の克服、あるいは精神的脆弱性に抑圧されている我が身の内省なのだ。

追憶は主観ではない。記憶の再生は、映像ではなく朗読なのだ。
種々の感情を、その起因とともに体験できるが、しかしそこには言語化という隔たりがある。
決して本心には近づけないが、それ故に自制、安寧を与えてくれる。
身の休まることのない、現実の世界から逃避すること。
それが、私に残された唯一の方法なのだ。\\

私は、廊下に立っていた。
理科室の前だ。ガラス張りの戸棚が、ショーウィンドウのようにビーカーや三角フラスコ、
その他諸々の実験器具を、これでもかと見せつけている。
ここが特異な場所に感じるのは、この仕掛のせいだろう。
自己主張の激しい理科室の中には、私が入ろうと試みている科学部の活動が行われている。
入部希望の動機は単純だ。楽だからだ。形骸化した部員名簿の中に、その名前を少し書き入れてもらおうと
思っただけだ。なぜか教師たちは子供に部活を強制する。それが心身の鍛錬になると、
本気で思っているのだろう。馬鹿らしい。私は興味のないことをやらされるために、
この学校に進学してきたわけではない。まあ、中学校なのだから、
自己意思云々は端から関係もないが。だからだ。不純と言われようともなんともない。

今日は入部希望をちらつかせるだけでいい。顔を少し出して、形だけでも興味があるように
見せてやろう。ほんの少しの保身。つくづく私は自分が嫌になる。
大人ぶっている自分が。

扉を叩いた。ぐらついている立て付けの悪い扉を、無理やり開けよとする音。

私は、中に案内された。

数人の着席している部員と、部長だろうか、長身の女生徒が立っていた。
部員たちは顕微鏡を使って、薄くスライスされた植物組織の、
その染色された細胞を見ていた。\\
「こんにちは」\\
向こうから話しかけてきた。\\
「あ、見学……いいですか?」\\
「良いわよ。適当に見てって」\\
第一印象は最悪だった。彼女の目はあまりにも機械的過ぎる。それでいて繊細な、薄氷のようなレンズ。
冷たい観測者。斥力を感じる。いや、目に見えない矢印を見ようとする私も、大して変わらないだろう。
私は気付いた。似た者同士、ここで惹かれ合ったのだと。
ただ互いに引かれ合う力が発生している以上、その反作用も存在して、最初はそれを敏感に感じたのだ。

ここで記憶は飛ぶ。

早回しに。

私はこの間の心情を補完する。

当初の目論見は完全に消え去って、私は、必ず部活に行くようになっていた。
部長\scalebox{3}[1]{―}いや先輩に、会いたくて。
部活のない日でも、玄関で彼女が出てくるのを待っていた。
ストーカーみたいだ。私の心情は現に、それに近いものだっただろう。
彼女は生まれて始めて、分かり合える人間だと思ったからだ。

私のこの白い髪と、周囲に同化出来ない風貌は、多くの斥力を生み出していた。
友達への斥力。周囲への斥力、家族への、教師への。そして、社会からの、私への斥力。
自分にも非があることぐらい、十分に承知している。
だがそれを克服してまで、他者に対して触れ合いたい思うことはなかったのだ。
けれど彼女は違った。最初こそは、冷徹な、それこそ人なんてものに微塵の関心もない、
心無い人だと感じていたが、それは、純粋な優しさだったのだ。
優しいからこそ、気の利いたことは言えないし、自分でどうすれば良いのか解っていないのだ。
それは普通、不器用だと嘲られることだろう。私はその意見が嫌だった。
彼女の時折見せる何気ない仕草や、物憂う表情は、
少なくとも、未だ幼稚園児のように学校を走り回ったり、
陰湿な陰口を自慢し合ったりする、大人の汚さを変に真似するその他の人間より、
よっぽど価値のあるものに見えた。

私は、そんな彼女に、いつの間にか依存していた。
私は彼女という存在を求めていたのだ。
彼女こそが、私に欠けていたものを埋め合わせてくれる、ただ一人の人間だと。
たとえ彼女がそこまで私を思っていなくても、それでよかった。

彼女は、私の密かな支柱となっていった。私は、それを律することなく、
むしろ推奨していた。性に合わないと、頑なに心を閉ざしていたことの、
なんと低俗だったことだろうか。

「先輩の家に遊びに行ってもいいですか?」\\
私は聞いてみた。
感慨もなく、ただ窓の外を彼女と一緒に見つめて、
あまり面白くなかったから、会話に起伏をつけようとした。\\
「だめですか?」\\
彼女は答えなかった。ずっと、外を見ている。校舎裏の景色に目を向けたまま、沈黙を貫いている。
考えあぐねているのだろうか。

家に遊びに行くのは、一種の到達点だと思っている。ごく個人的な領域、しかし家族との共有の場を
他人に提供できることは、相当な親しさか、下心がないと出来ないだろう。
私も、小学校の頃はよく遊びに行っていた。別段仲がいいというわけではなかったが、なんとなく、
付き合うべきだと思っていた。それが自分の身を助けるなにかになると、幼心に打算していたのだ。

ただ今回は違う。純粋な好気だ。彼女の家、そうでなくとも、一緒に遊びたかった。
外に出て、この教室以外の場所、そこに立つ彼女を見てみたかった。

けれど、「だめ」と返ってきた。

私はその時の顔を忘れない。
悲しい顔だった。私と同じ顔。動かない顔だ。
だから何かを抑え込んでいる。目尻も広角も、一切微動だにしない。
ただ音を出すために、必要最低限な筋肉の動きだけが、彼女を表現していた。

家庭環境の悩みは、誰にも相談できないだろう。

私は辞退した。

「いつか、遊びましょう」\\
彼女は言ってくれた。
その言葉を信じて、私は舞い上がったが、今振り返ってみれば、結局、
一緒に遊んだことは一度もなかった。

いつも、その事実に気が付いて、この場面は終りを迎える。

その瞬間、割れる窓ガラスと、ベランダへ続く扉。
きっと、はるか上空を飛ぶ気密された飛行機の、その窓や乗降口を一気に開ければ、
同じような光景になるだろう。
紙は乱れ飛び、ガラスは散乱して私達に襲いかかる。

現実にはありえない現象の記録。
一体どの部分が、この現象をエミュレートしているのだろうか。

擬似的な気圧差はついに壁まで砕いて、空に続く地面を生む。
そこは、つい最近の景色だった。

色彩は変わらず、夕焼けの世界。
本来空中であるはずの空間には、瓦礫とかしたコンクリート造りの住宅の残骸が
形成した土台が広がっていた。

あの時の記憶。

アヤメさんが腕を失った日のものだ。
私を悩ませる災厄の日。
その再生は些かに苦しく、しかし向き合わなければならないという焦りもある。

私は既に舞台に立っていた。

私は視点を動かす。

目覚めることのない、苦痛にうなされるだけのアヤメさんの姿は、目を背けたくなる。
カナンを探すが見つからない。彼女は既に目覚めていた。
だから、この先の出来事を彼女は知らないだろう。\\

「アンタたちどうするつもり」\\
私は聞いた。このまま彼女を放置するのか、それとも何か手を持っているのか。
彼女たちの余裕は、不自然に見えたからだ。
……あるいは、他者の生死に頓着しない、異常な人間だというだけか。

白フードの女と、金髪に染めた褐色の少年。
それらの接点が、一体どこにあるのかは探る必要もないし、おそらく不可能だろう。

私には何も術がない。
カナンは既に目覚めてしまったし、私は実質一人きりだ。
かつての私。頼るもののない孤独な世界。
だがそれがどれほど無力で、情けない状態であるかは、嫌というほど知っているし、
だからこそ、私は目の前の不審者に頼らざるを得ない。\\
「これ、携帯」\\
「携帯? 電話なんてどこに掛けるの」\\
「もう掛けてある」\\
「え?」\\
私は画面を見た。救急の番号。
彼女は救急車を呼んだのだ。\\
「こんなことして、どうにもならないでしょ」\\
「どうして? 怪我をしているのに、病院に行かないほうがおかしいでしょ?」\\
「どう説明するつもり。まさか、悪夢に食べられました、って素直に言うつもり。
信じてくれるはずない」\\
「放置でいい。勝手に連れてってくれる。理由がわからなくても、すぐ傍に血まみれの人間が
倒れていたら、助けようとするのが人の性」\\
「途中で、死んでしまう可能性もある」\\
「大丈夫よ。死にはしない。彼女はもう戻れない。
人にも、狩人にも。それは終われないということ。
人としての死も、狩人としての目覚めも、どこにもない。
待っているのは\scalebox{3}[1]{―}」\\
「聞きたくない」\\
「どうして。拒否しても、なにも変わらない。
あなたは真実を知るべきよ。私達と一緒。
きっとうまくやれる。
あのお友達より、少なくともあなたの本性には、理解を示せる自信があるのだけれど」\\
「本性? 私は何も隠してなんかいない。私は、カナンを……」\\
「守る。そんなこと言ってたら、自分の身も守れなくなるよ、お姉さん?」\\
いきなり口を出してきた少年。なにか、引っかかるものがあったのか。\\
「私は、アンタたちとは違う。アヤメさんも見捨てない。カナンもそう。
弱いから、役立たずだからって、簡単に人を裏切らない。私には、もっと理由が必要よ。
アンタたちが思っている以上に」\\
「そう……お姉さんって案外厳しいんだね」\\
「これ以上は無意味ね。それに、もうそろそろ救急車も到着するんじゃないの」\\
「そうだね。お姉さんも乗り気じゃないし」\\
「さようなら、時国さん。またどこかで会うでしょうね。
狩人を続けるならば」\\
「私は嫌だけどね。さようなら。もう二度と、会いたくない」\\
「手厳し―」\\
「ほら、行くよ」\\
「はーい」\\
姉が弟を引き連れていくような関係。
私は、彼女たちの立場がますます分からなくなった。
目覚める、というより、普通に徒歩で帰っていく。
まだ夢の中に居るのか。

私は、自分を鑑みた。

私も確かに、目覚めが来ていない。
悪夢を倒せば、今までも半ば自動的に目覚めが訪れていたのに、
今日はそれがない。だが、カナンは目覚めている。
なぜ私は目覚めていない。
自らの意思で選んだつもりもない。
深層心理の影響か。
……私には、今の状況を正確に解釈できる知識も知力もなかった。

だから他のことを考えよう。
一体、彼女たちは私達をどうするつもりなのか。
なぜ、私を仲間に入れようとするのか。
優先度を設定するならば、アヤメさんが最も適していたはずだろう。
私達二人よりも、ずっと強い。
私ならそうする。
腕の一本や二本を失っても、彼女なら戦えそうだった。
そんな気迫を感じた。
それとも、本当に腕の破損は、致命的な傷だということか。
だがなおのこと彼女は生きている、いや、死ねないという説明に、納得ができない。

夕日が闇に転化していく。日は完全に地の下にに沈んで、私を盲目にする。
ここで夢も終わりだ。記憶の目覚めとともに、現実の目覚めも訪れる。
まだ朝の五時だ。けれど、夏の朝は私が思う以上に明るくて、
錯覚を起こす。偽りの覚醒。シーツに染みた汗が、気持ち悪い。
臭いもひどい。だから夏は嫌いだ。自分の体から、汚い何かが滲み出ているという事実を、
こうも黙認させられる。

私は悩んだ。

そして朝日をちら見して、私は二度寝した。

\section{}
目覚めると、私はいつも天井を見るしかない。
そこにしか、私の場所はないからだ。
一般化された病室に、固有な私の部屋を重ねてみても窓の外はカーテンに仕切られた
隣人の気配に塗れているし、ドアを開けて出ていこうとしても、
それはただの大部屋の一部を出るに過ぎない。
常に、個を保つための境界線は揺らぐし、私は耐えられなかった。
だから天井を仰ぐ。そこは永遠に真っ白な座標に、動的な模様が蠢いている。
もちろん、実際は静的なパターンに他ならないが、錯覚がそれを動かしていた。
この模様の運動が、唯一私の持つ、私だけの内密の領域なのだ。

病室に囚われてから、早一週間が経とうとしていた。
私はその間に、多くのものを失った。
数え上げるときりがないくらい、私を構成する無限の要素、その大部分が蒸発してしまった。

なぜ、私は生きているのか。
なぜ、ここに居るのか。
主治医に聞いてみたが、彼は「申し訳ありませんが、お答えできません」と、
誰が救急を呼んだのかを知らなかった。

腕の痛みはまだある。

私は、原因不明の事故の犠牲者として、ここに運ばれた。
記憶はない。意識の焦点が、ずっと合わなくて、私は混乱の中に生きていた。
私は死ねなかった。最後に目を瞑った時、私は死を享受しようとした。
この苦しみから解放されるのなら、いっそ、喉を描き切ればよかった。
後悔している。私は自分が恨めしい。このベッドの上に横たえる、動かない肉の塊が。

夢も見れず、現実にも自由のない私。
左腕のない、無力な私。

失った片腕を、部屋の隅、その無限遠に伸ばす。
視界において、幻想であったとしても、それは有限であるはずなのだが、
無限分割によって、私は永劫に端に振れることは出来ない。
手を開いて、何かを掴むイメージ。
だけど、帰ってくるのは無限だ。私を内省させる、私の中の無。
無限であって、尽きることのない、生の源泉。
私はこの角が、対角線論法に等しいものであると気付いた。
私の中の無限は、私の外にある無限よりも、濃度が低いのだ。

より大きな無限を内包する私の、この空想の部屋は、しかし私に永久的な絶望を与える。
無限は確かに、私達に意味を与えてくれる。衝動、アノミー。

ああ、どうすれば良いのか。私にはわからない。

いま自分が何を考えているのかも、何を納得しようとしているのかも、
何もかもわからない。
わからないという無限。

永遠に降下する世界。
神の降下。私の上昇。相対的にそうだ。
そのはずだ。

だが、失った腕は帰ってこない。

その揺るぎない確定した世界の規律に、私は反逆を試みる。
常に無限を鑑みて、世界の測度を取ろうとする。
私の視界内において、有効なアプローチだ。
だがそれ以外において、それは全くの無意味だ。
私の背後には私の腕は伸びないし、地球の裏側にも私の腕は存在し得ない。
それは確定できない。私の世界は私が向く方向にしか伸びていない。
それなのに、そこには無限がある。
私の見る世界、つまりある場所からの光が、網膜に到達する限界の距離。

違う、これは全くの的外れだ。考える必要がない。
私は受け入れなければならない。
だがどうやって。

\scalebox{3}[1]{―}。

幻肢痛のシグナル。
線状の痛み。
過去に切り刻まれた証。
私は思い出し、だからこそ取り乱した。

もう二度と、その傷を享受できないという絶望に。\\
「あああああああ」\\
声が出ない。声帯を空気は流れているはずなのに。

またアレが襲ってくる。
今までの思考はすべてそれに対する防御機構だったのだ。
無意味だが、益はある無駄。\\

乖離していく意識を繋ぎ止めるために。

私は\scalebox{3}[1]{―}していた。

それは罪で。

\scalebox{3}[1]{―}が戒めた。

\scalebox{3}[1]{―}はもう、いない。\\

守るものなく、そもそも狩人を剥奪されて、私はみなしごになって、
暗闇を漂うしかないのか。嫌だ。それは酷だ。地獄だ。
魂の飢餓だ。喉を渇かす涙の流れ。
何もかも終わりだと、どうしようもないから溢れるしかない液体、そのちっぽけな慰め。

私は、私は、私は、私は、私は、なんなのだろうか。

\scalebox{3}[1]{―}こっちにきなよ。

どこかからか声が聞こえる。
窓の向こう\scalebox{3}[1]{―}これは現実のだ\scalebox{3}[1]{―}から、いや、その下だ。
まさかなにもない空中から、誰にも気づかれずに侵入することは出来ないだろう。
私は、ベッドの下に手を奔らせた。
髪の毛の感触。男の頭だろう。\\
「驚かせてごめんなさい」\\
「私にお見舞いに来てくれるなんてね。そんな人間だったかしら」\\
「彼女たちは来ていないの」\\
「さあ? 場所も知らないだろうから」\\
「寂しい?」\\
少年が来た。あからさまに子供だ。おそらく小学生か、可能な限り歳を大きく見積もっても、
中学校を卒業していないだろう。\\
「寂しくないわよ」\\
「へえ、本当ぽい、なんか怖いね」\\
「ああ、思い出した。あの時の子供ね。でも、おかしいわね。子供は寝る時間よ」\\
「俺はもう子供じゃねえ。もう背も伸びなくていいし。寝る必要はないんだよ」\\
「ふふっ、チビのくせに」\\
少年は不機嫌そうだ。煽り弱いのか。
まあ、いいリハビリになった気がする。
敵の存在は、常に心を律してくれる。\\
「それはそうとして、元気そうね。これだったら、一緒にやれるかもしれない」\\
「何を? 初対面の人間になんの説明もしない人間を、信用できるとも?」\\
「してくれるはずよ。私達も狩人だから」\\
「狩人だから信じるなんて、オレオレ詐欺に引っかかるみたいじゃない?」\\
「怪しむのは分かるよお姉さん。えーと架谷彩芽さんだったっけ」\\
私は眉をひそめた。その名前をどうして知っているのか。
他人に自分の情報を握られていることは、酷く人を不機嫌にさせる。
その量が非対称なほど、よりそれは深まる。\\
「気持ち悪い。名前で呼ばないで」\\
「じゃあなんて」\\
「呼ばれる必要はない」\\
「ライト、そこまでにして」\\
「……はい」\\
「ごめんなさいね。ライトはまだ子供だから」\\
「私からすれば、あなたも子供ね。まあ、私もまだ成人してないけれど」\\
「昔は十二歳で元服だった。だから十分よ。無駄に年をとっても、
いつまでも子供な人間は子供だし、早く大人になってしまうのもいる」\\
「あなた達そうだと」\\
「彩芽さん、あなたもよ。私達は似た者同士じゃない。
あなたはもう、これ以上狩人にはなれないけど、それでも執拗に狩人になりたがる。
こんなにも酷い怪我をしても、まだやりたいと心の底から願っている」\\
「それは\scalebox{3}[1]{―}否定できない」\\
「それは怖いからでしょ。誰かの為になれない自分が、怖くて逃げたがっている」\\
「……」\\
「私達はきっと力になれる。あなたがいないと、この夢は終わらない。
悪夢は永遠に生まれ続ける。元を断たないといけない。そのためには、あなたのような強い意思が必要なの」\\
「でも、私にはもう、腕がない」\\
私は傾いていた。少しでもと、可能性に身をやつして。\\
「大丈夫。腕はこっちでどうにかなる」\\
「それは、本当に?」

\scalebox{3}[1]{―}。

「そうだよ。君の腕を君に与えよう、架谷彩芽。それはきっと、君の力になるはずだ」\\
それはひどく聞き慣れた声に、似ていた。

\section{}
暑さも相まって、私達の気だるさは最高潮に達していた。
学校にはもう、アヤメの姿はなかった。

あの日からすでに一週間。

その間には一度も狩りの夜は訪れず、なおのこと私の焦燥は溜まっていった。
対価を以て、彼女の腕を取り戻すこと。
その願いを叶える為に、それほどの働きをすれば良いのかも、全く見積もりができない。
アオタも姿を見せないし、相談しようがない。

あの時の言葉が不安とともに蘇る。いや、そんな訳はない。
信じるべきだ。少なくとも、私達を狩人に導いてくれた。私は今、確信している。
これこそが、人を助けられる最も優れた方法であると。
私の労力を、最も効率よく正義に変換できる方法。
手放すことなどできるはずもない。セレナも、きっとそう思っているだろう。
若干の苛つきを溜め込んでいるように見えた。

「アヤメさん、大丈夫なのかな」\\
何度目の言葉だろう。\\
「わかんない。でも、生きてるって信じるしかない」\\
「そうだよね……それしかないよね」\\
もはや日常会話というものは存在せず、私達の周囲は何食わぬ顔で、いつも通りの生活を営んでいるが、
この二人の空間だけが、そっくりそのまま切り取られて、どこか別の次元に貼り付けられているような、
そんな異質な疎外感を私達は纏っていた。

娯楽を求めることもしない。最低限、生きるための行動をして、一日を早く終わらせる。
今はもう、かつて深夜まで起きていた生活習慣も身を潜めて、夜の八時にはもう就寝するようになっている。
勉強も、もちろん最低限だ。学生としての本文がどうこうと、誰かに言われそうだが、
そんなことを気にしている余裕などないし、必要もない。
早く寝て、早く狩りの夜に目覚めようとする。
それが今の私にできる、唯一の抵抗だった。この最悪な状況に対する、限られた手段の中で。

「アレって、どうなってるの」\\
「アレって何?」\\
「ほら、セレナのやってるやつ」\\
「ああ、アレね。全然話ないね。やっぱり、モエのときが特別だったってことなのかな」\\
「そうなんだ」\\
「悩んでる人って、どうやって相談すれば良いのかもわからないのかもしれない」\\
「確かに、今の私達もそうかもね」\\
「だから、もっと違う方法を見つけないといけないのかもしれない。
もっと根本的な解決方法」\\
「根本的?」\\
「例えば……悪夢の元を断つとか」\\
「生まれないようにするってこと」\\
「うん。それがさ、一番確実だよね」\\
「でもどうやって?」\\
「それは\scalebox{3}[1]{―}見当がつかないけど」\\
「そうだね。君の考え方は、ご尤もだと思うよ」\\
いきなりの声に、私は驚いた。背筋が伸びて、半開きだった瞼が完全に開く。
アオタだ。久しぶりにその姿を見た。\\
「アオタ……驚かせないでよ」\\
「ごめんね。いろいろと忙しかったんだ」\\
「忙しかったって? アヤメさんがあんなことになったのに……よくも」\\
「ちょっとセレナ」\\
セレナを静止する。もちろん手を出すわけにはいかないし、無論傷害を与えることも出来ないだろう。
私はただ、その攻撃的な態度を戒める。

と、言っても、私にも不満はいくらでもあった。
聞きたいこともだ。彼が一体、どこまで把握しているのかも、私達には重要な事項だ。\\
「ねえアオタ。あなたはどう思ってるの」\\
全てに対してだ。\\
「アヤメのことは\scalebox{3}[1]{―}どうしようもないよ。僕も、心の底から悲しんでいる。
自分の傲慢さを反省するべきだった。僕たちはあまりにも多くを知らなすぎた。
目を背けてきた僕の責任だ。こればかりは、どんな言い訳もできない」\\
「……」\\
私達は無言だった。アオタの言葉をただ待っていた。
彼の持つ全てを話してほしい。強要にも似た沈黙。\\
「ここ最近の狩りの夢は、不可解なことばかり起こっている。
修復不可能なアヤメの腕もそうだけれど、現実の物体に、明らかに干渉している悪夢が出てきた。
あの人型もそうだね。
僕たちの観測してきた限りでは、一度もなかった。
\scalebox{3}[1]{―}認識を改める必要があるかもしれない。
セレナの言う通りに」\\
「私の?」\\
「君の言う、根本を取り除くべきだということ。僕はそれに賛成する」\\
「方法は?」\\
「わからない。けれど、何も手がかりがないわけじゃない」\\
「どんなの?」\\
「それはまだ、君たちには伝えることは出来ない。不確かだし、そこにリソースを割いてほしくない。
幸い、君たちの働きによって、悪夢の数は減っているけど、その質は高まっている。
凶悪な個体が、生き残ってきているんだ。自然淘汰と言うべきなのかは判断しかねるけど。
とにかく、目下の危険性を排除するべきだと考えている」\\
「それは、私達もそう思ってる。でも……」\\
「私達はまだ未熟なの」\\
それは常に意識してきたことだ。
だが、今ここで弱音を吐いても意味がないと思った。
もっと建設的な課題の発見を目指すべきで、それは個人で解決するべきだと、私は考えていたのだ。
それが可能であるかどうかは置いといて、一刻も早く先に進むべきだと焦っていた。
「セレナ!」\\
「確かに、アヤメはまだ全てを伝えきれていないようだし、教育は完了していないようだ。
もちろん、今の君たちも十分に強力な狩人だけれど、現状を鑑みるに、まだ強くなる必要がある。
あのアヤメも負傷させて、再起不能にするほどの悪夢が居るんだ。
もっと慎重にならないとね」\\
「だとしたら、もっと訓練とかしないと」\\
「でもどうやって? 筋トレでもするの」\\
「大丈夫、それには手を打ってある。古い狩人を呼ぶ」\\
「古い狩人?」\\
私は聞いた。\\
「かつてアヤメと共に戦った狩人。きっと君たちの力になるよ」\\
その言葉は、妙に自信づいていた。
\end{document}