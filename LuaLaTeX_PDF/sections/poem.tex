\documentclass[../IHMain]{subfiles}
\onecolumn
\begin{document}
    \begin{verse}
        怖いんだ\\
        さめてしまうのが\\
        だからこの思いを小箱にしまって\\
        鍵をかけたい\\ \\

        けれど、恐怖はいずれ鍵穴から\\
        ひっそりと抜け出して\\
        戻ってくるんだ\\ \\

        だから君と一緒に閉じたい\\
        鍵穴を二つにわけて\\
        出口を細めて\\ \\

        そうすれば、わずかに溢れるだけだから\\ \\

        繰り返しくりかえし\\
        鍵を回して\\
        穴を狭めて\\ \\
        
        僕は安心して\\
        君と眠りたい\\

        \newpage

        思い出を絞首台にかけていく\\ \\
        
        お父さん、有罪\\
        お母さん、有罪\\
        お友達A、有罪\\
        お友達B、有罪\\
        ほか多数、有罪\\ \\

        そして私、有罪\\ \\

        空っぽの輪\\
        擦れてちぎれる\\
        最後に残ったのは、彼女だけだった

        \newpage

        何もかも捨てて、私は逃げたかった\\
        肉も骨も血も削ぎ落として、生まれ変わって\\
        遥かな景色を見に行きたかった\\ \\

        けれどそれはできなくて\\
        私は行き詰まっていた\\ \\

        生きていくのが辛い\\
        死ぬ理由もない\\
        ただ心を悪魔に差し出して\\
        私は天国に行きたい\\ \\
        ただ、それだけなんだ

        \newpage

        加わる力に負けて、折れてしまいそうだ\\
        両端のベクトルは、私の心臓に\\
        弾丸のように走る\\
        苦しさにおかしくなりそうだ\\
        だれかたすけて\\ \\

        貧乏ゆすりが止まらない\\
        冷たい足先を感じる\\
        無意味な動画の再生\\
        イヤホンの窮屈さ\\
        捻った不健康な体勢で座る私\\ \\

        響く鼓動\\
        明日を憂いて\\
        空を飛びたい

        \newpage

        イライザとおしゃべりして\\
        意味ある語句を見出す人がいるように\\
        無意味な世界にも、何かを見出すことができるのだろうか\\ \\

        世界に対するELIZA効果\\
        運命はときに、自らと『運命的に』何かを巡り合わせるという誤認、迷信\\
        人がするように、運命が事物を紹介してくれると思うこと\\
        論理誤謬に成り立った私達の生活\\ \\
        
        私は、外にでかけたかった。\\
        でも\\
        雨が降ってしまった\\
        なんて、ついていない私

        \newpage

        ステンレスの蛇口に写る顔\\
        曲面で反射する歪んだ実像\\
        笑っている\\
        体を前後に動かす\\
        前に、後ろに、近づいたり遠のいたりして、笑顔を繰り返す\\
        ひどい屈折だ\\
        鋭角な口角\\ \\

        私の代わりに笑ってくれる、唯一の私

        \newpage

        自由落下と上昇運動に何の違いがあるのだろうか\\
        おそらく習ったであろうニュートンの運動方程式\\
        その変数$t$を$-t$に変換したとしても、自然な答えを返してくれる\\
        上向きの『落下』運動は存在しうるのだ\\
        (適切な論理であることは保証しない)\\ \\

        だが私達の世界は低きに落ちていく\\
        巨視的な世界では必ずそうなる\\
        林檎は、空に向かえない\\
        人もまた、地面に叩きつけられるのみだ\\ \\

        時間の矢は私達の背後を常に狙って\\
        逆らうものを射殺す\\ \\

        では彼女たちは何を願って、空に向かったのか\\
        吸い込まれそうな青に、目を奪われたのか\\
        それとも、上向きの運動を確信していたからか\\
        束縛からの開放か\\
    \end{verse}
        \newpage
    \parbox<y>{\linewidth}{
    \begin{landscape}
    \thispagestyle{empty}
    \tt
    Listen,\\
    LoveMeTrueLoveMeTrueLoveMeTrueLoveMeTrueLoveMeTrueLoveMeTr\\
    oveMeTrueLoveMeTrueLoveMeTrueLoveMeTrueLoveMeTrueLoveMeTur\\
    veMeTrueLoveMeTrueLoveMeTrueLoveMeTrueLoveMeTrueLoveMeTrue\\
    eMeTrueLoveMeTrueLoveMeTrueLoveMeTrueLoveMeTrueLoveMeTrueL\\
    MeTrueLoveMeTrueLoveMeTrueLoveMeTrueLoveMeTrueLoveMetrueLo\\
    eTrueLoveMeTrueLoveMeTrueLoveMeTrueLoveMeTrueLoveMeTrueLov\\
    TrueLoveMeTrueLoveMeTrueLoveMeTrueLoveMeTrueLoveMeTrueLove\\
    rueLoveMeTrueLoveMeTrueLoveMeTrueLoveMeTrueLoveMeTrueLoveM\\
    ueLoveMeTrueLoveMeTrueLoveMeTrueLoveMeTrueLoveMeTrueLoveMe\\
    eLoveMeTrueLoveMeTrueLoveMeTrueLoveMeTrueLoveMeTrueLoveMeT\\
    LoveMeTrueLoveMeTrueLoveMeTrueLoveMeTrueLoveMeTrueLoveMeTr\\
    oveMeTrueLoveMeTrueLoveMeTrueLoveMeTrueLoveMeTrueLoveMeTur\\
    veMeTrueLoveMeTrueLoveMeTrueLoveMeTrueLoveMeTrueLoveMeTrue\\
    eMeTrueLoveMeTrueLoveMeTrueLoveMeTrueLoveMeTrueLoveMeTrueL\\
    MeTrueLoveMeTrueLoveMeTrueLoveMeTrueLoveMeTrueLoveMeTrueLo\\
    eTrueLoveMeTrueLoveMeTrueLoveMeTrueLoveMeTrueLoveMeTrueLov\\
    TrueLoveMeTrueLoveMeTrueLoveMeTrueLoveMeTrueLoveMeTrueLove\\
    rueLoveMeTrueLoveMeTrueLoveMeTrueLoveMeTrueLoveMeTrueLoveM\\
    ueLoveMeTrueLoveMeTrueLoveMeTrueLoveMeTrueLoveMeTrueLoveMe\\
    eLoveMeTrueLoveMeTrueLoveMeTrueLoveMeTrueLoveMeTrueLoveMeT\\
    \end{landscape}}

\newpage
\yoko
次章予告\\ \\
{\huge
助けて、と彼女は言った。\\
私は、気にもしなかった。}\\
他愛のない日々の呟きだと、そう勝手に思ったのだ。\\
\scalebox{3}[1]{―}九月の終わり、かつての友人が私を訪ねてきた。\\
何ら理由はない。ただ近くによっただけ。彼女はそう言った。\\
……\\
重力に服従するその手を掴む。引きずられる体。だがすぐに軽くなった。\\
彼女は飛んだ。開放を目指して。この世界から逃げたくて。\\
割れた頭蓋から覗く脳。タンパク質の塊。心の壁。\\ \\
私は吐き出した。

\end{document}