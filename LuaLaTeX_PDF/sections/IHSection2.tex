\documentclass[../IHMain]{subfiles}

\begin{document}
    
\chapter{狩人、\\その使命}
\section{\tt Where is my dream?}
\subsection*{(1)}
「行ってきます」\\
玄関を出た。
昨日と同じ寒い朝。
いつもより早めに家を出た。
コンビニで昼ごはんを買うためだ。
お母さんは昨日、どうやらかなり遅くまで起きていたらしく、
お弁当を作る気力がなかったのだろう。
いつまでも起きてこない、お母さんとお姉ちゃんをよそに、お父さんはいつもどおり
早朝に家を出ていた。

駅のコンビニに入る。
サラダサンドウィッチと、鮭おにぎり。
飲み物は水筒にお茶を入れているので、買う必要はない。
忙しい時間帯だ。
レジには多くの人が並んでいて、スタッフの人は忙しく働いている。
バーコードリーダーの音がより一層、その印象をくっきりとした形にしている。\\
「お待ちの方どうぞ」\\
隣のレジに、商品を置く。
お金を払って、レジ袋に入れられたおにぎりやパンを、そのままバッグに入れながら、
コンビニを出た。

改札を通って、まだ来ていない電車を待った。

\subsection*{(2)}
駅を出て、学校へ向かって歩いていた。
道すがら、ふと昨日の出来事が頭をよぎる。
あの後、意識的にこのことについて考えることを避けていたのだけれど、やはり気になってしまう。
夢のこともそうだが、何よりも、あの黒い物体と、それと共に現れた人間。
まじまじと見つめていた訳ではないが、その時の印象を思い起こしてみる。
眉を潜めた顔しか、記憶には残っていないが、それでも感じたことがある。
あの顔はかなり若かった気がする。
そう思えば、彼女の容姿が私たちの一つか二つ上の人たちによくいるような感じがした。
それに髪型も、かなり特徴的だった。
片方を伸ばしたショートボブというべきだろうか。
そしてなぜ、あの場所に現れたのか。
あそこに人が立ち入ることが、あるのだろうか。
あるとしても大人だろう。それも、かなり歳をとった人。
だとすれば、あれは同じ学校の人間。
\scalebox{3}[1]{―}いやそもそも、アレが現実であるという前提で話を進めるのもどうかと思う。

考えれば考える程に、私の頭はこんがらがる。

思考に頭が重くなって、うつむいたままに道を歩いていると、後ろから何かが肩にぶつかった。\\
「すみません」\\
通り過ぎようとする女性からの声だった。
ずいぶんと早歩きで力強い。
颯爽と過ぎていく彼女に、私は注目した。
横顔が見える。
なんだか見覚えのある顔、あの時の人に似ている。
もちろん、口元は隠されていたから確かめようがないが、
その鋭い目元や、片側だけを伸ばしたアシンメトリーな髪型は、どこか印象深い。
そうに違いない。
きっと彼女が、あの時の女性だ。
私は抑えきれなかった。\\
「あの、あなた、あの時の人ですよね」\\
自分に声がかけられたと気づいていないのか、それとも無視しているのだろうか、訝しんだりすら
ない、全くの無反応。\\
「すみません。あなたですよね! 私たちを助けてくれたの」\\
流石に気づいたのだろう、彼女は振り向いた。\\
「あ、あの。すみません」\\
「あなた誰。知らない人。
申し訳ないけど、人違いじゃないの」\\
冷たい声。
鬱陶しがっているのは明白だった。
だけどその顔は、はっきりと断言できる。
あの顔とそっくりだ。
怪訝な、けれど攻撃的ではない目つき。
しかし人そのものを否定するような、
はっきりとした拒絶を滲ませたそれに、私はそれ以上踏み込むことが出来なかった。

彼女と私の住む世界は全く違うのだと言わんばかりに、
彼女は私をいとも簡単に置き去りにして、瞬く間に遥か向こうへと去っていった。

\subsection*{(3)}
「あ、カナン」\\
休み時間。
トイレを済ませていたら、偶然にもセレナと一緒になった。
寝不足気味なセレナの目。
彼女も昨日の出来事に頭を悩ませていることは、すぐに分かる。
手を洗いながら、私たちは話し合った。\\
「私もうわかんないよ。カナンはさ、覚えてるの? 昨日のこと」\\
「覚えてるよ。やっぱり、本当にあったことなのかな、アレって」\\
「えーでも、ありえなくない? あんなにさ、アニメとか漫画みたいに人が
ビューンって飛んでくるとか、おかしいよね。物理的にありえない」\\
妙に堅苦しい語句を使うのは、セレナがその事柄について深く考えすぎている証拠だ。
確かに、彼女の言う通りだと思う。
あの真っ黒い玉はよしとしても、あの彼女の運動はありえない。
私たちの遥か頭上を飛び越えて、しかもかなりの時間地面に落ちず、剣を振り続けることなんて、
現実にできるはずがない。\\
「やっぱさ、私たちがどうかしてたんだよ。
集団ヒステリーってやつ、なのかも。化学薬品とかさ、そういうやつで幻覚をみてさ。
……それにしてもリアルすぎるってのもあるし、直前の夢とかもあるけど、
そうとしか言えないよね」\\
「ヒステリーってもっと病的なんじゃないの?」\\
「うーん、そんなこと言っても\scalebox{3}[1]{―}
ああもうわかんない。わかんなよなんにも。
ああ、イライラ」\scalebox{3}[1]{―}する。そう彼女がいいかけた時、
どこからか、遮るように声が聞こえた。\\
「そんなこと、簡単じゃないか。ただあるがままを受け入れる。
どうして人は、それが簡単にできないんだろう?」\\
声も子供っぽく、しかし憎たらしくないほど知性さを帯びている。
どこから聞こえるのだろうか。
洗面台の端。
何かが見える。
影のようだ。
いや、光だ。
私は目を疑った。
本当に、私の頭が狂ってしまったのかと、心配になった。
薬物の乱用者が、その禁断症状に見る光景、教科書で知ったそれと、
なんな変わりないものが見えたからだ。

落ち着いた青色の光球。
それに、まるでマスコットの様な極端にデフォルメされた目、
例えるならスマイルのマークといえばいいだろうか。
けれど口に喩えるパーツはない。
目だけだ。
それでも、それがただのぬいぐるみだといえば、ただの愛らしいものだろう。
だけどそれは、ゆらゆらと揺らめいて、動いている。
しかも、言葉を喋ったのだ。

「ねえ、今の私だけじゃないよね」\\
私はセレナに寄った。\\
「うん。やっぱ、カナンにも見えてるよね」\\
「見えてる」\\
「どんな形?」\\
「青くて、丸っこい」\\
「だよね」\\
これ以上、なんと返せばいいのか行き詰まってしまった。

混乱の静けさの中に、チャイムの音が響く。\\
「あ、時間だ」\\
白々しいセレナの言葉。\\
「遅れちゃう、いこうカナン」\\
「うん」\\
私たちは、見なかったふりをして、トイレから出ていこうとする。
その身振りを見て、焦るような表情
\scalebox{3}[1]{―}おそらくはほんの少し目の形が変わっただけだろうが\scalebox{3}[1]{―}
を見せる青い玉。\\
「ちょっとまってよ。僕は君たちに話したいことがあって\scalebox{3}[1]{―}」\\
何かを言いかけていたが、もうどうでもよかった。\\
「ほら、早く!」\\
「待ってよセレナ!」\\

「詠、遅刻」\\
残念だが、授業には間に合わなかった。

\subsection*{(4)}
「ねえ、なんなのコイツ」\\
「私が聞きたいよ」\\
私たちは食堂の端っこに、小ぢんまりとしていた。
昼休みだし、人は多い。
他人からの目を恐れて、私たちは交互にあたりを監視していた。
どう見たって、傍からは頭のおかしな連中にしか見られないだろうからだ。

キョロキョロとした二人。
これだけでも十分に怪しいが、それでも仕方ない。
木を隠すには森のなかに。
人混みに紛れれば、注目されることもないだろう。
人が全く来ない場所で、かつ話していても気づかれないような場所は、私たちには思いつかなかった。
下手をすれば、私たちは有りもしない虚空か何かに話しかけている、ヤバイ奴になってしまうのだ。

その原因は一つしかない。
先程の青玉は、未だしつこく、私たちに付き纏ってくるのだ。\\
「だ、か、ら! アンタがどうこう言ってても、私たちには一切関係ないから。そうだよねカナン」\\
うん、と同意する。\\
「そんなこと言っても、僕らにも僕らなりのちゃんとした理由があるんだよ。
それを訳がわからないの一点張りで拒絶するのは、はっきり言って酷いことだと思うよ」\\
私たちの強硬な姿勢に、まるで高圧的な客に対して、辟易とする接客業務員の姿を重ね見てしまう。
けれど当たり前のことだ。この物体は自分の身元を明かさないのだ。
初対面の人間に\scalebox{3}[1]{―}そもそもまともな人格がそれにあるのかわからないが\scalebox{3}[1]{―}
名乗らないまま、自らの言い分に耳を傾けることを求めるのは、筋が通っていない。
ただでさえ昨日の出来事で疲れ切った私たちに、おかしな格好と、それまた正反対な言動を、
受け入れろというは酷な話だ。

ただ、相手もなんだかんだで引くことを知らないようだ。
かれこれ、押し問答は三十分以上続いていた。\\

「じゃあ、あなたって何を私たちに『お願い』したいの?」\\
これ以上の繰り返しに、意味を見いだせなかった私は、思い切ってソレに聞いてみた。\\
「聞いてくれるのかい?」\\
一転して、その声色は明るいものになった。本当にわかりやすく、あざとく。\\
「カナン、付き合わなくてもいいよ」\\
「そうかもしれないけど、気になるでしょ。私たちに何があったとか。
このままでいても、どっちにしろ納得は出来ない。だったら、一度受け入れてみるのも
いいんじゃないのかなって」\\
口はつぐんだが、納得はしていないようだった。
だけど、セレナも私と同じ気持ちだろう。
ただ私よりも、自分の中に押し込めるのが上手なだけで。

「もう始めてもいいかな?」\\
私は黙って頷いた。\\
「僕たちはね、君たちに大切なお願いがあるんだ。
僕たちは、ある人達を探しているんだ。
とても重大で深刻な問題を解決できる、強い力を持った人たち。
才能を持っていると言ってもいい。
だけど、それは本人には気づけないし、僕たちもなかなか見出すことが出来ない。
いいかい、落ち着いて聞いてほしい。
\scalebox{3}[1]{―}僕たちはね、君たちにこの星を救ってほしいんだ」\\

……一瞬で体の気が抜け落ちていくような感覚がした。
彼の雄大な熱弁と違って、私たちの今の顔は正しく間抜け面だろう。
ソレ\scalebox{3}[1]{―}いや彼の言葉は、まるで童話のセリフだ。
現実にはそぐわない、空想の世界の言葉。
すべてがお膳立てされた道筋の、単なるきっかけに過ぎない文字列。
『世界を救ってほしい』その言葉のなんと幼稚なことだろうか。

セレナは、思わず笑いだしてしまった。
そんな彼女に、彼は不機嫌になったのだろう。
凹んだ目を、眉間に皺を寄せる仕草に見立てているらしい。\\
「君たちはもっと、物事の本質を見るべきだと思うよ。
見たんだろう。あの黒い化物を。
感じたんだろう? その時の恐怖を。
だとしたら、結び付けられるはずだ。
これは決して、笑い事じゃないんだ」\\
精一杯に語気を強くしたつもりなのだろう。
だがそれでも、あくまでも可愛げのあるただの、マスコット的発声だ。
しかしその懸命さは、私たちの関心を引くには十分なものだった。\\
「じゃあどうやってその、『星』を救うの。
宇宙人と戦う? 悪いやつを殺すの?」\\
セレナは、さっきまでの不満そうな顔つきと一変して、かなり興味津々だった。\\
「宇宙人じゃないよ。もちろん人間でもない。
もし君たちが僕たちと『契約』を結ぶなら、君たちは『悪夢』と戦ってもらうんだよ」\\
悪夢。なんとも抽象的な名称だ。
横目にセレナと見合いながら、首をかしげる。\\
「悪夢というのは、実は僕たちにもよく分かっていないものなんだけど\scalebox{3}[1]{―}」\\
「ちょっとまってよ」\\
セレナが会話を遮った。\\
「どうしてアンタにもわかんないような奴と、戦わないといけないの?」\\
それは私も同感だった。\\
「確かに、その意見は正しいと思う。けれど実害は発生しているんだ。
現に、君たちもそうだっただろう。
うなされる夢に侵され、妙に現実味のある夢想の末に、特定の場所に連れ出される。
君たちは自らの意思で行動したと思っているようだけど、あれは一種の捕食行為。
あのまま僕たちが助けなかったら、君たちは今頃死んでいただろうね」\\
死んでいたかもしれない。
その一言はかなり心にのしかかってきた。
おちゃらけた雰囲気は一気に凍えて、私たちの顔はどよめく
まったくどうしてその一言だけで、今どき私たちの様な中高生の間では、
たいして忌避すらされていない言葉を聞いて、こうも不安になってしまうのか。
記憶のせいだろう。
私たちは確かに、その恐怖を知っている。\\
「それじゃあ、死んだ人もいるってことなの」\\
私は聞いた。
息を整えることを、促すような間。
彼は見えない口を厳かに開くよう、話を繋げる。\\
「そうだよ」\\
明らかに重たい声。鎮魂の重みだろうか。\\
「死者は多くいる。
防げるものもあれば、どうしようも出来ない場合もある。
人口に対して、常に悪夢に対抗できる人間の絶対数は極小で、
すべてを守り切ることは不可能なんだ。
だからこそ、可能な限り、君たちのような適正ある人間に、協力してほしい」\\
私たちの顔はよりいっそう暗くなっていた。
なぜだろう、それが現実なのかを一切疑わずに、いたずらに死という単語に共鳴している。\\
「ねえ、さっきアンタは私たちを助けたって言ってたよね。
だとしたら、あの時の女の人は、死んじゃうかもしれなかったのに、
私たちのために戦ってくれたの」\\
「それが事実だよ。でもね、大丈夫。
君たちが僕たちと契約をすれば、必ず力を得る事ができる。
その力さえあれば、立ち向かうことができる。
死ぬ可能性は十分に抑えられるし、死の淵にある人々を救うことができる」\\
彼はもう、私たちが彼の言葉に同意することを前提に話を進めているようだった。
確かにこの状況で、もう結構ですと、席を立てる勇気はない。
ただそれでも、彼の話は信じ難いと同時に、魅力的でもあった。
高校生には、いや日々をただ惰性で過ごしている私たちには、おそらく一生叶うことのない
ことだろう。
誰かの為になる。
急にこんなものを、輝く宝石のような、
生きる意味を提示されれば、目が眩んでしまうのは当然だろう。
それに具体性も持っている。
その真偽はともかくとして。
少なくとも私はそう感じている。
ここに来てまた、自分の中から現実性が脱落した気がする。\\
「これがどれだけ、君たちの心を迷わすことなのか、僕たちは理解しているつもりだよ。
それでも、心に漂うものたちと向き合うことは、有意義なことだと思う。
もし君たちが、この願いを受け入れてくれるのなら、放課後、E科の三年生教室に寄ってみてくれないかな。
\scalebox{3}[1]{―}答えがどうであろうと、僕たちはそれを受け止めるよ」\\
時計を見れば、すでに12時50分。
もうすぐで昼休みは終わる。
気付けば、周りも人は疎らで、残っているのは私たちを除けば、片手で数えられるほどだった。\\
「まって」\\
セレナが叫んだ。\\
「アンタは何者なの」\\
「それは、君たちが決めることだよ。もし来てくれたのなら教えるよ。
でもそうじゃなかったら、全部忘れるといい。こんなこと、心に潜めておく必要なんてないからね。
その時は君たちの送りたい人生を送るべきだよ」\\
彼はそう言い残して、静かに消えていった。
無色に、溶け込んでいくように。\\

残された私たちは、ひとまずは現実に戻ることにした。
時間はまだある。
急いで弁当箱を片付けたり、皿を下げたりして、食堂を出ていった。
その一切は無言のままだった。
互いに、自分の心の奥底に沈殿する何かを、必死に見透かそうとしているのだろうか。

自分でもわからないが、とにかく、普通ではなかった。

\section{}
\subsection*{(1)}
陽は傾き、散乱した赤い光が横から指す。
廊下に人は見えず、電灯は飛び飛びに着けられている。
節電のためだろう。
まだ四時過ぎだ。
セレナが教室に来てくれた。
掃除が終わって、上げられた椅子を机から降ろして、私たちは話し合った。
私は椅子に、セレナは机に。
もちろん、昼のことについてだ。\\
「どうするの、セレナ」\\
「どうしよう」\\
「そうだよね。なんだか、しっくりこない。
星を守るって、なんか宙ぶらりんだし、そもそも私にそんなことできるわけない。
でも、あの子が言ってたみたいに、私たちがせめて、せめて誰かの役に立つなら、
それを拒否するのも、自分は許せない」\\
「言ってたもんね、死ぬことはないって」\\
「たぶん」\\
「だったら、カナンの言う通りだよ。
黙って知らんぷりするのは、許せない。
私たちは助けられたんだよね。
だったら、私たちも誰かを助けたいよ。
できるかどうかは、別だけど」\\
もはや二人とも、疑うことはなかった。
すべてを事実として、選択すべきを決めかねている。
心の整理はつかないが、なにか私の、
本質的なものがざわめいているのは確かだった。
たましいなのだろうか。
少なくとも、理性的な刺激ではなかった。\\
「行ってみようよ。三年の教室だっけ。
カナンと一緒なら、私行くよ」\\
セレナの目は、どこまでも澄んでいた。
彼女の瞳は、少し青みがかっている。
吸い込まれるような、それは決意の証だ。
彼女は私が行く行かないにしろ、すでに決めているのだろう。
そして、私のことをよく分かってくれている。
私は、一人では寂しいのだ。
自らで踏み出す一歩を、極端に嫌がるのだ。

結局、いつもそうだ。
私は全てに対して受動的なのだ。
だけど、それが自分であると認めたくなかった。
諦めたくなかった。
だから今こそ、自分から前へ進もう。
後ろを押されるのではなく、一緒に。

「うんわかった。行こう」\\
私は立ち上がった。\\
「ありがとう、カナン」\\
「そんなことないよ」\\
私たちは荷物を持って、廊下を出た。
三年の教室、正確に言えば、E科の学科棟は私たちの教室がある建物からは離れた場所にある。
外にでる必要があるのだ。
夕日が眩しい、学科棟に繋がる廊下を歩いていく。\\
「でも、まだ決まったわけじゃないし。そんな顔しなくてもいいよ」\\
「そんなに酷い顔なの?」\\
セレナに言われて、窓ガラスに反射する、僅かな私の姿に目を細める。\\
「ほら、もう真っ青」\\
ほら、とセレナは私の前にたった。
はっ、と思わず声が出た。\\
「これで赤くなるでしょ」\\
セレナの手が私の頬を覆っていた。\\
「ふふ、カナンのお肌スベスベ」\\
「やめてよ、くすぐったい」\\
「でも本当だもん。羨ましいなあ」\\
なんだか、心が穏やかさを取り戻していく。
やはりセレナは、私のことをよく分かっている。
軽くなった心は、足取りにも現れ、私たちは前を向きながら、進んでいった。

\subsection*{(2)}

学科の違うこともあって、教室にたどり着くまでに、かなり迷ってしまった。
空気が違うのだ。
まだ一年生の私たちが、この学校に三年近くすでにいる人間たちの領域に、
踏み込むためには、それ相当の勇気が必要だった。
気付けばすでに午後五時近く。
ようやく、目的地にたどり着いた。

薄暗い教室。

誰もいない。

「来てくれたんだね」\\
あの声だ。
でも、姿は見えない。\\
「どこにいるの?」\\
私は聞いた。\\
「ここよ」\\
女の声。
全くの部外者。\\
「カナン、あそこ」\\
セレナの指差す方向を見る。
そこには、廊下の壁にもたれかかった女性と、その肩に乗った、彼がいた。\\
「あっ」\\
それは、朝の彼女だった。\\
「ああ、貴女。朝の子ね。
あのときはごめんなさい」\\
いいえ、と頭を少し下げる。\\
「それにしても、あなた達を待つのは退屈だった。
できればもう少し早くしてもらいたかったんだけど」\\
「道に迷ったんで、すいません」\\
「そう、一年生なら無理もないか。
そもそも他学科なら尚更かもね」\\
彼女の印象は、今朝のそれに比べて、ずいぶんと穏やかに見える。
受け入れる、とすでに語っているようだった。\\
「ほら、いつまでも突っ立てないで、中に入りなさい。
いろいろ説明することがあるから」\\
はい、と私たちは教室に入っていった。\\
「てきとうに座って」\\
机の質感が違う。
それだけどそわそわする。
私とセレナは並んで、彼女は私たちの前に座った。
黒板に向かう私たちの前に、前の席の椅子を反転させて、足を大胆に組んでいる。
かっこよかった。\\
「それじゃあ、まず自己紹介かな。
私は\ruby{架谷|彩芽}{ハサタニ|アヤメ}。そっちは?」\\
彼女\scalebox{3}[1]{―}アヤメは、セレナに目線をやる。\\
「時国瀬玲奈です。こっちは\scalebox{3}[1]{―}」\\
「詠華南です。よろしくおねがいします」\\
アヤメは笑った。\\
「カナン、でいいのかな。そんなに堅苦しくしないでもいいよ。
今のうちに言っておくけど、私のことはどう呼んでもいいから。
ただし、私だと分かるように」\\
はい、と私たち二人は返事をした。\\
「それじゃあ、肝心なところを説明しよう」\\
アヤメは、肩に目を配る。\\
「こんにちは! 二人ともよく来てくれたね。
僕の名前は\scalebox{3}[1]{―}アオタ。
といっても、この名前はアヤメがつけた名前なんだけど」\\
「え、どうして?」\\
「\kenten[size=1, kenten=﹅]{あお}い\kenten[size=1, kenten=﹅]{た}ま、だからアオタ。
たいしたひねりもないわよ」\\
「でも、こういうのってだいたい固有の名前があるんじゃ」\\
「それがないんだよね。僕たちはとても特殊な存在なんだ。
\scalebox{3}[1]{―}君たちはガイア理論って知っているかな。
簡単に言えば、この地球には意思、まあ正確に言えばこれは正しくないけれど、
そういったものがあるという考え。
この星が一つの生命体だという考えだと言ってもいい。
僕たちはいわば、この星の代弁者。
それがいま、この地上で最も知性的な君たち人類の共通基盤を依り代に、
ここで会話可能なエージェントとして存在しているのが僕のおそらくの現状だと思う」\\
大したことでもないというように、
すらすらと情報量の過密な文章を垂れ流されても、戸惑うだけだ。
それでもとりあえず、私は語尾に着目した。\\
「だと思うって、ずいぶんと曖昧な言い方だと思うけど」\\
「その指摘はご尤もだね」\\
アオタはあっさりと認めた。\\
「僕たちは神様なんてものじゃない。
もちろん森羅万象全ての真理を知っているわけでもない。
正直に言うと、僕たちの認識する知識は人間に依存しているし、完全に君たちと同等だよ。
だから、君たちにわからないことは、僕たちにもわからない。
全ての思考は君たち人類が居なければ成り立たないし、
だからこそ君たちとなんら遜色なく会話することが実現できている。
裏を返せば、それ以上は不可能だってことだね。
結局、自分自身が客観的に何者であるかは、言及不可能なんだよ。
まあ、僕が現状の科学知識から大きく乖離した存在であることは自認しているけどね」\\
釈然としないことに変わりはないが、これ以上の追求が無意味であることはわかった。\\
「まあアオタのことは気にしないで。大切なことは他にある。ほら、説明して」\\
アオタをつつくアヤメ。\\
「そうだね。本題に入ろうか」\\
アオタは私たちを見つめた。
その粒のような目は、今では全てを飲み込む暗い穴だ。
私たちの意識を吸い取る。
凝視する。\\
「今日から君たちは『狩人』になる。
そしてそれには必ず危険が伴う。まずはそれを理解してほしい」\\
彼は神妙に語った。\\
「狩人? なにそれ」\\
セレナはその単語にいまいちな反応を示している。
きっとあまり意味を分かっていないのだ。
普段使う言葉でもないから。
私は聞いた。\\
「狩人って、何かを狩るってことでしょ。
さっき言っていた悪夢、だったっけ」\\
「そう。狩人は悪夢を狩る。
戦うと言ってもいいけど、形容するなら狩るという言葉が適切だね。
君たちは夜に、狩人となって悪夢を狩る。
悪夢というのは人の心を食いつぶす、悪性腫瘍のようなもので、
人類の集合無意識上に、時代を問わず絶えず存在し、人を脅かしてきた。
その行動原理は不明だけど、およそ知性的とも言えない。
おそらく、単なる自己保存のプログラムに過ぎないだろうね」\\
「つまり、獣のようなヤツってことよ」\\
「まあ狩りと言っている理由の一つがそれなんだけどね。
でも、奴らのいる場所が問題なんだ。
さっきも言ったけど、狩人になれる人間には適正が必要なんだ。
人の心の奥深くに侵入できる強固な自我が。
悪夢の生息領域は、人の集合無意識が現実世界に部分的に重なったもので、
普通の人間には認知することは出来ないし、干渉もできない。
けれど奴らは人の心理的な部分に依拠しているから、一方的に干渉ができるんだ。
君たちも夢を見たように、悪夢は他者のイメージに侵入して、自分の縄張りにおびき寄せる。
これは防ぎようがない。
防ぐとしたら、それは人をやめないといけないからね。
だけど僕たちは、人の中にもこの領域に存在できる個体がいるという事実を確認した。
だったら、ほかに手段はないよね」\\
「ちょっとまって。アオタはさ、星を守ってほしいんだよね。
だったら、どうして人間を守るってことになるの。
私たち人間が地球をめちゃくちゃにしていることだって、いっぱいあるのに」\\
「セレナ、確かに人間は地球環境を自ら調整する技術を得た。
でもそれは必ずしも悪いことではないよね。人間以外も、多くの生物が自らの都合の良く
環境を改変しているよ。ただ人と比べて、規模がごく僅かであるというだけで。
でも人間は多くを変えられる。
今はまだ、それは適切に管理されているし、人は自らを自制できている。
それがもし、悪夢たちが自らの生存のために、人を無造作に増やし、
その心を乱せばなにが起こるだろうか。
予想は多岐にわたるけど、おそらく良いものにはならないだろう。
僕たちはその可能性を潰すことこそが、重要だと考えているんだ」\\
「悪い結果をもたらす可能性を潰す。簡単なことよ。
でもそれだけじゃない。悪夢は今も誰かを誘っている。
自らのもとに、或いは死に。
心の弱い人や、感受性の高い人に被害者は多いの。
そして彼らは大抵、現実に居場所を失っている。
あなた達はまた別だろうけど、そういった人間が心の中にまで、
最後まで自分だけの砦だった心にまで入り込まれたらどうなるか」\\
「苦しい」\\
私がそういった時、私の心はあの悪夢の苦しみを再生していたのだろう。
無意識に、とっさに出たのだ。\\
「そう、それも自覚のない苦しみよ。
だったら、大局的なことはひとまず置いといて、
まずはそんあ人たちを助けているんだと思うべきね。
私たちは結局一人の人間で、世界を見渡すことは出来ない。
だからこそ、一つ一つの事物に意識を向けるべきね。
それに、ややこしいことは全部アオタがやってくれるから」\\
アヤメはアオタを叩いた。
ゴム毬のように弾んで、彼は冗談めいて怒っている。\\
「とにかく、難しいことはすぐに理解する必要はないよ。
君たちは誰かを助けたくて、ここに来たんだよね。
顔も、それこそ名前もわからないような他人を、
助けようと思ったそのこころを大切にしてほしい」\\
「ねえアオタ、聞いてもいい?」\\
「どうしたんだい、セレナ」\\
「私、本当に戦えるかな。なんだか怖くなってきたの」\\
私は驚いた。セレナがそんな弱音を吐くなんて。
いや、彼女だからだろう。物事を深く考える彼女は、
私よりもよほど現実的に物事を捉えているんだ。
その素晴らしさも、危険さも。
彼女は慎重に、天秤にかけているのだ。\\
「大丈夫だよ。君たちには力があるし、
まず大きなアドバンテージとして、その頭があるよね」\\
「そうよ、セレナ。私たちは考える剣よ。弱いわけない」\\
アヤメはセレナの肩に手をおいた。
そして私の方にも目を向ける。
今朝からのイメージが一新されていく。
私は彼女に対して、なにか強い心の強靭さと、優しさを感じた。

「君たちが不安がるのもよく分かるよ。
でも安心して、僕たちが精一杯サポートするから。
それに、死の可能性は殆どないよ。それは保証する」\\
「はい、いきなりってことはないから。
ちゃんと教えてあげる。
多分、今夜からでしょ」\\
「そうだね。決意が揺らぐ前に」\\
「今夜って、どうすればいいんですか」\\
「アヤメもセレナも、今日は早く寝てね」\\
「寝る? 家で寝てるんですか?」\\
「そう、眠っていて。遅くても十時までには、寝ておいた方がいい」\\
「集合場所とか、決めなくても\scalebox{3}[1]{―}」\\
「そんなものはないの。ただいつも通り、寝てね」\\
「わかりました」

「それと……」\\
アヤメはアオタに促すような仕草をする。\\
「もう一つ君たちに重要なことがあるんだ。
君たちは僕と契約をする。
契約は双方性を持っている。
つまり、君たちが狩りを行う対価を、僕たちは用意しなければならないということだね」\\
「対価? お金か何かなの」\\
茶化すような言葉。\\
「そんなわけないでしょ、セレナ」\\
「あはは。そうだよねーいくらなんでもそんな俗っぽい\scalebox{3}[1]{―}」\\
「構わないよ。お金でも物でも、実現できる可能な限りを尽くすよ」\\
私は耳を疑った。
有り体の場合、と言っても創作物上の話でしかないが、
こういった行為は、無償の奉仕ではないのだろうか。
或いは、奇跡のような人のみには有り余る神秘。
世俗から乖離した、純粋無垢な願いの成就なのではないだろうか。\\
「本当に? じゃあ億万長者になりたいとか、
あの店の服全部買い占めたいとかも?」\\
「実現可能性を考慮する必要はあるけど、基本的にどんな願いでも
僕たちは受け入れるよ。
もちろん、ある程度はそれに似合った働きは必要だけどね。
一人の人間が、その上に何十億もの人間の未来を背負う苦しみは大変だろうし、
そんな仕事をタダでしろ、という方が酷な話だよね」\\
「でも願いがなかったら」\\
私は聞いた。
私ははじめから、何も望まないつもりだった。
セレナもそうだろう。
ただあの時の恐怖から、突き動かされただけなのに。\\
「人には必ず、願望がある。その大小は関係ない。
どんな些細なことでもいい。これは、僕たちの気持ちなんだ」\\
「めっちゃいいやつじゃん! アンタ!」\\
セレナはアオタを掴んで持ち上げる。
小動物を可愛がるように、頭をくしゃくしゃに撫でる。\\
「願いは何だっていいの」\\
「アヤメさんにはあるんですか」\\
「私はもう、叶えてるの」\\
「それはどういう……」\\
「必ずしも狩りを全うしてから、ということでもないよ。
例えば一人暮らしをしたい! と願えば、僕たちは賃貸契約や光熱費、
食費などを十分にまかなえる金銭を用意する。
もちろん、実際に住む場所も、本人が望むなら用意するよ」\\
「へえー。でも料理は作ってくれないの?」\\
「流石にそこまではね。専属の料理人をつけるってことならできるかもしれないけど。
一般市民が召使いなんて、今の時代じゃああまり考えられないからね」\\
「ふーん」\\
「とにかく、いろいろな形態が許される。
望めば、今日からでもできるかもね」\\
「どうだい? カナン、セレナ?」\\
「うーん。今はいいかな。考えとく」\\
「そうね、私も」\\
「わかったよ。急ぐ必要はないしね」\\
時計を見ると、すでに五時を過ぎていた。\\
「私から一つ助言をすると、そんなに気負うことはないってことね」\\
「そうだね。リラックスするといいよ。適度な緊張も必要だけど、
一番大事なのは心のゆとりだからね」\\
「これからあなた達には、想像以上の世界が広がる。
その一つ一つに驚いたり気を滅入ったりしてる暇はない。
だから、覚えておいて。
\scalebox{3}[1]{―}目覚めれば、全ては夢だと分かることを」\\


それじゃあ、と帰るように促された。
駅まで一緒に歩いて、アヤメは上りで私たちは下りの電車に乗る。

金沢で降りる。

じゃあね、と別れた。

また暗い道。
私は夕暮れの会話を思い起こした。
名前も知らない誰かを死の淵から助け出す。
その響きは心地よく、私の心に共鳴する。
その分の反響もまた、私の心に影響する。
はたして、私に務まるのだろうか。
私は、何を望めばいいのだろうか。

\subsection*{(3)}
「ごめんねー今日弁当作れなくて」\\
「ううん別にいいよ」\\
「ほんと朝が辛いのよ。明日はちゃんと作るからね」\\
「うん。無理しないでね」\\
ごはんを食べながら、お母さんの弁明を聞く。
最悪、昼ごはんや朝ごはんが一食程度なかったとしても、死にはしないし、
気にすることはなかった。そもそもお小遣いももらっているのだから、それで賄えばいい。\\
「ただいま」\\
「おかえり」\\
「おかえりなさいあなた。どうする? ごはん食べる?」\\
「うー、あー後でいいわ。先に風呂入るわ」\\
「了解」\\
お父さんは風呂場へと消えていった。\\
「ごちそうさま」\\
お姉ちゃんは早々に食べ終わって、食器を下げる。
昔からまるで、飲んでいる様な速さでモノを食べている。\\
「相変わらず早いわね。もっと噛んで食べなさいよ」\\
「いいの。私は唾液の中の、アミラーゼが多いの」\\
「なにーそれ」\\
「消化酵素。小学校で習ったでしょ」\\
「お母さんの小学生時代なんていつだと思ってるのよ」\\
「はいはい」\\
「お姉ちゃんお風呂入ったの?」\\
「うん、もう入った」\\
「なんだか最近早くない?」\\
「仕事が早く終わるの。そんだけ私が優秀ってことね」\\
「ふーん」\\
お姉ちゃんは二階へと登っていった。
いつも、というか年明け前のお姉ちゃんは、私よりも相当遅い帰りだった。
午後八時過ぎごろだっただろうか。
だから、お姉ちゃんと一緒にごはんを食べたり、ましてや先にお風呂に入られて、
リビングでくつろいでいる姿を見るのは、なんだか落ち着かない。
いやというわけではないが、まあ、じき慣れるのだろうか。\\
「いただきます」\\
お姉ちゃんの食器を洗い終えたお母さんが、向かいに座ってごはんを食べ始めた。\\
「チャンネル変えてもいい?」\\
「いいよ」\\
流れている番組がことごとくつまらない。
チャンネルを数回回して、結局ニュース番組だ。\\
「なんだか物騒ね」\\
「なにが」\\
「テレビ」\\
「ほんとだ」\\
ニュースは丁度、地方局のほうに切り替わっていて、そこには行こうと思えば、
すぐに歩いて行けるくらいの近場で起こった、不審死事件についての報道をしていた。
それによると、なくなったのは身元不明の女性の遺体で、死後一ヶ月ほどは経っていた。
しかし、事件性はないという。
警察の推察は、彼女の死因を餓死であるとしてる。
今の時代、飢え死ぬ人間などいるのだろうか。
私は疑問に思った。
そしてすぐに、当然の帰着、今の私だからだろうが、
に至った。

彼女こそが、救うべきものであったのかもしれないと。

\section{}
\subsection*{(1)}
早く寝ろという言葉を思い出した。
私はその言葉の通りに、いつもの習慣を早送りにして、
普段よりも三時間早く布団に入った。
夜九時ごろだった。
眠れと言われたが、いざ時間が迫っている緊張を感じると、
余計な雑音が頭の中に響いて、声が聞こえる。
音のない声。
私は、枕を耳に押しつけて、何も考えないことにした。
あー、と叫び続ける。
意識的な無音の発声が、雑念を突き刺して、そのままに意識の領域から突き抜けていく。

そしていつの間にか、その叫びも衰えて、私は眠りに落ちていった。

\subsection*{(2)}
ぼやけた視界。
こもった音。
流れる空気の冷たさに、私は目覚める。

ここはどこなのか。
いや、そもそもどうなっているのか。

「あ、カナン。こんばんわ」\\
すぐ目の前にはセレナの背中。
私はだんだんと世界の輪郭を捉え始める。
街だ。ごく普通の住宅街。沈黙を貫く町並み。
私たちはその中の、電灯の下に立っていた。
だけど、これはただの夜ではなかった。\\
「こんばんわって」\\
「挨拶は大事だって、アヤメさんが言ってた」\\
「それはそうだけど」\\
「なんでも、普通でいるのが一番いいんだって。
だからほら、カナンも」\\
「わかったよ……こんばんわ」\\
「どうも」\\
なんだか照れくさい。
友達に対して意識的に挨拶をすることは普段ないからだ。\\
「アヤメさんはどこに?」\\
彼女の姿はなかった。\\
「カナンが遅いから、探しに行っちゃったの。
でもすぐ戻ると思うよ」

「セレナ。いつからここにいるの?」\\
「十分ぐらい前かな。それよりもさ! なんかすごくない? これ!」\\
ひらひらとマントのようなものをたなびかせながら、くるくると回るセレナ。
まるでこどもだ。
でもその気持ちはすぐにわかった。
私は自分の羽織るものに気づいた。\\
「カナンのやつもすごくない? なんか裏路地にいるアブナイ人みたい」\\
「なにそれ」\\
私はやっと自分の視界を制限していたものに気づいた。\\
「うわ、私なに着てるの」\\
大きなフードだ。
黒くて、これを目深にかぶっていれば、確かに危険な香りを匂わせる。
ポンチョのようなものにフードが付いている。
その中身は白いワイシャツで、なんだかイメージとは違う。
腹にはベストのようなものが巻かれている。内臓を守るためだろうか。
足には長いブーツが、ぎっちりと結ばれている。
普段ベルトは閉めないので、その圧迫感はには不慣れで、すこし動きづらい。\\
「カナンのかっこいいよね。でもどう? 私のも結構キマってると思うけど」\\
彼女の服装もまた独特だった。
ビクトリア調の礼服の上にケープコートを羽織っている。
裾や袖には細かな装飾が、目立たないように施されている。
彼女の整った顔には、かなり親和性の高いおしゃれな服だった。\\
「貴族っぽい。セレナに似合ってる」\\
「やめてよ貴族なんて」\\
言葉は否定的だが、まんざらでもないようだ。\\
「でもやっぱ慣れないね。ちょっと動きにくい」\\
「そうなんだよねえ。私なんてベルト初めてでちょっときついの」\\
この衣装が一体どういう基準で選ばれているのかはわからないが、
ここで文句を言っていてもどうにもならないことは、すぐにわかった。\\
「セレナ、カナン。子どもじゃないんだから、いつまでもはしゃいでない」\\
「すみません」\\
「ごめんなさい」\\
「まあ分からなくもないわ。いきなり格好そう簡単になれるわけじゃないしね」\\
アヤメの格好もまた特殊で、喩えるなら銃士、といったところだろうか。
ただ私たちの衣装全般に言えることは、ことごとく暗い色彩で、
暗闇に紛れることを第一としているらしい。
まだここは電灯の光が届くから、はっきりと境目を捉えられるが、確かに一歩光の外に出れば、
完全に同化して、悪夢というものにはどう映るのかは判断しようがないが、
人の目には捉えられないだろう。
ただセレナは、もう少し可愛さを求めているようだった。\\
「この服って好きにできないんですか?」\\
「さあ、あまり考えたことないし、わからない。
その必要な感じられないけど」\\
「そう……ですよね」\\
「おしゃれしたくなるのもしょうがないけど、
せいぜい小物にとどめておいた方がいいわ。
これはこれで、慣れれば柔軟に動けるから」\\
「わかりました」\\
「カナンもそうしてね」\\
「はい、わかってます」\\
「それじゃあ、すこし移動するから。ついてきて」\\
私たちは夜の街へ歩き出した。\\
「どこに行くんですか?」\\
「公園ね。ひと目につかないから」\\
「そういえば、ここってどこなんですかね。県内だとは思うんですけど」\\
車のナンバープレートが、よく私たちの見るものだった。\\
「さあ。私は地理に疎いから。まあどこであろうと、基本的やることは変わらないわ」\\
ということは私たちの任意ではないということだろうか。\\
「目覚める? って言っていいのかな\scalebox{3}[1]{―}」\\
私の言葉を訂正するように、アヤメは割り込んできた。\\
「私たちは、今の状態を『夢に生きている』と言っているの。
どうしてかはよく聞いていないけど、昔からそう言っているらしい」\\
「じゃあ、この夢に生きているときって、場所とかどうなってるんですか」\\
「場所は不定ね。悪夢が近くにいる場所から『始まる』。アオタが決めてるの。
あの子はいつも、悪夢が出そうな場所を監視しているの」\\
「へえーアイツって案外すごいヤツなんじゃ」\\
「そうよ。アオタが居ないと、私たちの狩りはままならない。
だからセレナも、あんまりキツく当たらないであげてね」\\
「いやー別にそんなつもりはないんですけどね」\\
ははと笑うセレナ。
やけに大きく聞こえる。
こんなに静かな外は久しぶりだった。

「ここね。入って」\\
側溝を飛び越えて、ロープを乗り越えて、公園へと入る。
あまり大きな公園ではなく、遊具も少ない。
こんな場所にいれば、かえって目立つのではないだろうか。\\
「こんなところ逆に目立つんじゃないですか」\\
「そうかしら。人はかえってコソコソしているものに目がつくんじゃないの」\\
考え込んでいるのだろうか、若干の間のあと言葉は続いた。\\
「ああ、ひと目につかないっていうのは、
人に見られるんじゃなく悪夢に勘づかれないって意味ね」\\
わかりにくてごめんなさいと、彼女は付け加えた。

「アオタいない? ここで待ち合わせるって約束したんだけど」\\
周囲を見渡しても、どこにも見当たらない。\\
「いませんね」\\
「こっちもいない」\\
「焦る必要もないから、少し待ちましょう」\\
「はい」\\
「でもただ突っ立ってるのも暇ね。
なにか質問とかない? 今のうちに答えられるものは答えるわ」\\

「悪夢って、普通どこにいるんですか」
セレナの質問だ。\\
「そうね、アイツらは普段というか昼間の時間帯は、実体を持たない。
つまり私たちの目には映らない。
どこにいるのかっていうと、人の心の中、精神と同調しているの。
たしか、アオタがガンって言ってたと思うけど、まさにそうで、
大して大きくない悪夢には何も影響はない。
コイツらはただ、人の精神活動のおこぼれに与っているだけね。
人の心を惑わしたり出来ない。
だからこの種類の悪夢は、私たちの夢\scalebox{3}[1]{―}今ここのことだけど\scalebox{3}[1]{―}
では小さいし非力。多分今日はこの手のヤツを狩ってもらうわ」\\
「じゃあ、私たちがあの時に見たやつは……」\\
「アレはかなり成長したタイプね。そうなってくると、人間の心に与える影響も大きくなるし、
ただのおこぼれじゃ満足しなくなる。
最初は夢を、それも食いつぶすと人そのものを喰らおうとする。
魂を喰らうの。
二三人も食べれば、かなりの大物になる。
ここまでくると、かなり個体差が出てきて一概に言えなくなる。
その相手都度に異なる戦い方を求められるわ。今の貴方たちには無理な話だけど、
いずれこれにも戦ってもらわないとね」\\
「アヤメさんはどれぐらいの時に、戦えるようになったんですか」\\
「私はたしか、二ヶ月ぐらいだったかしら。
その時は私も含めて、二人しか居なかったし、しかも二人とも素人だった。
だからかもね。普通はもっと時間をかけてもいいし、何度も言うけど、焦る必要はどこにもないの」\\
「心が弱くなるからですか?」\\
「そうねセレナ。焦りは心の隙を生む。悪夢は人の心に強い。
だから、私たちは狩人になってそれを防護する。
狩人という意識が、私たちを強くしてくれるの」\\
「本当にそうなんですかね……わたしなんだか怖くて」\\
「カナン。その気持ちは私にもあったわ。
でも大丈夫よ。どんなことでも、なにかを始めるっているのは度胸のいること。
カナンはもう一歩を踏み出せてる。
それを忘れないで。たまには根拠のない自信を持つのも大切よ」\\
彼女やアオタの言葉は、確かに精神的強度を上げようとするものばかりだ。
病は気からというように、人の精神力は物理的な肉体に作用していると思っている。
ただ、自分の精神をコントロールする術を私は知らない。\\
「根拠のない自信だったら、私の得意分野じゃん!」\\
セレナはそうだ。常に活気に満ちている。
少なくとも私の前ではそうだ。落ち込んでいたり不安がっていたりしても、短い時間で
それは溶けていく。\\
「私は……苦手です」\\
「カナン。そんなこと言わない」\\
アヤメは私の肩を掴んだ。
彼女の目線は、まっすぐに私の瞳に向いている。
何も言わない。
アイコンタクト、なのか。
目を逸らすことは出来ず、ただ無言のまま見つめ合うだけ。
だけどどこか、嫌ではない。
力強さを感じる。
年長者の落ち着きか、彼女の気高さかなにかが、私をどこかへ導いてくれる。\\
「うん、これでいい」\\
アヤメは手を離した。
掴まれた後はほんのりと温かい。
人の温かさだ。\\
「なんかすごい。絆ってやつですね!」\\
セレナが目を輝かせている。\\
「そんなものじゃないわ。これはカナンの力を引き出したの」\\
「ええ、そんな技があるんですか?」\\
「嘘よ。ただのプラシーボ」\\
「それって言ったら効き目ないんじゃ」\\
「いいのよ」\\
彼女は若干のいい加減さを持ち合わせているようだった。
それも、強くなる秘訣なんだろう。

「あそこ」\\
セレナが指を指した。
青い玉が浮遊している。
遠目に見ればまるで人魂だ。\\
「ごめんねみんな。どれ位待った?」\\
「ううん、全然」\\
「それはよかった。それと、今夜は楽な狩りになるだろうね。
あちこち探し回って、やっと三匹ほどの集団を見つけられたよ」\\
「その三匹って、小さい?」\\
「もちろん。おそらく生まれて間もないだろうね。
君たちにも十分倒せるだろう」\\
「よかったねーカンナ!」\\
喜ぶべきことかは判断しかねるが、
いきなり難しいことにはならないようだから、私はとりあえず安心して、彼女に同意した。\\
「どこらへんにいるの」\\
「そう遠くない。路地だね」\\
「そう、なら楽ね。いい、みんな。今回は追い込む方法でいきましょう。
行き止まりに、脅かして追い込むの」\\
「逃げられなくなったところを、一気に、ですね」\\
「そうよ」\\
作戦は至ってシンプルだった。
砂場に簡単な地図を描いて、どう動くべきかを打ち合わせした。
ただ肝心な点を理解していなかった。
「あのーところで、私たちって、武器とかないんですか」\\
「ああ、そうね」\\
すっかり忘れていた、という様な顔だ。\\
「ごめんなさいね。ほら、アオタ」\\
「二人ともこっちに来て」\\
アオタはベンチの方へ私たちを誘う。\\
「これが、今から君たちが狩人たる寄る辺になる。気に入ってくれるといいのだけど」\\
私たちはベンチの上を見た。
そこには、現代社会にはおおよそ居場所のない代物が鎮座してた。\\
「これ、どっちがどっち?」\\
「剣はセレナ。銃はカナンだね」\\
「じゃあ、この長いバッグが私のってこと?」\\
「そうだよ。鞘みたいなものだね。それも大事にしてよ」\\
「持ってみてもいい?」\\
「いいけど、気をつけてね。セレナは危なっかしそうだし」\\
あ? と威嚇するセレナ。
だか私もそう思う。
もっとも、それは私も同じだが。\\
「重い」\\
セレナの持ち上げた剣は、かなり長い。
それも一本ではない。
長く大きな剣と、短い剣。\\
「これ二刀流ってやつ?」\\
「そういう運用方法も可能だろうね。どう使うかは、その剣に聞くといい」\\
「剣に聞くって……」\\
「さあカンナも。取り出して持ってみて」\\
「うん……」\\
促されるままに、私も武器を調べる。
黒いナイロン製だろうか、よく映画で見るような装備、スナイパーの持ち物だ。
チャックを開いて、中身を見る。\\
「これが私の武器?」\\
中から出てきたモノは、夜目にも一際目立つ真っ黒な武器だった。

一つは小さい、というかよくテレビとかで見る典型的な拳銃のイメージそのもので、
ずっしりと重たいが不思議と持ちにくさはない。
まるで何年も使い古して、完全に自分の手に馴染んでしまっている様な感触。
引き金を引くまでの所作を違和感なく行える。
もう片方は私の身長の四分の三以上はあるかもしれない長さと、威圧感の装飾を纏った銃だった。
片方よりも更に重たく、素人目に見ても持ち上げて撃つものではないことは分かった。
長方体の集合、直線によってのみ構成された\ruby{番}{つがい}の武器は、
その銃把を手に握ると同時に思考が晴れ渡り、
個々の持つ特性、適切な運用方法が頭の中に入ってくる。
それをマニュアルによる情報と言うより、
\ruby{先天的}{アプリオリ}な事実として理解させられたことに、私はアオタの存在の根源性を感じた。

それによると、無機質な銃たちは外見を裏切らず、二つとも戦う以外の役割を一切捨てていた。
当然といえば当然だが、私たちの一般生活においてそんなものは一切触れない。
私は若干の困惑を、現実性と共に受け取ったのだ。
それは煽動にも似て、私の心をかき乱す。\\
「見てみてカナン」\\
セレナが剣を振り回していた。\\
「ほらこれ、かっこよくない?」\\
「おおーすごい」\\
彼女の剣舞は、見事なものだった。
熟練した剣さばき、私でも分かる。
美しいのだ。優雅な線を描いている。\\
「なんかこう、剣が教えてくれるっていうか、
どう動かせばいいのかわかるんだよ。すごくない?」\\
「なかなかね」\\
アヤメが近づいてきた。\\
「見てください! アヤメさん」\\
もう一度剣を振る。\\
「見事ね。動きにくいとか、違和感を覚えたりしない?」\\
「ないです、まったく」\\
「そう。カナンは? それは、使える?」\\
「大きい方はまだわからないですけど、こっちは」\\
私は自然に弾倉を確認し、スライドを引く。
セーフティを外して、あとは引き金を引きさえすれば、弾は発射されるだろう。\\
「これで大丈夫ですか」\\
「そうね、まずまずかしら。基本は押さえていると思うし、あとは実践できるかどうかね」\\
「はい」\\
「今はまだ、その大きい方\scalebox{3}[1]{―}狙撃銃かな、そっちはまた後にしたほうがいいかも。
今日はたぶん必要ないから」\\
「わかりました」\\
「それじゃあ、そろそろいくわよ」\\
セレナ、とアヤメは子供のように剣を振り回していた彼女も呼んで、私たちはみな、
アオタの前に集合した。\\
「二人とも今日が初めての狩りだね。こう何度も言われると鬱陶しいかもしれないけど、
本当に大切なのは、心のゆとりを保つこと。冷静でいることだね。
それさえ理解してくれれば、今日の狩りは、たとえ物量的な収穫がなくても、
意味のあるものになるよ」\\
「うん、わかった」\\
「気をつけるよ」\\
「アオタの言う通りね。何かあっても、焦らないでね。
私がいるから。自惚れるわけじゃないけど、たぶん今回の獲物は私一人でも十分だから。
無理しないこと。約束できるよね」\\
「はい! 約束します」\\
「私もです。アヤメさん」\\
「うん、いいわね。じゃあ、移動しましょう」

私たちはアヤメの後ろを付いていく。
アオタはどうやら同行しないらしい。
なぜ彼は行かないのかアヤメに聞いてみると、
「彼は後方支援担当なの。それに、お世辞にも戦えるとは言えないから」と返ってきた。
あたりを見渡せば、明かりは更に消え、夜は深みを増していく。
ただ、はじめ頃ほどの不安はなかった。
握る武器の頼もしさか、アヤメの背中の勇ましさか、
セレナといる安心感か、どれが私の恐怖を紛らわせてくれたのかはわからないが、
それでも、踏み出す一歩一歩が、普段にまして力強いのは、疑いようもない。
自分にこんな自信の源があったのか。
私は密かに、些細な喜びを噛み締めていた。

\section{}
狩りは思いの外簡単なものだった。
アヤメにはもちろん、セレナやカナンにもそう言えるだろう。
悪夢の外見はさほどおどろおどろしくもなく、どこかアオタに似た、
ぬいぐるみのような可愛らしさを持っている。
水滴のような体を引きずる姿は、彼女たちの気の緩みを誘う。
それを追う彼女たちは、遊戯感覚で狩りにあたっていた。
それほど、今夜の獲物は単純極まりない存在だった。

それらの行動原理は単調なものだ。
手続き的に記述できるだろう。
悪夢は、狩人を敵であると認識できている。
だから、目の前にそれらしき物があれば、
きた道をただ反転して、一直線に逃げる。
そして反対にそれが起これば、またしても反転して逃げるのみ。
ただ、記憶はあるようで、分岐路の存在する場合は、まだ一度も自らが
進んでいない道を、その最短距離から判断して選択している。
少し観察すれば、誰にでも理解できるものだった。
彼女たちが悪夢を行き止まりに追い込むのに、そう時間はかからなかった。

「向こうに行った!」\\
「了解」\\
カナンとセレナは、走り回っていた。
彼女たちの頭の中には、この地域周辺の寸分違わず正確な地図が浮かんでいた。
それがアオタの力であることは明白だ。星の代弁者を名乗るのも、今の彼女たちには頷ける。
精神的な高揚もあってか、彼女たちの思考は驚くほどに明瞭だった。
いまもし彼女たちが、学校の試験を受ければ、間違いなく満点を取るだろう。
それほどまでに冴えているのだ。
この問題をどう解くべきかは自明で、後はただ『上手く』やるだけだ。
アヤメはというと、彼女は、彼女がカナンたちと最初に接触したときのように、
家々の屋根を飛び、悪夢たちを威嚇していた。
地上のカナンたちが楽々と悪夢を追えるのも、彼女の助力があってこそだった。
アヤメは優れた先導者なのだろう。
自分がやるべきことをよく理解している。
ここはまず、自らの見せ場を捨てて、彼女たちに狩りへの恐怖心を減らすべき。
そう彼女は判断したのだ。

「あそこ、あそこに追い込もう」\\
「わかった。カナンはそっち塞いで」\\
「間に合うかな」\\
「頑張って!」\\
「うん」\\
カナンは全速力で走った。普段使うことのない筋肉を、酷使している痛みを感じる。
それでも不思議と苦痛はない。痛みはただ、傷つくことだけを知らせる手紙だ。
中身は理解できるが、そのものではない。
だからもっと、もっと速く。
叫ぶカナン。\\
「間にあって!」\\
「おら、こっちに来い!」\\
カナンの努力は功を奏し、悪夢たちは唯一の逃げ道を塞がれ、どうしようもなくなってしまった。
ただ安心しているのだろう。狩人の姿は見えない。
彼らはすっかり逃げるという思考を放棄した。
この場がもっとも安全であると評価したのだ。
それがまったくの間違い、それもかなり愚かな判断であることを、
彼らが自らを評価できるはずもなかった。

「アヤメさん! こっちです」\\
セレナが手招きをする。
二人は袋小路の入り口の、壁に背を寄せていた。
「あそこに居ます。四匹です」\\
指さされた先を、アヤメは慎重に覗き込む。
そこにはインク溜まりの様な悪夢たちが、月明かりで疲労を洗い流している。
「上出来ね。二人ともよくやったわ」\\
「ありがとうございます!」\\
「セレナ、まだ喜ぶのは早いわ」\\
「あ、そうですよね。仕留めてないし」\\
「そうよ。こういうところで気を抜くと、痛い目を見る。
鉄則よ、覚えておいて」\\
はい、と二人揃って頷いた。\\
「それじゃあ、カナン」\\
アヤメはカナンの腕を掴み、持ち上げた。\\
「あなたの出番よ。その銃で、ヤツらを撃ってみて」\\
「私がですか?」\\
「あなた以外に誰がいるの?」\\
「そうですけど」\\
カナンの顔は乗り気ではない。
彼女の臆病な部分が顕になる。
無理もないだとアヤメも承知しているが、これは絶好の機会だ。
逃すわけにはいかないだろう。\\
「頑張って! カナン」\\
セレナはカナンの背中を叩く。\\
「セレナ……」\\
「カナン、やるしかないわ。ほら、ちゃんと構えるのよ」\\
カナンは覚悟を決めた。
やるしかないのだ。
銃把を握りしめ、彼女は問う。
どうすればよいのかを。
どうやって照準を合わせるのか、衝撃に耐えるにはどうすればいいのか、
この状況に対する、適切な助言を求める。
返答はすぐに、彼女の思考に刻まれる。
具体的な映像イメージと、声によって修飾された情報を、彼女はすぐに理解した。

そう、理解したつもりだった。

いま! そう心の中で叫んで、彼女は物陰から飛び出した。
音に敏感なのか、悪夢たちは彼女の靴が擦れる音をすぐさま聞き取った。
ギイ、ギイ、と不協和な怒りの声を喚いて、激しく動く。
壁を登って逃げようとしているのだろうか。
しかし彼らには、壁をよじ登れる手足がない。
おそらく運動量も足りないだろう。
彼らもまた覚悟を決めた。
正面突破しかない、と。
地団駄を踏む悪夢達に、カナンは焦ってしまった。
早く打たなければ、そう思えば思うほど、照準のブレは大きくなっていく。\\
「ああー、当たれ!」\\
彼女は一瞬合致した悪夢にめがけて、弾を発射した。
びゅきゅ、と粟立つ音がする。
命中したのだ。\\
「あ、当たった」\\
喜ぶのもつかの間、残る悪夢たちは猛突進してくる。\\
「カナン伏せて!」\\
アヤメはカナンの頭をとっさに押し下げ、そのままに逃げ去っていった二匹の悪夢を追う。\\
「あう」\\
予告もなく無理矢理に動かされた頭を抱えながら、カナンは尻もちをつきかけていた。
たがまだ一匹いる。
しかし彼女はそれを忘れてしまっている。\\
「カナン後ろ!」\\
セレナの呼びかけは無意味だった。
悪夢の体は、セレナに直撃した。
突っ伏せるカナン。
悪夢も動きを鈍くしていた。\\
「カナン\scalebox{3}[1]{―}」\\
セレナは堪えた。
カナンに駆け寄るよりも前に、自分にはやるべきことがあると。\\
「くっそ、この野郎!」\\
剣を手に、悪夢を滅多打ちにする。
打たれるたびに、耳障りなノイズを放つ悪夢。
その光景は、この夢の始まりに、彼女が振るった剣技とは大違いだった。
いかにも素人で、ただ感情に任せているだけだ。
下手くそ。そう形容するほかない。\\
「はあ、はあ……」\\
息を荒げる彼女の前には、もはや形状をとどめていない、
無残な黒い染みが広がっているだけだった。\\

カナンやセレナと比べて、やはりアヤメの手際はよい。
無駄がない。

彼女の追っていた、死にものぐるいな悪夢の一匹は、なんと宙を浮いた。
しかしそれは悪夢にとって、先の袋小路となんら変わらなかった。
アヤメは一瞬の内に飛び上がり、自由落下する悪夢を貫いた。
跳んだはいいもの、落下中の自らを制御する方法など持っているはずもなく、
彼女がその行き先を予見することは簡単だった。

そしてもう一匹は愚直に、元来の行動表に従って、直線上を逃避していた。
最小限の音しか立てず、地面に着地したアヤメは、右手に握っていた剣を左手に持ち替えた。
そして驚くことに、剣は弓に形態を変化させた。
高度な仕掛けによるものだろう。
剣は鉛直方向に真っ二つに別れ、若干の曲線はまさに弓そのものに剣を仕立て上げる。
腰にかかった籠から、矢を取り出す。
金属の弦だろうか、固く、ずいぶんと大きな力を込めて、彼女は弓を引いた。
放たれた矢は空中を裂くように、悪夢めがけて飛んでいく。
それが当たることは、難しいことではなかった。
直線上に位置する悪夢に直撃。
息絶えた悪夢は、溶けるように消失した。
跡形もなく。

悪夢は完全に、敗北したのだ。\\

「大丈夫、カナン?」\\
倒れ込んだカナンを支えるセレナ。\\
「ああ、血が出てる」\\
「大丈夫だってセレナ。これくらいなんともないから」\\
鼻血だ。\\
「大丈夫じゃないよ! 鼻折れてたらどうするの? ちゃんと手当しないと」\\
「ただの鼻血だって、だから\scalebox{3}[1]{―}」\\
「二人とも頑張ったね」\\
アヤメが帰ってきた。その肩にはいつの間にか、アオタの姿もある。\\
「アヤメさん! カナンが血を!」\\
「あら、本当ね」\\
ただアヤメは見るだけで、何もしようとしない。
その態度に不満なのか、セレナは言い寄ろうとする\\
「血ですよ! 血! ちゃんとしないと\scalebox{3}[1]{―}」\\
「その心配は必要ないよセレナ。カナンは大丈夫だ」\\
「大丈夫って、アンタがどうして言えるのよ」\\
「セレナ、本当に大丈夫だから」\\
セレナを止めるアヤメ。\\
「ところで、どうだった。はじめての狩りは?」\\
「うーん……まあまあです」\\
「私は、だめだったと思います。こんな風になっちゃったし」\\
「二人とも自己採点が厳しいわね。私はそう思わないけど。ねえ、そうでしょアオタ」\\
「そうだね。君たち二人とも、しっかり一匹づつ悪夢を仕留めているじゃないか。
正直言って期待以上だよ」\\
「期待以上って、期待してなかったのかよ」\\
「まあ、そう怒らないで。その方が自然だろう。
どちらにせよ、君たちは僕の期待値を超えてくれた。十分に素質があると言えるね」\\
「二人とも、自信を持っていいわよ。
私の初めてなんて、こんな風にできなかった」\\
「そうなんですか……じゃあ、自信持ちます」\\
セレナは嬉しそうだった。
褒められ慣れてないのだろう。
途端に俯いて、アヤメやアオタと目を合わせようとしない。\\
「カナンもそうよ。これから、貴方達はもっと上を目指せる。だからもっと胸を張ってね」\\
「はい。がんばります」\\

カナンは冷静に自らの評価を受け入れていた。\\
「それじゃあ、さようならね」\\
「うん、みんなお疲れ様。しっかり朝は起きてね。早寝早起きは健康の秘訣だよ」\\
「はいはい、お疲れ様でした」\\
お疲れ様、というのもまた、彼女なりの工夫だった。
緊張感も程よく、適度な精神の弛緩を促すには、この狩りをより一般的な行動、
学校に通ったり、バイト先に務めるような感覚にするべきだという考えだろう。
ただ新入りの二人にとっては、若干の混乱の種かもしれないが。\\
「お、お疲れ様でした」\\
「お疲れ様です」\\
「うん、じゃあ。みんな良い目覚めを」\\
「バイバイー」\\
アオタの声を最後に、彼女たちの意識はふつりと途切れた。

\newpage
\section*{\tt after\_awakening}
夜の出来事がまるで嘘だったような、苦痛のない目覚め。むしろ、普段よりも快い。
ふと顔を触る。痛みも、血の跡もどこにもない。ベッドにも当然。
学校からの帰り道、傘を落としてしまった。
あの夜、初めての獲物を狩ったあの時と、同じ感覚。

まるで変わらない、アスファルトのザラザラとした痛み。


\end{document}