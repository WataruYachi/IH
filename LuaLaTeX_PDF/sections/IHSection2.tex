\documentclass[../IHMain]{subfiles}

\begin{document}
    
\chapter{狩人、\\その使命}
\section{\tt Where is my dream?}
\subsection*{(1)}
「行ってきます」\\
玄関を出た。
昨日と同じ寒い朝。
いつもより早めに家を出た。
コンビニで昼ごはんを買うためだ。
お母さんは昨日、どうやらかなり遅くまで起きていたらしく、
お弁当を作る気力がなかったのだろう。
いつまでも起きてこない、お母さんとお姉ちゃんをよそに、お父さんはいつもどおり
早朝に家を出ていた。

駅のコンビニに入る。
サラダサンドウィッチと、鮭おにぎり。
飲み物は水筒にお茶を入れているので、買う必要はない。
忙しい時間帯だ。
レジには多くの人が並んでいて、スタッフの人は忙しく働いている。
バーコードリーダーの音がより一層、その印象をくっきりとした形にしている。\\
「お待ちの方どうぞ」\\
隣のレジに、商品を置く。
お金を払って、レジ袋に入れられたおにぎりやパンを、そのままバッグに入れながら、
コンビニを出た。

改札を通って、まだ来ていない電車を待った。

\subsection*{(2)}
駅を出て、学校へ向かって歩いていた。
道すがら、ふと昨日の出来事が頭をよぎる。
あの後、意識的にこのことについて考えることを避けていたのだけれど、やはり気になってしまう。
夢のこともそうだが、何よりも、あの黒い物体と、それと共に現れた人間。
まじまじと見つめていた訳ではないが、その時の印象を思い起こしてみる。
眉を潜めた顔しか、記憶には残っていないが、それでも感じたことがある。
あの顔はかなり若かった気がする。
そう思えば、彼女の容姿が私たちの一つか二つ上の人たちによくいるような感じがした。
それに髪型も、かなり特徴的だった。
片方を伸ばしたショートボブというべきだろうか。
そしてなぜ、あの場所に現れたのか。
あそこに人が立ち入ることが、あるのだろうか。
あるとしても大人だろう。それも、かなり歳をとった人。
だとすれば、あれは同じ学校の人間。
\scalebox{3}[1]{―}いやそもそも、アレが現実であるという前提で話を進めるのもどうかと思う。

考えれば考える程に、私の頭はこんがらがる。

思考に頭が重くなって、うつむいたままに道を歩いていると、後ろから何かが肩にぶつかった。\\
「すみません」\\
通り過ぎようとする女性からの声だった。
ずいぶんと早歩きで力強い。
颯爽と過ぎていく彼女に、私は注目した。
横顔が見える。
なんだか見覚えのある顔、あの時の人に似ている。
もちろん、口元は隠されていたから確かめようがないが、
その鋭い目元や、片側だけを伸ばしたアシンメトリーな髪型は、どこか印象深い。
そうに違いない。
きっと彼女が、あの時の女性だ。
私は抑えきれなかった。\\
「あの、あなた、あの時の人ですよね」\\
自分に声がかけられたと気づいていないのか、それとも無視しているのだろうか、訝しんだりすら
ない、全くの無反応。\\
「すみません。あなたですよね! 私たちを助けてくれたの」\\
流石に気づいたのだろう、彼女は振り向いた。\\
「あ、あの。すみません」\\
「あなた誰。知らない人。
申し訳ないけど、人違いじゃないの」\\
冷たい声。
鬱陶しがっているのは明白だった。
だけどその顔は、はっきりと断言できる。
あの顔とそっくりだ。
怪訝な、けれど攻撃的ではない目つき。
しかし人そのものを否定するような、
はっきりとした拒絶を滲ませたそれに、私はそれ以上踏み込むことが出来なかった。

彼女と私の住む世界は全く違うのだと言わんばかりに、
彼女は私をいとも簡単に置き去りにして、瞬く間に遥か向こうへと去っていった。

\subsection*{(3)}
「あ、カナン」\\
休み時間。
トイレを済ませていたら、偶然にもセレナと一緒になった。
寝不足気味なセレナの目。
彼女も昨日の出来事に頭を悩ませていることは、すぐに分かる。
手を洗いながら、私たちは話し合った。\\
「私もうわかんないよ。カナンはさ、覚えてるの? 昨日のこと」\\
「覚えてるよ。やっぱり、本当にあったことなのかな、アレって」\\
「えーでも、ありえなくない? あんなにさ、アニメとか漫画みたいに人が
ビューンって飛んでくるとか、おかしいよね。物理的にありえない」\\
妙に堅苦しい語句を使うのは、セレナがその事柄について深く考えすぎている証拠だ。
確かに、彼女の言う通りだと思う。
あの真っ黒い玉はよしとしても、あの彼女の運動はありえない。
私たちの遥か頭上を飛び越えて、しかもかなりの時間地面に落ちず、剣を振り続けることなんて、
現実にできるはずがない。\\
「やっぱさ、私たちがどうかしてたんだよ。
集団ヒステリーってやつ、なのかも。化学薬品とかさ、そういうやつで幻覚をみてさ。
……それにしてもリアルすぎるってのもあるし、直前の夢とかもあるけど、
そうとしか言えないよね」\\
「ヒステリーってもっと病的なんじゃないの?」\\
「うーん、そんなこと言っても\scalebox{3}[1]{―}
ああもうわかんない。わかんなよなんにも。
ああ、イライラ」\scalebox{3}[1]{―}する。そう彼女がいいかけた時、
どこからか、遮るように声が聞こえた。\\
「そんなこと、簡単じゃないか。ただあるがままを受け入れる。
どうして人は、それが簡単にできないんだろう?」\\
声も子供っぽく、しかし憎たらしくないほど知性さを帯びている。
どこから聞こえるのだろうか。
洗面台の端。
何かが見える。
影のようだ。
いや、光だ。
私は目を疑った。
本当に、私の頭が狂ってしまったのかと、心配になった。
薬物の乱用者が、その禁断症状に見る光景、教科書で知ったそれと、
なんな変わりないものが見えたからだ。

落ち着いた青色の光球。
それに、まるでマスコットの様な極端にデフォルメされた目、
例えるならスマイルのマークといえばいいだろうか。
けれど口に喩えるパーツはない。
目だけだ。
それでも、それがただのぬいぐるみだといえば、ただの愛らしいものだろう。
だけどそれは、ゆらゆらと揺らめいて、動いている。
しかも、言葉を喋ったのだ。

「ねえ、今の私だけじゃないよね」\\
私はセレナに寄った。\\
「うん。やっぱ、カナンにも見えてるよね」\\
「見えてる」\\
「どんな形?」\\
「青くて、丸っこい」\\
「だよね」\\
これ以上、なんと返せばいいのか行き詰まってしまった。

混乱の静けさの中に、チャイムの音が響く。\\
「あ、時間だ」\\
白々しいセレナの言葉。\\
「遅れちゃう、いこうカナン」\\
「うん」\\
私たちは、見なかったふりをして、トイレから出ていこうとする。
その身振りを見て、焦るような表情
\scalebox{3}[1]{―}おそらくはほんの少し目の形が変わっただけだろうが\scalebox{3}[1]{―}
を見せる青い玉。\\
「ちょっとまってよ。僕は君たちに話したいことがあって\scalebox{3}[1]{―}」\\
何かを言いかけていたが、もうどうでもよかった。\\
「ほら、早く!」\\
「待ってよセレナ!」\\

「詠、遅刻」\\
残念だが、授業には間に合わなかった。

\subsection*{(4)}
「ねえ、なんなのコイツ」\\
「私が聞きたいよ」\\
私たちは食堂の端っこに、小ぢんまりとしていた。
昼休みだし、人は多い。
他人からの目を恐れて、私たちは交互にあたりを監視していた。
どう見たって、傍からは頭のおかしな連中にしか見られないだろうからだ。

キョロキョロとした二人。
これだけでも十分に怪しいが、それでも仕方ない。
木を隠すには森のなかに。
人混みに紛れれば、注目されることもないだろう。
人が全く来ない場所で、かつ話していても気づかれないような場所は、私たちには思いつかなかった。
下手をすれば、私たちは有りもしない虚空か何かに話しかけている、ヤバイ奴になってしまうのだ。

その原因は一つしかない。
先程の青玉は、未だしつこく、私たちに付き纏ってくるのだ。\\
「だ、か、ら! アンタがどうこう言ってても、私たちには一切関係ないから。そうだよねカナン」\\
うん、と同意する。\\
「そんなこと言っても、僕らにも僕らなりのちゃんとした理由があるんだよ。
それを訳がわからないの一点張りで拒絶するのは、はっきり言って酷いことだと思うよ」\\
私たちの強硬な姿勢に、まるで高圧的な客に対して、辟易とする接客業務員の姿を重ね見てしまう。
けれど当たり前のことだ。
ただでさえ昨日の出来事で疲れ切った私たちに、それまたおかしな格好と、正反対なその言動を、
受け入れろというは酷な話だ。

ただ、相手もなんだかんだで引くことを知らないようだ。
かれこれ、押し問答は三十分以上続いていた。

「じゃあ、あなたって何を私たちに『お願い』したいの?」\\
これ以上の繰り返しに、意味を見いだせなかった私は、思い切ってソレに聞いてみた。\\
「聞いてくれるのかい?」\\
一転して、その声色は明るいものになった。\\
「カナン、付き合わなくてもいいよ」\\
「そうかもしれないけど、気になるでしょ。私たちに何があったとか。
このままでいても、どっちにしろ納得は出来ない。だったら、一度受け入れてみるのも
いいんじゃないのかなって」\\
口はつぐんだが、納得はしてないようだった。
だけど、セレナも私と同じ気持ちだろう。
ただ私よりも、自分の中に押し込めるのが上手なだけで。

「もう始めてもいいかな?」\\
私は黙って頷いた。\\
「僕たちはね、君たちに大切なお願いがあるんだ。
僕たちは、ある人達を探しているんだ。
とても重大で深刻な問題を解決できる、強い力を持った人たち。
才能を持っていると言ってもいい。
だけど、それは本人には気づけないし、僕たちもなかなか見出すことが出来ない。
いいかい、落ち着いて聞いてほしい。
\scalebox{3}[1]{―}僕たちはね、君たちにこの星を救ってほしいんだ」\\

……一瞬で体の気が抜け落ちていくような感覚がした。
彼の雄大な熱弁と違って、私たちの今の顔は正しく間抜け面だろう。
ソレ\scalebox{3}[1]{―}いや彼の言葉は、まるで童話のセリフだ。
現実にはそぐわない、空想の世界の言葉。
すべてがお膳立てされた道筋の、単なるきっかけに過ぎない文字列。
『世界を救ってほしい』その言葉のなんと幼稚なことだろうか。

セレナは、思わず笑いだしてしまった。
そんな彼女に、彼は不機嫌になったのだろう。
凹んだ目を、眉間に皺を寄せる仕草に見立てているらしい。\\
「君たちはもっと、物事の本質を見るべきだと思うよ。
見たんだろう。あの黒い化物を。
感じたんだろう? その時の恐怖を。
だとしたら、結び付けられるはずだ。
これは決して、笑い事じゃないんだ」\\
精一杯に語気を強くしたつもりなのだろう。
だがそれでも、あくまでも可愛げのあるただの、マスコット的発声だ。
しかしその懸命さは、私たちの関心を引くには十分なものだった。\\
「じゃあどうやってその、『星』を救うの。
宇宙人と戦う? 悪いやつを殺すの?」\\
セレナは、さっきまでの不満そうな顔つきと一変して、かなり興味津々だった。\\
「宇宙人じゃないよ。もちろん人間でもない。
もし君たちが僕たちと『契約』を結ぶなら、君たちは『悪夢』と戦ってもらうんだよ」\\
悪夢。なんとも抽象的な名称だ。
横目にセレナと見合いながら、首をかしげる。\\
「悪夢というのは、実は僕たちにもよく分かっていないものなんだけど\scalebox{3}[1]{―}」\\
「ちょっとまってよ」\\
セレナが会話を遮った。\\
「どうしてアンタにもわかんないような奴と、戦わないといけないの?」\\
それは私も同感だった。\\
「確かに、その意見は正しいと思う。けれど実害は発生しているんだ。
現に、君たちもそうだっただろう。
うなされる夢に侵され、妙に現実味のある夢想の末に、特定の場所に連れ出される。
君たちは自らの意思で行動したと思っているようだけど、あれは一種の捕食行為。
あのまま僕たちが助けなかったら、君たちは今頃死んでいただろうね」\\
死んでいたかもしれない。
その一言はかなり心にのしかかってきた。
そして同時に、不安も湧き上がる。\\
「それじゃあ、死んだ人もいるってことなの」\\
私は聞いた。
息を整えることを、促すような間。
彼は見えない口を厳かに開くよう、話を繋げる。\\
「そうだよ」\\
明らかに重たい声。鎮魂の重みだろうか。\\
「死者は多くいる。
防げるものもあれば、どうしようも出来ない場合もある。
人口に対して、常に悪夢に対抗できる人間の絶対数は極小で、
すべてを守り切ることは不可能なんだ。
だからこそ、可能な限り、君たちのような適正ある人間に、協力してほしい」\\
私たちの顔は暗くなっていた。
なぜだろう、それが現実なのかを一切疑わずに、死という単語に共鳴していた。\\
「ねえ、さっきアンタは私たちを助けたって言ってたよね。
だとしたら、あの時の女の人は、死んじゃうかもしれなかったのに、
私たちのために戦ってくれたの」\\
「それが事実だよ。でもね、大丈夫。
君たちが僕たちと契約をすれば、必ず力を得る事ができる。
その力さえあれば、立ち向かうことができる。
死ぬ可能性は十分に抑えられるし、死の淵にある人々を救うことができる」\\
高校生には、いや日々をただ惰性で過ごしている私たちには、おそらく一生叶うことのない
ことだろう。
誰かの為になる。
急にこんなものを、輝く宝石のような、
生きる意味を提示されれば、目が眩んでしまうのは当然だろう。
少なくとも私はそうだ。
ここに来てまた、自分の中から現実性が脱落した気がする。\\
「これがどれだけ、君たちの心を迷わすことなのか、僕たちは理解しているつもりだよ。
それでも、心に漂うものたちと向き合うことは、有意義なことだと思う。
もし君たちが、この願いを受け入れてくれるのなら、放課後、E科の三年生教室に寄ってみてくれないかな。
\scalebox{3}[1]{―}答えがどうであろうと、僕たちはそれを受け止めるよ」\\
時計を見れば、すでに12時50分。
もうすぐで昼休みは終わる。
気付けば、周りも人は疎らで、残っているのは私たちを除けば、片手で数えられるほどだった。\\
「まって」\\
セレナが叫んだ。\\
「アンタは何者なの」\\
「それは、君たちが決めることだよ。もし来てくれたのなら教えるよ。
でもそうじゃなかったら、全部忘れるといい。こんなこと、心に潜めておく必要なんてないからね。
その時は君たちの送りたい人生を送るべきだよ」\\
彼はそう言い残して、静かに消えていった。
無色に、溶け込んでいくように。\\

残された私たちは、ひとまずは現実に戻ることにした。
時間はまだある。
急いで弁当箱を片付けたり、皿を下げたりして、食堂を出ていった。
その一切は無言のままだった。
互いに、自分の心の奥底に沈殿する何かを、必死に見透かそうとしているのだろうか。

自分でもわからないが、とにかく、普通ではなかった。

\section{}
\subsection*{(1)}
陽は傾き、散乱した赤い光が横から指す。
廊下に人は見えず、電灯は飛び飛びに着けられている。
節電のためだろう。
まだ四時過ぎだ。
セレナが教室に来てくれた。
掃除が終わって、上げられた椅子を机から降ろして、私たちは話し合った。
私は椅子に、セレナは机に。
もちろん、昼のことについてだ。\\
「どうするの、セレナ」\\
「どうしよう」\\
「そうだよね。なんだか、しっくりこない。
星を守るって、なんか宙ぶらりんだし、そもそも私にそんなことできるわけない。
でも、あの子が言ってたみたいに、私たちがせめて、せめて誰かの役に立つなら、
それを拒否するのも、自分は許せない」\\
「言ってたもんね、死ぬことはないって」\\
「たぶん」\\
「だったら、カナンの言う通りだよ。
黙って知らんぷりするのは、許せない。
私たちは助けられたんだよね。
だったら、私たちも誰かを助けたいよ。
できるかどうかは、別だけど」\\
もはや二人とも、疑うことはなかった。
すべてを事実として、選択すべきを決めかねている。
心の整理はつかないが、なにか私の、
本質的なものがざわめいているのは確かだった。
たましいなのだろうか。
少なくとも、理性的な刺激ではなかった。\\
「行ってみようよ。三年の教室だっけ。
カナンと一緒なら、私行くよ」\\
セレナの目は、どこまでも澄んでいた。
彼女の瞳は、少し青みがかっている。
吸い込まれるような、それは決意の証だ。
彼女は私が行く行かないにしろ、すでに決めているのだろう。
そして、私のことをよく分かってくれている。
私は、一人では寂しいのだ。
自らで踏み出す一歩を、極端に嫌がるのだ。

結局、いつもそうだ。
私は全てに対して受動的なのだ。
だけど、それが自分であると認めたくなかった。
諦めたくなかった。
だから今こそ、自分から前へ進もう。
後ろを押されるのではなく、一緒に。

「うんわかった。行こう」\\
私は立ち上がった。\\
「ありがとう、カナン」\\
「そんなことないよ」\\
私たちは荷物を持って、廊下を出た。
三年の教室、正確に言えば、E科の学科棟は私たちの教室がある建物からは離れた場所にある。
外にでる必要があるのだ。
夕日が眩しい、学科棟に繋がる廊下を歩いていく。\\
「でも、まだ決まったわけじゃないし。そんな顔しなくてもいいよ」\\
「そんなに酷い顔なの?」\\
セレナに言われて、窓ガラスに反射する、僅かな私の姿に目を細める。\\
「ほら、もう真っ青」\\
ほら、とセレナは私の前にたった。
はっ、と思わず声が出た。\\
「これで赤くなるでしょ」\\
セレナの手が私の頬を覆っていた。\\
「ふふ、カナンのお肌スベスベ」\\
「やめてよ、くすぐったい」\\
「でも本当だもん。羨ましいなあ」\\
なんだか、心が穏やかさを取り戻していく。
やはりセレナは、私のことをよく分かっている。
軽くなった心は、足取りにも現れ、私たちは前を向きながら、進んでいった。

\subsection*{(2)}

学科の違うこともあって、教室にたどり着くまでに、かなり迷ってしまった。
空気が違うのだ。
まだ一年生の私たちが、この学校に三年近くすでにいる人間たちの領域に、
踏み込むためには、それ相当の勇気が必要だった。
気付けばすでに午後五時近く。
ようやく、目的地にたどり着いた。

薄暗い教室。

誰もいない。

「来てくれたんだね」\\
あの声だ。
でも、姿は見えない。\\
「どこにいるの?」\\
私は聞いた。\\
「ここよ」\\
女の声。
全くの部外者。\\
「カナン、あそこ」\\
セレナの指差す方向を見る。
そこには、廊下の壁にもたれかかった女性と、その肩に乗った、彼がいた。\\
「あっ」\\
それは、朝の彼女だった。\\
「ああ、貴女。朝の子ね。
あのときはごめんなさい」\\
いいえ、と頭を少し下げる。\\
「それにしても、あなた達を待つのは退屈だった。
できればもう少し早くしてもらいたかったけど」\\
「道に迷ったんで、すいません」\\
「そう、一年生なら無理もないか。
そもそも他学科なら尚更かもね」\\
彼女の印象は、今朝のそれに比べて、ずいぶんと穏やかに見える。
受け入れる、とすでに語っているようだった。\\
「ほら、いつまでも突っ立てないで、中に入りなさい。
いろいろ説明することがあるから」\\
はい、と私たちは教室に入っていった。\\
「てきとうに座って」\\
机の質感が違う。
それだけどそわそわする。
私とセレナは並んで、彼女は私たちの前に座った。
黒板に向かう私たちの前に、前の席の椅子を反転させて、足を大胆に組んでいる。
かっこよかった。\\
「それじゃあ、まず自己紹介かな。
私は\ruby{架谷|彩芽}{ハサタニ|アヤメ}。そっちは?」\\
彼女\scalebox{3}[1]{―}アヤメは、セレナに目線をやる。\\
「時国瀬玲奈です。こっちは\scalebox{3}[1]{―}」\\
「詠華南です。よろしくおねがいします」\\
アヤメは笑った。\\
「カナン、でいいのかな。そんなに堅苦しくしないでもいいよ。
今のうちに言っておくけど、私のことはどう呼んでもいいから。
ただし、私だと分かるように」\\
はい、と私たち二人は返事をした。\\
「それじゃあ、肝心なところを説明しよう」\\
アヤメは、肩に目を配る。\\
「こんにちは! 二人ともよく来てくれたね。
僕の名前は\scalebox{3}[1]{―}アオタ。
といっても、この名前はアヤメがつけた名前なんだけど」\\
「え、どうして?」\\
「\kenten[size=1, kenten=﹅]{あお}い\kenten[size=1, kenten=﹅]{た}ま、だからアオタ。
たいしたひねりもないわよ」\\
「でも、こういうのってだいたい固有の名前があるんじゃ」\\
「それがないんだよね。僕たちはとても特殊な存在なんだ。
\scalebox{3}[1]{―}君たちはガイア理論って知っているかな。
簡単に言えば、この地球には意思、まあ正確に言えばこれは正しくないけれど、
そういったものがあるという考え。
この星が一つの生命体だという考えだと言ってもいい。
僕たちはいわば、この星の代弁者。
それがいま、この地上で最も知性的な君たち人類の共通基盤を依り代に、
ここで会話可能なエージェントとして存在しているのが僕のおそらくの現状だと思う」\\
大したことでもないというように、
すらすらと情報量の過密な文章を垂れ流されても、戸惑うだけだ。
それでもとりあえず、私は語尾に着目した。\\
「だと思うって、ずいぶんと曖昧な言い方だと思うけど」\\
「その指摘はご尤もだね」\\
アオタはあっさりと認めた。\\
「僕たちは神様なんてものじゃない。
もちろん森羅万象全ての真理を知っているわけでもない。
正直に言うと、僕たちの認識する知識は人間に依存しているし、完全に君たちと同等だよ。
だから、君たちにわからないことは、僕たちにもわからない。
全ての思考は君たち人類が居なければ成り立たないし、
だからこそ君たちとなんら遜色なく会話することが実現できている。
裏を返せば、それ以上は不可能だってことだね。
結局、自分自身が客観的に何者であるかは、言及不可能なんだよ。
まあ、僕が現状の科学知識から大きく乖離した存在であることは自認しているけどね」\\
釈然としないことに変わりはないが、これ以上の追求が無意味であることはわかった。\\
「まあアオタのことは気にしないで。大切なことは他にある。ほら、説明して」\\
アオタをつつくアヤメ。\\
「そうだね。本題に入ろうか」\\
アオタは私たちを見つめた。
その粒のような目は、今では全てを飲み込む暗い穴だ。
私たちの意識を吸い取る。
凝視する。\\
「今日から君たちは『狩人』になる。
そしてそれには必ず危険が伴う。まずはそれを理解してほしい」\\
彼は神妙に語った。\\
「狩人? なにそれ」\\
セレナはその単語にいまいちな反応を示している。
きっとあまり意味を分かっていないのだ。
普段使う言葉でもないから。
私は聞いた。\\
「狩人って、何かを狩るってことでしょ。
さっき言っていた悪夢、だったっけ」\\
「そう。狩人は悪夢を狩る。
戦うと言ってもいいけど、形容するなら狩るという言葉が適切だね。
君たちは夜に、狩人となって悪夢を狩る。
悪夢というのは人の心を食いつぶす、悪性腫瘍のようなもので、
人類の集合無意識上に、時代を問わず絶えず存在し、人を脅かしてきた。
その行動原理は不明だけど、およそ知性的とも言えない。
おそらく、単なる自己保存のプログラムに過ぎないだろうね」\\
「つまり、獣のようなヤツってことよ」\\
「まあ狩りと言っている理由の一つがそれなんだけどね。
でも、奴らのいる場所が問題なんだ。
さっきも言ったけど、狩人になれる人間には適正が必要なんだ。
人の心の奥深くに侵入できる強固な自我が。
悪夢の生息領域は、人の集合無意識が現実世界に部分的に重なったもので、
普通の人間には認知することは出来ないし、干渉もできない。
けれど奴らは人の心理的な部分に依拠しているから、一方的に干渉ができるんだ。
君たちも夢を見たように、悪夢は他者のイメージに侵入して、自分の縄張りにおびき寄せる。
これは防ぎようがない。
防ぐとしたら、それは人をやめないといけないからね。
だけど僕たちは、人の中にもこの領域に存在できる個体がいるという事実を確認した。
だったら、ほかに手段はないよね」\\
「ちょっとまって。アオタはさ、星を守ってほしいんだよね。
だったら、どうして人間を守るってことになるの。
私たち人間が地球をめちゃくちゃにしていることだって、いっぱいあるのに」\\
「セレナ、確かに人間は地球環境を自ら調整する技術を得た。
でもそれは必ずしも悪いことではないよね。人間以外も、多くの生物が自らの都合の良く
環境を改変しているよ。ただ人と比べて、規模がごく僅かであるというだけで。
でも人間は多くを変えられる。
今はまだ、それは適切に管理されているし、人は自らを自制できている。
それがもし、悪夢たちが自らの生存のために、人を無造作に増やし、
その心を乱せばなにが起こるだろうか。
予想は多岐にわたるけど、おそらく良いものにはならないだろう。
僕たちはその可能性を潰すことこそが、重要だと考えているんだ」\\
「悪い結果をもたらす可能性を潰す。簡単なことよ。
でもそれだけじゃない。悪夢は今も誰かを誘っている。
自らのもとに、或いは死に。
心の弱い人や、感受性の高い人に被害者は多いの。
そして彼らは大抵、現実に居場所を失っている。
あなた達はまた別だろうけど、そういった人間が心の中にまで、
最後まで自分だけの砦だった心にまで入り込まれたらどうなるか」\\
「苦しい」\\
私がそういった時、私の心はあの悪夢の苦しみを再生していたのだろう。
無意識に、とっさに出たのだ。\\
「そう、それも自覚のない苦しみよ。
だったら、大局的なことはひとまず置いといて、
まずはそんあ人たちを助けているんだと思うべきね。
私たちは結局一人の人間で、世界を見渡すことは出来ない。
だからこそ、一つ一つの事物に意識を向けるべきね。
それに、ややこしいことは全部アオタがやってくれるから」\\
アヤメはアオタを叩いた。
ゴム毬のように弾んで、彼は冗談めいて怒っている。\\
「とにかく、難しいことはすぐに理解する必要はないよ。
君たちは誰かを助けたくて、ここに来たんだよね。
顔も、それこそ名前もわからないような他人を、
助けようと思ったそのこころを大切にしてほしい」\\
「ねえアオタ、聞いてもいい?」\\
「どうしたんだい、セレナ」\\
「私、本当に戦えるかな。なんだか怖くなってきたの」\\
私は驚いた。セレナがそんな弱音を吐くなんて。
いや、彼女だからだろう。物事を深く考える彼女は、
私よりもよほど現実的に物事を捉えているんだ。
その素晴らしさも、危険さも。
彼女は慎重に、天秤にかけているのだ。\\
「大丈夫だよ。君たちには力があるし、
まず大きなアドバンテージとして、その頭があるよね」\\
「そうよ、セレナ。私たちは考える剣よ。弱いわけない」\\
アヤメはセレナの肩に手をおいた。
そして私の方にも目を向ける。
今朝からのイメージが一新されていく。
私は彼女に対して、なにか強い心の強靭さと、優しさを感じた。

「君たちが不安がるのもよく分かるよ。
でも安心して、僕たちが精一杯サポートするから。
それに、死の可能性は殆どないよ。それは保証する」\\
「はい、いきなりってことはないから。
ちゃんと教えてあげる。
多分、今夜からでしょ」\\
「そうだね。決意が揺らぐ前に」\\
「今夜って、どうすればいいんですか」\\
「アヤメもセレナも、今日は早く寝てね」\\
「寝る? 家で寝てるんですか?」\\
「そう、眠っていて。遅くても十時までには、寝ておいた方がいい」\\
「集合場所とか、決めなくても\scalebox{3}[1]{―}」\\
「そんなものはないの。ただいつも通り、寝てね」\\
「わかりました」

「それと……」\\
アヤメはアオタに促すような仕草をする。\\
「もう一つ君たちに重要なことがあるんだ。
君たちは僕と契約をする。
契約は双方性を持っている。
つまり、君たちが狩りを行う対価を、僕たちは用意しなければならないということだね」\\
「対価? お金か何かなの」\\
茶化すような言葉。\\
「そんなわけないでしょ、セレナ」\\
「あはは。そうだよねーいくらなんでもそんな俗っぽい\scalebox{3}[1]{―}」\\
「構わないよ。お金でも物でも、実現できる可能な限りを尽くすよ」\\
私は耳を疑った。
有り体の場合、と言っても創作物上の話でしかないが、
こういった行為は、無償の奉仕ではないのだろうか。
或いは、奇跡のような人のみには有り余る神秘。
世俗から乖離した、純粋無垢な願いの成就なのではないだろうか。\\
「本当に? じゃあ億万長者になりたいとか、
あの店の服全部買い占めたいとかも?」\\
「実現可能性を考慮する必要はあるけど、基本的にどんな願いでも
僕たちは受け入れるよ。
もちろん、ある程度はそれに似合った働きは必要だけどね。
一人の人間が、その上に何十億もの人間の未来を背負う苦しみは大変だろうし、
そんな仕事をタダでしろ、という方が酷な話だよね」\\
「でも願いがなかったら」\\
私は聞いた。
私ははじめから、何も望まないつもりだった。
セレナもそうだろう。
ただあの時の恐怖から、突き動かされただけなのに。\\
「人には必ず、願望がある。その大小は関係ない。
どんな些細なことでもいい。これは、僕たちの気持ちなんだ」\\
「めっちゃいいやつじゃん! アンタ!」\\
セレナはアオタを掴んで持ち上げる。
小動物を可愛がるように、頭をくしゃくしゃに撫でる。


それじゃあ、と帰るように促された。
駅まで一緒に歩いて、アヤメは上りで私たちは下りの電車に乗る。

金沢で降りる。

じゃあね、と別れた。

また暗い道。
私は夕暮れの会話を思い起こした。
名前も知らない誰かを死の淵から助け出す。
その響きは心地よく、私の心に共鳴する。
その分の反響もまた、私の心に影響する。
はたして、私に務まるのだろうか。
私は、何を望めばいいのだろうか。










\end{document}
