\documentclass[../IHMain]{subfiles}

\begin{document}
    
\chapter{狩人、\\その使命}
\section{\tt Where is my dream?}
\subsection*{(1)}
「行ってきます」\\
玄関を出た。
昨日と同じ寒い朝。
いつもより早めに家を出た。
コンビニで昼ごはんを買うためだ。
お母さんは昨日、どうやらかなり遅くまで起きていたらしく、
お弁当を作る気力がなかったのだろう。
いつまでも起きてこない、お母さんとお姉ちゃんをよそに、お父さんはいつもどおり
早朝に家を出ていた。

駅のコンビニに入る。
サラダサンドウィッチと、鮭おにぎり。
飲み物は水筒にお茶を入れているので、買う必要はない。
忙しい時間帯だ。
レジには多くの人が並んでいて、スタッフの人は忙しく働いている。
バーコードリーダーの音がより一層、その印象をくっきりとした形にしている。\\
「お待ちの方どうぞ」\\
隣のレジに、商品を置く。
お金を払って、レジ袋に入れられたおにぎりやパンを、そのままバッグに入れながら、
コンビニを出た。

改札を通って、まだ来ていない電車を待った。

\subsection*{(2)}
駅を出て、学校へ向かって歩いていた。
道すがら、ふと昨日の出来事が頭をよぎる。
あの後、意識的にこのことについて考えることを避けていたのだけれど、やはり気になってしまう。
夢のこともそうだが、何よりも、あの黒い物体と、それと共に現れた人間。
まじまじと見つめていた訳ではないが、その時の印象を思い起こしてみる。
眉を潜めた顔しか、記憶には残っていないが、それでも感じたことがある。
あの顔はかなり若かった気がする。
そう思えば、彼女の容姿が私たちの一つか二つ上の人たちによくいるような感じがした。
それに髪型も、かなり特徴的だった。
片方を伸ばしたショートボブというべきだろうか。
そしてなぜ、あの場所に現れたのか。
あそこに人が立ち入ることが、あるのだろうか。
あるとしても大人だろう。それも、かなり歳をとった人。
だとすれば、あれは同じ学校の人間。
\scalebox{3}[1]{―}いやそもそも、アレが現実であるという前提で話を進めるのもどうかと思う。

考えれば考える程に、私の頭はこんがらがる。

思考に頭が重くなって、うつむいたままに道を歩いていると、後ろから何かが肩にぶつかった。\\
「すみません」\\
通り過ぎようとする女性からの声だった。
ずいぶんと早歩きで力強い。
颯爽と過ぎていく彼女に、私は注目した。
横顔が見える。
なんだか見覚えのある顔、あの時の人に似ている。
もちろん、口元は隠されていたから確かめようがないが、
その鋭い目元や、片側だけを伸ばしたアシンメトリーな髪型は、どこか印象深い。
そうに違いない。
きっと彼女が、あの時の女性だ。
私は抑えきれなかった。\\
「あの、あなた、あの時の人ですよね」\\
自分に声がかけられたと気づいていないのか、それとも無視しているのだろうか、訝しんだりすら
ない、全くの無反応。\\
「すみません。あなたですよね! 私たちを助けてくれたの」\\
流石に気づいたのだろう、彼女は振り向いた。\\
「あ、あの。すみません」\\
「あなた誰。知らない人。
申し訳ないけど、人違いじゃないの」\\
冷たい声。
鬱陶しがっているのは明白だった。
だけどその顔は、はっきりと断言できる。
あの顔とそっくりだ。
怪訝な、けれど攻撃的ではない目つき。
しかし人そのものを否定するような、
はっきりとした拒絶を滲ませたそれに、私はそれ以上踏み込むことが出来なかった。

彼女と私の住む世界は全く違うのだと言わんばかりに、
彼女は私をいとも簡単に置き去りにして、瞬く間に遥か向こうへと去っていった。

\subsection*{(3)}
「あ、カナン」\\
休み時間。
トイレを済ませていたら、偶然にもセレナと一緒になった。
寝不足気味なセレナの目。
彼女も昨日の出来事に頭を悩ませていることは、すぐに分かる。
手を洗いながら、私たちは話し合った。\\
「私もうわかんないよ。カナンはさ、覚えてるの? 昨日のこと」\\
「覚えてるよ。やっぱり、本当にあったことなのかな、アレって」\\
「えーでも、ありえなくない? あんなにさ、アニメとか漫画みたいに人が
ビューンって飛んでくるとか、おかしいよね。物理的にありえない」\\
妙に堅苦しい語句を使うのは、セレナがその事柄について深く考えすぎている証拠だ。
確かに、彼女の言う通りだと思う。
あの真っ黒い玉はよしとしても、あの彼女の運動はありえない。
私たちの遥か頭上を飛び越えて、しかもかなりの時間地面に落ちず、剣を振り続けることなんて、
現実にできるはずがない。\\
「やっぱさ、私たちがどうかしてたんだよ。
集団ヒステリーってやつ、なのかも。化学薬品とかさ、そういうやつで幻覚をみてさ。
……それにしてもリアルすぎるってのもあるし、直前の夢とかもあるけど、
そうとしか言えないよね」\\
「ヒステリーってもっと病的なんじゃないの?」\\
「うーん、そんなこと言っても\scalebox{3}[1]{―}
ああもうわかんない。わかんなよなんにも。
ああ、イライラ」\scalebox{3}[1]{―}する。そう彼女がいいかけた時、
どこからか、遮るように声が聞こえた。\\
「そんなこと、簡単じゃないか。ただあるがままを受け入れる。
どうして人は、それが簡単にできないんだろう?」\\
声も子供っぽく、しかし憎たらしくないほど知性さを帯びている。
どこから聞こえるのだろうか。
洗面台の端。
何かが見える。
影のようだ。
いや、光だ。
私は目を疑った。
本当に、私の頭が狂ってしまったのかと、心配になった。
薬物の乱用者が、その禁断症状に見る光景、教科書で知ったそれと、
なんな変わりないものが見えたからだ。

落ち着いた青色の光球。
それに、まるでマスコットの様な極端にデフォルメされた目、
例えるならスマイルのマークといえばいいだろうか。
けれど口に喩えるパーツはない。
目だけだ。
それでも、それがただのぬいぐるみだといえば、ただの愛らしいものだろう。
だけどそれは、ゆらゆらと揺らめいて、動いている。
しかも、言葉を喋ったのだ。

「ねえ、今の私だけじゃないよね」\\
私はセレナに寄った。\\
「うん。やっぱ、カナンにも見えてるよね」\\
「見えてる」\\
「どんな形?」\\
「青くて、丸っこい」\\
「だよね」\\
これ以上、なんと返せばいいのか行き詰まってしまった。

混乱の静けさの中に、チャイムの音が響く。\\
「あ、時間だ」\\
白々しいセレナの言葉。\\
「遅れちゃう、いこうカナン」\\
「うん」\\
私たちは、見なかったふりをして、トイレから出ていこうとする。
その身振りを見て、焦るような表情
\scalebox{3}[1]{―}おそらくはほんの少し目の形が変わっただけだろうが\scalebox{3}[1]{―}
を見せる青い玉。\\
「ちょっとまってよ。僕は君たちに話したいことがあって\scalebox{3}[1]{―}」\\
何かを言いかけていたが、もうどうでもよかった。\\
「ほら、早く!」\\
「待ってよセレナ!」\\

「詠、遅刻」\\
残念だが、授業には間に合わなかった。

\subsection*{(4)}
「ねえ、なんなのコイツ」\\
「私が聞きたいよ」\\
私たちは食堂の端っこに、小ぢんまりとしていた。
昼休みだし、人は多い。
他人からの目を恐れて、私たちは交互にあたりを監視していた。
どう見たって、傍からは頭のおかしな連中にしか見られないだろうからだ。

キョロキョロとした二人。
これだけでも十分に怪しいが、それでも仕方ない。
木を隠すには森のなかに。
人混みに紛れれば、注目されることもないだろう。
人が全く来ない場所で、かつ話していても気づかれないような場所は、私たちには思いつかなかった。
下手をすれば、私たちは有りもしない虚空か何かに話しかけている、ヤバイ奴になってしまうのだ。

その原因は一つしかない。
先程の青玉は、未だしつこく、私たちに付き纏ってくるのだ。\\
「だ、か、ら! アンタがどうこう言ってても、私たちには一切関係ないから。そうだよねカナン」\\
うん、と同意する。\\
「そんなこと言っても、僕らにも僕らなりのちゃんとした理由があるんだよ。
それを訳がわからないの一点張りで拒絶するのは、はっきり言って酷いことだと思うよ」\\
私たちの強硬な姿勢に、まるで高圧的な客に対して、辟易とする接客業務員の姿を重ね見てしまう。
けれど当たり前のことだ。
ただでさえ昨日の出来事で疲れ切った私たちに、それまたおかしな格好と、正反対なその言動を、
受け入れろというは酷な話だ。

ただ、相手もなんだかんだで引くことを知らないようだ。
かれこれ、押し問答は三十分以上続いていた。

「じゃあ、あなたって何を私たちに『お願い』したいの?」\\
これ以上の繰り返しに、意味を見いだせなかった私は、思い切ってソレに聞いてみた。\\
「聞いてくれるのかい?」\\
一転して、その声色は明るいものになった。\\
「カナン、付き合わなくてもいいよ」\\
「そうかもしれないけど、気になるでしょ。私たちに何があったとか。
このままでいても、どっちにしろ納得は出来ない。だったら、一度受け入れてみるのも
いいんじゃないのかなって」\\
口はつぐんだが、納得はしてないようだった。
だけど、セレナも私と同じ気持ちだろう。
ただ私よりも、自分の中に押し込めるのが上手なだけで。

「もう始めてもいいかな?」\\
私は黙って頷いた。\\
「僕たちはね、君たちに大切なお願いがあるんだ。
僕たちは、ある人達を探しているんだ。
とても重大で深刻な問題を解決できる、強い力を持った人たち。
才能を持っていると言ってもいい。
だけど、それは本人には気づけないし、僕たちもなかなか見出すことが出来ない。
いいかい、落ち着いて聞いてほしい。
\scalebox{3}[1]{―}僕たちはね、君たちにこの星を救ってほしいんだ」\\

……一瞬で体の気が抜け落ちていくような感覚がした。
彼の雄大な熱弁と違って、私たちの今の顔は正しく間抜け面だろう。
ソレ\scalebox{3}[1]{―}いや彼の言葉は、まるで童話のセリフだ。
現実にはそぐわない、空想の世界の言葉。
すべてがお膳立てされた道筋の、単なるきっかけに過ぎない文字列。
『世界を救ってほしい』その言葉のなんと幼稚なことだろうか。

セレナは、思わず笑いだしてしまった。
そんな彼女に、彼は不機嫌になったのだろう。
凹んだ目を、眉間に皺を寄せる仕草に見立てているらしい。\\
「君たちはもっと、物事の本質を見るべきだと思うよ。
見たんだろう。あの黒い化物を。
感じたんだろう? その時の恐怖を。
だとしたら、結び付けられるはずだ。
これは決して、笑い事じゃないんだ」\\
精一杯に語気を強くしたつもりなのだろう。
だがそれでも、あくまでも可愛げのあるただの、マスコット的発声だ。
しかしその懸命さは、私たちの関心を引くには十分なものだった。\\
「じゃあどうやってその、『星』を救うの。
宇宙人と戦う? 悪いやつを殺すの?」\\
セレナは、さっきまでの不満そうな顔つきと一変して、かなり興味津々だった。\\
「宇宙人じゃないよ。もちろん人間でもない。
もし君たちが僕たちと『契約』を結ぶなら、君たちは『悪夢』と戦ってもらうんだよ」\\
悪夢。なんとも抽象的な名称だ。
横目にセレナと見合いながら、首をかしげる。\\
「悪夢というのは、実は僕たちにもよく分かっていないものなんだけど\scalebox{3}[1]{―}」\\
「ちょっとまってよ」\\
セレナが会話を遮った。\\
「どうしてアンタにもわかんないような奴と、戦わないといけないの?」\\
それは私も同感だった。\\
「確かに、その意見は正しいと思う。けれど実害は発生しているんだ。
現に、君たちもそうだっただろう。
うなされる夢に侵され、妙に現実味のある夢想の末に、特定の場所に連れ出される。
君たちは自らの意思で行動したと思っているようだけど、あれは一種の捕食行為。
あのまま僕たちが助けなかったら、君たちは今頃死んでいただろうね」\\
死んでいたかもしれない。
その一言はかなり心にのしかかってきた。
そして同時に、不安も湧き上がる。\\
「それじゃあ、死んだ人もいるってことなの」\\
私は聞いた。
息を整えることを、促すような間。
彼は見えない口を厳かに開くよう、話を繋げる。\\
「そうだよ」\\
明らかに重たい声。鎮魂の重みだろうか。\\
「死者は多くいる。
防げるものもあれば、どうしようも出来ない場合もある。
人口に対して、常に悪夢に対抗できる人間の絶対数は極小で、
すべてを守り切ることは不可能なんだ。
だからこそ、可能な限り、君たちのような適正ある人間に、協力してほしい」\\
私たちの顔は暗くなっていた。
なぜだろう、それが現実なのかを一切疑わずに、死という単語に共鳴していた。\\
「ねえ、さっきアンタは私たちを助けたって言ってたよね。
だとしたら、あの時の女の人は、死んじゃうかもしれなかったのに、
私たちのために戦ってくれたの」\\
「それが事実だよ。でもね、大丈夫。
君たちが僕たちと契約をすれば、必ず力を得る事ができる。
その力さえあれば、立ち向かうことができる。
死ぬ可能性は十分に抑えられるし、死の淵にある人々を救うことができる」\\
高校生には、いや日々をただ惰性で過ごしている私たちには、おそらく一生叶うことのない
ことだろう。
誰かの為になる。
急にこんなものを、輝く宝石のような、
生きる意味を提示されれば、目が眩んでしまうのは当然だろう。
少なくとも私はそうだ。
ここに来てまた、自分の中から現実性が脱落した気がする。\\
「これがどれだけ、君たちの心を迷わすことなのか、僕たちは理解しているつもりだよ。
それでも、心に漂うものたちと向き合うことは、有意義なことだと思う。
もし君たちが、この願いを受け入れてくれるのなら、放課後、E科の三年生教室に寄ってみてくれないかな。
\scalebox{3}[1]{―}答えがどうであろうと、僕たちはそれを受け止めるよ」\\
時計を見れば、すでに12時50分。
もうすぐで昼休みは終わる。
気付けば、周りも人は疎らで、残っているのは私たちを除けば、片手で数えられるほどだった。\\
「まって」\\
セレナが叫んだ。\\
「アンタは何者なの」\\
「それは、君たちが決めることだよ。もし来てくれたのなら教えるよ。
でもそうじゃなかったら、全部忘れるといい。こんなこと、心に潜めておく必要なんてないからね。
その時は君たちの送りたい人生を送るべきだよ」\\
彼はそう言い残して、静かに消えていった。
無色に、溶け込んでいくように。\\

残された私たちは、ひとまずは現実に戻ることにした。
時間はまだある。
急いで弁当箱を片付けたり、皿を下げたりして、食堂を出ていった。
その一切は無言のままだった。
互いに、自分の心の奥底に沈殿する何かを、必死に見透かそうとしているのだろうか。

自分でもわからないが、とにかく、普通ではなかった。

\section{}
\subsection*{(1)}
陽は傾き、散乱した赤い光が横から指す。
廊下に人は見えず、電灯は飛び飛びに着けられている。
節電のためだろう。
まだ四時過ぎだ。
セレナが教室に来てくれた。
掃除が終わって、上げられた椅子を机から降ろして、私たちは話し合った。
私は椅子に、セレナは机に。
もちろん、昼のことについてだ。\\
「どうするの、セレナ」\\
「どうしよう」\\
「そうだよね。なんだか、しっくりこない。
星を守るって、なんか宙ぶらりんだし、そもそも私にそんなことできるわけない。
でも、あの子が言ってたみたいに、私たちがせめて、せめて誰かの役に立つなら、
それを拒否するのも、自分は許せない」\\
「言ってたもんね、死ぬことはないって」\\
「たぶん」\\
「だったら、カナンの言う通りだよ。
黙って知らんぷりするのは、許せない。
私たちは助けられたんだよね。
だったら、私たちも誰かを助けたいよ。
できるかどうかは、別だけど」\\
もはや二人とも、疑うことはなかった。
すべてを事実として、選択すべきを決めかねている。
心の整理はつかないが、なにか私の、
本質的なものがざわめいているのは確かだった。
たましいなのだろうか。
少なくとも、理性的な刺激ではなかった。\\
「行ってみようよ。三年の教室だっけ。
カナンと一緒なら、私行くよ」\\
セレナの目は、どこまでも澄んでいた。
彼女の瞳は、少し青みがかっている。
吸い込まれるような、それは決意の証だ。
彼女は私が行く行かないにしろ、すでに決めているのだろう。
そして、私のことをよく分かってくれている。
私は、一人では寂しいのだ。
自らで踏み出す一歩を、極端に嫌がるのだ。

結局、いつもそうだ。
私は全てに対して受動的なのだ。
だけど、それが自分であると認めたくなかった。
諦めたくなかった。
だから今こそ、自分から前へ進もう。
後ろを押されるのではなく、一緒に。

「うんわかった。行こう」\\
私は立ち上がった。\\
「ありがとう、カナン」\\
「そんなことないよ」\\
私たちは荷物を持って、廊下を出た。
三年の教室、正確に言えば、E科の学科棟は私たちの教室がある建物からは離れた場所にある。
外にでる必要があるのだ。
夕日が眩しい、学科棟に繋がる廊下を歩いていく。\\
「でも、まだ決まったわけじゃないし。そんな顔しなくてもいいよ」\\
「そんなに酷い顔なの?」\\
セレナに言われて、窓ガラスに反射する、僅かな私の姿に目を細める。\\
「ほら、もう真っ青」\\
ほら、とセレナは私の前にたった。
はっ、と思わず声が出た。\\
「これで赤くなるでしょ」\\
セレナの手が私の頬を覆っていた。\\
「ふふ、カナンのお肌スベスベ」\\
「やめてよ、くすぐったい」\\
「でも本当だもん。羨ましいなあ」\\
なんだか、心が穏やかさを取り戻していく。
やはりセレナは、私のことをよく分かっている。
軽くなった心は、足取りにも現れ、私たちは前を向きながら、進んでいった。

\subsection*{(2)}




\end{document}
