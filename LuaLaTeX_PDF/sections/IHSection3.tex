\documentclass[../IHMain]{subfiles}

\begin{document}

\chapter{双極の爪痕}
\section{}
私たちの生活は\scalebox{3}[1]{―}それほど変わらなかった。\\
「もうすぐ春休みですねえ」\\
私たち二人と、アヤメの仲はすっかりよくなって、
いまはこうして、一緒にお昼ご飯を食べられるぐらいになっている。\\
「そうね。まだ貴方達は一年だからわからないと思うけど、
春休みは死ぬほど退屈よ」\\
「そうなんですか? 春休みなんてすぐ終わる、って感覚しかないんですけど」\\
「そうそう、中学校の時なんてそうだったよねー」\\
「ここの春休みは一ヶ月以上あるの」\\
「一ヶ月!」\\
「そうよ。それに、宿題も多分一年生のときは大して出ないから、
本格的にやることないわよ」\\
「自分のやるべきことを、見つけろってことですかね」\\
「一応、自主学習しろとか言われるけど、私はやらない」\\
「そうですよ、勉強なんて面倒くさいだけ」\\
「そんなこと言ってるか、セレナはテストで点取れないんだよ」\\
「赤じゃないから、いいの」\\
「セレナ、油断大敵よ。いまはまだいいかもしれないけど、二年はもっと厳しくなるわよ。
実際に落ちるときは落ちるから。取れる点は取っておくべきよ」\\
「うっ、はい」\\
私たちの会話は、なんら普通の人間と変わらない、
他愛のない日常の風景だった。

初めての狩り、あれからすでに二ヶ月ほど経ったが、
その後の狩りは両手で数え切れるほどしかなく、
どれもまた最初の獲物と多少大小する程度のものを相手にするだけで、
私たちが出会った一番最初の悪夢、あの恐怖の塊に比べれば、
可愛いものだった。

だが、経験は溜まっていった。
私たちはアヤメから、狩りの手ほどきを受け、それなりの狩人になれたと思っている。
睡眠学習とはよく言ったものだが、本当に、狩りの夜に学んだことは、たとえそれが一度だけで
それも呟き程度のものであっても、一語一句覚えているのだ。
きっとこの夜に勉強できたら、どんなにいいだろうか。
まあ、そんな余裕はないのだが。

しかし、狩りの夜から目覚めても、不思議と疲労はないのだ。
私はよく勉強疲れをするタイプで、暗記する時に顕著なのだけれど、
狩りの学習に、倦怠感を伴ったことは一度もない。
むしろ、目覚めはよく、頭の中が綺麗さっぱり洗い流されている。
だるさはなく。
それは生理のときもそうだった。
精神的な満足感によるものだろうか。
それもまた、私たちが狩りの夜に馴染んでいくのに、大いに役立った。


\end{document}