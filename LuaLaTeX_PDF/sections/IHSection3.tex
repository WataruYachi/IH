\documentclass[../IHMain]{subfiles}

\begin{document}

\chapter{双極の爪痕}
\section{}
\subsection*{(1)}
私たちの生活は\scalebox{3}[1]{―}それほど変わらなかった。\\
「もうすぐ春休みですねえ」\\
私たち二人と、アヤメの仲はすっかりよくなって、
いまはこうして、一緒にお昼ご飯を食べられるぐらいになっている。\\
「そうね。まだあなた達は一年だからわからないと思うけど、
春休みっていう時間ほど、死ぬほど退屈な時間なんてないの」\\
「そうなんですか? 春休みなんてすぐ終わる、って感覚しかないんですけど」\\
「そうそう、中学校の時なんてそうだったよねー」\\
「ここの春休みは一ヶ月以上あるの」\\
「一ヶ月!」\\
「そうよ。それに、宿題も多分一年生のときは大して出ないから、本格的にやることないわよ」\\
「自分のやるべきことを、見つけろってことですかね」\\
「一応、自主学習しろとか言われるけど、私はやらなかったわね」\\
「宿題もないから、私もそうなっちゃいそう」\\
「そんなこと言ってるから、セレナはテストで点取れないんだよ」\\
「赤じゃないから、いいの」\\
「セレナ、油断大敵よ。いまはまだいいかもしれないけど、二年はもっと厳しくなるわよ。
実際に落ちるときは落ちるから。取れる点は取っておくべきよ。
私の周りには、それで後悔した人間が幾らでもいるから」\\
「うっ、はい」\\
私たちの会話は、なんら普通の人間と変わらない、
他愛のない日常の風景だった。

初めての狩り、あれからすでに二ヶ月ほど経ったが、
その後の狩りは両手で数え切れるほどしかなく、
どれもまた最初の獲物と多少大小する程度のものを相手にするだけで、
私たちが出会った一番最初の悪夢、あの恐怖の塊に比べれば、
可愛いものだった。

だが、経験は溜まっていった。
私たちはアヤメから狩りの手ほどきを受け、今やそれなりの狩人になれたと思っている。
睡眠学習とはよく言ったものだが、本当に、狩りの夜に学んだことは、たとえそれが一度だけで
それも呟き程度のものであっても、一語一句覚えているのだ。
きっとこの夜に勉強できたら、どんなにいいだろうか。
まあ、そんな余裕はないのだが。

しかし、狩りの夜から目覚めても、不思議と疲労はない。
私はよく勉強疲れをするタイプで、暗記する時には顕著なのだけれど、
狩りの学習に、倦怠感を伴ったことは一度もない。
むしろ、目覚めはよく、頭の中が綺麗さっぱり洗い流されている。
だるさはなく。
それは生理のときもそうだった。
精神的な満足感によるものだろうか。
それもまた、私たちが狩りの夜に馴染んでいくのに、大いに役立った。

今や狩りの夜は、現実の私たちに大きな影響を持つ。
自信が伴うのだ。
無力感に打ちひしがれて、未熟な自分を抱えていたあの時と、
夜に紛れ人々を救い、だがそれは誰にも知られずに、
しかしそれでいいのだと満足するヒロイズムに酔う私たちは、
姿勢が全く違う。

華やかさを求めるのも、その一つなのだろうか。\\
「そういえば、アヤメさんって次から私服でしょう?」\\
セレナは特にそうだった。\\
「そうだけど、セレナには関係ないでしょ」\\
「ありますよ! アヤメさんなんかそういうのガサツっぽいから」\\
「そう? 私ってそう見えるの?」\\
私に求められても、こう答えるしかない。\\
「はい……見えます」\\
待ち合わせ場所に学校の体操服であるジャージ姿で現れたときには、
さすがに彼女の装いへの無頓着さを実感しなければならなかった。\\
「そう……まあ確かに、服なんて買ったことないからね」\\
「自分で買ったことないんですか?」\\
「だって着ないから。外に出るのは学校かバイトだけだから」\\
「これは、重症だなあ……」\\
「でも流石に、四年生で制服は」\\
「そうよね、ただのコスプレよね」\\
「じゃあ! 一緒に買いに行きませんか?」\\
「いつ?」\\
「春休みのどこかで」\\
「買い物に行く服がないんだけど」\\
「春休み中なら、まだ制服でもオーケーですよ!」\\
「ええ、セレナそれは……」\\
「それもそうね。まだ『三年生』の春休みだものね。
いいわ、どうせみんなやることないんでしょう。
付き合いなさい」\\
「はい!」
「わかりました」\\
「断っておくけど、私のセンスは壊滅的よ」\\
「……わかってます」\\
「任せてくださよ! 私がしっかり、コーディネートさせてもらいます!」\\
後は、具体的な日にちを打ち合わせて、それからはまた違う話題に移り変わっていった。

\subsection*{(2)}
彼女の言う通りに、春休みは瞬く間に過ぎ去っていった。
冷蔵庫の横に貼ってある、新聞販売店の配るカレンダーを見ながら、
すでにこの長い休みの半分近くを消費した事実に気づいた。
三月の半ばだった。

それに、何かに打ち込んだりもなく、そもそもの勉強すらままならず、
怠惰な一日を過ごしていくだけの現状。
狩りの夜も、めっきりで刺激に欠ける。

眠たい時に眠り、起きたい時に起きる生活は、
いずれ社会からの孤立を自覚させる。
椅子にだらしなく座って、ただぼーっと天井を見上げるとき、
私はそう感じざるを得ない。

お昼を食べようと、二階に降りたが特に動いてもいないので、
用意された分だけのごはんは必要ないとしか言えない。
それでも残すのはどうかと思うし、いつも食べ切るが
そのせいで最近は若干太りつつある。
尤も、量を減らせと言えばいいのだが、それすらも面倒なのだ。
受動でいる日々に慣れきって、いつの間にか行動すること自体を忌避するようになる。
今の私の日課は、動画を見ることだけだ。
ただ受け身のまま。
楽しみだったゲームすらも疲労のもとになってしまう。

そんな日々が続くと、忙しい忙しいと嘆いていた、学校が待ち遠しい。
勉強がしたいわけではない、友達と会いたいのだ。
誰かとまともな会話をしたいのだ。

あ、そうだ。

明日が約束の日であることを思い出した。
ちょうどいい。
久しぶりに友達と会えることに、私の心は踊る。

早急だろうが、私は明日へ向けた準備を始めた。

着ていく服を選んで、バッグを用意して、財布を確認したり、
久々の労働だ。

やはり何かに向かって運動するということは、
人生に必要なことであると、私は改めて実感した。

\subsection*{(3)}
「あ、こっちこっち!」\\
セレナが遠くから手を振ってくれた。
人混みは激しくて、なかなかたどり着けない。
ひと目をそれなりに気にして、わざわざ平日に設定したのだけれど、
大して効果はなかったようだ。
学校でもないのに制服だなんて。
恥ずかしい。
流れをかき分けて、私はやっとたどり着いた。\\
「おはようございます」\\
「おはようカナン!」\\
「おはよう」\\
「みんな早いですね」\\
「電車の時間がこれしかないから」\\
「私も、バスがね」\\
「そうなんだ」\\
待ち合わせの時間は二十分ほど後だったが、全員揃った。
私は歩いて来れる距離、というか駅前なので、ほぼ駅と等しい距離だったのだが、
こういう場合はいつも、時間よりも早く来てしまう。
時間の見積もりがいつもより長くなってしまうのだ。
幸い今回は二人とも居てくれたので、一人で待ちぼうけする必要はなかった。

以前私はアヤメが自分で服を買いに行ったことがない、ということに対して、
若干の驚きを感じていた。
けれど、よくよく考えると私も自分の意思で、服を買いに行こうとなったことはあまり記憶にない。
ショッピングモールに服を買う目的で、たった一人で自発的に入ったことは一度もないのだ。
この中で慣れているのは、セレナだけ。
私とアヤメは、両親の後ろに縮こまって付いていく子供のように、セレナの後を追った。
平日だからか、想像していたよりも人は少ないが、それでも賑わっていると言えるだろう。
ここで降りますよ、とセレナが言った。
看板を読むに、ヤング・レディース向けの階に到着したらしい。
エスカレーターに乗っていたが、私たちのように制服でいる女性は一人も見かけなかった。

「あ、ここ」\\
セレナが指を指した。\\
「私いつもここで服を買ってるんですよ」\\
その店で売っている服は、カラフルなもので、
いかにもおしゃれ系な女性たちに人気のようなものだ。\\
「はあ、こういうのってアヤメさんに似合いますか?」\\
私は本心だった。\\
「似合うって。ほらアヤメさん、何か好きなの選んでみてくださいよ」\\
「う、うん」\\
明らかに困惑している。
彼女でも、自分にこの系統の服装が似合わないことを理解しているようだった。

「似合わない」\\
結局、この店で探すのは諦めることにした。\\
「いいとおもったんだけどなぁ」\\
さすがのセレナも難儀しているようだった。
次の店はアヤメが選ぶことになった。
私とセレナは彼女にただついていくだけ。\\
「ここでいいでしょ」\\
彼女が見つけたのは、言い方が悪いかもしれないが、
まあ当たり障りのない大手の売り場だ。\\
「ここですか?」\\
どうしてこんな場所で、と言いたげなセレナ。\\
「ほら、アヤメさんがここって言うんだから」\\
私は突っ立つ不動のセレナを押す。\\
「これどうかな」\\
「地味、じゃないですか」\\
「私はいいと思いますよ」\\
「だめ。絶対ダメです。
そんなんじゃ婚期を逃しますよ!」\\
「まだ二十歳ですらないのに、そんなこと考える暇はないわよ」\\
じゃあこれで、と彼女はレジに向かおうとする。\\
「ちょっと、試着しないんですか?」\\
「別にいいかなって」\\
「それこそ本当にだめですよ。ちゃんと来てみないと」\\
「サイズは合ってるし、既製品だからどれも変わらないわよ」\\
私たちがちょっかいを出したばかりに、彼女は躍起になったのだろうか。
手にとった服の、色違いや柄違いを適当に選んで、会計を通してしまった。\\
「本当にそれでいいんですか?」\\
「いいの。二着あればローテーションには十分でしょ。三着あるから、
洗濯が間に合わなくてもバックアップがある」\\
「無地とストライプと水玉って……」\\
「十分よ。読めない英文を着飾ってるより、よっぽどマシ」\\
「……そうですか」\\
返す言葉はなかったが、まあ、彼女は最低限地肌を隠せればそれでいいのだろう。
それに、どれもまあまあ似合いそうだ。
彼女の端正な顔が、逆に強調されるかもしれない。

「よし、じゃあお昼食べましょう」\\
「もうそんな時間ですか?」\\
「ほんとだ、丁度十二時だし、いこうよ」\\
「確か上にたくさんあったはずよね」\\
「はい。六階です」\\
「じゃあいこ」\\
セレナはさっぱりアヤメのファッションに興味をなくしてしまったようだ。
彼女の意思を曲げることは、誰にも出来ないだろう。\\
「何食べます?」\\
エスカレーターを登りながら、下の方向から聞こえる声に答える。
「なんでもいいわよ。あなたたちが選んで」\\
「カナンは?」\\
「私もなんでいいかな」\\
「ええー逆に困るなあ。……じゃあなんか良さそうなとこで」\\
ショーウィンドウのサンプルを品定めして決める、ということだ。
それでいいだろう。\\
「ここは?」\\
「これおいしそう」\\
セレナはハンバーグのサンプルを指さした。
その他にもオムライスやスパゲッティーなど、いろいろある。
洋食のレストランだ。\\
「ここでいいんじゃない」\\
最終的な決定権は、アヤメにあった。
一応年長者でもあるし、当然だろう。
私たちは店の中へと入っていった。

「これ見てもらえますか?」\\
セレナは自分の携帯をアヤメに渡した。\\
「なにそれ」\\
私が聞いてもセレナははぐらかす。
仕方なく私は注文したハンバーグを\scalebox{3}[1]{―}セレナと同じものだ\scalebox{3}[1]{―}
口に運んでいた。\\
「ふーん。こんなもので、集まるとは思わないけど」\\
「何なんですかそれ?」\\
彼女の見せてくれたのはSNSのアカウントだった。\\
「悪夢実体験……」\\
それは自らの見た悪夢の叙述をしているアカウントだ。\\
「なにこれ、中に私たちのじゃん」\\
「そうだよ。だって私が作ったもん」\\
「これセレナが?」\\
「うん。これでさ、私たちみたいに悪夢を見てる人を探すの」\\
「でも……」\\
このアカウントをフォローしている人数は十人に満たなかった。\\
「これじゃあただの雑談ネタ供給じゃないの」\\
「ついこのあいだ作ったばっかりだし、これからだよ。絶対来るって」\\
「その可能性は低いでしょうね」\\
アヤメは断言した。\\
「でもでも、可能性はあるんですよね」\\
「まあ、低いけど」\\
「じゃあもしそれで見つけられたとしても、どうするの?」\\
「それは\scalebox{3}[1]{―}」\\
彼女は答えなかった。
こんなことをしたのも、ある種の好奇心なんだろう。
ある程度物事に慣れれば、それ以上を求めようとする。
自分の能力に関係なしにだ。
それは私も同じだし、だからこそ慎重さに欠けると思った。
いざその方法で悪夢を見る人を見つけも、
それが私たちの手に負えない存在であったらどうするのだろうか。\\
「カナンもきつく言わないで。セレナは、セレナなりに方法を考えたのだから、
まずはそれを評価してあげようか」\\
「それはそうですけど」\\
「お願い! もうちょっとだけ続けさせて」\\
「ほら」\\
「わかった」\\
「ありがとう! なんか見つかったら連絡するから」\\
セレナは満足げだが、私とアヤメは猜疑心に溢れていた。
本当にそれが、私たちの行動に結びつくのか。

だが結果は、すぐにわかった。

\subsection*{(4)}
「それ本当なの?」\\
休み時間、セレナが押しかけてきた。\\
「本当だって! 今日直接会いたいって言ってるの」\\
「ここで?」\\
「うん」\\
「ええ……」\\
私は急いでいた。次は移動教室なのだ。
それに春休み明け最初の授業日で、慣れてもいない。
クラスメイトもあたふたしている。
二年生になって初日。
とにかく忙しいのだ。\\
「わかったから。後でいい?」\\
「いいけど、ちゃんと来てよ!」\\
「わかってる」\\
私はセレナを置いて走った。
幸い遅れることもなく\scalebox{3}[1]{―}正確に言えば間に合わなかったが、
先生が寛容だったのだ\scalebox{3}[1]{―}二コマ目の授業が始まった。\\

「ねえ、もう一回聞くけど、それ本当なの? 誰かのイタズラとかじゃないの」\\
「違うって絶対。文がそれっぽいもん」\\
「文章ぐらい誰だって書けるよ」\\
「もう、信じてよ!」\\
彼女が言うには、春休み中に始めたあのアカウントに、
直接学校で会いたいというメッセージが届いたというのだ。
それも、自身の夢について相談したいという。
セレナはそのアカウントで、『自らも悪夢を見て、それをある方法で解決した。
その方法をみんなにも教えたい、相談に乗りたい』
というスタンスで振る舞っていたようだ。
そして運良く、臨んでいたものが舞い込んだ。
そう信じたい。

「ねえどこにいるの?」\\
「私たちと同学年らしいから、どっかそこらへんにいると思うけど」\\
そんなことを言われても、時間は昼休みだ。
それに春休み明けだから、久しぶりの友達と会えて嬉しい、
という人間がわんさか溢れかえっている。
いつもより活気のある状況だ。
こんな人でごった返した廊下の、どこにいるだろうか。\\
「あ、あれそうじゃない」\\
私は人を待っていそうな、と言っても完全に私の決めつけだが、
そんな女子を発見した。
冷水機などが置いてある、小さいホール。
その片隅に立っている。\\
「あれだよ。一人ぼっちで立ってますって書いてあるから」\\
早とちりかもしれないが、まあいい。
私たちは彼女に話しかけた。\\
「あの? あなたが連絡くれた……」\\
「\ruby{中村|萌}{なかむら|もえ}です。はじめまして」\\
大人しめな少女だ。
ショートボブの髪型と彼女の可愛らしい小顔は親和性が高い。\\
「私は時国瀬玲奈。でこっちが」\\
「詠華南です」\\
「どうも……よろしくおねがいします」\\
「それじゃあちょっと移動してもいい?」\\
「はい、いいですけど\scalebox{3}[1]{―}どこに?」\\
「食堂」\\
「わかりました」\\
「あ、お昼とか大丈夫?」\\
「大丈夫です。私寮生だから」\\
「ああそうって、えっ、でもそれじゃあ尚更じゃない?」\\
「まずいから食べたくないんです。どうせ残すし」\\
「昼食抜き?」\\
「はい」\\
「はーよくもつね」\\
「ほら、セレナいこう」\\
「ああ、ごめんごめん」

私たちは食堂に移動した。
途中同じ建物の二階にある、購買に一度よった。
昼食抜き、といっても全く食べないわけではないようだ。
それにしても、私からしたら少ないとしか言えないのだが。
彼女は菓子パンを一つ買っただけだった。

「私は架谷彩芽」\\
「私らの先輩ね」\\
「……よろしくおねがいします」\\
私とセレナは隣り合わせに、モエとアヤメは私たちの向かいに座った。\\
「それじゃあ、単刀直入に聞かせてもらうけど、
あなたの見た悪夢はどんなものだった?」\\
初対面の人間に、いきなり夢といった極めて個人的な事柄を
打ち明けるのは、相当な負担を強いるだろう。
それでも、彼女がそれを望んできたのだから、何かあるはずだ。
私たちは彼女の言葉に耳を傾けた。
しばらくは無言だった。
きっと話の内容を整理していたのだろう。
彼女はゆっくりと話し始めた。\\
「夢っていうか、妄想っていうか、その\scalebox{3}[1]{―}
変に思わないでくださいよ\scalebox{3}[1]{―}私、好きな人といつも寝てるんです」\\
「寝てるって?」\\
「夢の中でですよ! 現実じゃないですよ、で\scalebox{3}[1]{―}」\\
「あの、寝てるって、どういう?」\\
私はマズい質問をしてしまったと後悔した。
モエは口を噤んでしまった。
私は本当に、彼女の『寝ている』という意味を厳密にしたかっただけだといことを、
念押ししようとした。決して下世話な気持ちはないのだと。\\
「あ、あの、ごめんなさい。本当に変な意味じゃなくて、その、しっかりしておきたいというか\scalebox{3}[1]{―}」\\
「やってるんです!」\\
彼女が叫んだ。
幸い、周囲のざわめきに吸収されて、その声は私たちの範囲に留めることが出来たらしい。\\
「恥ずかしがる必要なんてないわよ。私だってそれぐらいある」\\
だから続けて、とアヤメはモエの背中をさすった。
アヤメの落ち着きは、彼女の鎮静剤にもなったようだ。
私は胸をなでおろした。\\
「その、あの、やってるんですね。それで、それがすごくリアルで、
あそこに書いてあったみたいな」\\
私たちの体験を指しているのだろう
「すごく幸せで、気持ちよくて、
寝るのが楽しかったんです。
その夢を見るって言うのが、次第に人生の意味みたいになっていって。
私、前はすっごく遅かったんですよ、寝るのが。
でも年明けぐらいからは九時前に寝るのが普通になって、
しかも起きるのも遅くって、寝坊が当たり前というか。
だけど、最近怖くなってきたんです」\\
彼女は言葉を詰まらせた。\\
「怖いって、なにが?」\\
「首を……」\\
「首?」\\
「首を締めてくるんです。それが、すごく怖くて」\\
「殺そうとしてくるってこと?」\\
彼女は首を横に振った。\\
「わからないんです。苦しいって言ったら、離してくれることもあるし。
だから、それが嫌で」\\
「どんな風に締めてくるの。状況を教えて?」\\
「あの、私の夢、リアルって言いましたよね。
本当に細かくて、いきなり始まるってわけじゃないんです。
ホテルに行くところからとか、家に入るところからとか、
同棲しててごはんを食べ終わってからとか、だらだらテレビを見てたりとか」\\
「それはわかったから」\\
「あ、ごめんなさい。それで、彼は優しいんです。
いつも私の嬉しことをしてくれる。
\scalebox{3}[1]{―}なのに、最近は違って、いきなり押し倒してくるんです」\\
「なにそれ、怖い」\\
「ですよね。それで、最初はふざけているのかなって。
男の子ってそんなことしたくなるのかなって、思ってたんですけど」\\
「耐えてたのね」\\
「え……あ、そうです。でもそれもキツくなってきて、なんだか本気になって首を締めてくるというか」\\
彼女は明らかに現実と夢の垣根を取っ払っていた。
彼女の言葉の明瞭さは、もはや夢の記憶の範疇を逸脱していた。
妙に現実味を帯びた彼女の語りに、私は引き込まれていく。
悲しげな、行き詰まった顔だ。
彼女にとって、夢の中の彼が豹変した事件は、
いまここにいる私たちよりも、よっぽど関心の高い事柄なのだろう。\\
「それで? 首を締められるってどんな感覚なの?」\\
ただアヤメは違うようだった。
彼女はあくまで冷静に、どこか一つ高い領域から私たちを見下ろしているような物言いだ。
彼女のところどころ迷走する話を、なんとか矯正しようとしている。\\
「真っ白になるんです。眩しくて、苦しくて、でも、でもいつの間にか
気持ちよくなるんです。
もうこれ以上明るくならない、っていう風になったら、
心の底から幸せがじわじわ染み出してくるっていうか。
それで、気づいたら目が覚めてたって感じで、いつも終わるんです」\\
「でも今日は違った」\\
彼女は驚いたようだ。
秘密を暴かれた様な、自分の領域を侵されたという恐怖心を顕にした表情。\\
「なんで、それ」\\
「包帯が見えてる。新しい」\\
袖口を確認するモエ。その動作は驚くほど機敏だった。
\scalebox{3}[1]{―}最悪。彼女は呟いた。本当に小さな独り言。
彼女はそのことについて、端から話すつもりはなかったようだ。
隠し通せないと、諦めて澱んだ声色。\\
「\scalebox{3}[1]{―}今日はただ苦しかったんです。
首の閉まる音って、わかります?
聞こえるんですよ。
ほんの小さな隙間から、喉を通る空気の音。
骨がミシミシ言う音が耳に響いて\scalebox{6}[1]{―}」\\

煙たい部屋。

仄暗い部屋。

輪郭の部屋。

実体の部屋。

彼との部屋。

甘い吐息を嗅ぎ分けて、私は唇を探す。
ねっとりとした感触。
口を開いて全てを受け入れる。
彼の舌。彼の息。彼の唾液。
彼の魂。

こうしていることだけが、私の幸せ。

この漂う世界に残された、唯一私のみの場所。
安息の地。

私は安堵する。

またここに流れ着いたことに、この夢に埋没することに。

\scalebox{3}[1]{―}彼はさらに求めてくる。

制服を掻き分ける。

肌が露わになる。

恥ずかしいが、嬉しくもある。

彼が本当に私を欲して、私が本当に彼を欲している。

そして、彼はいつも同意を求めてくれる。
私の言葉を待つのだ。
それだけで、彼以外の人間など無に等しくなる。
私の言葉に耳を傾けてくれる人間が、この世界のどこにいるのというのだろうか。
私は知らない。
知りたくもないが、これは事実だった。

だからこそ、彼が際立つのだ。

沈んだ心をすくい上げてくれる。

私は、いいよと答えた。

優しくするよ。

彼も答えた。

目を瞑って、意識を集中する。
下腹部の圧迫感は、けれど苦痛ではない。
こみ上げる快楽。
これがまやかしというのなら、感覚も所詮、麻薬と電流に過ぎない。
これだけが真実、これだけが私。
そう語りかけてくる何かに身を任せ、私は目まぐるしく回る世界の中心に座すのだ。

シーツを掴む。

激しくなる。

もっと、もっと、もっと、もっと……。

\scalebox{3}[1]{―}不意に首筋から寒気が走る。

私は目を啓いた。

彼の顔が目の前に。

あれ、見えない。

暗くてよく見えない。

彼の顔、端正な顔。

……端正。端正って、どんな?

力が入る。

息が苦しい。

照明が眩しい。

真っ白な綿が、目にこびりついているような、
そんな視界に輝く星芒形を見る。

ようやく私は、これが窒息への凋落であることを悟った。

瞬間、衝動が体を貪る。
力が表皮の下を這いずり回って、私の手に集中する。
掻きむしる。それが誰の腕かもわからずに。

どうして。

私は叫んだ。塞ぎかかった気道に流れる空気は微小で、僅かに喉笛を鳴らすだけ。

ひゅーひゅー。

悲しい。怖い。痛い。苦しい。助けて。死にたくない。どうして。裏切ったの。
私が悪かったの。私を嫌いになったの。

考えることすら出来ない。

\scalebox{3}[1]{―}。

俯瞰する。

ベッドに打ち捨てられた体。

醜い、淫らな女。

嫌だ。

違う。

私は、私は、私は\scalebox{3}[1]{―}あなたと一つになりたい。\\

「\scalebox{3}[1]{―}骨が折れたんですよ! 折れて、痛くて、苦しくて、涙が出て、
だけど手は離してくれなくて。それで、それで藻掻いて、暴れて\scalebox{3}[1]{―}」\\
「そんな腕になった」\\
「抉ってました。布団に赤いシミがついてました。
爪の間に肉が挟まってて、それを洗い流すのが大変だった……」\\
性根尽き果てた彼女の、ぐったりとした姿だけが残っている。
その魂の訴え。
私はそれを聞き逃さない。\\
「モエさんは、もう嫌なんですよね。そんな夢は」\\
「\scalebox{3}[1]{―}はい」\\
「その前のやつも?」\\
それは違う、と彼女は言いたげだった。
けれど飲み込んだのだろう。\\
「いいえ」\\
「わかった。ありがとう。私たちがなんとかするから」\\
私はモエの肩に手をおいた。
だいぶ前に、アヤメにされたことだ。
いきなりの温さに、モエの体は少し粟立ったのか、振動を感じた。\\
「そうだよ! 任せて、私たちがきっと\scalebox{3}[1]{―}」\\
「あなたの夢を解決してくれる人を紹介してあげるから」\\
「えっ、アヤメさん」\\
かろうじて、隣に座る私にだけ聞こえる、小さな声だった。
困惑するセレナをよそに、アヤメはモエの前にかがみ込んだ。\\
「まあ正確には、夢を治してくれる人に、あなたを紹介するって感じかな」\\
「夢を、治す?」\\
「そう、だから心配しないで」\\
「……お願いします」\\
モエは涙目になっていた。
もはやそれがどんなデタラメであろうと、それにすがるしかない。
そんな彼女を、優しく抱くアヤメ。
私たちとは全く違う。
同年代の人間には出来ない、彼女の励まし方。
それは私のそれよりも、よほど効果的なのだった。\\
「あと、部屋の鍵は開けといてね」\\
「どうして……」\\
「鍵は、夢の中に入るのを妨げるの。その人が言ってた」\\
「わかりました」\\
赤目を掻きながら、彼女は答えた。\\
「よし、それじゃあ、今日はここまで。明日また会いましょう。
明日はきっと、いい夢を見れるはずよ」\\
「はい」\\
その声は、震えていた。\\

「今日の夜、やるんですか」\\
帰り道、私はアヤメに聞いた。\\
「ええそうよ」\\
「わかりました」\\
彼女は少し黙って、それから話を続けた。\\
「それと、みんなに心してもらいたいことがある」\\
「強いって、ことですか」\\
「セレナは感がいいのかしら。そうよ。今夜の悪夢はかなり手強いかもしれない」\\
「私たちに務まりますか」\\
率直な感想だった。
今の私たちは、それなりの経験があると言っても、未だ彼女の足元にも及ばない。
屠ることのできるのは、せいぜい雑魚程度だ。\\
「分からない。やってみないとね」\\
「そんな……」\\
「大丈夫よ。そんな顔しないで。私たちは何度も狩りをやってきた。
一人よりも二人、二人よりも三人。今の私たちならできる。
私は、貴方達を信頼してるから」\\
いつもより慎重な言葉選びだと感じた。\\
「がんばります。私、今日は覚悟決めます」\\
セレナの物言いは重々しかった。\\
「私もです。精一杯、努力します」\\
「うん、ありがとう」

それじゃあ、と彼女は改札をくぐった。
電車の時間が来たのだ。
アヤメの乗る電車を見て、流れていく窓から彼女を探そうとした。
残ったのは背景だけだったが。

行こうよ。セレナが促した。

私たち二人はそれぞれの帰路についた。

\section{}
\subsection*{(1)}
今回はかなりの手応えがある。
彼女はそう思った。
狩りの夜に没入するための儀式。
ベッドがことごとく液化したように、沈んでいく体を抱いて、目を強く瞑る。
漂白されていく意識が、彼女の心を強くしていく。
現実の雑念を払い、彼女は狩人になるのだ。

今夜の空気は冷たい。
春の夜だと油断していたようだ。
彼女が目覚めたのはもちろん校内で、教員用の駐車場だった。

「誰も居ない」\\
珍しく、彼女が先着だった。
いつもは、偶然かもしれないが、彼女は二人の後に目覚めていた。
それだけ彼女がこの狩りに意気込んでいるという現れか。
見回りとはいかないものの、試しに周囲を見渡してみる。
動いているものはないか、ひと目を気にする必要はないが、
悪夢が今どこにいるのかを注意する必要は十分にある。
だが見つけられなかった。
このまま一人で、モエの部屋まで行こうかとも考えたが、それは過ちだろう。
彼女は待つことにした。\\

「早いわねカナン」\\
アヤメが来たようだ。\\
「こんばんは、アヤメさん」\\
「こんばんは」\\
習慣的な挨拶は、やはり心の平穏を期待してのものだと、彼女はようやく理解できた。
何気ない言葉だが、その響きはやはり安心できるのだろう。
今夜は尚更だろう。\\
「少し探してみたんですけど、悪夢はどこにも見つかりませんでした」\\
「でしょうね。こんな見通しのいい場所には隠れられない。
おそらくは、モエの部屋ね」\\
「私も、行こうとしたんですけど、やめたんです」\\
それとも行くべきでしたか、と彼女は付け加えようとした。\\
「賢明な判断ね」\\
「モエさんは、いまどうなってるんですか。今もやっぱり、悪夢を見ているんですか」\\
「さあね。まあ、よくはないでしょうね。けど焦らないで」\\
「はい、気をつけます」\\
「こんばんわー」\\
セレナもやっと来たようだ。

それに合わせて、いつの間にかアオタがアヤメの肩に乗っている。
カナンは狩りを繰り返すほど、アオタという存在のどこか不自然な
感覚を覚えていた。と言っても、彼を不信しているわけではない。
ただどこか、会話や行動が人間というのもに噛み合わない気がするだけだと。
一個の人格として受け止めていいのか、考えあぐねているのだ。
だがそれは、問題にするべきではない。
カナンのアオタに対する好感度は決して低いものではないし、
むしろ並大抵の人間よりは高いだろう。
だから今はやめよう。
そう思った。

「前も言ったと思うけど、今夜の獲物は厳しいものかもしれない」\\
アオタが続いた。\\
「アヤメの言う通りだね。今夜は不吉な予感がする。
並々ならぬ気配を感じるよ」\\
これも、最近わかったことなのだが、どうやらアオタは気配を感じられるようだ。
それにアヤメも。
熟練した夢の住人にとっては、それは基本的なものなのだろう、
そう彼女も考え、密かに憧れていた。
それはセレナも同じで、だから二人は感覚に頼ることを良しとしようとしない。
しかし気配というものが、経験則に基づいているのではないかという考慮はないようだった。\\
「これはあくまでも憶測に過ぎないのだけど、あなた達の出会った最初の悪夢、覚えてる?」\\
「あれは、忘れるほうが難しいです」\\
「うん。今でもすぐに思い出せる……」\\
「その片割れ。それがモエに取り憑いた悪夢かもしれない。
双極性の悪夢、といえばいいかしら。
本当は二体で一つみたいなものだったのかもしれない。
飴と鞭ね。
急速に心を溶かして、それを補食していたのかも」\\
「実際、双子というかそういう関係性をもつ悪夢はこれまでも居たんだ。
ただ大抵の場合は対存在で、片方の消滅はもう一方の死に直結していた。
アヤメの推論を支持するのなら、今回の悪夢は共生関係というよりは、
協力関係に居たのかもしれない。
だとすれば、その力関係は均衡であったはず」\\
「強いってことよ。下手すると凶暴化してる可能性もある。
相棒が消えて、生きるのに必死になってるかもしれないから」\\
「私たち、どうすればいいの」\\
「そうだね……セレナとカナンは、うーん難しいね。
経験は十分だと思うけど、今夜の敵は\scalebox{3}[1]{―}」\\
「ついてきなさい」\\
「アヤメ!」\\
「いいでしょ。彼女たちももう素人じゃない。
その武器の使い方も十分わかってる。
使えない戦力ではないわ」\\
「それはそうだけど、早すぎる」\\
「何事も経験でしょ。
貴方達も、大丈夫よね」\\
「はい!」\\
「はい、大丈夫です」\\
「ほら、こう言ってる。それはいざとなれば私でなんとかする」\\
「……そうかい。そこまで言うなのなら、止める理由はないね。
僕はもう何も出来ないから、せめて君たちの狩りの成就を願ってるよ。
\scalebox{3}[1]{―}がんばって」

彼女たちは、女子寮へと向かった。

\subsection*{(2)}
女子寮は最近建て替えられたらしく、男子寮に比べれば綺麗だった。
新築から数年たった様な、まだかろうじて若いアパートの様な外観だ。
タイル張りの外観が、他の建築物との毛色の違さを引き立たせている。
その中の二階に、モエの部屋はあるらしい。
三人はもちろん寮生ではないので、寮の中には一度も入ったことはない。
いわば別世界だ。
同じ敷地内にありながら、立ち入ることを禁じられていた場所だ。
カナンとセレナの、若干の興奮はしょうがないだろう。
寝静まった夜だ。
物音を立てないよう、慎重に足を運ぶ。
もちろん声も囁き程度だ。\\
「ここじゃないですか?」\\
カナンが指をさす。\\
ドアノブを回す役割は、やはりアヤメが担うようだ。\\
「開けるわよ」\\
ゆっくりとドアを開く。
鍵はかかっていない。
彼女の部屋で間違いなかった。
徐々に見える間取りは、単純な構造だ。
しかし、肝心の彼女の姿は見えなかった。
丁度死角になっているのだ。
アヤメは先陣を切って、モエの部屋の中へ進む。
慎重に、慎重に、中腰のまま移動していく。
それに続く二人。
ここまで来ると、緊張はごまかしきれない。
あの時、一番はじめの遭遇の記憶が、再び脳裏に過るのだ。
それはあからさまに歩調へと現れていく。
アヤメは着実に進んでいくのに、二人との距離は離れるばかり。\\

アヤメが静止を促した。
音を聞けと合図をする。
私たちは耳に集中する。
カサカサと音が聞こえる。
布団が擦れる音だ。
私は思い切って壁を離れ、モエの姿を捉えようとした。

私はそれが過ちであることを、すぐに自覚した。

\scalebox{3}[1]{―}目が合った。

モエの上に浮かぶ、水滴を逆さにしたような体の悪夢。
その体から生える細長いては、彼女の首を締めていた。
\scalebox{3}[1]{―}いや違う。彼女は彼女の手によって、首を締められている。
悪夢はその腕を掴んでいるだけだ。
彼女の傷の理由がわかった。
まるで操り人形の糸を手繰るように、彼女の手を右往左往させている悪夢。
私は戦慄した。
ここまで人間的な動きを、悪夢がするという事実に。
いままでの個体は、すべからく獣だった。
動きも、感覚も。
それが私たちの『獲物』だ。
だがあれは違う。
知性を感じる。
それも狡猾なもの。

今それと目を見合わせてしまったことは、十分に私の体を凍結する理由になりえた。\\

先に動いたのは悪夢の方だった。
自らの危険性を察知したのか、浮遊した体を地面に落とし、手足を更に生やした。
怒涛の進撃が始まった。
悪夢はアヤメを躱し、ドアを開けて出ていこうとする。\\
「撃ちなさい!」\\
アヤメが叫んだ。\\
しかし動転したカナンが、その震える手を抑え、引き金を引ける体制に持っていくまでに、
悪夢はドアを開けきり、外に飛び出していった。
バン、バン、バン。
三発撃ったが、まるっきり当たらなかった。\\
「ああ、どうしよう」\\
外したこと、音を立てたこと、彼女は身を晒してしまうことに慌てふためいていた。\\
「武器の音は聞こえない。ほら、早く!」\\
アヤメは二人を急かした。
セレナはそれを汲み取り、カナンを立たせ、悪夢を追おうとする。\\
「ほらカナン、立って」\\
押されるカナンも、気を取り戻し、走る足に力を込める。
こんなことで、ヘナヘナしている暇はないのだと。\\
「セレナ! これ」\\
板状の物体が投げられた。\\
「おっと」\\
なんとか落とさずに取ることが出来た。\\
「携帯?」\\
「連絡用。何かあったら電話して。こっちからもするかもしれない」\\
「わかりました」\\
携帯をポケットに入れ、セレナはカナンを追いかけて、寮を飛び出した。

残されたアヤメは、部屋を見渡していた。
眠っているモエ。
その腕を縛った。
自傷を防ぐためだろう。

そしてなぜか、窓を開けた。
冷たい風が流れ込み、カーテンがなびく。

そしてアヤメは、わざわざその窓から外へと出ていった。

\subsection*{(3)}
逃走劇は長く、二人は走りっぱなしだった。
悪夢は明らかに、撒こうとしている。
角をいきなり曲がったり、わざわざ室内に入り、彼女たちに行方を撹乱しようとしている。
だがそれで諦める二人ではなかった。

横腹が痛む。
その痛みを抑えながら、なおも走り続ける。

半屋外のような通路で、そのときは満ちた。

先にしびれを切らしたのは、悪夢だった。
急回転し、彼女たちの方向を見る。
大きな手を広げ、ムチのようにそれを振るう。
\scalebox{3}[1]{―}受けきれない。
そう判断した二人は滑り込み、攻撃を避けた。
セレナは無事だが、カナンは違った。
腕と脚。
腕の方は悪夢のせいで、脚の方は擦りむいただけだ。
だがそれは十分に、空きとなりえた。

「クソが!」\\
カナンには珍しい暴言とともに、銃を乱射する。
擦り傷など構わず、感情に任せて撃っている。
一発一発、手に反動が返ってくる。
連射は相当な負担だった。
けれどその効果はあったようだ。
何発かは悪夢に命中し、それは大きな悲鳴を上げている。
耳鳴りの様な、不快な高音。

またとない好機だ。\\
「まだまだ!」\\
セレナが走り出した。
それはもう、全速力で。
黒い染みをたどっていく。
どうやら血のようなものを流しているようだった。
痕跡をたどり、前方を見れば、そこはまたあの寮だった。\\
「アイツ!」\\
悪夢は壁をよじ登っていた。
必死に、向かう先は決まっている。
モエの部屋だ。
セレナは迷うことなく、寮の入り口へ向かおうとする。\\
「待って」\\
それを、アヤメは引き止めた。\\
「アヤメさん! そんな場合じゃないですよ。
早く助けに行かないと! モエちゃんが」\\
「もう手遅れよ。どうもできない。
それより次の一手を考えるべきよ」\\
「そんな……」\\
「カナンは?」\\
「M科の扉の前に、たぶん」\\
「急いで行って、早く」\\
「はい、行きます」\\
明らかに不服だった。
それでも、アヤメには何か魂胆があると信じているようだ。
彼女は来た道を折り返し、カナンの元へと走っていった。\\

「あれ、どう思う」\\
「どうしようもないね。手遅れだ。
でも完全には消化できてないから、今夜中に仕留められれば、なんとかなるかな」\\
またいつの間にか、アヤメの肩にはアオタが乗っていた。
それぞれの視線は空いた窓に向かっている。\\
「でもそれで、やりやすくなるでしょ」\\
「まあ、あれだけ太ればね」\\
その瞬間、夜空には大きな音が響いた。
女性の悲鳴に似たそれは、けれど無機的だった。\\
「出てきた」\\
悪夢が窓枠からはみ出してきた。
まるで、カタツムリがその貝殻から顔をだすように、
或いは太りきった体を狭い空間から逃がすように、
悪夢はその図体を背負い、這い出て、見上げる星空へ吠えるたてた。

まるでカエルだ。
だらしなく飛び出た腹を抱えて、その巨体は動き出す。

その姿は当然、セレナやカナンにも捉えられていた。\\
「なにあれ」\\
「わかんない」\\
だが悪夢であることは明白だし、彼女たちも理解していた。
それでも信じようとしない。
あの巨体なら仕方のないことだろう。
あれを相手にできる狩人などいるのだろうか、そうカナンは疑問に思った。
強い意思だの心の平穏など、もはやどうでもよかった。
放心した心には、あの悪夢は刺激的、印象的すぎる。\\
「ねえ、どうすればいいのカナン!」\\
「知らないよ、わかんないよ」\\
「そんな、でも\scalebox{3}[1]{―}」\\
うわ、と二人は飛び上がった。
携帯のヴァイブレーションだった。
震える手を抑えて、セレナは電話に出た。\\
「聞こえる?」\\
「はい、聞こえます」\\
「それじゃあ一回カナンに変わってくれないかしら」\\
「わかりました」\\
「えっ、わたし?」\\
「うん、はい」\\
「もしもし、カナンです」\\
「よく聞いてくれるカナン?」\\
「あ、はい。聞きます」\\
「カナンにはもう一つ銃があったでしょ」\\
「はい、あります」\\
でも一度の使ったことがない。
彼女に不安がよぎる。\\
「それを使うときが来たのよ」\\
「え、でも私一回も……」\\
「使い方ぐらいもうわかってるでしょ」\\
「そうれは、そうですけど」\\
「シャキッとしないさい!」\\
「は、はい」\\
「いい、これを頼めるのはあなただけよ。
私の弓じゃ威力がないの。カナンの狙撃じゃないと、アレを倒すことは出来ない。
カナンしか居ない。それをわかって」\\
「でも、私、出来ません。あんな大きいやつに」\\
「でかいだけよ。見なさいあのノロマを!」\\
ちらりと悪夢に目をやるカナン。\\
「見ました」\\
「どう?」\\
「怖いです、大きいです!」\\
「それだけよ。すばやくもない。あなたは遠くから銃を撃つだけ。
誘導は私たちがやるから。ね? やる気をだして!」\\
回答が返ってこない。
だめか、とアヤメは諦めかけようとした。
だがカナンは、期待は裏切らなかった。\\
「はい、やります。がんばります。だからどうすればいいんですか!」\\
力いっぱい叫んでいる。
彼女は決めているのだ。狩人になった時から。
自分を変えてみたいと。意気地なしな己を、乗り越えたいのだと。\\
「いいわよ、その調子。それじゃあ、カナンは駅に行ってくれない?」\\
「駅って、あそこですか」\\
カナンが想像したのは、登下校に使う最寄りの駅だ。\\
「そう。そこの連絡橋。そこで待ってて」\\
「連絡橋って? 高くないですか? 私まだ」\\
「飛べるか飛べないかなんて考えない。やってみなさい。
それからよ。それに飛べなくてもよじ登れるでしょ。
とにかくそこで準備してて。
アオタが一緒にいるから」\\
「そうだよ」\\
携帯を当てている耳の、反対の耳からの音。紛れもない彼の声\\
「う、うわあ」\\
アオタが肩によしかかっていたのだ。\\
「いるでしょ。それじゃあお願いね」\\
セレナに変わって、そう言ってカナンへの指示は終わった。

「……わかりました。すぐ向かいます!」\\
通話は終わった。
セレナへの指示は、簡潔なものだったようで、二三言葉をやり取りしただけだった。\\
「カナンは行って。私はアヤメさんのところに行くから」\\
「わかった」\\
「それじゃあ、頑張ってね」\\
彼女の目は憂いていた。
カナンを一人にすることへの罪悪感と、自らへの不信感。

カナンは笑った。

「うん、頑張るね」

二人は走り出した。

セレナも、吹っ切れたのだ。

やるしかないと。

未だ夜空を仰ぐ悪夢は、全てを吐き出すように叫び、
それは涙に濡れる声と似ていた。

\subsection*{(4)}
「これどうやって登ればいいの……」\\
終電はとっくに過ぎて、駅には誰も居ない。
足音が交響して、私は孤独を実感する。
世界は今ただ一つ暗闇の中に滞留して、私を取り巻いている静けさに、
私の心は揺れ動く。
はたして、本当にできるのだろうか。
今更の焦燥。自分でもうんざりする。
一体、何度立ち止まれば気が済むのだろう。\\
「落ち着いて」\\
聞き慣れた声。
アオタだ。私は彼の存在をすっかり忘れていた。\\
「ねえアオタ。私、本当にできるのかな」\\
「それは君次第だよ。これ以外はどうともいえないよ」\\
「そう、だよね。私次第だよね。やるしかないんだ」\\
「何がどうあれ、結果を責めることができる人間は誰も居ないよ。
アヤメは君を信じて、セレナは君を心配してくれて、僕は君を選んだ責任がある。
だからこれは君の問題だ。君だけのもの。君が決めて、君が歩みだすべきなんだ」\\
私は黙って鉄骨をよじ登りだした。
手すりに足をかけて、勢いをつける。
冷たい鉄が手のひらに打ち付けられる。
そんなことどうでもいい。
なんだか心が軽くなった気がする。
そう、これは私の問題だ。
私だけが抱えるべきものなんだ。
誰にも、私の歩みを止める権利はない。
もちろん、私自身にもだ。

頂上の風景は、いいものだった。
ちょっとした達成感。

屋根の上は風が酷い。
フードを目深に被って、私は準備を始めた。

ジッパーを下ろして、三脚を取り出す。
三本足を開いて、バランスを保たせる。
アタッチメントに銃本体を取り付けようとする。
重たい。
ずっしりとした質量と威圧感。
設置は二分ほどで終了した。
本来はさらに短くできるはずだが、今の状況を考えると上出来だろう。

寝そべって、試しにアイサイトを覗き込む。
スコープによって拡大された風景は、いつもとは違う顔色を見せる。
点滅する信号。哀愁漂う電灯の並。

白い靄が通り過ぎる。
一つ二つと続き、更に多くの白い影が流れ落ちていく。
驚いた。
春なのに雪が降っている。
それほどの量はないが、自分の目を疑うには十分だった。
幻想的だ。輝くわけでもないが、私の目にはどこか星のように映る。
肩に落ちる結晶。
すぐに溶けて、見えなくなる。

感傷に浸る私を、携帯の振動が呼び戻す。
アヤメからの電話だろう。
メッセージが来ていたのだ。
また後で電話をすると。
その時間きっかりにかかってくる。身構える暇もなかった。\\
適当な挨拶を交わして、確認に入った。\\
「\scalebox{3}[1]{―}いま、セレナと一緒にアイツを追い詰めてる。結構すばしっこくて、
もう少し時間がかかるかもしれない」\\
了解です、と私は返した。\\
「それじゃあ、準備お願いね。それと\scalebox{3}[1]{―}余計なことは考えなくていいから\scalebox{3}[1]{―}」\\
彼女の励まし。
私は前を向いた。それが精神的な高揚に付随した現象であると、私は思った。
彼女の言葉は、いつもより柔らかかった。\\
「はい、わかりました。……信じてみます。自分を」\\
頑張って、と通話は途切れた。

大きく深呼吸をして、私は興奮を治める。
握る手の蒸し暑さに濡れて、引き金は滑り落ちていきそうになる。
それを握り直して、ただじっと待つ。
覚悟に寄り添い寂しさを堪えて、寒空の下にただ一人。

だが恐れはない。

そう、信じたい。

\subsection*{(5)}
悪夢の抵抗は激しさを増していた。
貪欲に魂を飲み込んだ怪物は、しかしその早急さ故に自らの足を引きずりざるを得なかった。
足元を執拗に攻撃され、たまらずに逃げ出す。
校庭を走り回り、支えにもならないフェンスにもたれ、結局道路へと転がり落ちていく。
敷地は嵩上げされていて、それなりの高さはあったが、それでも悪夢の大きさはそれをゆうに超える。
街中に出すわけにはいかない。
その進路を線路沿いに向ける必要があった。\\
「まだですか! アヤメさん!」\\
「まだ足りない、もっと駅に近づけて」\\
「了解!」\\
ずっしりと構え、腰を落とし、剣先に力を溜めるセレナ。
彼女の我流だろうか。
大きく、けれどそっと息を吸い、そして吐く。
はあっ、と声を出し、悪夢を突く。
かなりの質量を伴った衝撃が足元に命中して、悪夢の足取りはぐらつく。
倒れそうになる体を引っ張って、逆らえぬ前進を続ける。
建物をすり抜けて、線路に突っ込む。

もう少しだ。アヤメはカナンに合図を送ろうとする。
もうすでに、彼女が視認できる距離にいるはずだ。
手を振って、彼女の注目を引く。\\

微かな光の反射。
それが人の手だと分かるまで、少し時間を必要としたが、私の気を引くのには十分だった。
こんな距離で届くのだろうか。
もちろん、今の距離が有効射程内であることは理解しているが、
それそも本来射撃というものは観測手を伴うもので、その機能を自分一人で負担できるのかは、
この期に及んでまだ信じることが出来なかった。
いらぬ詮索だったと後悔している。下手にネットで検索をしたことを、今になって後悔している。

いや関係ない。何度も自分に言い聞かせているが、やるしかないのだ。

呼吸を整えて、手の揺れを抑える。
息を止める。正確には呼吸をする暇がない。

スコープを覗く。
その片方の目でも視界を確保して、悪夢との正確な空間関係を把握することに務める。
夜よりも深い闇、けれど完全に真っ黒ではない、薄っすらと青みがかった、宇宙のような表面。
何かがおかしい。
私は全体を精査する。
アヤメとセレナの攻撃によって、立ち竦んでいる悪夢は、けれど人や動物のように、
わかりやすい急所を晒してはいなかった。だがそれはいつものことだ。

知識の不足でもあると思うが、悪夢に身体の組織構造などあるのだろうかと時たま疑問に思う。
基本的に、言葉通りの蜂の巣になるまで、弾丸を叩き込んだり、
殴り潰すのが、私たちの経験している狩りの方法だった。
時に液体のように地面を流れ逃れようとする悪夢を見たときは、
私は彼らの『生』というものに不信を抱かざるを得なかった。

\scalebox{3}[1]{―}気づいた。違うのだ。私はスコープから目を外し、
悪夢の上に視線を移動させる。
雫のような曲線上の輪郭に、ただ一箇所だけの出っ張りがある。
遠くからはただぼやけた粒でしかないが、私はよく観察しようと思った。

拡大された光景の先には、趣味の悪い、しかしそこはかとなく耽美な彫像が鎮座していた。\\
「なにあれ……」\\
思わず漏れてしまった。
私の見たものは女性の裸体。形は完全に人だった。膨らんだ乳房に、高い鎖骨。
表情はわからないが、顔と認識できるほどのパーツは揃っている。
ただ下半身は見えず、鼠径部の少し上から下は、完全に悪夢の海に沈んでいる。
同様に手首も拘束されているように同化している。
しかし完全な美ではない。
そこにはやはり普遍な体型を感じる。
どこにでもいる、少し華奢な青年。
\scalebox{3}[1]{―}モエだ。あれは、彼女の体を模しているんだ。

生々しい人間の体は、私の握る手を揺らがせた。
私は、私の頭の中の発想に恐怖したのだ。
あれが、モエの体が、まさに悪夢の急所なのだと。\\
「カナン、アレはモエじゃないよ」\\
アオタが喋った。\\
「彼女の形をしているだけだ。あれは楔だよ。彼女を悪夢に打ち付けている。
君は、それを打ち砕くんだ」\\
「あれを撃って、モエは死なない?」\\
私は確認したかった。誰かの証明が欲しかった。
お世辞にも厳密なものとは言えないが、
それでも後ろ盾を求める気持ちは誰にでも理解できるだろう。\\
「死なないよ。開放されるんだ。モエの魂は今、悪夢に囚われている。
でもね、人の魂っていうのは案外強いものなんだ。
悪夢が頬張っても、そう簡単に消化できるものじゃない。
やせ我慢しているんだ。だからあんなに太ってる。
でも時間は問題を解決する。だから急ぐべきだ。ほら、勇気を持って、引き金を引くんだ」\\
わかった、そう言って今度こそ狙いを定める。
私は思い出したのだ。モエは今どこにいるのか。
そう、寮の自分の部屋だ。彼女の生きた体は、温かい実体は今そこで眠っているのだ。
それをやっと、私はあの言葉を振り返ってわかった。
だから信じよう。
どのみちそれ以外に方法はない。

息を止める。これは自らの意思だ。
横隔膜の運動は消えて、私の意識する運動それ以外の一切が、一時的に死に絶える。
指先だけが私の『生きている』であって、この感覚だけが存在の意味だ。

揺れる標的。
ゆさゆさと髪はたなびく。
その体の右胸。人の心臓の場所。
理由は思いつかなかったが、とにかく狙いをつける。

風の音も、寒さも、感情も、全て消え去る。
自浄作用はおそらくこの銃によるものだろう。
たった今その考察すら捨て去って、思考も閉ざし、私は機械の一部に成り代わる。

撃て。

頭蓋に交響する命令。

私は引き金を引いた。

命中したかどうかは、すぐに目に見える形で返ってきた。

\scalebox{3}[1]{―}アアアアアアアアァ!

激しい咆哮と黒い飛沫。耳をつんざく高音と振動。
粟立つ体を抑え、私はまだ狙いを保ち続ける。
油断はできない。その余裕もない。

それがまったくの本能であることは、私には分からなかった。\\

悲鳴を上げて倒れる悪夢。\\
「やったのね。カナン」\\
呟くアヤメ。
その横でセレナは、ただ呆然と崩れ落ちていく悪夢を見上げている。
機械音声のように、ノイズ混じった声。
女性の声だ。だがどこか嬉しそうでもある。

しかしアヤメはすぐに自らの過ちを知ることになった。

油断しまいと常に構えていた彼女の、その一瞬の喜びが、命取りに成りかけたのだ。

「はっ、しまった!」\\
アヤメは走り出した。\\
「どうしたんですか!」\\
セレナは後を追おうとする。\\
「セレナはそこに居て!」\\
叫ばれて怯むセレナ。

倒れ込む悪夢は、しかし最後の抵抗を試み、
その崩れ行く体と慣性を利用して、兇弾の発射元、
カナンの居場所への体当たりを目指していた。

\subsection*{(6)}

パズルピースの欠けるように、ぼろぼろと消滅していく光景を見て。
なおもカナンは銃把を投げ出さなかった。
それはもはや偏執にも似て、貪欲な捕食者の心理でもあった。
じっと構えて、ただ時を待っている。
それ以外の選択などとっくに捨ている。

影がカナンを覆う。

\scalebox{6}[1]{―}。

輝きを見た。

カナンはそれを逃さなかった。
光に溢れ、夜に拓けた夜明けのような温かさ。
その中に彼女は耳を澄ませた。
微笑みと笑い声。幼い童たちの、夕日に向かって走り去っていく憧憬。

刹那、彼女は揺らいだ。
心と体、その乖離をかろうじて留め、彼女はさよならを交わした。

バン。引き金を引いた。

それでも自然法則には逆らえられない。
悪夢もまた、隷属する世界の歯車にしか過ぎないのだ。

成れの果てに下敷きになりかける。

だが、カナンの体は放物線を描いて中空を飛んでいく。
何が起こったのかは本人すらもよく理解していないようだった。

ふと後ろを見る。

アヤメもまた飛び上がっている。
彼女が投げてくれたのだ。

カナンを間一髪ですくい上げて、彼女は難を逃れる事ができた。

\subsection*{(7)}
無重力かと錯覚した。
三半規管がここは空中であるということを教えてくれたのは、そのすぐ後だった。
直下の風景を見ようと、体を捻ってみるが、うまくいかない。
そのまま落ちていく。酷くゆっくりに。
悪夢の消失を見て、私は充足感に浸る。
飛沫を上げて消えていく。
圧巻だった。それらを含めた全ての現象、夜空の星、セレナの声、アヤメの運動、私の墜落、
その全てが私の思考よりも遥かに遅く、或いは私のそれが速すぎるだけなのか。
多くの情報を処理するために、私の世界は遅延しているのだろう。

音が聞こえる。
籠もった音声。

リリリリリ、と規則的に鳴り響く鐘を模した音。
携帯の目覚ましだと気づくまで大した時間はかからなかった。

鮮明になっていく騒音。

夜景は明け方へと変遷していく。

そこは、天井だった。

\section{}
朝の電車の窓を眺めながら、夜のことを思う。
あの景色が印象的で、私の頭から離れない。
夕暮れの光景。どこか懐かしくもあって、けれどもうどうやっても届くことの出来ない場所。

でもそれだけだ。私は考えることをやめた。

電車が止まろうとする。
減速して、左右に揺れる中、慣性に揺さぶられながら立ち上がり、出口の前に立つ。
まだ新学期が始まって間もないから、車内は相変わらず新入生で溢れかえっている。
だけど彼らもいつか、こんなに早い時間に来なくても間に合うことを知って、
幾ばくかは空いてくれるだろう。

電車を降りる。

連絡橋を渡って、私は上を見る。
私の頭上に、さっきまで私が居たのだ。
そんなこともつゆ知らず、多くの人たちがそそくさと移動していく。

それが私たちの狩りだ。
いくら騒音を撒き散らしても、建物は壊れないし、傷つくのは私たちの体だけ。
それもまた目覚めれば一切が消えてなくなる。
証明はただ私の記憶に依存して、だから誰にも説明することは出来ないし、
信じてもらうことも出来ない。
私が救った人も、私を知らずに生きていく。

ただ今回に限って、私は直接な充足感を得るのだろう。
学校への道すがら、ささやかな自尊心を満たす妄想は、許されてもいいだろう。

「ありがとうございます!」\\
モエが頭を下げた。\\
「ああ、ええ、どうも」\\
褒められ慣れていないセレナはあたふたしているが、私も彼女のことをどうこう言える
わけではなかった。\\
「私たちは別に……紹介、しただけだから」\\
アヤメの持ち出した設定を覚えていたのは、我ながらよくやったと思う。\\
「でも、私の話を聞いてくれたのは、セレナさんとカナンさんだけだし、
本当に、本当に感謝してます」\\
「じゃあ今日はスッキリ起きれた?」\\
「はい。でもなんか腕がタオルで縛られてたんですけど……」\\
「まあまあ、それは気にしないで、ね?。\scalebox{3}[1]{―}よかった。これで安心だね」\\
「でも、まだ今日だけだよね、もし次もまた悪夢を見たら\scalebox{3}[1]{―}」\\
「次はもうないです。だって、彼とはお別れしたんです。はっきり」\\
「お別れ?」\\
「さようならって。私はこっちに戻れって。もう二度と会わないからって。
寂しいけど、それでいいんだと思ってます」\\
「そうなんだ。よかった」\\
どうあれ、彼女は彼女の中で踏ん切りをつけたのだろう。
これ以上は私たちが立ち入る必要はない。
悪夢は去った。他ならぬ私がとどめを指したのだ。
それは、私が保証するべきだろう。\\
「モエさん。ありがとう」\\
「え? お礼を言うのは私の方ですよ」\\
「そうだけど。でもなんとなく、言いたくて」\\
「ああ、わかりますよ。なんだかお礼を言われると、こっちもありがとうって言いたくなるの。
私もなんか、恥ずかしくって言っちゃうし」

それが真実なのだろうか。本当に意図せず言った言葉だったが、確かに褒められるというのはこそばゆい。
ありがとうを言うの苦手なのだろうか。
この地域の人間は、なかんずく顕著だと、どこかで見聞きした気がする。
だが感謝の気持ちがないわけではない。
彼女の直球な言葉は嬉しいし、私は狩人になってよかったと思う。
そうしよう。素直に受け止めて、今後の糧にして、前へ進もう。

「ううん。やっぱり違う。私は嬉しくて言ったの」\\
彼女は驚いたような顔をしたが、すぐにそれを縦に振った。
その後、若干のはにかみを挟んで、こう言ってくれた。\\
「ありがとう」\\
\newpage
\section{}
太田誠は刑事だ。署内での評価はそこそこで、至って普通の人間だ。
しかし、少々俗物な側面を持ち合わせ、金に糸目がつかない。
時に一般人相手に探偵ごっこを演じたり、お友達と口裏を合わせて
あくまで合法の範囲で小遣いを稼いだりなど、私腹を肥やしていた。
そんな彼はいま、最悪の気分だった。

なんせ、見ず知らずの人間に誘い出されたからだ。

昼、着信があった。
メールだ。自分のアドレスを知っている人間が、こんな時間帯に連絡をよこすなど、ありえない。
彼はそれを確信している。家族は仕事中だと分かっているし、幾人かの友人は、みな夜行性だ。
メールを開く。
嫌な予感が的中した。
『夜十時。、ここに来い』ただそうとだけ書かれて、下には住所が記されていた。

彼も人の子、後ろめたさを持っている。
黙って従うしかない。

脅迫罪でしょっぴいてやろうかとも考えたが、未遂罪だから問うことは出来ない。

「なんだここ」

捨てられたラブホテルだろうか。
廃れたネオンが見苦しい建物。\\
「こんなところなんざ選んで、頭おかしいんじゃねえか」\\
苛立ちを孕んだ言葉。
渋々中に入っていく。
南京錠は無理矢理に破壊されていた。

音が流れる。
まさか、と思ったがそれが自分の携帯からの音だと、すぐに気づいた。

『中に入れ』

受付の中に入れという意味か。

埃っぽい部屋。

扉を閉じる。

コンコンと、窓口の板を誰かが叩いた。\\
「誰だ」\\
返事はない。\\
「おい、なんか言ったらどうだ」\\
それでも無言を貫く、向こう側の人間の顔を太田は見ようとしたが、
如何せんこういう場所の受付は、顔を見られないような工夫が施されている。
見えなかった。
だが代わりに、紙を差し出された。
便箋、だろうか。
ルーズリーフだ。

『柏木研究所』と『調べろ』の文字。

「なんだよこれ。調べろって、俺は刑事だぞ、こんなこと出来る訳\scalebox{3}[1]{―}」\\
またノックだ。同時に紙が出てくる。

『君の得意技だろう』

「くそ。何を知ってるのか知らんが、俺は合法活動しかしないんだよ。
研究所か会社か知らんが、そんなもんぐらい自分で調べろよ。
それとも、なにか、潜入捜査でもしろっていうのか」\\
またまた紙だ。
どうやら口頭で答えるつもりは更々ないようだった。

『報酬は君の好きに決めるといい。そこで、何があったのかを知りたい』

「あのな\scalebox{3}[1]{―}」\\
太田は口ごもった。\\
「なあ、金はいくら出せる?」\\
何も、返ってこない。\\
「\scalebox{3}[1]{―}わかったよ。一応、調べてやる。
んで、どうすればいい。結果は? 俺が自由に決めていいのか?」\\
着信の合図。\\
『こちらから連絡する。
今日は、もう、さようならだ』\\
「ちっ、身勝手なやつだ」\\
足音が響く。
どうやら相手はそそくさと帰っていったようだ。

立て付けの悪い扉を開けて、太田もホテルを出ていく。

まずは、市役所だな。

太田はやはり、俗物だった。

\newpage
\onecolumn
\parbox<y>{\linewidth}{
\begin{landscape}
\thispagestyle{empty}
\jfontspec{851tegaki_zatsu_normal.ttf}
付き合おう。

彼女は言った。
それがどれだけ自然な流れの内にある言葉なのか、私は意識しなかったが、少なくとも私は、
不意打ちを食らった。

だがこう答えた。

いいよ、と。

断る理由もない。

彼女も好きだ。

たとえそれが、どれだけ空虚なものだとしても、
私は、彼女を守りたかった。ただ、それだけのことだ。
\end{landscape}
}



\twocolumn
\end{document}