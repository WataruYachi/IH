
\documentclass[../IHMain]{subfiles}

\setcounter{chapter}{0}
\begin{document}
    

\chapter{夜の\\始まりへ}
\markboth{\tt To the beginning of the night}{}
\section{}
暗闇はひどく人を不安にさせる。
未だかつてない、これほどまでに明るい夜を手に入れた私達でも、その恐怖は変わらない。
この世界から、浮き上がってしまっているような居場所のなさ。
そんな夜に、しっとりと落ちてくる雪は、これが夢の続きであるような錯覚を与えてくれる。

でもこれは現実だ。
絶対的な証拠はどこにもないが、この空気の肌を刺す風が、吐き出す白い息が、確信させてくれる。

それに、たった数メートル地面から離れただけで、
駅の連絡橋の上は凍え死んでしまいそうなくらい寒いし、心細い。
うつ伏せになって、もう一時間ほどは経っている。
待ち続けるだけというのは、かえって神経をすり減らしていくのだ。

傍らに横たわる、黒く重たい塊。
やけに長い。
狙撃銃というものか。
あまり詳しくないからよくわからないけれど、信用できる強さを感じる。
私はこれで、いずれやってくるであろう獲物を、仕留めなくてはならないのだ。
もちろん、銃を撃ったことも、握ったことも、そもそも今まで本物を見たことすらなかった。
それでもやらなければならないという緊張は、凄まじい。

\scalebox{3}[1]{―}悴む手が携帯で震えた。
いきなりの音と振動に、心臓がすこしドキッとなった。
彩芽さんからの電話がかかってきたのだ。
ポケットから取り出して、私は電話に出た。\\
「もしもし、聞こえる」\\
「はい、聞こえてます」\\
当然のことだけど、確かめておこうと決めていた。
この夜のなかでは、普通であることすらも心強い。\\
「良かった。それじゃあ確認するわね」\\
「はい」\\
「いま、瀬玲奈ちゃんと一緒にアイツを追い詰めてるとこ。
結構すばしっこくて、もう少し時間がかかるかもしれない」\\
片手間にスコープを覗き込む。
確かに動く影は一つもない。\\
「だから、慌てないでいいから」\\
「了解です」\\
「それじゃあ、準備お願いね。それと\scalebox{3}[1]{―}」呼吸を整える間の後
「\scalebox{3}[1]{―}余計なことは考えなくていいから。
自分はできるんだぞって、思い込めば案外なんとかなるって、さっき言ったでしょ。
本当にその通りだから。自分を信じれば、後はあの子達がバックアップしてくれる。
自分を信頼して。本当に、それしかないから」\\
大人びて、けれど柔らかい声は、とても大きな心の安らぎを与えてくれる。\\
「はい、わかりました。……信じてみます。自分を」\\
だけどその返事から、自信のなさがにじみ出ていることぐらい、自分でもわかっていた。\\
「うん、じゃあ、頑張って」

電話は切れた。

静かな暗闇で、私は彼女の言葉を反芻する。
先の言葉は、彼女が本当に、たった二年ほど早く生まれてきただけなのかを疑いたくなるぐらいだ。
でも彼女は私の想像を超える出来事を、今までずっと、たった一人で乗り越えてきた人なんだ。
だから、こんなにも強くて優しくなれるんだろう。
身勝手な納得だけれど、私はそれで満足した。

だから後は自分のやるべきことをするだけ。

そう覚悟して、私は時を待った。

\section{}
\subsection*{\gt \centering(1)}
酷い目覚め。
悪い夢を見ていた。
何か心の奥底から這い上がってくる、得体の知れない恐怖に顔を叩かれたような気がした。
枕を見れば、汗でぐっしょりと濡れていた。
いくら寒くて毛布を三枚重ねて寝ていたからといって、こんなにも汗をかくなんて。
窓を見ても、まだ外は真っ暗だった。
時計は午前五時前。
カチカチとなる秒針の音。
二度寝しようにも、もう一度あの夢を見るのかと思うと、寝られなかった。

なのに、肝心の内容は何一つ覚えていなかった。

\subsection*{\gt(2)}
結局、目が覚めてからずっと、ただ布団に包まっていただけだった。
薄っすらと明るくなってきた空を見て、私は二階から降りた。

「おはよー、\ruby{華南}{カナン}」\\
いつものことだが、お母さんが弁当を作っていた。\\
「うん、おはよう」\\
適当に返事をして、顔を洗う。
冷たい。
だから冬は嫌なんだ。
冷たさは痛い。
でもお湯が出るのを待つのも面倒だし、結局我慢する。

けれど、それのおかげで目も覚めた。

髪を整えて、制服をハンガーから取って、そのままストーブの前を占領する。
パジャマを脱いで、直に肌に当たる熱は、
すこしピリピリした感覚だから、長くは当たっていられない。
寒いし痛いしで、だらだらと着替える暇はないのだ。

「お母さーん。体操服どこ」\\
パジャマを洗濯物のかごに入れて、ついでに干してあるはずの体操服を探したのだが、
見つからなかった。\\
「ええ、しらんよー。どっか棚に入ってない?」\\
「棚?」\\
お母さんはいつもそんな手間のかかることはしない。
基本的に自分の服は自分で片付けるのが、我が家の暗黙の了解だった。
だから、まさかとは思いながら、下着やら靴下しか入っていないはずの、姉妹共用の引き出しを漁る。\\
「あ、あった」\\
ぐちゃぐちゃに丸められて、無理矢理に押し込まれていた。
しわしわなジャージ。
お姉ちゃんの仕業だ。
間違いない。
こんなにがさつなのは、この家では姉しかいないのだ。
でもなんで……。

まあ、どうでもいいか。

カバンをとってくるために、二階にまた上がろうとする。
その時「華南、ついでにお姉ちゃん起こしてきて。
もう時間だぞって」\\
お母さんからの指令が飛んできた。

こんこん、ノックをしても反応はない。
「お姉ちゃん、朝だよ。起きて」\\
扉越しでも十分に聞こえると思う大きさで言ってもても、起きてくる気配がない。
仕方なく、入ることにした。\\
「入るよ、お姉ちゃん」\\
ひどい寝相。
ベッドから体の半分が飛び出ている。\\
「ほら、起きて」\\
ばっと、布団をはがす。
あああああ、と呻く姉。\\
「あ、それ、私の体操服じゃん」\\
あそこに体操服が合ったのは、お姉ちゃんが使ってたからなのか。\\
「ねむい」\\
「眠いじゃない。起きて。仕事でしょ」\\
「まだ冬休み」\\
「うそつかないでよ。あと、なんで私の服着てるの?
それパジャマじゃなくて体操服なんだけど」\\
「使ってなかったから」\\
「使います」\\
「それは今日からでしょ」\\
「ああ、もういい。ちゃんと降りてきてよ」\\
うんうんと適当に返事をされれば、あまり気持ちは良くない。

ああ、冬休み明け初日から、なんだか嫌だな。

\subsection*{\gt(3)}
忘れ物は、ない。
ポケットやバッグの中を何度も確かめて、お弁当もしっかり持った。
経験上、長期休暇明けは忘れ物が多い。
小学校の頃は雑巾だったり、中学校のときは教科書だったり筆箱だったりで、
大変な思いをしてきたのだ。

「いってきます」\\
「いってらしゃい」\\
洗い物をしながらお母さんは返事を返してくれたが、お姉ちゃんからは何もない。
テレビを見てるだけだった。

はあ、寒い。

まだ薄暗い朝。

人通りも少ない道。

凍結した道路で滑りそうになるが、かろうじて回避できた。
夏だったら駅まで自転車に乗って行けたのだけれど、冬は歩きで行くしかない。
冬休みをぐーたら過ごしていたせいだろうか、少し歩いただけでも疲れる。

十分ほど歩いて、駅に着いた。
それと同時に、駅から大勢の人が出てくる。
向こうからの電車が到着した合図だった。
バス停に向かう人の流れをかいくぐりながら、私は改札をくぐり、
エスカレーターに乗って、駅のホームに上った。

エスカレーターを降りて左側に止まっている電車に乗る。
出発時刻は7時40分頃。
今の時刻は25分頃。
この時間帯に来れば、確実に席に座れるのだ。
駅から近いところに家があるから、もっとゆっくりしてもいいんじゃないかと、よく言われる。
でも家にいて時間を潰すのも、ここで座って待つのも大して変わらないのだから、早く来ているのだ。

いつもの席、立ち上がる時のことを考えて、私は通路側の席に座ることにしている。
窓側に座ると、席を立つために通路側の人の足を避けないと行けないし、
そのときに足と足がぶつかったりするのが、気まずいのだ。
幸いに誰にも座られていなかった。
車両の先頭から数えて、二つ目の出入り口が、ちょうど到着駅のホーム階段の目の前になる。
ここに座ればスムーズに降りることができて、列に巻き込まれることがない。
目の前の人の足の遅さに、イライラせずに済むのだ。
断っておくが、私はせっかちなわけではない。
他人の歩調に束縛されるのが嫌なだけなのだ。
特に朝は。

何人かの乗客だけで、その殆どは高校生だ。
みんな手元に集中している。
私もそうだ。
入学当初は、文庫本を読んでいたりしたが、今では携帯を触っている。
段々と、取り出したりするのが面倒になったり、少し周りの目が気になってしまったのだ。
自分だけ本を読んでいる疎外感。
感じなくてもいいものを、感じてしまったのだ。

ふと時計を見るとすでに40分になっていた。
乗り換えの人たちで、いつの間にか車内はいっぱいで、少し窮屈。
がたんと、音がなった。
電車が、動き出した。

動き出してからもう10分ほど経った。
二つの駅を過ぎて、そろそろ私が降りる駅に着く。

バイブレーション。
通知がきたのだ。
携帯のロックを外す。
誰からなのかは検討がついている。
セレナだ。\\
『おはよー』\\
『いまおきた』\\
いつも彼女はこうやっていちいちメッセージを送ってくる。
友達がいつ起きたとか、あまり興味はないから、いつも無視している。
まあ、あっちもそれを承知でやっているのだろうけど。

揺れる電車。
ちらほらと立ち上がる人は、おそらくその殆どが同じ学生だろう。
私も携帯をしまって、右の扉の前で待つ。

甲高い音を立てて、電車は止まった。
『開く』のボタンを押して、私は電車を降りた。
階段を登って、連絡橋を渡って、降りて、改札をでる。

冬の空。

寒い。

ここから15分ほど、学校まで歩く。
駅を出て左を行く。
少し前の、富山方面から来たであろう人たちを越していく。
程なくして、脇道に入る。
ここまでくれば、人も少なくなる。
途中、何人かの人に抜かれながら、やっと学校の目の前までたどり着く。
しかし、ここからが問題なのだ。
玄関までの坂道。
しかもかなりの傾斜があって、登るのも一苦労。
羽織っているコートが邪魔に思えるほど、じんわりと汗をかきながら、
私は教室にたどり着いた。

「おはよう」\\
返事を返してくれるのは、耳の空いている二人ぐらい。
そもそも人の少ない朝の教室は、驚くほど静かだ。
みんな携帯で動画を見たり、音楽を聞いたり、ゲームをしたりしている。
私もその一人だ。

ちょうど真ん中らへんの机が、今の私の席だ。
前過ぎず、後ろ過ぎない、先生の目も手薄な席で満足している。

バッグを机の横にかけて、席に座る。
教科書や筆箱を取り出して、環境を整える。

あとは、8時50分の一コマ目の開始時間まで、また携帯で暇つぶし。
音楽の趣味はないので、もっぱらゲームかニュースの閲覧。
イヤホンを取り出して、ジャックに差し込む。
両耳を塞いで、ゲームを始める。
最近は周りの影響もあって、リズムゲームをやっている。
面白いのか面白くないのかよくわからないが、キャラクターが魅力的なのでやっている。
けれど肝心の才能は、これっぽちもないのであった。

何曲かやり終わった後、チャイムが鳴った。
五分後には、またチャイムが鳴って授業が始まった。
国語の授業。
内容は、現代文。
一コマ90分は、やはり長い。
一つの科目で普通校の二時間分を潰すのは、無理があるのではないか。
時折、というか最近はそう愚痴を吐きたくなる。

結局、ぼーっとしている間に授業は終わってしまった。

\subsection*{\gt(3)}

二コマ目の数学。
数学それ自体は、あまり得意でもなく不得意でもない。
なんと言うか、平均点の少し上をふらふらしているという感じだった。
でも授業は単調というか、端的というか、とくに過不足のない平凡なもので、
時たまに居眠りをしてしまう。

まさに今、まぶたは重く、閉じかかっている。
早起きのツケが回ってきたのだ。
耐え難い睡魔が私を襲う。
締め切った教室の、こもった空気。
汗ばむ熱気。
頭が、沈む。

……。

……。

「起きてください」\\
肩を叩かれた。
私はとっさに顔を上げて、
「あ、はい、起きてます」と言った。
足音が遠ざかる。
また眠気が。
目がショボショボしてきた。
抗えない。
生理現象なのだからしょうがないと、半ば開き直って、もう寝てしまおうと思った。
ほんの五分だけ。
そう決めた。

うとうと。

ウトウト。

「起きてください」\\
また頭上で声がする。\\
「起きてください」\\
段々と大きくなる。\\
黙って顔をあげる。
目は閉じたまま。
それでも、先生は起きたと判断したのだろう、前に戻っていく。
机に突っ伏す。
限界だった。
どうしてこんなにも眠たいのだろうか。
考えることもできない。

眠い。眠い。眠い。ねむい。ねむい。ねむい。
ねむ……。ね\scalebox{3}[1]{―}。


寂しさに潰れそうな私を、\ruby{瀬玲奈}{セレナ}がそっと手を握ってくれる。
友達同士で手をつなぐことなんて、今までになかったから、その暖かさに驚く。

そして、寒さに身を寄せ合う私たち二人を前に、彼は語り始めた。\\
「君たちには、悪夢を狩ってもらいたいんだ」\\
子ども番組に出てくるキャラクターのような愛くるしい声と、その異質な姿。
「悪夢?それって、あの時のヤツみたいな?」\\
漂う光る球体に、瀬玲奈がそう聞いた。

テスト\colorbox{black}{adsf}
\end{document}