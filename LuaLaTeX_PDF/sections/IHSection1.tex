
\documentclass[../IHMain]{subfiles}

\setcounter{chapter}{0}
\begin{document}
    

\chapter{夜の\\始まりへ}
\markboth{\tt To the beginning of the night}{}
\section{}
暗闇はひどく人を不安にさせる。
未だかつてない、これほどまでに明るい夜を手に入れた私達でも、その恐怖は変わらない。
この世界から、浮き上がってしまっているような居場所のなさ。
そんな夜に、しっとりと落ちてくる雪は、これが夢の続きであるような錯覚を与えてくれる。

でもこれは現実だ。
絶対的な証拠はどこにもないが、この肌を刺す風が、口から吐き出る白い息が、確信させてくれる。

たった数メートル地面から離れただけで、
駅の連絡橋の上は凍え死んでしまいそうなくらい寒いし、心細い。
うつ伏せになって、もう一時間ほどは経っている。
待ち続けるだけというのは、かえって神経をすり減らしていくのだ。

傍らに横たわる、黒く重たい塊。
やけに長い。
狙撃銃というものか。
あまり詳しくないからよくわからないけれど、信用できる強さを感じる。
私はこれで、いずれやってくるであろう獲物を、仕留めなくてはならないのだ。
もちろん、銃を撃ったことも、握ったことも、そもそも今まで本物を見たことすらなかった。
それでもやらなければならないという緊張は、凄まじかった。

\scalebox{3}[1]{―}悴む手が携帯で震えた。
いきなりの音と振動に、心臓がすこしドキッとなった。
アヤメさんからの電話がかかってきたのだ。
ポケットから取り出して、私は電話に出た。\\
「もしもし、聞こえる」\\
「はい、聞こえてます」\\
当然のことだけど、確かめておこうと決めていた。
この夜のなかでは、普通であることすらも心強い。\\
「良かった。それじゃあ確認するわね」\\
「はい」\\
「いま、瀬玲奈ちゃんと一緒にアイツを追い詰めてるとこ。
結構すばしっこくて、もう少し時間がかかるかもしれない」\\
片手間にスコープを覗き込む。
確かに動く影は一つもない。\\
「だから、慌てないでいいから」\\
「了解です」\\
「それじゃあ、準備お願いね。それと\scalebox{3}[1]{―}」呼吸を整える間の後
「\scalebox{3}[1]{―}余計なことは考えなくていいから。
自分はできるんだぞ、って思い込めば案外なんとかなるって、さっき言ったでしょ。
本当にその通りだから。自分を信じれば、後はあの子達がバックアップしてくれる。
自分を信頼して。本当に、それしかないから」\\
大人びて、けれど柔らかい声は、とても大きな心の安らぎを与えてくれる。\\
「はい、わかりました。……信じてみます。自分を」\\
だけどその返事から、自信のなさがにじみ出ていることぐらい、自分でもわかっていた。\\
「うん、じゃあ、頑張って」

電話は切れた。

静かな暗闇で、私は彼女の言葉を反芻する。
先の言葉は、彼女が本当に、たった二年ほど早く生まれてきただけなのかを疑いたくなるぐらいだ。
でも彼女は私の想像を超える出来事を、今までずっと、たった一人で乗り越えてきた人なんだ。
だから、こんなにも強くて優しくなれるんだろう。
身勝手な納得だけれど、私はそれで満足した。

だから後は自分のやるべきことをするだけ。

そう覚悟して、私は時を待った。

\section{}
\subsection*{\gt \centering(1)}
酷い目覚め。
悪い夢を見ていた。
何か心の奥底から這い上がってくる、得体の知れない恐怖に顔を叩かれたような気がした。
枕を見れば、汗でぐっしょりと濡れている。
いくら寒くて毛布を三枚重ねて寝ていたからといって、こんなにも汗をかくなんて。
窓を見ても、まだ外は真っ暗だった。
時計は午前五時前。
カチカチとなる秒針の音。
二度寝しようにも、もう一度あの夢を見るのかと思うと、寝られなかった。

なのに、肝心の内容は何一つ覚えていなかった。

\subsection*{\gt(2)}
結局、目が覚めてからずっと、ただ布団に包まっていただけだった。
薄っすらと明るくなってきた空を見て、私は一階へ降りた。

「おはよー、\ruby{華南}{カナン}」\\
いつものことだが、お母さんがお弁当を作っていた。\\
「うん、おはよう」\\
適当に返事をして、顔を洗う。
冷たい。
だから冬は嫌なんだ。
冷たさは痛い。
でもお湯が出るのを待つのも面倒だし、結局我慢する。

けれど、それのおかげで目も覚めた。

髪を整えて、制服をハンガーから取って、そのままストーブの前を占領する。
パジャマを脱いで、直に肌に当たる熱は、
すこしピリピリした感覚だから、長くは当たっていられない。
寒いし痛いしで、だらだらと着替える暇はないのだ。

「お母さーん。体操服どこ」\\
パジャマを洗濯物のかごに入れて、ついでに干してあるはずの体操服を探したのだが、
見つからなかった。\\
「ええ、しらんよー。どっか棚に入ってない?」\\
「棚?」\\
お母さんはいつもそんな手間のかかることはしない。
基本的に自分の服は自分で片付けるのが、我が家の暗黙の了解だった。
だから、まさかとは思いながら、下着やら靴下しか入っていないはずの、姉妹共用の引き出しを漁る。\\
「あ、あった」\\
ぐちゃぐちゃに丸められて、無理矢理に押し込まれていた。
しわしわなジャージ。
お姉ちゃんの仕業だ。
間違いない。
こんなにがさつなのは、この家では姉しかいないのだ。
でもなんで……。

まあ、どうでもいいか。

カバンをとってくるために、二階にまた上がろうとする。
その時「華南、ついでにお姉ちゃん起こしてきて。
もう時間だぞって」\\
お母さんからの指令が飛んできた。

こんこん、ノックをしても反応はない。
お姉ちゃんの部屋は、私の部屋の横にある。
ちょうど、ドアの位置関係は直角だ。
返事はまだ返ってこない。
仕方なく、私は先に自分の部屋で用意を始める。
教科書を詰め込んで、何度か今日の時間割と合致しているか確認した後、
バッグを担いで出た。
結局起きてくる気配は微塵もなかった。
だから、もっと激しくドアを叩いた。\\
「お姉ちゃん、朝だよ。起きて」\\
扉越しでも十分に聞こえると思う大きさで言ってもても、一向に反応がない。
しょうがないので、中に入ることにした。
姉妹の部屋へ入ることに抵抗感のない人は、
この年になると少なくなるんじゃないだろうか。\\
「入るよ、お姉ちゃん」\\
ひどい寝相。
ベッドから体の半分が飛び出ている。\\
「ほら、起きて」\\
ばっと、布団をはがす。
物の散乱した床に足の踏み場はないも同然で、その動作も一苦労だ。
あああああ、と呻く姉。\\
「あ、それ、私の体操服じゃん」\\
あそこに体操服があったのは、お姉ちゃんが使ってたからなのか。\\
「ねむい」\\
「眠いじゃない。起きて。仕事でしょ」\\
「まだ冬休み」\\
「うそつかないでよ。あと、なんで私の服着てるの?
それパジャマじゃなくて体操服なんだけど」\\
「使ってなかったから」\\
「使います」\\
「それは今日からでしょ」\\
「ああーもういい。ちゃんと降りてきてよ」\\
部屋を出る。
返事が返ってきたのは、私が階段を降りかけたときだった。
それに、うんうんと適当な返事をされると、あまり気持ちは良くない。

ああ、冬休み明け初日から、なんだか嫌だな。

\subsection*{\gt(3)}
忘れ物は、ない。
ポケットやバッグの中を何度も確かめて、お弁当もしっかり持った。
経験上、長期休暇明けは忘れ物が多い。
小学校の頃は雑巾だったり、中学校のときは教科書だったり筆箱だったりで、
大変な思いをしてきたのだ。

「いってきます」\\
「いってらしゃい」\\
洗い物をしながらお母さんは返事を返してくれたが、お姉ちゃんからは何もない。
テレビを見てるだけだった。

はあ、寒い。

まだ薄暗い朝。

人通りも少ない道。

凍結した道路で滑りそうになるが、かろうじて回避できた。
夏だったら駅まで自転車に乗って行けたのだけれど、冬は歩きで行くしかない。
冬休みをぐーたら過ごしていたせいだろうか、少し歩いただけでも疲れる。

十分ほど歩いて、駅に着いた。
それと同時に、駅から大勢の人が出てくる。
向こうからの電車が到着した合図だった。
バス停に向かう人の流れをかいくぐりながら、私は改札をくぐり、
エスカレーターに乗って、駅のホームに上った。

エスカレーターを降りて左側に止まっている電車に乗る。
出発時刻は7時40分頃。
今の時刻は25分頃。
この時間帯に来れば、確実に席に座れるのだ。
駅から近いところに家があるから、もっとゆっくりしてもいいんじゃないかと、よく言われる。
でも家にいて時間を潰すのも、ここで座って待つのも大して変わらないのだから、早く来ているのだ。

いつもの席、立ち上がる時のことを考えて、私は通路側の席に座ることにしている。
窓側に座ると、席を立つために通路側の人の足を避けないと行けないし、
そのときに足と足がぶつかったりするのが、気まずいのだ。
幸いに誰にも座られていなかった。
車両の先頭から数えて、二つ目の出入り口が、ちょうど到着駅のホーム階段の目の前になる。
ここに座ればスムーズに降りることができて、列に巻き込まれることがない。
目の前の人の足の遅さに、イライラせずに済むのだ。
断っておくが、私はせっかちなわけではない。
他人の歩調に束縛されるのが嫌なだけなのだ。
特に朝は。

まだ車内にいるのは、何人かの乗客だけで、その殆どは高校生だ。
みんな手元に集中している。
私もそうだ。
入学当初は、文庫本を読んでいたりしたが、今では携帯を触っている。
段々と、取り出したりするのが面倒になったり、少し周りの目が気になってしまったのだ。
自分だけ本を読んでいる疎外感。
感じなくてもいいものを、感じてしまったのだ。

ふと時計を見るとすでに40分になっていた。
乗り換えの人たちで、いつの間にか車内はいっぱいで、少し窮屈。
がたんと、音がなった。
電車が動き出した。

動き出してからもう10分ほど経った。
二つの駅を過ぎて、そろそろ私が降りる駅に着く。

バイブレーション。
通知がきたのだ。
携帯のロックを外す。
誰からなのかは検討がついている。
セレナだ。\\
『おはよー』\\
『いまおきた』\\
いつも彼女はこうやっていちいちメッセージを送ってくる。
友達がいつ起きたとか、あまり興味はないから、いつも無視している。
まあ、あっちもそれを承知でやっているのだろうけど。
\ruby{時国|瀬玲奈}{トキクニ|セレナ}、それが彼女だった。
中学校の頃から、同じ塾に通っていた友達。
いつからかはよく覚えていないけれど、一番親しい人。
彼女以外に、友達と言える人は正直言っていなかった。
でもそれでいいと思っている。

揺れる電車。
ちらほらと立ち上がる人は、おそらくその殆どが同じ学生だろう。
私も携帯をしまって、右の扉の前で待つ。

甲高い音を立てて、電車は止まった。
『開く』のボタンを押して、私は電車を降りた。
階段を登って、連絡橋を渡って、降りて、改札をでる。

冬の空。

寒い。

ここから15分ほど、学校まで歩く。
駅を出て左を行く。
少し前の、富山方面から来たであろう人たちを越していく。
程なくして、脇道に入る。
ここまでくれば、人も少なくなる。
途中、何人かの人に抜かれながら、やっと学校の目の前までたどり着く。
しかし、ここからが問題なのだ。
玄関までの坂道。
しかもかなりの傾斜があって、登るのも一苦労。
坂を登り終わった後は、
羽織っているコートが邪魔に思えるほどの、じんわりとした汗をかきながら、
四階の教室を目指す。
やっとこさ、私は教室にたどり着いた。

「おはよう」\\
返事を返してくれるのは、耳の空いている二人ぐらい。
そもそも人の少ない朝の教室は、驚くほど静かだ。
みんな携帯で動画を見たり、音楽を聞いたり、ゲームをしたりしている。
私もその一人だ。
冬休み明けだからといって、この時間帯の人たちは騒ぎ立てるようなことはしないのだ。
冷たいのか大人びているのか、はたまた面倒なだけなのか。
私は、ただ面倒くさいだけなのだが。

ちょうど真ん中らへんの机が、今の私の席だった。
前過ぎず、後ろ過ぎない、先生の目も手薄な席で満足している。

バッグを机の横にかけて、席に座る。
教科書や筆箱を取り出して、環境を整える。

あとは、8時50分の一コマ目の開始時間まで、また携帯で暇つぶし。
音楽の趣味はないので、もっぱらゲームかニュースの閲覧。
イヤホンを取り出して、ジャックに差し込む。
両耳を塞いで、ゲームを始める。
最近は周りの影響もあって、リズムゲームをやっている。
面白いのか面白くないのかよくわからないが、キャラクターが魅力的なのでやっている。
だけど肝心の才能は、これっぽちもないのであった。

何曲かやり終わった後、チャイムが鳴った。
五分後には、またチャイムが鳴って授業が始まった。

国語の授業。
内容は、現代文。
一コマ90分は、やはり長い。
一つの科目で普通校の二時間分を潰すのは、無理があるのではないか。
時折、というか最近はそう愚痴を吐きたくなる。

結局、ぼーっとしている間に授業は終わってしまった。

\subsection*{\gt(3)}

二コマ目の数学。
数学それ自体は、あまり得意でもなく不得意でもない。
なんと言うか、平均点の少し上をふらふらしているという感じだった。
組み合わせ、順列の授業。
PやCやら新しい記号がどんどんと導入され、こんがらがってしまう。
わかりやすくするために、板書をマーカーペンで色分けする。
どんどんと出来上がってくるノートに、私はほんの少しの満足感を味わう。
ここ最近、やっと自分の勉強が、中学から先の高校の勉強だという感じが出てきた。
けれど授業は単調というか、端的というか、とくに過不足のない教科書通りなもの。
しかも、いかにも寝てくださいと言わんばかりのもの柔らかい先生の声のせいで、
時たまに居眠りをしてしまうのだ。

まさに今、まぶたは重く、閉じかかっていた。
早起きのツケが回ってきたのだ。
耐え難い睡魔が私を襲う。
締め切った教室の、こもった空気。
汗ばむ熱気。
頭が、沈む。

……。

……。

「起きてください」\\
肩を叩かれた。
私はとっさに顔を上げて、
「あ、はい、起きてます」と言った。
足音が遠ざかる。
また眠気が。
目がショボショボしてきた。
抗えない。
教科書を盾にしてごまかそうとする、そんな余裕すらなかった。
だからもう生理現象なのだからしょうがないと、半ば開き直って、もう寝てしまおうと思った。
ほんの五分だけ。
そう決めた。

うとうと。

ウトウト。

「起きてください」\\
また頭上で声がする。\\
「起きてください」\\
段々と大きくなる。\\
黙って顔をあげる。
目は閉じたまま。
それでも、先生は起きたと判断したのだろう、前に戻っていく。
机に突っ伏す。
限界だった。
どうしてこんなにも眠たいのだろうか。
考えることもできない。

眠い。眠い。眠い。ねむい。ねむい。ねむい。
ねむ……。ね\scalebox{3}[1]{―}。

「起きてください」

起きて。

起きなさい。

起きろ。

起きます。

「じゃあ、詠さん。前に出て答えを書いてください」\\
\ruby{詠華南}{ナガメカナン}\scalebox{3}[1]{―}私の名前。
呼ばれるままに、前に出る。
ふわふわとした意識が、足元をふらつかせる。
教壇を上がり、チョークを持って黒板の前に立つ。
深い緑色。
なんだか薄くなっていく。\\
「じゃあ、そこらへんに答えを書いてください」\\
答え……そもそも問題が分からない。\\
「えっと」\\
前の席の人に見せてもらおうと思った。

後ろを振り向く。

ざわざわと音が聞こえるだけだった。

「えっ、ここ、どこ」

思わず口から溢れる。

\scalebox{3}[1]{―}。

しばらくの内、やっとここがどこか理解できた。

学校の裏の竹林だ。

確かに、円柱の数々は微かに茶色がかった緑色をしていて、
先端には葉っぱみたいなものが付いている。
でもただそれだけだった。
クラスメイトも、先生も、教室も消えて、雪の被った竹林にただ一人。

ざわざわ。

ざわざわ。

ざわざわ。

ざわざわ。

葉の擦れる音。
だんだんと大きくなってくる。私を取り囲むように、反響して増幅して交響していく。
うるさい。
うるさすぎる。
耳を塞ぐ。
塞いでもまだ聞こえる。
耳と手の、ほんの僅かな隙間から入り込み、外耳の中で増進していく。

息が荒い。
なぜか焦りを感じている。
怖い。

耐えられなくて、私は叫んだ。\\
「誰かいませんか」\\
何度も、何度も、喉が破れるくらいに大きな声で。
けれど何も帰ってこない。
ざわざわとうるさいだけ。
なんで。
なんで、なんなのこれ。
誰かどうか、どうか返事をください。\\
「起きてください」\\
聞こえた。
確かな人の声。\\
「起きてください」\\
起きている。私は起きている。

目は覚めている。
これほどまでにはっきりした意識を感じたことは、ないかもしれない。

それとも夢なのか。

分からない。

ほっぺをつねってみる。

\scalebox{3}[1]{―}痛い。

「起きてください」\\
真後ろから聞こえる。

私は、振り向いた。

「起キテくだサイ」\\
真っ白な世界に、真っ黒でまんまるな、影があるだけだった。

ああ。

あああ。

ああああ。

アアアアアアアア\scalebox{3}[1]{―}。

「あっ」\\
「それじゃあ、詠さんに……ああ、お休み中ですね。じゃあ\scalebox{3}[1]{―}」\\
紛れもない先生の声。
目が覚めた。
汗でノートが濡れていた。
ゆっくりと顔を上げて、周りを見てみる。
何も変わってない。
あの風景は、結局夢だったのか。
でも、あんなにも現実味を帯びた夢、記憶にこびりつくような夢は、
今まで見たことがなかった。
額に手を当てる。
少し熱いが、風邪を引いているほどではない。
ぐったりとした体。

時計を見れば、後少しで授業は終わりそうだった。

\subsection*{(4)}
チャイムが鳴った。
一斉に立ち上がるみんな。
私も立とうと思ったが、なんだかふらつくし、少し落ち着いてからにしようと思った。\\
「おーい」\\
聞き慣れた声がする。
教室の後ろのドアから身を乗り出して、セレナが私を呼んでいた。\\
「ごはん、いこ」\\
うん、と返事をして、一度深呼吸をして、バッグから弁当箱を取り出して、彼女の方へ向かった。
セレナはいつも弁当じゃなくて、学食で昼ごはんを食べている。
だから私は彼女に付き合って、一緒に食堂でごはんを食べる。
食堂は一旦外に出ないと行けない。
薄暗い廊下を歩いて、階段に向かおうとする。
バカ騒ぎしている男子の、いくつかのグループをかき分けて、私達は前に進んでいった。\\
「あ、セレナ、トイレ行ってきてもいい」\\
「うん、わかった」\\
彼女はそう言ったが、やっぱりあたしも行くと、一緒についてきた。

「なんか顔赤くない」\\
セレナが聞いてきた。\\
「えー、そうかな」\\
鏡を見て確認する。
そんなに赤いかな、と思っていたら、セレナが何かに気づいたような顔をした。
「あ、よだれ付いてるじゃん。居眠りしてたんだ。
あれれー、居眠りはしないものなんじゃないの?」\\
私のほっぺをグリグリしながら、セレナは笑った。
手洗いしたての手についた水が冷たい。\\
「セレナが言えることじゃないでしょ。
寝すぎて怒られた人に、言われたくない」\\
「私はしょうがないの。バイトしてるから」\\
「学生でしょ。本分は勉強。私は、勉強しすぎで疲れて寝ちゃったの。
私のほうがエラい」\\
「はあ、もうそんなことで偉そうぶるなんて、子どもだな、カナンくんは」\\
確かに、私の言葉は子供じみていた。
だから、私達は笑いあった。\\
「はいはい。じゃあ行こう」\\
ハンカチで手を拭きながらトイレを出て、階段を降りて、
私達は校舎をでた。\\

学食にはすでに多くの人間が並んでいた。
食券機に並んでるから、と列に付いた彼女を置いて、先に席を探す。
窓際の机の端に、ちょうど向かい合って座れる席が空いていたから、そこに座った。
しばらくすると、セレナはきつねうどんを持ってきた。
安いが、それ相応の味らしい。
私もお弁当を取り出して、食べ始めた。
今日の献立は卵焼きと野菜炒め、それにおにぎりだった。

もぐもぐと口を動かしながら、セレナは突然聞いてきた。\\
「そういえばさ、カナンってなんか夢とか見るの?」\\
「なに急に」\\
「いや、居眠りするってことはさ、夜に寝られないってことでしょ。
ていうことは、怖い夢とか見るのが嫌だとか、そんな感じかなって。
私は小さい頃そうだったの。
お化けのでる夢を見るのが怖くて、寝たくないって大泣きしたこともあったんだって。
お母さんが言ってた」\\
「夢が怖くて寝れないって、あったとしても小学生まででしょ。
私は一度もなかったけど」\\
「じゃあ怖い夢を見て、起きたりとかは」\\
「それは……たまにあるね。今日もそうだったの。内容は何も覚えていないけど」\\
「やっぱりあるよね。あと、なんか見覚えあるなーっていう夢を見たことってある? 
私は何回かあるの」\\
「ふーん、でもさ夢は記憶を整理しているだけだって、どっかに書いてあったよ。
だからおんなじ夢って見ないんじゃないの。毎日記憶はさ、新しく追加されていくんだし」\\
「そうかもしれない。けどカナンは知ってる?
あのさ、ずっと夢日記を書いてた人がいたの。
で、その人がなくなった後に、旦那さんだったかな、その人の日記を読んだら、
おんなじ夢の内容が周期的に書いてあったんだって」\\
「へぇー、じゃあセレナの夢もそうなの?」\\
「覚えてないから分かんないけど、たまに今日はこんな夢を見そうって思うことはあるよ」\\
「覚えてないのに?」\\
「うん。だから小さいときに寝たくないって泣いてたのかも」\\
なんだか話が脱線しているようだった。
お互いに何を聞きたかったのかを忘れた様な、無言の間。

「あ、でどうなの。カナンの夢って」\\
セレナのうどんはもうなくなって、彼女は暇そうに割り箸の先を噛んでいた。\\
「うーん。まあ、さっきのは怖い夢だったのかな」\\
「さっきのって、居眠りしてた時の?」\\
「うん。すごく短いんだけど、すごく怖かった」\\
「怖い夢かあ。居眠り中に夢なんて、私は見たことないなあ。
……どんな夢だったの?」\\
箸が止まった。
意図的に思い出すことを防ごうとしているのか、
一瞬、頭の中が真っ白になった。\\
「えっと、どんなのだったかな。
すごい変な夢なんだけどね、学校の裏にさ、竹林ってあるでしょ」\\
「うん、あるね」\\
「そこに突然、立たされたっていうか、気づいたらそこにいるって感じで、
立ってたの。不思議なのが、そこがどこか無意識にわかってたみたいで、
周りに竹しか見えないのに、ここが学校の裏の竹林だってことを受け入れてたの。
で、ざわざわって音がうるさくて、うるさいなって思った瞬間に
先生の声が聞こえて、後ろを振り向いたら、
真っ黒な球体みたいな感じのやつが浮いてたの。
それを見た瞬間に目が覚めて、なんか、すごい汗かいてたの」\\
「それ普通に怖くない?なんか憑かれてるのかも」\\
「やめてよ。私幽霊とか信じてないから」\\
「でもなんか妙にリアルだよね。やっぱり何かあるんだよ」\\
「偶然だって」\\
神妙な顔で、セレナは私を見つめていた。
確かに不自然な夢だが、夢とはそういうものなのじゃないのだろうか。
「そうだ、宿題あったんだ」\\
とっさに立ち上がって、セレナは食堂を出ていこうとする。\\
「えーもう行くの?」\\
「ごめん、宿題やってないから。またあとでね」\\
セレナは騒がしく走り去って行った。

なんだかもう、食欲が失せてしまった。
残りかけのごはんを残して、私は弁当を片付けた。
食堂を出て、教室へ帰る道すがら、裏の山を見る。
あそこはどうなっているんだろう、不意に思った。
でもすぐに、どうでもよくなった。

\section{}
授業が終わった。
一斉に帰りだす人の流れ。
私もそれに乗って、教室を出て、階段を降りていく。
入り口の前。
そのいくつかの溜まりの中で、セレナは待っていた。
向こうも気づいたのだろう、私の方へ駆け寄ってきた。
でもその顔はなんだか、いつもと違っていた。
何か予期せぬことが起こったと、わかりやすく顔に書いてある。
案の定、彼女の一声はおかしなものだった。\\
「見たんだよ!」\\
興奮と焦りが入り混じった声色。\\
「見たって何を」\\
「夢だよ夢。
カナンと全く同じの!」\\
「嘘でしょ。そんなわけ……」\\
「でも見たんだよ。私だって信じられないよ。でも、でも、あの、なんて言えばいいんだろう。
こう、気づいたらぱっと場所が変わってて。それがね、そこがねどこか分かるんだよ。
竹やぶ?あ、竹林か。分かるんだよ、行ったことも見たこともなにいのに、多分違うのに
そこがどこなのか理解させられるんだよ。やばいよねこれ。
……そもそもおんなじ夢見たって事自体、おかしいよね」\\
「ちょっとまってよセレナ。セレナと私の夢が、本当に同じ夢かなんて誰にも確かめられないじゃん」\\
「それはそうだけど。でも絶対そうだよ。絶対なんかあるよ。ねえ、見に行こうよ」\\
待ってよ。そういっても彼女は聞かなかった。
どこか子供じみたはしゃぎ方に、私は違和感を覚える。
でもいかないと。
一人きりになるのが嫌だった。
行き先は決まっている。
私も急いで、彼女を追いかけた。
\section{}
\subsection*{(1)}
初めてこんな場所にまで来た。
学校の裏、夏のプール授業の時に来ただけで、どうなっているのか今まで詳しく見ることはなかった。

ツタの絡まったフェンスを飛び越えようとする。
その先は完全に学校の敷地外だ。
先に、セレナが登った。
続いて私も登ろうとしたけれど、スカートがトゲに引っかかった。
そのまま勢い余って、セレナにもたれかかってしまった。
「大丈夫?」\\
「うん、大丈夫」\\
そう言ったけれど、自分の顔がどれほど暗い顔をしているのかを、いますぐ見てみたい。
きっと真っ青だ。
暗く淀んでいるはず。

でも今更引き返す訳にはいかない。
引き返せないのだ。
恐怖と同時に私の心をかき乱す、底知れぬ好奇。
それは、セレナも同じなんだろう。

彼女の腕を掴む。
二人一緒に、農道らしきコンクリートの道をそって歩いていった。
所々に落ちているタバコの吸殻。
ここが隠れた喫煙所であるという噂は、かなり有名だった。
日当たりも悪くて、しかも冬だ。
あたりはすでに薄暗く、気味が悪い。
伸び切った雑草と、整備の行き届いていない古い道。
どんどんと急になっていく坂道を登った先、なにか小屋らしき建物を見つけた。

その先は完全に藪。

行き止まりだった。

チャイムの音が聞こえる。

セピアな景色。

立ちすくむ私達。\\
「ねえ、カナン。帰ろうよ。
ここ入ったらダメなんじゃないの。
誰かの土地だよ。不法侵入だよ」\\
疲れ切った声だった。
私達はただ、夢に踊らされただけなのだろうか。

けれど、私はそうは思わなかった。

ふと、何かに呼ばれた気がした。

「カナン、カナン、帰るよ」\\
肩を叩かれる。
彼女の声は、耳に入っている。
だけど、頭の中には入ってこなかった。

あの時と同じだった。

ざわざわとうるさい。

これも夢の中なのだろうか。

明晰夢の中に居るような、居心地のもどかしさ。

揺すられる体は無気力で、今にも崩れ落ちそう。

\scalebox{3}[1]{―}私は目を疑った。

「ねえ、あれ、ヤバくない?ヤバイってホントに、ねえ、ねえ」\\
セレナの声。
恐怖に震える、確かな声。
でも、私は違った。
出せないのだ。声どころか体すら動かない。

目の前の歪みを、直視させられる。

現実にあるべきでないもの。

それはとても、あの時の夢に、似ていた。

\subsection*{(2)}

まるで抽象画の世界からひょっこり出てきたような化物。
緩やかな楕円と鋭利な三角形が組み合わさった胴体に、波動のように幾何学的な模様が絶えず動き回って、眼が痛い。
そして現実離れした異型から、ところどころに生えたヒトの手足。
ただそれだけが纏う現実感が、紛れもなく今襲いかかる、私達の危機的状況をまじまじと誇張してくる。\\
「カナン、ねぇカナン!」\\
引っ張る力はさらに強く、その声も耳をつんざく。
でも、セレナの必死さに反比例するかのように、私の意識は薄れていく。
なんだろう、何も言えない。返事をしたくても声が出ない。足が動かない。
金縛りにかかったように、自分の意志で体を動かすことができない。
終いには手足の感覚が、じわじわと消えていく。

だけど目が離せない。あの異物から瞬きすら拒ませる何かを感じる。\\
「ねぇカナン! 逃げよう、逃げるんだよ!」\\
張り詰める言葉に伴って、化物はこちらに歩み寄ってくる。
歩いているのか走っているのかも分からない歩幅で、しかし確実に私達を捉えながら。\\
「どうしちゃったのカナン! ヤバイよあれ、早く、早く!」\\
ダメだ、何も出来ない。本当に何も出来ない。
震える脚は歩くことを忘れて、立つことすらもままならない。

怖い、怖い怖い怖い、怖いよ、誰か助けて。

やっと、恐怖心だけでも取り戻せた。

けれど、遅すぎた。

いつしか目の前には、どこからか開いた大口迫っている。

ああダメだ。そう観念したその時。

背後から飛び込んできた人影が、怪物を、恐怖を、彼方へと吹き飛ばして行った。
見慣れない格好をしたその人は、そのまま追撃の手を緩めることなく、手にした武器のようなもの
\scalebox{3}[1]{―}剣だろうか\scalebox{3}[1]{―}で化物を薙いでいく。
一撃、また一撃と共に、寒空に響く嬌声は人の物ではなく、喩えるならばノイズがかったラジオの高音だった。

圧巻の一言では済まない、その異常な光景に、私達二人はただ立ち竦んでいるだけだった。

それでも分かるのは、さっきまでのおどろおどろしさなど微塵も感じられないぐらい、
あの化物には為す術なく、ただ攻撃を受け続けることしか出来ないということ。
そしてその攻撃は、私達と同じ『人間』によってなされているということ。

何かが光った。

それが、この戦いの終末だということは、一瞬の遅れを伴って、理解することができた。

音叉から鳴っているような、均一な高音。

\scalebox{3}[1]{―}そして、静かに消えていく化物の骸。

清廉とそれを目視する女性の姿。私達には目もくれず、立ち去ろうとする。\\
「あの!」\\
とっさに声が出た。\\
「ありがとうございました」\\
なぜこんなにも自然に、感謝の言葉が出てくるのだろうか。
自分でも分からなかった。
そして直後にこう思った。
彼女なら何か知っているのだろうかと。
いやそもそもこれは現実なのだろうか。
彼女もあの化物も、全部\scalebox{3}[1]{―}だとしたらそれもおかしいだろうが
\scalebox{3}[1]{―}全部セレナと一緒に見ている夢に過ぎないのだろうかと。
遅すぎる疑問の洪水。
一瞬、声に振り向いた彼女の目を見る。
口元は隠されていたが、彼女がどんな表情をしているのかは、なんとなくわかる気がした。

怪訝な顔だった。

\section{}
\subsection*{(1)}
「あれ、何だったんだろう」\\
窓枠にもたれかかったセレナが、心底疲れ切ったというような声で、そう呟いた。
電車の揺れさえも、強い衝撃として苦痛を感じてしまうほど、私も疲れていた。\\
「さあ、わかんない」\\
「夢だったのかな」\\
「さあ、わかんない」\\
「でも夢だとしたら、今も夢見てるんだよね」\\
「さあ」\\
「ねえ、つねってみてよ。目が覚めるかも」\\
そんなわけがない。
そう思いながら、彼女の頬をつねった。\\
「痛い。爪食い込んでる」\\
「ごめん」\\
どうやら、夢でもなんでもないらしい。
だとしたら、私達の頭がおかしいだけで、あれはただ幻覚を見ていただけなのだろうか。
セレナはただ、バッグを抱き枕のように抱えながら、夜の街を見つめているだけだ。
私もそうするべきなのだろうか。\\
「このこと、誰かに言うべきなのかな。
オカルト研究家とか、大学の先生とか」\\
「忘れたほうがいいんじゃないの」\\
セレナは、もううんざりしているようだった。
考えても意味のないことを、延々と考え続けるのは無駄に体力を消耗するだけで、
今の私にはまったく必要を感じない。

「間もなく、終点\scalebox{3}[1]{―}」

電車は止まった。
ぞろぞろと降りていく乗客たちに混じって、私達も降りた。

「じゃあね」\\
セレナに手を振る。\\
「おつかれ」\\
改札を抜けたあと、セレナは東口に、私は西口に別れた。
彼女はこの後、東口のバスターミナルからバスに乗って帰るのだ。

私は歩いて帰る。

冬の夜。

小さな雪が落ちてくる。

電灯も疎らで、記憶と相まった静寂は恐ろしい。

それでも恐怖は倦怠感に勝てず、グラグラと力なく歩みながら、私は家へと帰っていった。

\subsection*{(2)}
「ただいま」\\
「おかえり」\\
奥からお姉ちゃんの声。
台所に居るんだ。\\
「あれ、お母さんは?」\\
「なんか習い事に行くって」\\
「習い事? なんの」\\
「何だったかなぁ……編み物だったっけなあ。
働いてる人向けの習い事なのかな? 友達に誘われたって言ってた」\\
「そう。じゃあごはんは」\\
「これ。レンジで温めて食べてねって」\\
ラップのかかった皿が二つ、キッチンに並べてあった。
あれ、一つ足りない。
確か玄関にはお父さんの靴があったはずだが。\\
「お父さんは?」\\
「ああ、父さんは飲み会だって。新年会かなんかじゃないの? 
二人とも遅いかもだから、寝るときは鍵閉めといてね」\\
「うんわかった」\\
バッグをその場に下ろして\scalebox{3}[1]{―}いつもならお母さんに怒られているだろうが
\scalebox{3}[1]{―}私は脱衣所に向かった。\\
「温めておこうか?」\\
お姉ちゃんが聞いてきた。\\
「いい。あとで自分でする。先にお風呂はいるし」\\
「あっそう。じゃあ置いとくね。あと、お皿は自分で洗っといてねー」\\
「わかってる」\\
ドアを閉めた。\\
空っぽの浴槽。
そういえば、お湯を張っていなかった。
まあいいや。
もともと風呂に入るのはあまり好きではない。
水の圧迫感を感じて、胸が苦しくなるのだ。
冬だからと言っても、シャワーを浴びれば十分に温かくなる。
椅子に座りながら、数分間ずっと流れるお湯に打たれ続ければ、いつの間にか
体は芯までポカポカしている。
シャンプーにこだわりはない。
家族が買ってきたものをそのまま使っている。
ボトルのポンプを押して、シャンプーを手に出す。
泡立つ頭。
ただシャンプーが髪の毛に残らないようにすることだけは、気をつけている。
若い内に禿げたくないのだ。
頭の次は、体をボディーソープでしっかり洗う。
あとは丁寧に洗いで、髪の毛の水を切る。
入っていた時間は、大凡二十分ぐらいだった。

バスタオルで体を拭いて、パジャマに着替える。
洗面台の鏡の前に立って、ドライヤーを取り出す。
熱い風が髪の毛を乾かしてくれる。

お風呂を上がると、お姉ちゃんはとっくに夜ご飯を食べ終わって、自分の部屋に戻っていた。
ラップを剥がして、電子レンジの中に皿を突っ込む。
温まるまでの間に、炊飯器から白米をよそう。
適当に時間を設定して、温まったごはんを、リビングに持っていって食べる。
料理に興味がないから、今自分が何を食べているのか正確にはわからないが、
鶏肉を焼いたやつ、ソテーなのかな、としか言いようがない。
とりあえず不味くはないし、食べられるのもなのだからどうってことない。
食べ終わった皿を、さっと洗い流す。
そのまま食洗機に入れて、洗ってしまおうと思ったが、
よくよく考えれば、こんな少ない枚数で使うのはもったいないとわかった。
洗剤をたわしに着けて、食器と弁当箱を洗った。\\

部屋の後片付けなんかをしていたら、あっという間に夜の十時を過ぎていた。
いつもならゲームとか読書とか、趣味の時間なのだが、今の状態ではとてもその気にはならない。
もう寝よう。
玄関の鍵を閉めて、電気を消して上へ登る。
いつもより、ベッドに体が沈んでいる気がした。

目を瞑ればあっという間に、一日は終わった。

\newpage
{\onecolumn
\markboth{}{}
\begin{center}
    \ding{"76}
\end{center}

仏教の部派、\ruby{Sarvāstivādin}{サルヴァースティヴァーディアン}
(\ruby{説一切有部}{せついっさいうぶ})
の中には、意識に関する定量的な記述が見られるという。

それによれば、人の意識は、二四時間に六百四十八万個の「瞬間」によって成り立ち、
その平均の長さは約十三・三ミリ秒になる。
\begin{center}
    \ding{"76}
\end{center}}

我々はついに見つけたのだ。
上に落ちる林檎を。
アイソレーションタンク内の被験者の網膜上に投射した、
リンゴの自由落下運動の逆再生映像は、我々の期待を遥かに超える
働きを見せてくれた。\\

映像に使用したリンゴを、映像と同地点の研究室に設置した。
しばらくするとそれはまるで魔法のように宙を浮き出し、天井へ向かって上昇していった。
そしてリンゴは、映像の端と同じ高さに到達した途端、
ここが地球の重力圏を思い出したかのように地面へと落下していった。

我々はこれを多角的にカメラで収めていた。フレーム数は200で、
つまり一秒間に200回の「瞬間」を撮る事ができる。
チェックすると、なんとリンゴが写っていないフレームがあるのだ。
それはもう、影も形もない、全くの空。
私は興奮した。
これはもしかすれば、新しい世界を紡ぐ、始まりの一歩なのではないだろうか。\\

かの遠隔作用の如く、謎めいた運動を振る舞う林檎を、
我々はニュートンが古典力学を発見したように(かの逸話の真偽はここでは問わないが)、
その大いなる導きであることを信じている。
\twocolumn
\end{document}