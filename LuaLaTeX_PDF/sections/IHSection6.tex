\documentclass[../IHMain]{subfiles}

\begin{document}
\chapter{古い狩人(前)}
\section{}
\subsection*{(1)}
蒸し暑さにうなされて、私はなかなか寝付けなかった。
喉が渇く。発汗して蒸発していく水分量と、補給した量が明らかに釣り合っていない。
何度も下に降りて、コップにほうじ茶を注いで、一気に喉へ流し込む。
食道には物体が自然に落ちていくわけではないという。
冷蔵庫で綿密に冷やされたお茶に、更に氷を添加して、可能な限り冷却した液体は、
食道を冷たさとともに這いずり、痛いぐらいに自己主張する。
もはや飲み込むことすら辛いほど、熱帯夜の最悪な時間帯。
それを睡眠の彼方、意識の避暑地でやり過ごすことが出来ないのは、私にとって大きな精神的負担だった。
もうすぐ夜の十時を回ろうとしている。どうしよう。今夜がもし、狩りの夢だとしたら。

精神的な沈みこみは、感覚的なそれに直結している。
ベッドが堅く、けれど飲み込まれていく。
硬質な箱の中に閉じ込められそうになるように。
拘束具。降下運動。こころが重力に囚われて、ベッドをすり抜けて落ちていきそう。
トンネル効果。起こるはずもないが、しかし恐れもある。今日はこんなにも不安定だった。

儚い瞑想の中に、私は手足を拘束される。
感覚が違う。いつもの没入感とは違う、追体験を強制されるような、
催眠状態にあるような感覚だ。鋭いナイフが、私の衣服を切り裂いていく。
なにかを隠していないか、それを隈なく確認する。
厳しい取り調べ。他人を受け入れようとしない、疑心暗鬼な心を感じる。
この心は誰のものなのか。

その違和感を感じることは、私が狩りに慣れてきたという証拠かもしれない。

遠い先に光が見える。

気付けば、私は夢の中に居たようだった。
長い流れを遡って、私は目を覚ます。
そこは大通りだった。
駅西の長く幅の大きな道路。
その割に、周囲の開発は進んでいない。
かつての好景気が続けば、きっとこの場所にも多くのビルが立ち並ぶはずだったが、
あっけなくその夢は潰えて、虚しい広々さが残るだけの場所。
私の家の近所だ。
少し目を凝らせば、駅も見えるだろう。
疎らな電灯の、その一つに二人の人影が見える。
私の待ち人。
二人の狩人。
私は、大きく手を振ることも考えたが、彼女たちの視線が私に向いていないことを察知した。
互いに顔を背けあって、手持ち無沙汰に空や地面に顔を向けている。

私は、少し早足で歩いた。

足音に感づいて、二人は同一の方向に向き合った。
一人は見知った顔、セレナのものだ。
そしてもうひとりの顔、全く知らない、しかしどこかで既視感のある情報を持った姿。
長いポニーテールと両端の房、濡鴉の靡く長髪は、どこか人形のようだ。
意識してそうしているのか。彼女の表情は動かない。
どこか、あの時の女。私を眼前で見下した女、その雰囲気と似ているが、
彼女はそれを完全に自分のものにしていた。
あまりにも孤独に慣れすぎて、そのものになってしまっている。
それが第一印象だった。

その解析後に、私はやっと思い出した。
彼女の髪型と言うか輪郭は、いつか見たアヤメと口論していた女性そのものだ。\\
「あなたは……」\\
私は独り言つ。\\
「\ruby{黛}{まゆずみ}エナ。よろしく、ね」\\
伝わっているのかを不安視する首の傾き。
彼女の性格は、まだはっきりしない。\\
「ああ、よろしくお願いします」\\
えっと、名前は\scalebox{3}[1]{―}と言いかけると、「セレナから聞いているから、大丈夫」と返ってきた。
どうやらセレナと先に情報を交換しあっていたようだ。\\
「カナンも来たから、行きましょうよ。エナせ\scalebox{3}[1]{―}さん」\\
それにしては、どうもセレナの彼女への態度は、妙によそよそしかった。

少し歩くと言われて、私はエナの後に続く。
セレナは変わらず、彼女を見ようとしない。
なぜだろう。
緊張か不信か、そのどちらでもないのか。
その判断はできないが、私は、彼女のためにも、多くのことを目の前の人間から引き出そうと思った。\\
「エナさんは、前にも狩人をやってたんですよね」\\
「そうだね」\\
「アヤメさんと一緒に」\\
「そう……ね」\\
「強かったんですよね」\\
「まあ、そうかもしれない……」\\
「なんかすいません。偉そうに言って。でも、私達\scalebox{3}[1]{―}」\\
「不安がるのは理解できる。私も久しぶりだし。もう二度と狩人になる気はなかった」\\
「そうなんですか。それは、災難? ですねよね」\\
「確かに災難。でも、私は求められるなら、答える義務がある。あなた達がまだ強くなりきれないなら、
サポートするだけ」\\
「ありがとうござます。そう言ってくれて。私達も、がんばります」\\
「自惚れないでね。いつも気を引き締めて」\\
「はい」\\
「そうじゃないと、アヤ\scalebox{3}[1]{―}アヤメみたいになる」\\
アヤ、と親しく呼んだ彼女は、しかし躊躇していた。
彼女の事を、話したくない。私はそう感じ取った。\\
「エネさんは、どんな願いを叶えたんですか」\\
「願い? ああ、対価のことね。\scalebox{3}[1]{―}自分の生活」\\
「生活?」\\
「アパートの家賃だったり、電気代・ガス代・水道代諸々の生活費。
それを全部、私がもうこれ以上いらないと思うまで」\\
「そんなこと、できるんですか」\\
「その分、働く期間は長いけどね。アオタも多分そんなこと言ってたと思うけど」\\
「どれ位やってたんですか」\\
「一年以上。高一の夏ぐらいから、二年の冬ぐらいまで」\\
「そんなに、長いんですか」\\
「高望みしなければ、いつでもやめられる。肝心なのは、何を叶えたいか。
あなたは\scalebox{3}[1]{―}」\\
「私は、アヤメさんの腕を、治したいんです」\\
前を向いていたエナは、不意に私の方に目をやった。
ほんの僅かだが、その仕草は、まるで発言の源を確認する動作に見えた。
本当に言っているのか、と。\\
「そう\scalebox{3}[1]{―}優しいのね。あなたは」\\
「おかしいですか?」\\
「否定してるつもりはないよ。けど、私もアヤメも、自分のことしか頭になかった」\\
「私は、そうは思いません。アヤメさんも、もちろんエナさんも誰かのために、何かをできる人です」\\
「あなたがそう思っているのなら、そのままにしておいて。
ただ覚えておいてほしいの。私達は、決して綺麗な存在でいられないということ。
いずれ分かる。けれどそのときになっても、その心を大切にして。
……ああ、辛気臭くなったね。さっきのは忘れてもいいから」\\
「いえ、大切にします。私は、誰かのために戦うべきなんです」\\
彼女は、うなずくだけだった。

そしてこの会話の中で、セレナは一度も声を発しなかった。
それが、とても気がかりだった。
彼女はなにか、抱えているものがあるのだろうか。
私は、どこかしらの疎外感を押し付けられているような感覚を覚える。
二人に何があるのか。

だがそれは、詮索すべきではない。

\subsection*{(2)}
電灯の下、垂直な光に照らされた、コントラストの激しい表面。
私は、振り向いたエナの姿を初めて注視した。
すっぽりと長い外套に包まれた体。
その端々からは、かなり凝ったスーツのような服が見え隠れする。
襟の部分は普段見るものと同じに見えるが、その下は、胸を大きく強調するような、
かなり極端に短いジャケット、と言えるかはわからないが、そんな服を着ていた。
しかし大胆な服装を、結局はコートが隠しているし、第一目につくのは黒い手袋だ。
厚さは薄く、ピッタリと肌に張り付いているように見えるが、その重厚感は計り知れない。
柔らかいはずの手を、硬質な工具に変容させている。

多くの身体的な装飾に比べて、
ただ顔を隠すものはなく、しかも目立たない。
彼女の表情がそもそも、動かない彫像のようなもので、一般の風景に同化していた。
なびく房はたゆたう草木で、目を細めがちな顔面は壁の染みだ。適切な喩えではないと思うが、私の印象はそうだった。
どこか遠くに、自分の魂を飛ばしているような、影のなさ。
このだだっ広い歩道においても、彼女の存在感は電灯のそれと変わらなかった。

「ここで、何をするんですか」\\
私は聞いた。\\
「あなた達が今までどうやって狩りをしていたのかは、知らないけれど、
多分、同じよね」\\
「えっと、基本、追いかけてました。追いかけて、どっちかがへたるまで」\\
「……それ本当? 根比べしてたってこと?」\\
「大きかったりしたら、また違う感じですけど、普通はそうやってます」\\
「無駄ね」\\
「はあ……」\\
「追いかけるってことは、見つけないといけないってこと。こんな広い街の中で、
小さな点みたいな悪夢を、どうやって見つけるの?」\\
「それは、三人に分かれて」\\
「虱潰しに探すってことか……」\\
「でも、私達は悪夢の近くに目覚めるんですよ」\\


\section{}
性的な感覚を感じるということは、自身の体温の変化である。
人は絶えず体温を上昇させ続けている。
絶えず下がっていく自らの熱を、保ち続けるためにだ。
その永遠の作用の中に、温もりが介在すると、人は心理的な動悸を覚える。
温もりは実際、物質的な作用に依らないが、しかし多くの場合はそれに依拠している。
肌が触れ合う、あるいは極限まで接近すること、それが温もりの伝達である。
温もりが与えられ、その熱さ、あるいは冷たさに見を縮こまらせる行動原理。
肌の温度と乖離した\scalebox{3}[1]{―}熱さと冷たさのどちらか\scalebox{3}[1]{―}
驚くほどの温もりを得て、それに近づこうとする発汗や呼吸の乱れ、
つまり、体温をその温もりと同化させようとする働きが、恋である。
逆に、その温もりを排除して、本来の体温を取り戻そうとする働きが、嫌悪である。
肌に触れる熱さに震え、冷たさに身を静止させて、自らの熱を捨てることができるのは、愛し合うことの動物的な側面であろう。
そして嫌悪すること、自らの領域を保全しようとする心理的な作用は、愛し合うことの人間的な側面だろう。

そのどちらにも一致しない行動。
依存とは温もりを永遠に無へと還す諸作用のことである。
肌に触れ続ける、或いは延々と粘膜の接触を維持し続けること。
己の熱すらも忘れるほど、混じり続ける温もり。エントロピーの増大。
煩雑さをまして、しかしその回復をしない。
溜め込み続けるエントロピーは、いずれ熱的な死をもたらすように、
依存によって癒着した肌、温もりは、いずれ両者の破滅的な最後をもたらすだろう。
\end{document}