\documentclass[../IHMain]{subfiles}

\begin{document}
\chapter{古い狩人(前)}
\section{}
\subsection*{(1)}
蒸し暑さにうなされて、私はなかなか寝付けなかった。
喉が渇く。発汗して蒸発していく水分量と、補給した量が明らかに釣り合っていない。
何度も下に降りて、コップにほうじ茶を注いで、一気に喉へ流し込む。
食道には物体が自然に落ちていくわけではないという。
冷蔵庫で綿密に冷やされたお茶に、更に氷を添加して、可能な限り冷却した液体は、
食道を冷たさとともに這いずり、痛いぐらいに自己主張する。
もはや飲み込むことすら辛いほど、熱帯夜の最悪な時間帯。
それを睡眠の彼方、意識の避暑地でやり過ごすことが出来ないのは、私にとって大きな精神的負担だった。
もうすぐ夜の十時を回ろうとしている。どうしよう。今夜がもし、狩りの夢だとしたら。

精神的な沈みこみは、感覚的なそれに直結している。
ベッドが堅く、けれど飲み込まれていく。
硬質な箱の中に閉じ込められそうになるように。
拘束具。降下運動。こころが重力に囚われて、ベッドをすり抜けて落ちていきそう。
トンネル効果。起こるはずもないが、しかし恐れもある。今日はこんなにも不安定だった。

儚い瞑想の中に、私は手足を拘束される。
感覚が違う。いつもの没入感とは違う、追体験を強制されるような、
催眠状態にあるような感覚だ。鋭いナイフが、私の衣服を切り裂いていく。
なにかを隠していないか、それを隈なく確認する。
厳しい取り調べ。他人を受け入れようとしない、疑心暗鬼な心を感じる。
この心は誰のものなのか。

その違和感を感じることは、私が狩りに慣れてきたという証拠かもしれない。

遠い先に光が見える。

気付けば、私は夢の中に居たようだった。
長い流れを遡って、私は目を覚ます。
そこは大通りだった。
駅西の長く幅の大きな道路。
その割に、周囲の開発は進んでいない。
かつての好景気が続けば、きっとこの場所にも多くのビルが立ち並ぶはずだったが、
あっけなくその夢は潰えて、虚しい広々さが残るだけの場所。
私の家の近所だ。
少し目を凝らせば、駅も見えるだろう。
疎らな電灯の、その一つに二人の人影が見える。
私の待ち人。
二人の狩人。
私は、大きく手を振ることも考えたが、彼女たちの視線が私に向いていないことを察知した。
互いに顔を背けあって、手持ち無沙汰に空や地面に顔を向けている。

私は、少し早足で歩いた。

足音に感づいて、二人は同一の方向に向き合った。
一人は見知った顔、セレナのものだ。
そしてもうひとりの顔、全く知らない、しかしどこかで既視感のある情報を持った姿。
長いポニーテールと両端の房、濡鴉の靡く長髪は、どこか人形のようだ。
意識してそうしているのか。彼女の表情は動かない。
どこか、あの時の女。私を眼前で見下した女、その雰囲気と似ているが、
彼女はそれを完全に自分のものにしていた。
あまりにも孤独に慣れすぎて、そのものになってしまっている。
それが第一印象だった。

その解析後に、私はやっと思い出した。
彼女の髪型と言うか輪郭は、いつか見たアヤメと口論していた女性そのものだ。\\
「あなたは……」\\
私は独り言つ。\\
「\ruby{黛}{まゆずみ}エナ。よろしく、ね」\\
伝わっているのかを不安視する首の傾き。
弱々しく差し出された握手。私は手を取る。
強い振りだった。
彼女の性格は、まだはっきりしない。\\
「ああ、よろしくお願いします」\\
えっと、名前は\scalebox{3}[1]{―}と言いかけると、「セレナから聞いているから、大丈夫」と返ってきた。
どうやらセレナと先に情報を交換しあっていたようだ。\\
「カナンも来たから、行きましょうよ。エナせ\scalebox{3}[1]{―}さん」\\
それにしては、どうもセレナの彼女への態度は、妙によそよそしかった。

少し歩くと言われて、私はエナの後に続く。
セレナは変わらず、彼女を見ようとしない。
なぜだろう。
緊張か不信か、そのどちらでもないのか。
その判断はできないが、私は、彼女のためにも、多くのことを目の前の人間から引き出そうと思った。\\
「エナさんは、前にも狩人をやってたんですよね」\\
「そうだね」\\
「アヤメさんと一緒に」\\
「そう……ね」\\
「強かったんですよね」\\
「まあ、そうかもしれない……」\\
「なんかすいません。偉そうに言って。でも、私達\scalebox{3}[1]{―}」\\
「不安がるのは理解できる。私も久しぶりだし。もう二度と狩人になる気はなかった」\\
「そうなんですか。それは、災難? ですねよね」\\
「確かに災難。でも、私は求められるなら、答える義務がある。あなた達がまだ強くなりきれないなら、
サポートするだけ」\\
「ありがとうござます。そう言ってくれて。私達も、がんばります」\\
「自惚れないでね。いつも気を引き締めて」\\
「はい」\\
「そうじゃないと、アヤ\scalebox{3}[1]{―}アヤメみたいになる」\\
アヤ、と親しく呼んだ彼女は、しかし躊躇していた。
彼女の事を、話したくない。私はそう感じ取った。\\
「エネさんは、どんな願いを叶えたんですか」\\
「願い? ああ、対価のことね。\scalebox{3}[1]{―}自分の生活」\\
「生活?」\\
「アパートの家賃だったり、電気代・ガス代・水道代諸々の生活費。
それを全部、私がもうこれ以上いらないと思うまで」\\
「そんなこと、できるんですか」\\
「その分、働く期間は長いけどね。アオタも多分そんなこと言ってたと思うけど」\\
「どれ位やってたんですか」\\
「一年以上。高一の夏ぐらいから、二年の冬ぐらいまで」\\
「そんなに、長いんですか」\\
「高望みしなければ、いつでもやめられる。肝心なのは、何を叶えたいか。
きみは\scalebox{3}[1]{―}」\\
「私は、アヤメさんの腕を、治したいんです」\\
前を向いていたエナは、不意に私の方に目をやった。
ほんの僅かだが、その仕草は、まるで発言の源を確認する動作に見えた。
本当に言っているのか、と。\\
「そう\scalebox{3}[1]{―}優しいのね。きみは」\\
「おかしいですか?」\\
「否定してるつもりはないよ。けど、私もアヤメも、自分のことしか頭になかった」\\
「私は、そうは思いません。アヤメさんも、もちろんエナさんも誰かのために、何かをできる人です」\\
「あなたがそう思っているのなら、そのままにしておいて。
ただ覚えておいてほしいの。私達は、決して綺麗な存在でいられないということ。
いずれ分かる。けれどそのときになっても、その心を大切にして。
……ああ、辛気臭くなったね。さっきのは忘れてもいいから」\\
「いえ、大切にします。私は、誰かのために戦うべきなんです」\\
彼女は、頷くだけだった。

そしてこの会話の中で、セレナは一度も声を発しなかった。
それが、とても気がかりだった。
彼女はなにか、抱えているものがあるのだろうか。
私は、どこかしらの疎外感を押し付けられているような感覚を覚える。
二人に何があるのか。

だがそれは、詮索すべきではない。

\subsection*{(2)}
電灯の下、垂直な光に照らされた、コントラストの激しい表面。
私は、振り向いたエナの姿を初めて注視した。
すっぽりと長い外套に包まれた体。
その端々からは、かなり凝ったスーツのような服が見え隠れする。
襟の部分は普段見るものと同じに見えるが、その下は、胸を大きく強調するような、
かなり極端に短いジャケット\scalebox{3}[1]{―}と言えるかはわからないが\scalebox{3}[1]{―}そんな服を着ていた。
しかし大胆な服装を、結局はコートが隠しているし、第一目につくのは黒い手袋だ。
厚さは薄く、ピッタリと肌に張り付いているように見えるが、その重厚感は計り知れない。
柔らかいはずの手を、硬質な工具に変容させている。

多くの身体的な装飾に比べて、
ただ顔を隠すものはなく、しかも目立たない。
彼女の表情がそもそも、動かない彫像のようなもので、一般の風景に同化していた。
なびく房はたゆたう草木で、目を細めがちな顔面は壁の染みだ。適切な喩えではないと思うが、私の印象はそうだった。
どこか遠くに、自分の魂を飛ばしているような、影のなさ。
このだだっ広い歩道においても、彼女の存在感は電灯のそれと変わらなかった。

「ここで、何をするんですか」\\
私は聞いた。\\
「君たちが今までどうやって狩りをしていたのかは、知らないけれど、
多分、同じよね」\\
「えっと、基本、追いかけてました。追いかけて、どっちかがへたるまで」\\
「……それ本当? 根比べしてたってこと?」\\
「大きかったりしたら、また違う感じですけど、普通はそうやってます」\\
「無駄ね」\\
「はあ……」\\
「追いかけるってことは、見つけないといけないってこと。こんな広い街の中で、
小さな点みたいな悪夢を、どうやって見つけるの?」\\
「それは、三人に分かれて」\\
「虱潰しに探すってことか……」\\
「でも、私達は悪夢の近くに目覚めるんですよ」\\
「確かにそうだけれど、それは大雑把な位置関係でしかない。もっと確実な方法がある」\\
「どんなのなんですか」\\
「おびき出すの。餌を使って」\\
「餌?」\\
「アヤ\scalebox{3}[1]{―}メは使ってなかった?」\\
「いえ、全然」\\
「多分彼女なら、自分を餌にしていたんじゃないの」\\
「そんなことは……」\\
「彼女ならそうする。絶対に。
だから君たちは気づかない。どうして自分の周りに悪夢が出てくるのか」\\
「だとしても、私達は出来ません。どうすれば良いのかもわからないし、正直、怖いし」\\
「それはそうだね。でも、心配する必要はない。私はもっと安全にやるから」\\
そう言いながら彼女は、どこからか取り出した紙袋を私に手渡した。\\
「これ、なんですか」\\
「そのまま」\\
「あの、私はどうやってこれを夢の中に持ち込んだのか、気になるですけど」\\
「きみは変に堅苦しいところがあるのね。……聞いてないの? アオタかアヤメが説明してると思ってたんだけど」\\
「いいえ。知りません」\\
「そう、だったらまあ、簡単だし教えておくかぁ」\\
エナは私の手を掴んで、左腕に持った紙袋を奪って、頭の横に持ち上げた右手に持たせた。
まるでこれを枕にして寝ろ、と言いたげな動作。\\
「枕元に置いて寝る。置くだけじゃなくて、手に持って寝るのでもいい」\\
「それだけで、いいですか?」\\
「多分。だけど、ある程度大きなものは駄目みたい。家は持ってこれないしね」\\
面白いと思ったのか、頬を緩めてにやけるエナ。\\
「それは……そうですね」\\
「持ち込みたいものなんて、適当にすればいいから。お菓子とか、ゲームとか」\\
「あの、狩りって遊ぶ時間じゃないと思うんですけど」\\
「待ち時間は暇よ」\\
空気感が違うのか。彼女はもっと弛緩した意識で狩りに挑んでいるような気がした。\\
「無駄口はここまでね」\\
エナはそう言って、紙袋を開いた。
私も自分の手に持つそれを覗き込んだ。
暗い夜の、更に暗い部分を見透そうとすると、私は紙吹雪に目を潰されかけた。

細切れになった紙面の流れは、最小限に留めることが出来た。
驚きとともに手元が勝手に紙袋を閉じていてくれた。
それでも、浮遊した紙片は、一向に地面に落ちてこない。

その一つが、セレナの目の前に落ちた。
紙は無地ではなかった。何か情報を持っている。
その一片をセレナは持ち上げる。しなびたコピー用紙に印刷された何か。
セレナはその内容に、拒絶を示す。
彼女の顔を見て、私は好奇心に負けた。
閉じていた袋を開けて、大雑把に四角形たちを掴み上げた。

指の間から垣間見える、人の顔の部品。
グリッドにそって裁断されたと思われる、写真の一片たち。
その一つ、笑った右目はまっすぐに私の虹彩を突き抜けて、網膜に自らの姿を焼き付ける。
喜んでいるような、恥ずかしがっているような目つき、眉の歪み。
まだあどけない、私と同年代程だろうか、そんな女性のパーツ。
明らかに彼女の持つべきものではないだろう。
不安がよぎった。
その衝動のままに他の紙も調べる。新聞紙の記事、そのコピーや週刊誌の下品な見出し、
そしてさっきみたいな少女や男子の情事の残片が、大量に混ざり合っている。
紙の角が、私の肌を刺す。それは湿り気に似ている。
私は持っていられなくなった。
この掴み上げた塊が、おぞましい肉の集合体に思えたからだ。
白黒の筋肉質、或いは脂肪の塊。灰色のインク、或いはトナーの血、体液に滴った、人の縮合。
ただそれを開放することは、更に望まれないことだろう。
私はエナを問いただした。\\
「なんなんですかこれ! これって……」\\
「見れば分かるよ。新聞とか雑誌とか」\\
「これ、見てくださいよ。こんなの、使って良いものじゃない」\\
私は右目の紙片をエナの眼前に突き出した。\\
「ああ、ネットから拾ってきたやつね。掲示板とかサムネとかから引っ張ってきた」\\
「そういうことじゃなくて、これヤバイやつですよね」\\
「まあ、リベンジポルノ? っていうのかな。そういうのもあったと思うけど……でも、どうでもいいよね。
名前も顔も、覚えることのない人間なんだから」\\
「でも、これは\scalebox{3}[1]{―}」\\
漠然とした義憤。私はこの被害者たちをバラバラにして、それを手に握っている。
汗ばんで湿る紙の感覚は、じっとりと私の表皮を冒していく。\\
「こんなもの、どう使うんですか?」\\
「これは餌。おびき寄せるための罠に使う。なにか感じるでしょ。気持ち悪さみたいなものを。
それは悪夢も一緒なんだよ。でも悪夢は、その気持ち悪さを好む。
人の不幸、その塊が人に不幸をもたらす悪夢の源泉。だったら、これは最高の素材でしょ?
誰かが死んだり、死ぬほど恥ずかしい格好で写ったり」\\
彼女の頭には、同情という感情はないようだった。
目的のためならば、手段を問わない。
非情といえばそれまでだが、確かに、彼女の言うことには何かしらの説得力があった。
私は、こんなことで憂う必要はないのだろうか。
良心が痛む。裸を晒されて、今ここで全く知らない私に見られてしまうこの人物たちの、
私の勝手な妄想上の悲しみ。抑えるべきか、迷うことしか出来ない。
ただそんな私など、彼女は気にもしなかった。
説明は済んだと、エナはセレナにも『餌』を渡して、それを巻いてこいと命令する。\\
「これ、お願い。あとできるだけ、人目につくところに置いといて」\\
「はい」\\
セレナは黙って従った。
私は、任務を放棄した。
掴んだ塊を紙袋の中に戻す。手汗にふやけた紙片は、掌にしがみついてなかなか離れない。

気色悪い塊との分離に、私が手間取っている間に、セレナは手際よく作業をこなしていった。
彼女によって暗い夜道にばら撒かれた紙切れは、何かの卵のようにも見える。
腐った木の中に、びっしりと敷き詰められた虫の卵の整列。
乱雑さに紛れ込んだ、奇妙な一致、法則性が、一際異質な世界を演出していた。
その集合、電灯の根本をねぐらにした切片たちは、風になびくこともなく、地面に張り付いていて、人の興味を誘う。\\
「少し隠れるから」\\
エナは私達を連れて、物陰に隠れた。ビルとビルの狭間。
窮屈な空間に押し込まれて、私の胸が圧迫される。
ズルズルと布を引っ掛ける壁。
こんな場所に潜み続けるなんて、我慢できない。
それでも、エナが私達の逃げ道を塞いでいる以上、逃げることは出来ない。

そして彼女は、静止を求めた。息すら止めろと言わんばかりの、彼女の静けさに同調して、セレナと私の呼吸も弱まっていく。
エナの影、その後ろから見える光景に目を凝らせば、数人の通行人が観察できた。
皆足早に過ぎ去っていくが、時折に紙切れを発見して、訝しんだり、足で蹴飛ばそうとしたりしていた。
だが誰も、拾い上げようとはしなかった。そこまで意識を集中させる必要はないということだ。
彼らにとって、所詮それはごみに過ぎない。不届き者な誰かが道端に捨てた、ただの紙くずなのだ。

軽い落胆を、何故か覚えた私に、エナは肩を叩いて視線を誘導させた。
彼女の指す方向に、私は大きな物体を見つけた。

儚い夜空、流動的な表装をした悪夢の姿。
ただそれは多くの場合と違って、ひどく動物的だった。
草食動物特有の、隆起した足や胴の筋肉たち。
鹿に似た体つきに、だが頭を失っている。
断頭された首先、胴体と首の接続部は真っ平らで、本来あるべきと予想される形状を逸脱している。
だが視線を感じる。はっきりと。
その欠落した目で、草花を探るように、散乱した紙片を漁っている。
臭いを嗅いでいるように見えて、気持ち悪い。
妙な現実感が、動物的な動きの突発性を持って、私を度々驚かせる。
なにかに気付いたのか、顔を\scalebox{3}[1]{―}そう思えるのは胴や脚の筋肉の動きからだ\scalebox{3}[1]{―}
しきりに持ち上げて、周囲を見渡す。
それは極めて本能に従った警戒行動だと思うのは、自然なことだろう。
新しい型の悪夢だ、私はそう思った。

そして同意を求めるために、
振り向いた先のエナとセレナの目つきは、完全に狩人のそれだった。

\subsection*{(3)}
険しい表情をしたエナは、私にそっと話しかけた。
聞こえギリギリの音量で、彼女はこう囁いた。\\
「足を撃って」\\
ホルダーにしまい込まれていた私の銃を指さしながら「できるでしょ」と聞いてくる。\\
「この距離だったら、一発でいけますよ」\\
私はそんな必要はないと思っていた。間違いなく仕留められる。
今までがそうだったからだ。

だが彼女の質問は、私の反論を尽くねじ伏せた。\\
「あれのどこが急所なのか、君は分かるの?」\\
確かに、わからない。アレに心臓があるとは限らない。小さいものだったら、ほぼ確実に
急所らしい場所にダメージを負わせることができるが……その通りだ、私には確証がない。
下手に撃って逃げられるより、あからさまな駆動部を破壊するほうが賢明だろう。
大きなエネルギーは、きっと脚を吹き飛ばしてくれるはず。
私は頷いて、彼女の言葉に従った。

狙いを定めて、私は脇を締める。
ブレを無くすために、息を止める。
大きく息を吸って、瞬時に、自らを殺す。
静寂に耳が慣れきって、ささやかな車の駆動音すら煩わしく感じる。
そんな夜に、うごめく悪夢を撃つ。

一発、二発、三発、四発、ちょうど四肢を破壊するつもりで、私は発射した。
そのうちの二つ、右の前足と後ろ足を狙った初弾と二つ目の弾丸が、命中した。

自分が標的であることを理解した時、悪夢はもう遅かった。
私の銃声を合図に、エナとセレナは物陰から飛び出した。
私を置いて。
その速さは凄まじく、一歩一歩の動きは逆に地面が後ろに流れていくように、
彼女達は私の瞬きが終わるまでに、悪夢へ到達していた。

鞘から取り出した剣を、まだ健在な二脚に叩きつける。
エナは左前に、セレナは左後に。
切れて飛んでいく脚の残骸を目で追って、迸る血溜まりを踏み越して、私も前に走る。
彼女たちの連撃は、私の介入を拒んでいた。
激しく切り刻まれて、砕けた脚をばたつかせながら、悪夢は抵抗を試みる。
神経の伝達か、それとも反射の時間を経過する。
その途端に蹴りは精度をまして、セレナの顔、エナの腹を撲る。
吹き飛ばされる二人。
セレナが私の真横を通り過ぎていった。
エナは真逆に。
私は振り向き、その後を追うべきか悩んだ。
結局、私は自分が何を持っているのかにも気付かずに、二人の介護に向かった。
まずは近いセレナから。

そして当然、それは間違いだった。

死に物狂いな生命は、何よりも凶器になりうる。
私はそれを学ばなかった。

だから私は、後ろから突進する悪夢に衝突され、あっけなく地面に倒れ込んだ。
顔面をアスファルトに打ち付けて、鼻血が出る。
息が苦しい。脊髄への衝撃が自律神経を混乱させている。

眩しい。

電灯の灯り。

苦しみに悶ながら、私は立ち上がるエナを見つけた。

いきり立った悪夢の、その後ろで、墓場から起き上がる死者のような気だるさ緩慢さを伴って、昏睡からの蘇生。
彼女の姿は影が喰ってしまった。

私に割かれた注意を良いことに、エナは奇襲に成功したようだ。

私は目を見張った。アヤメの剣もそうだったが、エナの得物も、かなり凝った機構を備えていた。
刃は二つに分離して鋏のように振る舞うかと思えば、完全に根本から分かれ、二つの剣になった。
二刀流か、私はそう判断したが、しかしそれはまた独創的な攻撃方法の前哨だったのだ。
エナは分かれた片刃の剣を、悪夢を挟み込むように構えた。
そのままに振り下ろして、肉を断つ。
ただ、簡単に切れるわけでもない。太い筋繊維\scalebox{3}[1]{―}あくまでも仮の話だが\scalebox{3}[1]{―}
は用意に互いを手放さない。
そこで先の機構の逆順が役立つのだ。
彼女は留め具に再び二つの刃を重ねて、そのまま梃子のように動かした。
それこそ鋏だ。てこの原理に支えられて、刃はあっさりとはいかないが、肉を断ち切った。
皮膚の反対側まで突っ切った刃には、ドロドロとした気味悪い粘着質な物体が纏わりついていた。
脂肪の塊みたいだった。汚れた刃先を、エナは自らの長いコートの裾で拭った。

歴戦の狩人と言っても、てこの力を借りたとしても、エナはやはり人間的な範囲での力しか出せないようだった。
息が荒く、剣を杖にして辛うじて起立している。

だが彼女の得た対価に比べれば、十分なものだろう。
無残にも破断した悪夢の胴体。精肉店に吊るされている、豚肉のような惨めさ。
滴る黒いインクは、夜の深い影を更に深化させていく。墨汁を一滴垂らした水たまり。
水浸しになっていく路上。

私はとりあえず、セレナの方に向かった。
目を瞑って、深い息で眠っている。痛みに対しては、これが一番効果的な処方だと、
彼女の体が判断したのだろうか。私はそっと抱え起こして、ビルの壁に彼女をかけた。

一旦の安全を確保して、私はもう一度悪夢に注目した。

……消えない。

いつまでも、悪夢の死体は消えることがない。

私は嫌な予感がした。

呼吸を整えるために、空を仰いで深呼吸を繰り返すエナの眼の前で、
彼女が昏睡から目覚めたと同じように、悪夢もまた、再びの生を得ようとしていた。

目につく肌色。露出した腕は、悪夢の両断面から生えるものだった。
それぞれが対となって、ヒューマンチェーンの手の掴み方、
互いの手首と手首をがっしりと握る結合の仕方をしている。
そして互いを引っ張り合って、離れ離れになった胴を接合する。
断面の密着と同時に、接着剤のように肌色のペーストは噴出して、一瞬で固形化、次第に暗い青へと変色していく。
止血を兼ねた機構だろう。
そして最後に、使い物にならない自らの足をパージして、新しい脚を生やす。
生えたのは脚だが腕だ。
血色の良い肌色は最初だけで、ほんの数秒後には、
青い線、静脈を観察できるほど、白く透き通った不健康な骨張った手に成長していった。
四本腕の化物の再誕。それを間近にして、正気でいられる人間など居るのだろうか。
私は後退りした。全くの無意識のうちに、逃げ道へ脚を動かしていた。
薄膜を破って再生した手には、ところどころ湯気だって、暖かなその中を連想させる。
気持ち悪さに支配された空間。オブジェの奇抜さに目を閉じたい。

おぼつかない腕の運び。まだ歩くことを、膨大な遺伝情報の彼方から思い出せていない子鹿のように、
震えながら前進する悪夢。戦うことなど放棄して、ただ自らの破滅から逃げるだけ。
ついさっきまで死に体だったはずなのに、悪夢は一つの場所に向かって、生きることにしがみついていた。

私は、濁った水の流れを読む。
悪夢がそれを辿っているように見えたからだ。

チョロチョロと、傾斜もない平面を、コンクリートの僅かな凸凹を頼りに、爬行する液体の終点は、
先に餌として巻かれた紙切れたちだった。

流れに身を任せて、悪夢は負の集合へと導かれる。

初めて乳房から母乳を与えられる赤子のようだった。

僅かな水に群れる、乾ききった人々の水飲み。
インクの滲みを吸い出して、ふやけた繊維を啜って、苦しみを舐め取っていく。

這いつくばる悪夢に、セレナは気付いた。

私の後ろで、いつの間にか目覚めて、その長い剣先を悪夢に据えている。
再誕後の悪夢は、私達の目を釘付けにしていたが、セレナは例外だった。
一歩、また一歩と静かに近づいていく。
よろける脚をその度に、地面に押し付けながら。

音もなく接近して、セレナは剣を持ち上げる。

掲げる剣は、力の象徴だ。

握る腕を緩めて、地面に下す。

その時だった。

赤子の泣き声が聞こえる。
耳鳴りに近い不確かさ。
苦しみに泣く無力な声。
不快な音声。
遠くからなのか、私はそう思って後ろを振り返ったり、はるか前方に目を凝らしたりした。
だがどこにも音源はない。私には検知できなかった。

しかし、セレナには自明だったようだ。

震えるセレナの手。重力に抵抗する内に、血流が乏しくなる。
それだけではないだろう。彼女の震えは明らかに怯えだった。

聞こえる音に変化が生じる。

本当に微妙な、けれど、確かに心を狂わせる差異。
赤子とそうでないものの啼泣に、私は内臓を鷲掴みにされる。

少女の泣き声。

啜り泣く雑音。

ズルズルと鼻水を啜ったり、狂った呼吸を矯正する喘ぎ。

怒りも喜びも悲しみもない、虚空への落涙。

私には理解できない、しかし強制的に共鳴させられる声だった。
心の固有振動数、それを後ろ盾に、体を引き裂くぐらいに私を振動させる。

それ以外に何も思考できなくて、自分の手に持つものすら忘れて、私はただこの声に逆らおうとした。

ただその渦の中で、辛うじて見つけたもの。
セレナの顔は、酷く歪んでいた。
歯をむき出しにして、目をむき出しにして、自分を自分で抑えていた。

ガタガタと膝は笑って、もう限界に近づいていた。

セレナを助けたい。

だが、この声から離れられない。
少しでも気を許せば、一気に流れ込んでくる。
苦しみも悲しみもない無が。
人を一番に苦しめる、永遠という苦痛。
その恐れに抗えず、私は我が身可愛さにその場で留まるしかない。

もつれる手足に、剣は支えきれなくなるのも時間の問題だった。

\subsection*{(4)}

……悪夢がセレナを見る。

有りもしないつぶらな瞳で、しかしその視線を切に感じる眼差しを。
冷たい餞別。自らを殺すものに対する侮蔑。
完全に人のそれだった。

だからセレナは躊躇してしまう。

あの声に、その目に、何かを重ねてしまう。

紙くずを漁り終わって、悪夢は大きな伸びをした。
まるで、セレナはもはや己の敵ではないと高をくくった態度に、彼女はさらなる怒りを覚える。
だが事実だった。セレナは剣をそのままに、悪夢を『守って』いた。
振り下ろせば一撃で、おそらく死に至るはずなのに、彼女はそれを出来ない。

必死な食事で肥え太った悪夢は、その膨れ上がったゼリー状の腹を引きずりながら、
四本の腕を器用に動かして、セレナの元を去っていく。
しかし喉元に突き立てられた死をやり過ごした開放感が、彼の体を軽くしていく。
引きずり、黒い尾を引く水風船。

その後に倒れ込むセレナ。
動悸に苦しんでいる。

何が彼女をここまでにするのか。
きっと、彼女以外に分かる人間は、この場所に存在しない。

顔を覆って、塞ぎ込んでしまう。

……。

闇に薄れる悪夢。
真綿の寝床に埋まる肢体を想像しながら、それは自らを希釈させていく。
全てを忘れて、何事もなかったかのように。
また再び、世界に巣食う\ruby{蝗}{イナゴ}となるために、人々の夢の彼方に流れて、盲いた俗世へと。

\section*{\tt Clingy\_Rain(drop)}
雨音に傘が響いて、私は彼女の言葉に余計、意識を割かなければいけなかった。

大きな雨粒が、張ったビニールの膜に衝突して、際立つ音を鳴らす。

酷い雨だ。建物の外に出たくない。それでも、私は自動ドアをくぐらなければならなかった。
外に、彼女が待っていたからだ。彼女の教室がある一般科棟から、わざわざI科棟まで来るのだ。
校門前で待っていればいいのだと、いつも言っているが、彼女は頑なに通い続けた。

「エナ先輩も\scalebox{3}[1]{―}狩人だったんですね」\\
セレナの言葉は、どこか失望に染まっていた。今日出会って一言目がそれだ。
私は、少し悲しくなった。\\
「私も、セレナが狩人になってたなんて知らなかった」\\
「言っても分からないと思って」\\
「私も同じだよ」\\
彼女は黙った。私はここで理解できた気がする。彼女は失望しているのではない。
やり場のない迷いに閉じ込められているんだ。私が狩人だったこと。
それが彼女にとって、少なからず不満な結果をもたらすことなのだろう。

雨に濡れた白金の長髪は、綺麗に輝いている。
傘を持っていないようだった。朝は快晴で、雨が降るなど誰が予想できたか。
だが、この街には『弁当忘れても傘忘れるな』なんていう言葉もあるぐらいだ。
予期しない雨には、常に備えるべきなのだが、だからといって誰かを責められる理由にはならない。
私は、自分の体を気持ち細めて、彼女のための場所を作った。
もともと大きな傘ではないし、彼女を入れれば、確実に私の半身は濡れるが、
彼女をそのままにしておくよりは、よっぽど賢明なことだ。

入って、と私は言った。

失礼しますと、彼女は縮こまって、私に寄り添った。

彼女の告白を受けてから、初めて私達はこんなにも接近した。
あの接吻は除いてだ。
雨に透けた肌が、粘りつく雨を錯覚させる。
だが実際、夏の雨はそんなものだろう。ジメジメとした湿度と、生温かい雫が絶えず降り注ぐ。
今日みたいな豪雨だと、地面からの反射もあって、足元は無防備に温さに襲われる。
靴下が雨水を吸いきって駄目になる前に、或いは耐え難い湿り気と痒みに苛まれる前に、家に帰ろう。
私は提案した。\\
「家って、先輩の家ですか」\\
「そこしかないよ」\\
「……上がっても、良いんですか」\\
「恋人なのに?」\\
「でも\scalebox{3}[1]{―}まだ早い気がして」\\
「遠慮しないでよ。一人暮らしだし」\\
しばらくして「はい」と少し赤らむセレナ。俯く顔は照れ隠しか。
校門前まで続く長い坂を下って、何度も滑りかける。
水の流れは低きに、人の流れも低きに。
開いた傘は水たまり模様。
運動する円形。
その流れから、私たちは程なくして離れていった。

「私のやり方に、カナンは不満を持ってる」\\
「そう、みたいですね」\\
「セレナはどうなの」\\
「私は……先輩のやり方で良いと思ってます」\\
「本当に?」\\
「\scalebox{3}[1]{―}正直、嫌になります。
やり方もそうだけど、先輩がこんなことをやらなくちゃいけないってことに対しても」\\
「私を心配してくれるんだ」\\
「当たり前です」\\
「それじゃあ、狩人を続けることも」\\
「はい。私は、なってほしくない」\\
「……そうかもしれないけど、これは私の決めたことだから」\\
「はい」

彼女はそれきり、黙り込んでしまった。

私の住んでいるアパートは、こじんまりとした町の一部を構成している。
特に主張もない、平凡な設計と外見。けれど値段の割には、快適だと私は満足している。

鍵をカバンから取り出して、私はドアを開けた。

いつもの帰宅だが、随伴する人間が居るのは初めてだ。
自分の内緒な部分をさらけ出すことに、私は慣れていなかった。
だから、間違いに気付くのが遅すぎた。
片付けをしていなかったのだ。私は彼女を家に呼ぶべきではなかったと後悔した。
こんなにもゴチャゴチャした部屋は、堕落の象徴以外の何物でもない。
雨に晒される彼女を見かねて、安易に自分の部屋へ招待した己の浅はかさを恥じる。
彼女は私をどう思うだろうか。こんな部屋を見て。
だが彼女は、私の想像を裏切った。
もちろん、たぶんいい意味で。\\
「なんだか、落ち着きますね」\\
それがただのお世辞だとしても、私は少しだけ安心できた。

カーテンを開けようと思ったが、外は雨だ。
曇天を見上げても、気分は沈むだけだろう。だったら、青いカーテンのままでいい。\\
「座って」\\
私は座布団をベッドの下から引っ張り出して\scalebox{3}[1]{―}埃かぶってしまっていたが、それを払って\scalebox{3}[1]{―}
適当にテーブルの前に置いた。レイアウトはごく普通の、一人暮らしの狭い部屋だ。
人一人を座らせるだけで、他人の居場所がなくなってしまう。
彼女を座らせて、私はベッドの上に腰掛けた。いつもベッドの上で生活しているようなものだし、
だからこそ、なおさらに汚れてしまっているはずだ。

「タオルとかって、ありませんか?」\\
彼女は遠慮がちに聞いてきた。\\
「ああ! ごめんね。気が利かなくて。ちょっと待って」\\
私は完全に無関心だった。彼女が濡れていること、そんなことにすら気配りできなかった。
動転しそうだ。\\
「ごめん、バスタオルしかないや」\\
こんなに大きな布を渡されて、逆に困るんじゃないんだろうか。
そう思って口が滑った。\\
「全然、大丈夫ですよ。ありがとうございます」\\
受け取ったバスタオルで、髪を拭き始める。
夏の雨、温いはずの雫。だけど彼女は凍えているように見えた。
小刻みに震える手足。赤くしもやけている。\\
「寒い?」\\
「えつ\scalebox{3}[1]{―}あの」\\
はい、と恥ずかしがりながら言った。どうしてだろう。私は疑問に思った。
けれど、すぐに考える。そして私は、彼女が手を握ってほしいと思っている、そう結論づけた。\\
「じゃあ、温めてあげる」\\
彼女の手を包み込む。冷たかった。想像以上に、冷たい手先に驚きながら、私は自らの温もりを与える。

白い肌に赤い血の気は極端に映える。循環する血液の、保温をするために、私は暫くの間彼女のの手を握りしめていた。
汗ばむぐらいまで。

\section{}
性的な感覚を感じるということは、自身の体温の変化である。
人は絶えず体温を上昇させ続けている。
絶えず下がっていく自らの熱を、保ち続けるためにだ。
その永遠の作用の中に、温もりが介在すると、人は心理的な動悸を覚える。
温もりは実際、物質的な作用に依らないが、しかし多くの場合はそれに依拠している。
肌が触れ合う、あるいは極限まで接近すること、それが温もりの伝達である。
温もりが与えられ、その熱さ、あるいは冷たさに見を縮こまらせる行動原理。
肌の温度と乖離した\scalebox{3}[1]{―}熱さと冷たさのどちらか\scalebox{3}[1]{―}
驚くほどの温もりを得て、それに近づこうとする発汗や呼吸の乱れ、
つまり、体温をその温もりと同化させようとする働きが、恋である。
逆に、その温もりを排除して、本来の体温を取り戻そうとする働きが、嫌悪である。
肌に触れる熱さに震え、冷たさに身を静止させて、自らの熱を捨てることができるのは、愛し合うことの動物的な側面であろう。
そして嫌悪すること、自らの領域を保全しようとする心理的な作用は、愛し合うことの人間的な側面だろう。

そのどちらにも一致しない行動。
依存とは温もりを永遠に無へと還す諸作用のことである。
肌に触れ続ける、或いは延々と粘膜の接触を維持し続けること。
己の熱すらも忘れるほど、混じり続ける温もり。エントロピーの増大。
煩雑さをまして、しかしその回復をしない。
溜め込み続けるエントロピーは、いずれ熱的な死をもたらすように、
依存によって癒着した肌、温もりは、いずれ両者の破滅的な最後をもたらすだろう。
\end{document}